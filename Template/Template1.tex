% !TEX program = lualatex
%==============================================================================
% プリアンブル (Preamble)
%==============================================================================

% ===== ドキュメントクラス =====
% LuaLaTeX向けの日本語縦書きにも対応した標準的なクラス
\documentclass[
  a4paper,
  11pt,
]{ltjsarticle}

%------------------------------------------------------------------------------
% パッケージ読み込み
%------------------------------------------------------------------------------

% ===== フォント・言語設定 (LuaLaTeX専用) =====
\usepackage{luatexja-fontspec} % fontspecを日本語向けに拡張し、モダンなフォントシステムを有効化
% \usepackage{luatexja-preset} % (代替案) システム標準の和文フォントを自動設定する場合に利用

% ===== レイアウト関連 =====
\usepackage[margin=2.5cm]{geometry} % 余白を簡単に設定
\usepackage{graphicx}          % 画像の挿入 (\includegraphics)
\usepackage{booktabs}          % 見栄えの良い表 (\toprule, \midrule, \bottomrule)
\usepackage{float}             % 図表の位置調整 (`[H]`オプション)
\usepackage{wrapfig}           % 文章中への図の回り込み

% ===== 数式・物理単位関連 =====
\usepackage{amsmath}           % 高度な数式環境 (align, pmatrixなど)
\usepackage{amsthm}            % 定理環境 (proofなど)
\usepackage{newtxmath}         % 数式用フォント (Times系)
\usepackage{siunitx}           % 物理単位の記述を統一 (\SI{100}{\ohm})
\usepackage{cancel}            % 数式の打ち消し線 (\cancel)

% ===== 図表・グラフ描画関連 (TikZ/PGF) =====
\usepackage{tikz}
\usepackage{svg}               % SVGファイルの直接挿入を可能にする
\usepackage{circuitikz}        % 電気回路図の描画
\usepackage{pgfplots}          % 高機能なグラフ描画
\usepackage{pgfplotstable}     % グラフ描画のための表データ操作
\pgfplotsset{compat=1.18}      % pgfplotsのバージョン互換性を保証
\usepgfplotslibrary{statistics} % 統計計算(回帰分析など)ライブラリ
\usetikzlibrary{positioning}   % ノードの相対配置

% ===== プログラミング・アルゴリズム関連 =====
\usepackage{listings}          % ソースコードの表示
\usepackage{algorithm}         % アルゴリズム記述のフロート環境
\usepackage{algpseudocode}     % algorithm環境内で使う疑似コード

% ===== その他 =====
% hyperrefは、原則として最後に読み込むことで他のパッケージとの競合を避ける
\usepackage[
  colorlinks=true,      % リンクに色を付ける
  linkcolor=blue,         % 内部リンクの色 (目次など)
  citecolor=green!60!black, % 参考文献リンクの色
  urlcolor=cyan,          % URLリンクの色
  hidelinks,              % PDF上でリンクの枠を非表示にする (colorlinks=trueと併用推奨)
]{hyperref}

%------------------------------------------------------------------------------
% 各種設定
%------------------------------------------------------------------------------

% ===== フォント設定 =====
% --- 欧文フォント ---
% TeX Liveに標準で同梱されており、環境依存が少ないLatin Modernフォントを指定
\setmainfont{Latin Modern Roman}
\setsansfont{Latin Modern Sans}
\setmonofont{Latin Modern Mono}

% --- 和文フォント ---
% 注意: ご利用の環境にインストールされているフォント名を指定してください。
% 指定したフォントが存在しない場合、コンパイルエラーの原因となります。
% (例: "IPAexMincho", "Noto Serif CJK JP", "Hiragino Mincho ProN")
\setmainjfont[Renderer=HarfBuzz]{Yu Mincho}
\setsansjfont[Renderer=HarfBuzz]{Yu Gothic}
% \setmonojfont[Renderer=HarfBuzz]{Yu Gothic} % 等幅和文フォントが必要な場合

% ===== ドキュメント情報 =====
\title{実験レポート:高機能テンプレート (改訂版)}
\author{氏名}
\date{\today}

% ===== listings (ソースコード) のスタイル設定 =====
\lstset{
  language=Python,
  basicstyle=\small\ttfamily,
  keywordstyle=\color{blue},
  commentstyle=\color{green!50!black},
  stringstyle=\color{purple},
  showstringspaces=false,
  frame=tb,
  captionpos=b,
  breaklines=true,
  numbers=left,
  numberstyle=\tiny\color{gray},
  xleftmargin=2em, % 左側のマージン
  framexleftmargin=1.5em, % フレームと行番号のマージン
}

% ===== pgfplots (グラフ) の共通スタイル設定 =====
\pgfplotsset{
  % レポート内の全グラフに適用する共通スタイルを定義
  report-style/.style={
    xlabel style={yshift=0.5em}, % x軸ラベルと軸の間のスペース
    ylabel style={yshift=-0.5em},% y軸ラベルと軸の間のスペース
    legend pos=north west,
    grid=major,
    ticklabel style={font=\small},
    label style={font=\small},
    legend style={font=\small},
  }
}

% ===== 数式用カスタムコマンド =====
% Make differential operator definition conditional to avoid double definitions
\providecommand{\dd}{\mathrm{d}} % 微分演算子 d
\newcommand{\mi}{\mathrm{j}} % 虚数単位 j

% ===== algorithmicx のスタイル調整 =====
% Procedure をスモールキャピタルではなく太字で表示(フォント警告回避)
\renewcommand{\textproc}[1]{\textbf{#1}}

%==============================================================================
% ドキュメント本体 (Body)
%==============================================================================
\begin{document}

\maketitle
\tableofcontents
\clearpage

% ===================================================================
\section{はじめに}
% ===================================================================
このドキュメントは,LuaLaTeXを使用した技術レポートのテンプレートです。
特に,\verb|pgfplots|パッケージを用いて,測定データの表と回帰直線付きグラフを自動で生成する機能に焦点を当てています。
技術文書の作法に従い,句読点には「.」と「,」を使用します。
また,和文中のカッコには全角(例:VDEC(VLSI Design and Education Center))を用い,物理量の添え字(例:$V_{\mathrm{pp}}$,ppはpeak-to-peakの略)はローマン体で記述します。
数式 $V_{\mathrm{pp}} = \SI{3}{\volt}$ のように,単位の前には半角スペースを入れます (\verb|\SI|コマンドが自動で処理します)。

図\ref{fig:rlc_circuit}や表\ref{tab:regression_data}のように,図表の相互参照が可能です。
参考文献の引用も簡単です\cite{Scherz2006}。

% ===================================================================
\section{実験方法}
% ===================================================================
ここでは,実験に用いた装置や手順について説明します。

\subsection{実験装置}
実験に用いた回路を以下に示します。

% --- 回路図の例 ---
\begin{figure}[H]
  \centering
  \begin{circuitikz}[american currents]
    \draw (0,0)
    to[sV=$E$] (0,2)
    to[short] (2,2)
    to[european resistor=$R$] (2,0)
    to[short] (0,0);
    \draw (2,2)
    to[short] (4,2)
    to[L=$L$] (4,0)
    to[short] (2,0);
    \draw (4,2)
    to[short] (6,2)
    to[C=$C$] (6,0)
    to[short] (4,0);
  \end{circuitikz}
  \caption{RLC並列回路}
  \label{fig:rlc_circuit}
\end{figure}

図\ref{fig:rlc_circuit}に示す回路を用いました。各素子のパラメータを表に示します。

% --- 表の例 ---
\begin{table}[H]
  \centering
  \caption{素子のパラメータ例}
  \label{tab:params}
  \begin{tabular}{llr}
    \toprule
    素子 & 記号 & \multicolumn{1}{c}{値} \\
    \midrule
    抵抗 & $R$ & \SI{100}{\ohm} \\
    インダクタ & $L$ & \SI{10}{\milli\henry} \\
    キャパシタ & $C$ & \SI{1}{\micro\farad} \\
    \bottomrule
  \end{tabular}
\end{table}

表\ref{tab:params}に示すパラメータを用いました。

% --- 回り込み図のサンプル ---
\begin{wrapfigure}{r}{0.4\textwidth}
  \centering
  \vspace{-15pt} % 上方向のスペースを微調整
  \begin{circuitikz}
    \draw (0,0) to[R=$R_1$] (2,0) to[R=$R_2$] (4,0);
  \end{circuitikz}
  \caption{回り込み図の例}
  \label{fig:wrapped_circuit}
  \vspace{-10pt} % 下方向のスペースを微調整
\end{wrapfigure}
回り込み図のサンプルを図に示します。図\ref{fig:wrapped_circuit}に示すように、文章の途中に図を配置することで,紙面を有効に活用できます。
回り込みを終えたい場所で,適宜改行を挟んでください。

\subsection{測定手順}
測定の手順をここに記述します。
\begin{enumerate}
  \item 手順1
  \item 手順2
  \item 手順3
\end{enumerate}

% ===================================================================
\section{結果と考察}
% ===================================================================
ここでは,測定結果のデータと,それをグラフ化したものを提示し,考察を述べます。

% --- ダミーの実験データを生成 ---
% 目的: 毎回コンパイルするたびに新しいランダムなデータを作成する
% y = 2x + 5 を基本とし、-1から1の範囲でランダムな誤差を加える
\pgfplotstablenew[
    columns={x,y},
    create on use/x/.style={create col/expr={\pgfplotstablerow+1}},
    create on use/y/.style={create col/expr={(2*(\pgfplotstablerow+1) + 5) + rand*2-1}},
]{15}{\regressiondata} % 15行のデータテーブル \regressiondata を作成

\subsection{測定データ}
測定結果を表に示します。この表は,上で生成されたデータから自動的に作成されます。

\begin{table}[H]
  \centering
  \caption{測定データと回帰分析の結果}
  \label{tab:regression_data}
  \pgfplotstabletypeset[
    columns={x,y},
    columns/x/.style={column name=$V$ (\si{\volt})},
    columns/y/.style={column name=$I$ (\si{\milli\ampere}), precision=2},
    every head row/.style={before row=\toprule, after row=\midrule},
    every last row/.style={after row=\bottomrule},
  ]{\regressiondata}
\end{table}

表\ref{tab:regression_data}に示すデータです。

\subsection{グラフと回帰分析}
測定データをプロットし,線形回帰分析を行った結果を図に示します。
グラフと回帰直線は,表\ref{tab:regression_data}のデータを用いて自動的に描画されます。

\begin{figure}[H]
  \centering
  \begin{tikzpicture}
    \begin{axis}[
      report-style, % プリアンブルで定義した共通スタイルを適用
      xlabel={電圧 $V$ (\si{\volt})},
      ylabel={電流 $I$ (\si{\milli\ampere})},
    ]
      % --- 測定データのプロット ---
      \addplot+[only marks, mark=*, blue] table[x=x, y=y] {\regressiondata};
      \addlegendentry{測定データ}

      % --- 回帰直線の計算と描画 ---
      \addplot[red, thick, no marks] table[
        y={create col/linear regression={y=y}},
      ] {\regressiondata};
      
      % --- 回帰式の凡例 ---
      % \pgfplotstableregressiona (傾き) と \pgfplotstableregressionb (y切片) を整形して表示
      \addlegendentry{
        回帰直線: $I = \pgfmathprintnumber[fixed, precision=2]{\pgfplotstableregressiona} V + \pgfmathprintnumber[fixed, precision=2]{\pgfplotstableregressionb}$
      }
    \end{axis}
  \end{tikzpicture}
  \caption{測定結果のグラフと回帰直線}
  \label{fig:regression_graph}
\end{figure}

図\ref{fig:regression_graph}に示すグラフです。

\subsection{考察}
図\ref{fig:regression_graph}から,電圧$V$と電流$I$の間には強い正の相関が見られます。
回帰直線の傾きは$\pgfmathprintnumber[fixed, precision=2]{\pgfplotstableregressiona}$であり,これは回路のコンダクタンスに相当すると考えられます。

% ===================================================================
\section{CSVデータからの表とグラフの自動生成}
% ===================================================================
% 注意: このセクションをコンパイルするには、以下の内容で
% "my_data.csv" というファイルをTeXファイルと同じフォルダに作成する必要があります。
%
% --- my_data.csv の内容例 ---
% voltage,current
% 1,7.1
% 2,8.9
% 3,11.2
% 4,13.5
% 5,14.8
% 6,17.2
% 7,18.9
% 8,21.1
% -----------------------------
\pgfplotstableread[col sep=comma]{\detokenize{my_data.csv}}{\csvdata}

\subsection{CSVデータから生成した表}
外部CSVファイル \verb|my_data.csv| のデータを元に,自動的に生成された表です。

\begin{table}[H]
  \centering
  \caption{my\_data.csv から読み込んだデータ}
  \label{tab:csv_data}
  \pgfplotstabletypeset[
    columns={voltage,current},
    columns/voltage/.style={column name=$V$ (\si{\volt})},
    columns/current/.style={column name=$I$ (\si{\milli\ampere}), precision=2},
    every head row/.style={before row=\toprule, after row=\midrule},
    every last row/.style={after row=\bottomrule},
  ]{\csvdata}
\end{table}

表\ref{tab:csv_data}に示すデータです。

\subsection{CSVデータから生成したグラフ}
\verb|my_data.csv| のデータを元に,自動的に生成されたグラフと回帰直線です。

\begin{figure}[H]
  \centering
  \begin{tikzpicture}
    \begin{axis}[
      report-style, % 共通スタイルを適用
      xlabel={電圧 $V$ (\si{\volt})},
      ylabel={電流 $I$ (\si{\milli\ampere})},
    ]
      % --- CSVデータのプロット ---
      \addplot+[only marks, mark=*, blue] table[x=voltage, y=current] {\csvdata};
      \addlegendentry{測定データ (CSV)}

      % --- 回帰直線の計算と描画 ---
      \addplot[red, thick, no marks] table[
        x=voltage,
        y={create col/linear regression={y=current}},
      ] {\csvdata};
      
      % --- 回帰式の凡例 ---
      \addlegendentry{
        回帰直線: $I = \pgfmathprintnumber[fixed, precision=2]{\pgfplotstableregressiona} V + \pgfmathprintnumber[fixed, precision=2]{\pgfplotstableregressionb}$
      }
    \end{axis}
  \end{tikzpicture}
  \caption{my\_data.csv のデータから作成したグラフ}
  \label{fig:csv_graph}
\end{figure}

図\ref{fig:csv_graph}に示すグラフです。

% ===================================================================
\section{結論}
% ===================================================================
本レポートでは,〇〇の実験を行い,△△という結果を得た。
考察から,□□ということが示唆された。

% ===================================================================
% 付録
% ===================================================================
\clearpage
\appendix
\section{補足情報:数式とグラフのサンプル}

この付録では,レポート作成時に役立つ様々な数式やグラフの記述例を示します。

\subsection{様々な数式の表現 (amsmath)}

\subsubsection{微分・積分}
常微分と偏微分:
\begin{equation}
  \frac{\dd^2 f(x)}{\dd x^2} + \frac{\partial^2 f(x,y)}{\partial y^2} = 0
\end{equation}
時間積分と周回積分:
\begin{equation}
  \Phi = \int_S \vec{B} \cdot \dd\vec{S}, \quad \oint_C \vec{E} \cdot \dd\vec{l} = -\frac{\dd\Phi_B}{\dd t}
\end{equation}

\subsubsection{行列・ベクトル}
\begin{equation}
  \begin{pmatrix} a & b \\ c & d \end{pmatrix}
  \begin{pmatrix} x \\ y \end{pmatrix} =
  \begin{pmatrix} ax+by \\ cx+dy \end{pmatrix},
  \quad
  \det(A) = 
  \begin{vmatrix} a & b \\ c & d \end{vmatrix} = ad - bc
\end{equation}

\subsubsection{場合分け}
\begin{equation}
  f(x) = \begin{cases}
    x^2 & (x \ge 0) \\
    -x^2 & (x < 0)
  \end{cases}
\end{equation}

\subsubsection{連分数}
\begin{equation}
  \cfrac{1}{1 + \cfrac{1}{\sqrt{2}}}
\end{equation}
一般形:
\begin{equation}
  \cfrac{1}{a_0 + \cfrac{1}{a_1 + \cfrac{1}{a_2 + \cfrac{1}{a_3 + \cdots}}}}
\end{equation}

\subsubsection{数式の打ち消し線 (cancel)}
\begin{equation}
  \frac{(x+1)\cancel{(x-1)}}{\cancel{(x-1)}} = x+1 \quad (x \neq 1),
  \quad
  \frac{\cancelto{1}{x}}{\cancelto{2}{2x}} = \frac{1}{2}
\end{equation}

\subsubsection{電気電子工学における数式表現}
\paragraph{複素数}
交流回路の解析では,複素数を用いた表現が不可欠です。
\begin{equation}
  Z = R + \mi X = |Z|e^{\mi\phi}
\end{equation}
ここで,$\mi$ は虚数単位 ($\mi^2 = -1$) です。

\paragraph{ラプラス変換}
線形時不変システムの解析に用いられます。
\begin{equation}
  F(s) = \mathcal{L}\{f(t)\} = \int_0^\infty f(t)e^{-st} \dd t
\end{equation}

\paragraph{伝達関数}
システムの入出力関係を表す伝達関数 $H(s)$ の例です。
\begin{equation}
  H(s) = \frac{V_{\mathrm{out}}(s)}{V_{\mathrm{in}}(s)} = \frac{1}{RCs + 1}
\end{equation}

\paragraph{デシベル表現}
ゲインや減衰量を対数スケールで表現します。
\begin{equation}
  G_{\mathrm{dB}} = 20 \log_{10} \left( \frac{V_{\mathrm{out}}}{V_{\mathrm{in}}} \right)
\end{equation}

\subsubsection{数学記号の表現}

\paragraph{極限}
\begin{equation}
  \lim_{x \to 0} \frac{\sin x}{x} = 1
\end{equation}

\paragraph{総和と総積}
\begin{equation}
  \sum_{k=1}^{n} k = \frac{n(n+1)}{2}, \quad \prod_{k=1}^{n} k = n!
\end{equation}

\paragraph{コンビネーション}
\begin{equation}
  \binom{n}{r} = \frac{n!}{r!(n-r)!}, \quad {}_{n} \mathrm{C}_{k} = \frac{n!}{k!(n-k)!}
\end{equation}

\paragraph{階乗}
\begin{equation}
  n! = 1 \times 2 \times \cdots \times n
\end{equation}

\paragraph{順列}
\begin{equation}
  P(n,k) = \frac{n!}{(n-k)!}, \quad {}_{n} \mathrm{P}_{k} = \frac{n!}{(n-k)!}
\end{equation}

\paragraph{重複組合せ}
\begin{equation}
  H(n,k) = \binom{n+k-1}{k}, \quad {}_{n} \mathrm{H}_{k} = \binom{n+k-1}{k}
\end{equation}

\paragraph{重複順列}
\begin{equation}
  {}_{n} \Pi_{k} = n^{k}
\end{equation}

\subsubsection{代数}

\paragraph{方程式と不等式}
\begin{align}
  x^2 - 5x + 6 = 0 \\
  x^2 + y^2 = r^2 \\
  a^2 + b^2 = c^2 \\
  x^2 < 4
\end{align}

\paragraph{関数}
\begin{align}
  f(x) = ax + b \\
  g(x) = x^2 + 2x + 1 \\
  h(x) = e^x \\
  i(x) = \log x \\
  j(x) = \sin x + \cos x
\end{align}

\subsubsection{数列}

\paragraph{等差数列と等比数列}
\begin{align}
  a_n = a + (n-1)d \\
  a_n = a r^{n-1}
\end{align}

\subsubsection{幾何}

\paragraph{ベクトル}
\begin{align}
  \vec{a} = (a_1, a_2, a_3) \\
  \vec{a} \cdot \vec{b} = a_1 b_1 + a_2 b_2 + a_3 b_3 \\
  \vec{a} \times \vec{b} = (a_2 b_3 - a_3 b_2, a_3 b_1 - a_1 b_3, a_1 b_2 - a_2 b_1)
\end{align}

\paragraph{図形の面積と体積}
\begin{align}
  S = \pi r^2 \\
  V = \frac{4}{3} \pi r^3 \\
  S = a^2 \\
  V = a^3
\end{align}

\subsubsection{確率・統計}

\paragraph{確率}
\begin{equation}
  P(A) = \frac{n(A)}{n(U)}
\end{equation}

\paragraph{統計量}
\begin{align}
  \bar{x} = \frac{1}{n} \sum_{i=1}^{n} x_i \\
  \sigma^2 = \frac{1}{n} \sum_{i=1}^{n} (x_i - \bar{x})^2
\end{align}

\subsubsection{数学的帰納法}
数学的帰納法の証明例:
\begin{proof}
  \textbf{基底ステップ:} $n=1$ のとき成立。 \\
  \textbf{帰納ステップ:} $n=k$ のとき成立と仮定し、$n=k+1$ のときも成立することを示す。
\end{proof}

\subsubsection{論理式}
命題論理や述語論理で用いられる基本的な記号と数式の例です。
\paragraph{命題論理}
\begin{align}
  P \land Q & \quad (\text{P かつ Q}) \\
  P \lor Q & \quad (\text{P または Q}) \\
  \bar{P} & \quad (\text{P でない}) \\
  P \to Q & \quad (\text{P ならば Q}) \\
  P \leftrightarrow Q & \quad (\text{P と Q は同値}) \\
  P \equiv Q & \quad (\text{P と Q は同値}) \\
  P \Leftrightarrow Q & \quad (\text{P と Q は同値})
\end{align}

\paragraph{ド・モルガンの法則}
\begin{align}
  \overline{(P \land Q)} & \leftrightarrow (\bar{P} \lor \bar{Q}) \\
  \overline{(P \lor Q)} & \leftrightarrow (\bar{P} \land \bar{Q})
\end{align}

\paragraph{述語論理}
\begin{align}
  \forall x \in S, P(x) & \quad (\text{集合Sの全ての要素xに対しP(x)が成り立つ}) \\
  \exists y \in T, Q(y) & \quad (\text{集合Tのある要素yに対しQ(y)が成り立つ})
\end{align}

\subsection{様々なグラフの描画 (pgfplots)}

\subsubsection{片対数・両対数グラフ}
\begin{figure}[H]
  \centering
  \begin{minipage}{0.48\textwidth}
    \begin{tikzpicture}
      \begin{semilogyaxis}[report-style, title={片対数グラフ}, xlabel={$x$}, ylabel={$y=e^x$}, height=5cm]
        \addplot[blue, thick, domain=0:5, samples=100] {exp(x)};
      \end{semilogyaxis}
    \end{tikzpicture}
    \caption{y軸を対数スケールにしたグラフ}
    \label{fig:semilog_graph}
  \end{minipage}\hfill
  \begin{minipage}{0.48\textwidth}
    \begin{tikzpicture}
      \begin{loglogaxis}[report-style, title={両対数グラフ}, xlabel={$x$}, ylabel={$y=x^2$}, height=5cm]
        \addplot[red, thick, domain=1:100, samples=100] {x^2};
      \end{loglogaxis}
    \end{tikzpicture}
    \caption{両軸を対数スケールにしたグラフ}
    \label{fig:loglog_graph}
  \end{minipage}
\end{figure}

\subsection{アルゴリズムの記述 (algorithmicx)}
\verb|algorithm|と\verb|algorithmicx|パッケージを用いると,論文などで見られるような疑似コードを記述できます。

\begin{algorithm}[H]
  \caption{Euclidの互除法}
  \label{alg:euclid}
  \begin{algorithmic}[1] % "1"で行番号を表示
    \Procedure{Euclid}{$a, b$}
      \State $r \gets a \pmod b$
      \While{$r \neq 0$}
        \State $a \gets b$
        \State $b \gets r$
        \State $r \gets a \pmod b$
      \EndWhile
      \State \Return $b$
    \EndProcedure
  \end{algorithmic}
\end{algorithm}

% ===================================================================
% 参考文献
% ===================================================================
% BibTeX/biblatexを使用すると、参考文献の管理がより簡単になります。
\begin{thebibliography}{9}
\bibitem{Scherz2006}
P. Scherz and S. Monk, \textit{Practical Electronics for Inventors}, 2nd ed., McGraw-Hill, 2006.
\end{thebibliography}

\end{document}