\documentclass{ltjarticle}
\usepackage{amsmath, amssymb}
\usepackage{graphicx}
\usepackage{url}
\usepackage{hyperref}
\usepackage{geometry}
\usepackage{luatexja}
\usepackage{booktabs} % 表の罫線をきれいにするため

\geometry{a4paper, margin=25mm}

\title{実験報告書:抵抗の測定}
\author{国立長野高専 1年1組1番 電気電子 太郎}
\date{\today}

\begin{document}
\maketitle
\pagenumbering{arabic} % ページ番号を開始

\section{目的}
本実験は、既知の抵抗および未知の抵抗の抵抗値をデジタルマルチメータを用いて測定し、抵抗器のカラーコードの読み方、および測定値の誤差について理解することを目的に実施した。

\section{原理}
\subsection{抵抗カラーコード}
炭素皮膜抵抗などの抵抗器には、抵抗値と誤差の許容範囲(公差)を示す色の帯が印刷されている。本実験で用いる4本帯の抵抗では、第1色帯と第2色帯が抵抗値の最初の2桁の数値を、第3色帯が乗数を、第4色帯が公差を表す。

\subsection{測定誤差}
測定値と真の値との差を誤差という。本報告書では、公称値を真の値とみなし、測定値とのずれを評価するために相対誤差を用いる。相対誤差は式(\ref{eq:relative_error})で算出される。
\begin{equation}
    \text{相対誤差} [\%] = \frac{|\text{測定値} - \text{公称値}|}{\text{公称値}} \times 100
    \label{eq:relative_error}
\end{equation}

\section{実験方法}
まず、用意された3本の抵抗器について、カラーコードからそれぞれの公称値と公差を読み取った。次に、デジタルマルチメータを用いて各抵抗器の抵抗値を3回測定し、その測定値の平均を算出した。最後に、式(\ref{eq:relative_error})を用いて、測定値の平均と公称値との間の相対誤差を計算した。

\section{使用機器}
本実験で使用した機器を、表\ref{tab:equipment}にまとめる。
\begin{table}[h]
    \centering
    \caption{使用機器一覧}
    \label{tab:equipment}
    \begin{tabular}{llll}
        \toprule
        機器名 & メーカー名 & 型番 & 備品番号 \\
        \midrule
        デジタルマルチメータ & HIOKI & DT4256 & E-123 \\
        抵抗器セット & (不明) & (不明) & (なし) \\
        \bottomrule
    \end{tabular}
\end{table}

\section{結果および考察}
表\ref{tab:results}に、抵抗値の測定結果を示す。この表には、カラーコードから読み取った公称値、3回の測定値とその平均値、そして公称値に対する相対誤差をまとめた。

\begin{table}[h]
    \centering
    \caption{抵抗値の測定結果}
    \label{tab:results}
    \begin{tabular}{crrrrrr}
        \toprule
        抵抗器 & 公称値 [k$\Omega$] & 測定値1 [k$\Omega$] & 測定値2 [k$\Omega$] & 測定値3 [k$\Omega$] & 平均値 [k$\Omega$] & 相対誤差 [\%] \\
        \midrule
        R1 & 1.0 & 1.01 & 1.02 & 1.01 & 1.01 & 1.00 \\
        R2 & 4.7 & 4.68 & 4.69 & 4.68 & 4.68 & 0.43 \\
        R3 & 10 & 10.2 & 10.1 & 10.1 & 10.1 & 1.00 \\
        \bottomrule
    \end{tabular}
\end{table}

測定された3本の抵抗器の相対誤差は、すべて公称値の公差である$\pm 5\%$の範囲内に収まっており、測定は正しく行われたと考えられる。測定値のばらつきは小さく、デジタルマルチメータによる測定が安定していることがわかる。誤差の主な原因としては、抵抗器自体の製造公差が考えられる。

\section{報告事項}
\subsection{抵抗のカラーコードの読み方について説明する}
抵抗のカラーコードは、抵抗値と公差を色の帯で示したものである。例えば、本実験で用いた抵抗R1のカラーコードは「茶黒赤金」であった。これはそれぞれ数値の1、0、乗数$10^2$、公差$\pm 5\%$に対応する。したがって、抵抗値は $10 \times 10^2 = 1000\Omega = 1.0\text{k}\Omega$、公差は$\pm 5\%$と読み取ることができる。

\subsection{なぜ測定を複数回行う必要があるのか考察する}
測定には常に誤差が伴う可能性がある。測定を複数回行い、その平均値をとることで、偶然誤差の影響を低減し、より信頼性の高い測定値を得ることができる。また、測定値のばらつき(標準偏差など)を評価することで、測定の安定性を評価することも可能となる。

\begin{thebibliography}{9}
    \bibitem{manual} 国立長野高専 電気電子工学科, 実験報告書の書き方 詳細分析, 平成30年12月版.
    \bibitem{web} アールエスコンポーネンツ, 抵抗器のカラーコード, \url{https://jp.rs-online.com/web/generalDisplay.html?id=ideas-and-advice/resistor-colour-code-guide} (2025年7月7日閲覧).
\end{thebibliography}

\end{document}
