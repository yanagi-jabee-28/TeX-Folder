% ===== ドキュメントクラス =====
\documentclass[
  a4paper,
  11pt,
]{ltjsarticle}

% ----- フォント・数式関連 -----
\usepackage{newtxtext}
\usepackage{newtxmath}
\usepackage{amsmath}
\let\Bbbk\relax
\usepackage{amssymb}
\usepackage{bm}

% ----- 画像・単位・レイアウト関連 -----
\usepackage{graphicx} 
\usepackage[export]{adjustbox}
\usepackage{siunitx}
\usepackage{float}
\usepackage{placeins}
\usepackage[margin=25mm]{geometry}
\usepackage{booktabs}
\usepackage{url}

% ----- グラフ描画機能 -----
\usepackage{tikz}
\usepackage{pgfplots}
\pgfplotsset{compat=1.18}
\usetikzlibrary{arrows.meta, positioning, calc, shapes.geometric}

% 画像の高さ制限設定
\setkeys{Gin}{keepaspectratio}

\newcommand{\includefig}[2][]{\includegraphics[max size={\textwidth}{0.45\textheight},#1]{#2}}
\raggedbottom

% ----- 参考文献の上付き表示設定 -----
\makeatletter
\def\@cite#1#2{\textsuperscript{[{#1\if@tempswa , #2\fi}]}}
\makeatother

\usepackage[colorlinks=true,linkcolor=black,citecolor=black,urlcolor=black,anchorcolor=black]{hyperref}

% ===== ドキュメント情報 =====
\title{4. PIDによる温度制御}
\author{電気電子工学科 X年 X組 番号 氏名} 
\date{2025年12月4日}

% ===== ここからドキュメント本体 =====
\begin{document}

\maketitle

\section{目的}
ワンボードマイコンを用いてヒーターをPID制御する方法について理解する.

\section{PID制御}
三つのパラメータ proportional(比例),integral(積分),derivative(微分)のゲインを直感的にチューニングするだけで,ある程度の性能が確保できる制御方法であるため,広く使われている.

\subsection{PID制御の伝達関数}
PID制御における操作量 $u(t)$ と偏差 $e(t)$ の関係は式(1)のようになる.また,式(1)の伝達関数は式(2)となり,伝達関数のブロック図は図1となる.ここで $K_P$ は比例ゲイン,$K_I$ は積分ゲイン,$K_D$ は微分ゲイン,$T_I$ は積分時間,$T_D$ は微分時間となる.

\begin{equation}
  u(t) = K_P e(t) + K_I \int_{0}^{t} e(\tau) d\tau + K_D \frac{de(t)}{dt}
  \label{eq:pid_time}
\end{equation}

\begin{equation}
  K_C(s) = K_P + K_I \frac{1}{s} + K_D s = K_P \left( 1 + \frac{1}{T_I s} + T_D s \right)
  \label{eq:pid_tf}
\end{equation}

% --- 図1 PID制御ブロック図(修正版) ---
\begin{figure}[H]
  \centering
  \begin{tikzpicture}[auto, node distance=2.0cm, >=Latex]
    % Styles
    \tikzstyle{block} = [draw, rectangle, minimum height=3em, minimum width=4em]
    \tikzstyle{sum} = [draw, circle, inner sep=1pt]
    \tikzstyle{input} = [coordinate]
    \tikzstyle{output} = [coordinate]
    \tikzstyle{branch} = [circle,inner sep=0pt,minimum size=1mm,fill=black,draw=black]

    % Nodes definition
    \node [input] (input) {};
    \node [sum, right=1cm of input] (sum1) {};
    \node [left=0.1cm of sum1] {$+$};
    
    % Branch point
    \node [branch, right=1cm of sum1] (b1) {};
    
    % PID Blocks (Layout: KP in center, KI above, KD below)
    \node [block, right=1.5cm of b1] (kp) {$K_P$};
    \node [block, above=1.8cm of kp] (ki) {$\displaystyle \frac{K_I}{s}$}; 
    \node [block, below=1.8cm of kp] (kd) {$K_D s$}; 

    % Summing point 2
    \node [sum, right=2cm of kp] (sum2) {};
    
    % Dashed Frame for Controller
    % Calculate coordinates to enclose the blocks nicely
    \draw[dashed] ($(ki.north west)+(-0.8,0.6)$) rectangle ($(kd.south east)+(0.5,-0.6)$);
    \node at ($(ki.north)+(0,0.4)$) {\small PID制御器 $K_C$};

    % Connections
    % Input to Sum1
    \draw [->] (input) -- node {\small 目標値} (sum1);
    \draw [-] (sum1) -- (b1);
    
    % Branching to P, I, D
    \draw [->] (b1) -- (kp);
    \draw [->] (b1) |- (ki);
    \draw [->] (b1) |- (kd);
    
    % Output from P, I, D to Sum2
    \draw [->] (kp) -- (sum2);
    \draw [->] (ki) -| (sum2);
    \draw [->] (kd) -| (sum2);

    % Plant and Output
    \node [block, right=1.5cm of sum2] (plant) {制御対象};
    \node [output, right=1.5cm of plant] (output) {};

    \draw [->] (sum2) -- (plant);
    \draw [->] (plant) -- node [name=y] {\small 出力} (output);
    
    % Feedback Loop
    % Increased downward shift to avoid collision with KD block
    \draw [->] (y) -- ++(0,-3.5) -| node[pos=0.99] {$-$} (sum1);
    
  \end{tikzpicture}
  \caption{PID制御のブロック図(並列型)}
  \label{fig:pid_block1}
\end{figure}

% --- 図2 標準形(修正版) ---
\begin{figure}[H]
  \centering
  \begin{tikzpicture}[auto, node distance=1.5cm, >=Latex]
    \tikzstyle{block} = [draw, rectangle, minimum height=2.5em, minimum width=3.5em]
    \tikzstyle{sum} = [draw, circle, inner sep=1pt]
    \tikzstyle{input} = [coordinate]
    \tikzstyle{output} = [coordinate]
    \tikzstyle{branch} = [circle,inner sep=0pt,minimum size=1mm,fill=black,draw=black]

    % Frame
    \draw[dashed] (1.2,-2.5) rectangle (7.0, 2.5);
    \node at (4.1, 2.7) {\small PID制御器(標準形)};

    % Nodes
    \node [input] (input) {};
    \node [sum, right=1cm of input] (sum1) {};
    \node [left=0.1cm of sum1] {$+$};
    \node [block, right=0.8cm of sum1] (kp) {$K_P$};
    
    \node [branch, right=0.6cm of kp] (b2) {};
    
    \node [block, right=1.8cm of kp] (one) {$1$};
    \node [block, above=1.2cm of one] (ti) {$\displaystyle \frac{1}{T_I s}$};
    \node [block, below=1.2cm of one] (td) {$T_D s$};
    
    \node [sum, right=1.5cm of one] (sum2) {};
    \node [block, right=1.2cm of sum2] (plant) {制御対象};
    \node [output, right=1.5cm of plant] (output) {};

    % Connections
    \draw [->] (input) -- node {\small 目標値} (sum1);
    \draw [->] (sum1) -- (kp);
    \draw [-] (kp) -- (b2);
    
    \draw [->] (b2) -- (one);
    \draw [->] (b2) |- (ti);
    \draw [->] (b2) |- (td);
    
    \draw [->] (one) -- (sum2);
    \draw [->] (ti) -| (sum2);
    \draw [->] (td) -| (sum2);
    
    \draw [->] (sum2) -- (plant);
    \draw [->] (plant) -- node [name=y] {\small 出力} (output);
    \draw [->] (y) -- ++(0,-3.5) -| node[pos=0.99] {$-$} (sum1);
  \end{tikzpicture}
  \caption{PID制御のブロック図(標準型)}
  \label{fig:pid_block2}
\end{figure}

PID制御器では三つのゲイン $K_P, K_I, K_D$ が用いられるが,制御系におよぼすそれぞれの要素の影響や効果は次の通りである\cite{toyohashi}.

\begin{description}
    \item[比例ゲイン:$K_P$] \mbox{} \\
    偏差の現在値に比例した制御入力を求める項.このゲインを上げるとシステムの応答性が増すと同時に,目標値の変化や外乱などによる偏差を抑制することができる.
    \item[積分ゲイン:$K_I$ ($K_P / T_I$)] \mbox{} \\
    偏差の積分値(偏差の累積)に応じた制御入力を求める項.このゲインを上げると低周波外乱の出力への影響を効果的に抑制することができる.
    \item[微分ゲイン:$K_D$ ($K_P T_D$)] \mbox{} \\
    偏差の微分値(増減の動向)に応じた制御入力を求める項.このゲインを上げるとシステムの速応性を増すことができる.
\end{description}

\subsection{パラメータ調整}
\textbf{ステップ応答法} \\
式(3)のような1次遅れ要素とむだ時間要素で近似的に表せるシステムを対象とし,開ループにおける制御対象のステップ応答波形を用いる.手順は以下の通りである.

\begin{equation}
  G_P(s) = \frac{K}{1+sT}e^{-Ls}
  \label{eq:plant}
\end{equation}

\begin{enumerate}
    \renewcommand{\labelenumi}{(\roman{enumi})}
    \item ステップ応答の測定結果,またはステップ応答の計算値から図2のむだ時間 $L$,変曲点の傾き $R$ を読み取る.
    \item 表1に示す値に各パラメータを設定する.
\end{enumerate}

% --- 図2 ステップ応答波形 ---
\begin{figure}[H]
  \centering
  \begin{tikzpicture}[>=Latex, xscale=5, yscale=2.5]
    % Axes
    \draw[->] (-0.1,0) -- (1.2,0) node[below] {$t$};
    \draw[->] (0,-0.1) -- (0,1.2);
    
    % Step Input
    \draw[thick] (0,0) -- (0,1) -- (1.1,1);
    \node at (0.3, 1.3) {\small ステップ入力};
    \node at (-0.05, 1) {\small 1};

    % Step Response Curve
    \draw[thick, domain=0:1.1, smooth, samples=100] plot (\x, {1 / (1 + exp(-20*(\x-0.3))) * 0.6});
    \node at (0.6, 0.4) {\small ステップ応答};
    
    % K level
    \draw[dashed] (0,0.6) node[left] {$K$} -- (1.1,0.6);
    
    % Tangent line
    \draw[thin] (0.15, 0) -- (0.45, 0.7);
    
    % L and T dimensions
    \draw[dotted] (0.2, 0) -- (0.2, -0.2);
    \draw[dotted] (0.4, 0) -- (0.4, -0.2);
    
    \draw[<->] (0, -0.1) -- node[fill=white] {$L$} (0.2, -0.1);
    \draw[<->] (0.2, -0.1) -- node[fill=white] {$T$} (0.4, -0.1);
    
    % R indication
    \node at (0.7, 0.25) {$\displaystyle R = \frac{K}{T}$};
    
    % Point indication
    \node at (0.3, 0.3) [circle,fill,inner sep=1pt]{};
    \node at (0.25, 0.35) {\footnotesize 変曲点};

  \end{tikzpicture}
  \caption{ステップ応答波形}
  \label{fig:step_response_main}
\end{figure}

\section{実験方法}
KENTAC3522S 温度制御実習装置を使用し,ヒーターの温度を \SI{100}{\degreeCelsius} に制御する.

\begin{table}[H]
\centering
\caption{ステップ応答法のパラメータ調整}
\label{tab:zn_parameter}
\begin{tabular}{lccc}
\toprule
 & $K_P$ & $T_I$ & $T_D$ \\
\midrule
P制御 & $\displaystyle \frac{1}{RL}$ & --- & --- \\[1.5ex]
PI制御 & $\displaystyle \frac{0.9}{RL}$ & $3.3L$ & --- \\[1.5ex]
PID制御 & $\displaystyle \frac{1.2}{RL}$ & $2L$ & $0.5L$ \\
\bottomrule
\end{tabular}
\end{table}

\subsection{ON・OFF制御}
まずは手動でON・OFF制御を行い,特徴を理解する.
\begin{enumerate}
    \renewcommand{\labelenumi}{(\roman{enumi})}
    \item PidMonitorを起動し,温度制御を選択する.
    \item Autoボタンのチェックを外し,開始ボタンを押す.
    \item MV(ヒータの出力)の値を50.0\%程度に設定し,出力する.
    \item PV(ヒータの温度)の値が \SI{100}{\degreeCelsius} になるまで待つ.
    \item \SI{100}{\degreeCelsius} を超えたらMVの値を0.0\%に設定し,出力する.
    \item \SI{100}{\degreeCelsius} を下回ったら(iii)へ戻る.
\end{enumerate}

\subsection{P制御}
$T_I = 0$(理論上は $\infty$ だが実験機器の設定のため0とする),$T_D = 0$ とし,比例制御のみを行い,特徴を理解する.
\begin{enumerate}
    \renewcommand{\labelenumi}{(\roman{enumi})}
    \item $K_P$ の値を1にした場合の制御系の温度変化を観測する.
    \item $K_P$ の値を100にした場合の制御系の温度変化を観測する.
    \item 3分以内に目標温度 $\pm \SI{10}{\degreeCelsius}$ となる適切な $K_P$ の値を探し,温度変化を観測する.
\end{enumerate}

\subsection{PI制御}
$K_P$ を上記で求めた値,$T_D = 0$ とし,比例・積分制御を行い,特徴を理解する.
\begin{enumerate}
    \renewcommand{\labelenumi}{(\roman{enumi})}
    \item $T_I$ の値を1にした場合の制御系の温度変化を観測する.
    \item $T_I$ の値を100にした場合の制御系の温度変化を観測する.
    \item 3分以内に目標温度 $\pm \SI{0.5}{\degreeCelsius}$ となる適切な $T_I$ の値を探し,温度変化を観測する.
\end{enumerate}

\subsection{PID制御}
$K_P, T_I$ を上記で求めた値とし,比例・積分・微分制御を行い,特徴を理解する.
\begin{enumerate}
    \renewcommand{\labelenumi}{(\roman{enumi})}
    \item $T_D$ の値を1にした場合の制御系の温度変化を観測する.
    \item $T_D$ の値を100にした場合の制御系の温度変化を観測する.
    \item 2分以内に目標温度 $\pm \SI{0.5}{\degreeCelsius}$ となる適切な $T_D$ の値を探し,温度変化を観測する.
\end{enumerate}

\subsection{ステップ応答法}
ステップ応答法により各パラメータを算出する.
\begin{enumerate}
    \renewcommand{\labelenumi}{(\roman{enumi})}
    \item Autoボタンのチェックを外し,開始ボタンを押す.
    \item MVの値を10\%から20\%の間で設定し,出力する.この時の温度を初期温度とする.
    \item PVが一定値になったら停止ボタンを押す.
    \item 温度変化のグラフから $L, T$ を求める.
    \item $K, R$ は以下の式(4)(5)(6)(7)から求める.
    \begin{align}
        Y &= \frac{(\text{最終温度} - \text{初期温度})}{200} \times 100 \, [\%] \label{eq:Y} \\
        U &= (\text{出力値}) \, [\%] \label{eq:U} \\
        K &= \frac{Y}{U} \label{eq:K} \\
        R &= \frac{K}{T} \label{eq:R}
    \end{align}
    \item $K_P, T_I, T_D$ を表1から求め,温度変化を観測する.
\end{enumerate}

\section{考察}
\begin{enumerate}
    \renewcommand{\labelenumi}{(\roman{enumi})}
    \item P制御・PI制御・PID制御の特徴と,各制御に関する比例・積分・微分の各パラメータの影響についてまとめよ.
    \item 3.4と3.5で求めたパラメータと,その値を用いたP制御・PI制御・PID制御による温度変化について,比例・積分・微分のパラメータの違いをもとに比較せよ.
\end{enumerate}

\section{報告事項}
\begin{enumerate}
    \renewcommand{\labelenumi}{(\roman{enumi})}
    \item 以下の制御対象に対するPID制御のパラメータ $K_P, T_I, T_D$ をステップ応答法によって設計せよ.
    \begin{equation}
        G(s) = \frac{5}{2s+4}e^{-3s}
    \end{equation}
\end{enumerate}

\begin{thebibliography}{9}
\bibitem{toyohashi} 豊橋技術科学大学・高等専門学校制御工学教育連携プロジェクト (2012),制御工学,実教出版,pp.146-147
\end{thebibliography}

\end{document}