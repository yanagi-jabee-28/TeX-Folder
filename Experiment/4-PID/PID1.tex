% ===== ドキュメントクラス =====
% LuaLaTeXでのコンパイルを前提としたクラス設定
\documentclass[
  a4paper,
  11pt,
]{ltjsarticle}

% ----- フォント・数式関連 -----
\usepackage{newtxtext}
\usepackage{newtxmath}
\usepackage{amsmath}
% newtxmathとの競合回避
\let\Bbbk\relax
\usepackage{amssymb}
\usepackage{bm} % ベクトル太字用

% ----- 画像・単位・レイアウト関連 -----
% 【重要】コンパイラの自動判定設定
\usepackage{graphicx} 
\usepackage[export]{adjustbox}
\usepackage{siunitx}
\usepackage{float}
\usepackage{placeins}
\usepackage[margin=25mm]{geometry}
\usepackage{booktabs} % 表の罫線を美しくするため追加
\usepackage{url}      % URL表示用

% ----- グラフ描画機能 (TeX機能による数式の可視化) -----
\usepackage{tikz}
\usepackage{pgfplots}
\pgfplotsset{compat=1.18}
\usetikzlibrary{arrows, arrows.meta, positioning, calc, shapes.geometric}

% 画像の高さ制限設定
\setkeys{Gin}{keepaspectratio}

% 便利マクロ
\newcommand{\includefig}[2][]{\includegraphics[max size={\textwidth}{0.45\textheight},#1]{#2}}

% ページ下部の余白調整
\raggedbottom

% ----- 参考文献の上付き表示設定 -----
\makeatletter
\def\@cite#1#2{\textsuperscript{[{#1\if@tempswa , #2\fi}]}}
\makeatother

% ----- ハイパーリンク -----
\usepackage[colorlinks=true,linkcolor=black,citecolor=black,urlcolor=black,anchorcolor=black]{hyperref}

% ===== ドキュメント情報 =====
\title{4. PIDによる温度制御}
\author{電気電子工学科 X年 X組 番号 氏名} 
\date{2025年12月4日}

% ===== ここからドキュメント本体 =====
\begin{document}

\maketitle

% 目次(必要に応じて有効化)
% \tableofcontents
% \newpage

\section{目的}
ワンボードマイコンを用いてヒーターをPID制御する方法について理解する.

\section{PID制御}
三つのパラメータ proportional(比例),integral(積分),derivative(微分)のゲインを直感的にチューニングするだけで,ある程度の性能が確保できる制御方法であるため,広く使われている.

\subsection{PID制御の伝達関数}
PID制御における操作量 $u(t)$ と偏差 $e(t)$ の関係は式(1)のようになる.また,式(1)の伝達関数は式(2)となり,伝達関数のブロック図は図1となる.ここで $K_P$ は比例ゲイン,$K_I$ は積分ゲイン,$K_D$ は微分ゲイン,$T_I$ は積分時間,$T_D$ は微分時間となる.

\begin{equation}
  u(t) = K_P e(t) + K_I \int_{0}^{t} e(\tau) d\tau + K_D \frac{de(t)}{dt}
  \label{eq:pid_time}
\end{equation}

\begin{equation}
  K_C(s) = K_P + K_I \frac{1}{s} + K_D s = K_P \left( 1 + \frac{1}{T_I s} + T_D s \right)
  \label{eq:pid_tf}
\end{equation}

% --- 図1 PID制御ブロック図のTikZ再現 ---
\begin{figure}[H]
  \centering
  \begin{tikzpicture}[auto, node distance=1.5cm, >=Latex]
    % Styles
    \tikzstyle{block} = [draw, rectangle, minimum height=2em, minimum width=3em]
    \tikzstyle{sum} = [draw, circle, inner sep=1pt]
    \tikzstyle{input} = [coordinate]
    \tikzstyle{output} = [coordinate]
    \tikzstyle{branch} = [circle,inner sep=0pt,minimum size=1mm,fill=black,draw=black]

    % Frame for PID Controller
    \draw[dashed] (1.2,-2.2) rectangle (5.5, 2.2);
    \node at (3.35, 2.4) {\small PID制御器 $K_C$};

    % Nodes
    \node [input] (input) {};
    \node [sum, right=1cm of input] (sum1) {};
    \node [left=0.1cm of sum1] (plus) {$+$};
    
    % Gain Blocks
    \node [block, right=1.5cm of sum1] (kp) {$K_P$};
    \node [block, above=1cm of kp] (ki_block) {$\displaystyle \frac{K_I}{s}$}; % 積分
    \node [block, below=1cm of kp] (kd_block) {$K_D s$}; % 微分

    % Summing Point for PID
    \node [sum, right=2cm of kp] (sum2) {};
    
    % Plant
    \node [block, right=1.5cm of sum2] (plant) {制御対象};
    \node [output, right=1.5cm of plant] (output) {};

    % Connections
    \draw [->] (input) -- node {目標値} (sum1);
    \draw [->] (sum1) -- (kp);
    
    % Branching to I and D
    \node [branch, right=0.5cm of sum1] (b1) {};
    \draw [->] (b1) |- node[above, xshift=-0.5cm] {比例ゲイン} (ki_block) node[midway, above] {積分ゲイン} node[pos=0.9, above] {積分};
    \draw [->] (b1) |- node[below, xshift=-0.5cm] {微分ゲイン} (kd_block) node[midway, below] {微分};
    
    % To Sum2
    \draw [->] (ki_block) -| (sum2);
    \draw [->] (kp) -- (sum2);
    \draw [->] (kd_block) -| (sum2);

    % To Plant and Output
    \draw [->] (sum2) -- (plant);
    \draw [->] (plant) -- node [name=y] {出力} (output);
    
    % Feedback
    \draw [->] (y) -- ++(0,-3) -| node[pos=0.99] {$-$} (sum1);
    
  \end{tikzpicture}
  \caption{PID制御のブロック図(第1形式:並列型)}
  \label{fig:pid_block1}
\end{figure}

% 2つ目の形式(標準形)も描画
\begin{figure}[H]
  \centering
  \begin{tikzpicture}[auto, node distance=1.5cm, >=Latex]
    \tikzstyle{block} = [draw, rectangle, minimum height=2em, minimum width=3em]
    \tikzstyle{sum} = [draw, circle, inner sep=1pt]
    \tikzstyle{input} = [coordinate]
    \tikzstyle{output} = [coordinate]
    \tikzstyle{branch} = [circle,inner sep=0pt,minimum size=1mm,fill=black,draw=black]

    \draw[dashed] (1.2,-2.2) rectangle (6.5, 2.2);
    \node at (3.85, 2.4) {\small PID制御器(標準形)};

    \node [input] (input) {};
    \node [sum, right=1cm of input] (sum1) {};
    \node [left=0.1cm of sum1] {$+$};
    \node [block, right=0.8cm of sum1] (kp) {$K_P$};
    
    \node [branch, right=0.5cm of kp] (b2) {};
    
    \node [block, right=1.5cm of kp] (one) {$1$};
    \node [block, above=1cm of one] (ti) {$\displaystyle \frac{1}{T_I s}$};
    \node [block, below=1cm of one] (td) {$T_D s$};
    
    \node [sum, right=1.5cm of one] (sum2) {};
    \node [block, right=1.0cm of sum2] (plant) {制御対象};
    \node [output, right=1.5cm of plant] (output) {};

    \draw [->] (input) -- node {目標値} (sum1);
    \draw [->] (sum1) -- node {比例ゲイン} (kp);
    \draw [-] (kp) -- (b2);
    
    \draw [->] (b2) -- (one);
    \draw [->] (b2) |- node[midway, above] {積分時間} node[pos=0.9, above] {積分} (ti);
    \draw [->] (b2) |- node[midway, below] {微分時間} node[pos=0.9, below] {微分} (td);
    
    \draw [->] (one) -- (sum2);
    \draw [->] (ti) -| (sum2);
    \draw [->] (td) -| (sum2);
    
    \draw [->] (sum2) -- (plant);
    \draw [->] (plant) -- node [name=y] {出力} (output);
    \draw [->] (y) -- ++(0,-3) -| node[pos=0.99] {$-$} (sum1);
  \end{tikzpicture}
  \caption{PID制御のブロック図(第2形式:標準型)}
  \label{fig:pid_block2}
\end{figure}

PID制御器では三つのゲイン $K_P, K_I, K_D$ が用いられるが,制御系におよぼすそれぞれの要素の影響や効果は次の通りである\cite{toyohashi}.

\begin{description}
    \item[比例ゲイン:$K_P$] \mbox{} \\
    偏差の現在値に比例した制御入力を求める項.このゲインを上げるとシステムの応答性が増すと同時に,目標値の変化や外乱などによる偏差を抑制することができる.
    \item[積分ゲイン:$K_I$ ($K_P / T_I$)] \mbox{} \\
    偏差の積分値(偏差の累積)に応じた制御入力を求める項.このゲインを上げると低周波外乱の出力への影響を効果的に抑制することができる.
    \item[微分ゲイン:$K_D$ ($K_P T_D$)] \mbox{} \\
    偏差の微分値(増減の動向)に応じた制御入力を求める項.このゲインを上げるとシステムの速応性を増すことができる.
\end{description}

\subsection{パラメータ調整}
\textbf{ステップ応答法} \\
式(3)のような1次遅れ要素とむだ時間要素で近似的に表せるシステムを対象とし,開ループにおける制御対象のステップ応答波形を用いる.手順は以下の通りである.

\begin{equation}
  G_P(s) = \frac{K}{1+sT}e^{-Ls}
  \label{eq:plant}
\end{equation}

\begin{enumerate}
    \renewcommand{\labelenumi}{(\roman{enumi})}
    \item ステップ応答の測定結果,またはステップ応答の計算値から図2のむだ時間 $L$,変曲点の傾き $R$ を読み取る.
    \item 表1に示す値に各パラメータを設定する.
\end{enumerate}

% --- 図2 ステップ応答波形のTikZ再現 ---
\begin{figure}[H]
  \centering
  \begin{tikzpicture}[>=Latex, xscale=5, yscale=2.5]
    % Axes
    \draw[->] (-0.1,0) -- (1.2,0) node[below] {$t$};
    \draw[->] (0,-0.1) -- (0,1.2);
    
    % Step Input
    \draw[thick] (0,0) -- (0,1) -- (1.1,1);
    \node at (0.3, 1.3) {ステップ入力};
    \node at (-0.05, 1) {$1$};

    % Step Response Curve (Sigmoid-like approximation)
    \draw[thick, domain=0:1.1, smooth, samples=100] plot (\x, {1 / (1 + exp(-20*(\x-0.3))) * 0.6});
    \node at (0.6, 0.4) {ステップ応答};
    
    % K level
    \draw[dashed] (0,0.6) node[left] {$K$} -- (1.1,0.6);
    
    % Tangent line at inflection point (approx at t=0.3, y=0.3)
    % Slope R is roughly K/T. Let's draw a line.
    % Line passing through (0.2, 0) and (0.4, 0.6)
    \draw[thin] (0.15, 0) -- (0.45, 0.7);
    
    % L and T dimensions
    \draw[dotted] (0.2, 0) -- (0.2, -0.2);
    \draw[dotted] (0.4, 0) -- (0.4, -0.2);
    
    \draw[<->] (0, -0.1) -- node[fill=white] {$L$} (0.2, -0.1);
    \draw[<->] (0.2, -0.1) -- node[fill=white] {$T$} (0.4, -0.1);
    
    % R indication
    \node at (0.7, 0.25) {$\displaystyle R = \frac{K}{T}$};
    
    % Point indication
    \node at (0.3, 0.3) [circle,fill,inner sep=1pt]{};
    \node at (0.25, 0.35) {\small 変曲点};

  \end{tikzpicture}
  \caption{ステップ応答波形}
  \label{fig:step_response_main}
\end{figure}


\section{実験方法}
KENTAC3522S 温度制御実習装置を使用し,ヒーターの温度を \SI{100}{\degreeCelsius} に制御する.

\begin{table}[H]
\centering
\caption{ステップ応答法のパラメータ調整}
\label{tab:zn_parameter}
\begin{tabular}{lccc}
\toprule
 & $K_P$ & $T_I$ & $T_D$ \\
\midrule
P制御 & $\displaystyle \frac{1}{RL}$ & --- & --- \\[1.5ex]
PI制御 & $\displaystyle \frac{0.9}{RL}$ & $3.3L$ & --- \\[1.5ex]
PID制御 & $\displaystyle \frac{1.2}{RL}$ & $2L$ & $0.5L$ \\
\bottomrule
\end{tabular}
\end{table}

\subsection{ON・OFF制御}
まずは手動でON・OFF制御を行い,特徴を理解する.
\begin{enumerate}
    \renewcommand{\labelenumi}{(\roman{enumi})}
    \item PidMonitorを起動し,温度制御を選択する.
    \item Autoボタンのチェックを外し,開始ボタンを押す.
    \item MV(ヒータの出力)の値を50.0\%程度に設定し,出力する.
    \item PV(ヒータの温度)の値が \SI{100}{\degreeCelsius} になるまで待つ.
    \item \SI{100}{\degreeCelsius} を超えたらMVの値を0.0\%に設定し,出力する.
    \item \SI{100}{\degreeCelsius} を下回ったら(iii)へ戻る.
\end{enumerate}

\subsection{P制御}
$T_I = 0$(理論上は $\infty$ だが実験機器の設定のため0とする),$T_D = 0$ とし,比例制御のみを行い,特徴を理解する.
\begin{enumerate}
    \renewcommand{\labelenumi}{(\roman{enumi})}
    \item $K_P$ の値を1にした場合の制御系の温度変化を観測する.
    \item $K_P$ の値を100にした場合の制御系の温度変化を観測する.
    \item 3分以内に目標温度 $\pm \SI{10}{\degreeCelsius}$ となる適切な $K_P$ の値を探し,温度変化を観測する.
\end{enumerate}

\subsection{PI制御}
$K_P$ を上記で求めた値,$T_D = 0$ とし,比例・積分制御を行い,特徴を理解する.
\begin{enumerate}
    \renewcommand{\labelenumi}{(\roman{enumi})}
    \item $T_I$ の値を1にした場合の制御系の温度変化を観測する.
    \item $T_I$ の値を100にした場合の制御系の温度変化を観測する.
    \item 3分以内に目標温度 $\pm \SI{0.5}{\degreeCelsius}$ となる適切な $T_I$ の値を探し,温度変化を観測する.
\end{enumerate}

\subsection{PID制御}
$K_P, T_I$ を上記で求めた値とし,比例・積分・微分制御を行い,特徴を理解する.
\begin{enumerate}
    \renewcommand{\labelenumi}{(\roman{enumi})}
    \item $T_D$ の値を1にした場合の制御系の温度変化を観測する.
    \item $T_D$ の値を100にした場合の制御系の温度変化を観測する.
    \item 2分以内に目標温度 $\pm \SI{0.5}{\degreeCelsius}$ となる適切な $T_D$ の値を探し,温度変化を観測する.
\end{enumerate}

\subsection{ステップ応答法}
ステップ応答法により各パラメータを算出する.
\begin{enumerate}
    \renewcommand{\labelenumi}{(\roman{enumi})}
    \item Autoボタンのチェックを外し,開始ボタンを押す.
    \item MVの値を10\%から20\%の間で設定し,出力する.この時の温度を初期温度とする.
    \item PVが一定値になったら停止ボタンを押す.
    \item 温度変化のグラフから $L, T$ を求める.
    \item $K, R$ は以下の式(4)(5)(6)(7)から求める.
    \begin{align}
        Y &= \frac{(\text{最終温度} - \text{初期温度})}{200} \times 100 \, [\%] \label{eq:Y} \\
        U &= (\text{出力値}) \, [\%] \label{eq:U} \\
        K &= \frac{Y}{U} \label{eq:K} \\
        R &= \frac{K}{T} \label{eq:R}
    \end{align}
    \item $K_P, T_I, T_D$ を表1から求め,温度変化を観測する.
\end{enumerate}

\section{考察}
\begin{enumerate}
    \renewcommand{\labelenumi}{(\roman{enumi})}
    \item P制御・PI制御・PID制御の特徴と,各制御に関する比例・積分・微分の各パラメータの影響についてまとめよ.(各制御の制御原理をもとに理由を示すこと)
    \item 3.4と3.5で求めたパラメータと,その値を用いたP制御・PI制御・PID制御による温度変化について,比例・積分・微分のパラメータの違いをもとに比較せよ.
    ($K_P, K_P/T_I, K_P T_D$ を計算し,比較を行うこと)
\end{enumerate}

\section{報告事項}
\begin{enumerate}
    \renewcommand{\labelenumi}{(\roman{enumi})}
    \item 以下の制御対象に対するPID制御のパラメータ $K_P, T_I, T_D$ をステップ応答法によって設計せよ.
    \begin{equation}
        G(s) = \frac{5}{2s+4}e^{-3s}
    \end{equation}
\end{enumerate}

\begin{thebibliography}{9}
\bibitem{toyohashi} 豊橋技術科学大学・高等専門学校制御工学教育連携プロジェクト (2012),制御工学,実教出版,pp.146-147
\end{thebibliography}

\newpage
% =============================================
% 補足資料 (PDFの最初のセクション)
% =============================================
\appendix
\section*{補足説明}
\addcontentsline{toc}{section}{補足説明}

\subsection*{1. KENTAC3522S 温度制御実習装置について}
\subsubsection*{1.1 (3章) KENTAC3522S 温度制御実習装置の表記について}
\begin{itemize}
    \item \textbf{SV (Setting Variable)}: 目標となる値 $\to$ (赤) 目標温度
    \item \textbf{PV (Process Variable)}: 測定値 $\to$ (緑) センサ計測温度
    \item \textbf{MV (Manipulated Variable)}: 操作量の大きさ $\to$ (青) 出力 (0~100\%)
    \item \textbf{Auto}: 制御モード $\to$ \checkmark (制御モード 自動)
    \item \textbf{Pid設定 P}: $\to$ 比例係数
    \item \textbf{Pid設定 Ti}: $\to$ 積分時間 (sec)
    \item \textbf{Pid設定 Td}: $\to$ 微分時間 (sec)
\end{itemize}

\subsubsection*{1.2 (3.5章) ステップ応答法について}
ステップ応答法による各パラメータ ($L, T, R$) の算出手順

ステップ応答の画像データをInkscape等の画像編集ソフトに読込み,下記の手順で作図の後,各パラメータを読取り,算出する.

\begin{enumerate}
    \renewcommand{\labelenumi}{(\roman{enumi})}
    \item 最終値 ($K$) に直線を引く.
    \item 開始値 (始めの値の平なところ) に直線を引く.
    \item ステップ応答の傾きが最大となるような接線を,開始値の直線(ii)と最終値の直線(i)とまで引き,開始値の直線との交点を a 点,最終値の直線との交点を b 点とする.
    \item (iii)の b 点から垂線を開始値の直線まで引き,交点を c 点とする.
    \item ステップ入力の開始~a 点までを $L$,a 点~c 点までを $T$ とし,一分間の目盛りとpixel値の比例計算から,$L, T$ を読取り、読取った $T$ から $R$ を算出する.
\end{enumerate}

% --- 図: 作図法の説明用TikZ ---
\begin{figure}[H]
  \centering
  \begin{tikzpicture}[>=Latex, scale=1.2]
    % Coordinates based on image geometry
    \coordinate (O) at (0,0);
    \coordinate (EndK) at (6,3); % K level height
    \coordinate (StartLine) at (0,0.5); % Initial value level
    
    % Draw Axes
    \draw[->] (0,0) -- (6.5,0) node[below] {$t$};
    \draw[->] (0,0) -- (0,4) node[left] {1};
    
    % Step Input
    \draw[thick] (0,0.5) -- (0.5,0.5) -- (0.5, 3.5) -- (6, 3.5);
    \node at (2, 3.7) {\small ステップ入力};
    
    % Response Curve
    \draw[thick] plot [smooth] coordinates {(0.5,0.5) (1,0.5) (1.5,0.6) (2,1.2) (2.5,2.5) (3.5,2.9) (5,3) (6,3)};
    \node at (4.5, 2.5) {\small ステップ応答};
    
    % (i) Line at K (Final value)
    \draw (0,3) -- (6,3);
    \node at (-0.3, 3) {$K$};
    \node at (5.8, 3.2) {\footnotesize ①};

    % (ii) Line at Start (Initial value)
    \draw (0,0.5) -- (6,0.5);
    \node at (5.8, 0.7) {\footnotesize ②};
    
    % Tangent Line Construction
    % Assume tangent goes from a(1.5, 0.5) to b(2.8, 3)
    \coordinate (PointA) at (1.5, 0.5);
    \coordinate (PointB) at (2.8, 3);
    
    % (iii) Tangent line
    \draw (1.0, -0.46) -- (3.3, 3.96); % Extended line
    \node at (2.0, 1.8) {\footnotesize ③}; % Tangent label
    \node at (1.4, 1.5) {\footnotesize 変曲点};
    
    % Points a and b
    \fill (PointA) circle (2pt) node[below] {a};
    \fill (PointB) circle (2pt) node[above] {b};
    
    % (iv) Perpendicular from b to c
    \coordinate (PointC) at (2.8, 0.5);
    \draw (PointB) -- (PointC);
    \fill (PointC) circle (2pt) node[below right] {c};
    \node at (2.9, 1.5) {\footnotesize ④};
    
    % (v) L and T
    % Step start at t=0.5
    \coordinate (StepStart) at (0.5, 0.5);
    \draw[<->] (StepStart) -- node[below] {$L$} (PointA);
    \draw[<->] (PointA) -- node[below] {$T$} (PointC);
    
    \node at (2.2, -0.5) {$\displaystyle R = \frac{K}{T}$};
    
    % Number circles
    \node[draw, circle, inner sep=1pt] at (0.2, 0.2) {5}; % Just placing loosely based on figure
    
  \end{tikzpicture}
  \caption{ステップ応答波形とパラメータ算出の作図}
  \label{fig:step_response_method}
\end{figure}

\subsection*{2 実際のPID調節器 (温度調節器) について}
\url{https://www.fa.omron.co.jp/products/category/control-components/temperature-controllers/general-purpose/}

\end{document}