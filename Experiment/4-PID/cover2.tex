\documentclass[a4paper,11pt]{ltjsarticle}

% ===== 基本設定 =====
% 1ページに美しく収めるための余白設定
\usepackage[top=25mm, bottom=25mm, left=20mm, right=20mm]{geometry}
\usepackage{array}      
\usepackage{multirow}   
\usepackage{fancyhdr}   
\usepackage{amssymb}    % 矢印用

% ===== カスタムコマンド定義 =====

% 1. 下線付きテキストボックス
% 使い方: \UnderlineBox[線の長さ]{表示する文字}
% 文字を空にすれば書き込み用の線になります
\newcommand{\UnderlineBox}[2][3cm]{%
  \underline{\makebox[#1][c]{\vphantom{lp}\large #2}}%
}

% 2. 均等割り付けラベル(太字)
% 使い方: \JustifiedLabel{幅}{テキスト}
\newcommand{\JustifiedLabel}[2]{%
  \makebox[#1][s]{\large\bfseries #2}%
}

% 3. 通常の太字ラベル
\newcommand{\BoldLabel}[1]{%
  {\large\bfseries #1}%
}

% ===== フッター設定(学校名) =====
\pagestyle{fancy}
\fancyhf{} 
\renewcommand{\headrulewidth}{0pt} 
\renewcommand{\footrulewidth}{0pt} 
\cfoot{\vspace{5mm}\Large \bfseries 国立長野高専 電気電子工学科}

\begin{document}

% --- タイトル ---
\begin{center}
    \vspace*{5mm}
    {\Huge \bfseries 電気電子工学実験報告書}
    \vspace{20mm}
\end{center}

% --- テーマ名 ---
\noindent
\begin{tabular}{@{}ll}
  \BoldLabel{テーマ名} & \UnderlineBox[13cm]{} \\[1.5em] % ここにテーマ名を記入
\end{tabular}

% --- 報告者情報 ---
\noindent
\BoldLabel{報告者} \hspace{0.5em}
\UnderlineBox[1.5cm]{} {\large \textbf{年}} \hspace{0.2em}     % 年
(\UnderlineBox[1.5cm]{} {\large \textbf{組}}) \hspace{0.2em} % 組
{\large \textbf{番号}} \UnderlineBox[2.0cm]{} \hspace{0.5em}   % 番号
\UnderlineBox[1.5cm]{} {\large \textbf{班}} \hspace{1em}       % 班
\UnderlineBox[4.5cm]{}                                         % 氏名
\vspace{2.5em} 

% --- 実験場所・指導担当 ---
\noindent
\begin{tabular}{@{}p{0.48\textwidth} p{0.48\textwidth}}
  \BoldLabel{実験場所} \hspace{1em} \UnderlineBox[5.5cm]{} & % 実験場所
  \BoldLabel{指導担当} \hspace{1em} \UnderlineBox[5.5cm]{}   % 指導担当
\end{tabular}
\vspace{2.5em} 

% --- 共同実験者 ---
\noindent
\BoldLabel{共同実験者} \hspace{1em} \UnderlineBox[12.5cm]{} % 共同実験者
\vspace{4em} 

% --- 日付セクション ---
% PDFの絶妙なバランスを再現
\noindent
\renewcommand{\arraystretch}{2.0}
\setlength{\tabcolsep}{0pt}
\begin{tabular}{l l l l}
  % 1行目: 実験日
  \JustifiedLabel{5em}{実験日} & 
  \hspace{0.5em} 令和 \UnderlineBox[1cm]{} 年 \UnderlineBox[1cm]{} 月 \UnderlineBox[1cm]{} 日 & & \\
  
  % 2行目: 提出期限 -> 提出日
  \JustifiedLabel{5em}{提出期限} & 
  \hspace{0.5em} 令和 \UnderlineBox[1cm]{} 年 \UnderlineBox[1cm]{} 月 \UnderlineBox[1cm]{} 日 & 
  \hspace{1em}$\Rightarrow$\hspace{1em} \JustifiedLabel{4em}{提出日} & 
  \hspace{0.5em} 令和 \UnderlineBox[1cm]{} 年 \UnderlineBox[1cm]{} 月 \UnderlineBox[1cm]{} 日 \\
  
  % 3行目: (再提出期限 -> 再提出日)
  ( \JustifiedLabel{5em}{再提出期限} & 
  \hspace{0.5em} 令和 \UnderlineBox[1cm]{} 年 \UnderlineBox[1cm]{} 月 \UnderlineBox[1cm]{} 日 & 
  \hspace{1em}$\Rightarrow$\hspace{1em} \JustifiedLabel{4em}{再提出日} & 
  \hspace{0.5em} 令和 \UnderlineBox[1cm]{} 年 \UnderlineBox[1cm]{} 月 \UnderlineBox[1cm]{} 日 )
\end{tabular}

\vfill 

% --- 評価テーブル ---
\renewcommand{\arraystretch}{1.6}
\noindent
% 表全体の幅を用紙幅に合わせて調整
\begin{tabular}{|c|m{12.5cm}|c|}
\hline
\multicolumn{2}{|c|}{\JustifiedLabel{10em}{評 価 項 目}} & \JustifiedLabel{3em}{評 価} \\
\hline
\multirow{3}{*}{\shortstack{\large\bfseries 実 習\\[0.3em]\large\bfseries 評 価}} 
 & (1) 自ら積極的に実験に取り組めた &  \\ \cline{2-3}
 & (2) 実験装置を適切に使用でき,正確に実験を行なえた &  \\ \cline{2-3}
 & (3) グループ内で協力的に実験が行なえた &  \\
\hline
\multirow{4}{*}{\shortstack{\large\bfseries 報告書\\[0.3em]\large\bfseries 評 価}} 
 & (1) 結果のまとめかた(図表を含む) &  \\ \cline{2-3}
 & (2) 結果に対する考察 &  \\ \cline{2-3}
 & (3) 報告事項/課題(正しい解答や適切な引用など) &  \\ \cline{2-3}
 & (4) 報告書としての体裁が整っているか &  \\
\hline
\end{tabular}

\end{document}