\documentclass[a4paper,11pt]{ltjsarticle}

% ===== 基本設定 =====
% 余白設定(左右20mm確保)
\usepackage[top=20mm, bottom=20mm, left=20mm, right=20mm]{geometry}
\usepackage{array}      
\usepackage{multirow}   
\usepackage{fancyhdr}   

% ===== PDFフォーム機能 =====
\usepackage[hidelinks]{hyperref}

% ===== 入力欄コマンド =====
\newcommand{\InputLine}[2][3cm]{%
  \underline{%
    \makebox[#1]{%
      \TextField[name=#2, width=#1, height=1.2em, borderwidth=0, backgroundcolor={1 1 1}, align=1]{}%
    }%
  }%
}
\newcommand{\InputBox}[2][1.8cm]{%
  \TextField[name=#2, width=#1, height=1.2em, borderwidth=0, backgroundcolor={1 1 1}, align=1]{}%
}

% ===== ラベル用コマンド =====
\newcommand{\JustifiedLabel}[2]{%
  \makebox[#1][s]{\large\bfseries #2}%
}
\newcommand{\BoldLabel}[1]{%
  {\large\bfseries #1}%
}

% ===== フッター設定 =====
\pagestyle{fancy}
\fancyhf{} 
\renewcommand{\headrulewidth}{0pt} 
\renewcommand{\footrulewidth}{0pt} 
\cfoot{\Large \bfseries 国立長野高専 電気電子工学科}

\begin{document}

\begin{Form}

% --- タイトル ---
\begin{center}
    \vspace*{-5mm}
    {\Huge \bfseries 電気電子工学実験報告書}
    \vspace{10mm}
\end{center}

% --- テーマ名 ---
\noindent
\begin{tabular}{@{}ll}
  \BoldLabel{テーマ名} & \InputLine[12.5cm]{Theme} \\[1.5em]
\end{tabular}

% --- 報告者情報 ---
\noindent
\BoldLabel{報告者} \hspace{0.5em}
\InputLine[1.2cm]{Year} {\large \textbf{年}} \hspace{0.2em}
(\InputLine[1.2cm]{Class} {\large \textbf{組}}) \hspace{0.2em}
{\large \textbf{番号}} \InputLine[2.0cm]{Number} \hspace{0.5em}
\InputLine[1.2cm]{Group} {\large \textbf{班}} \hspace{1em}
\InputLine[4.2cm]{Name}
\vspace{2em} 

% --- 実験場所・指導担当 ---
\noindent
\begin{tabular}{@{}p{0.47\textwidth} p{0.47\textwidth}}
  \BoldLabel{実験場所} \hspace{1em} \InputLine[5.2cm]{Location} & 
  \BoldLabel{指導担当} \hspace{1em} \InputLine[5.2cm]{Instructor}
\end{tabular}
\vspace{2em} 

% --- 共同実験者 ---
\noindent
\BoldLabel{共同実験者} \hspace{1em} \InputLine[12.0cm]{Partners}
\vspace{2.5em} 

% --- 日付セクション ---
\noindent
\renewcommand{\arraystretch}{1.8}
\setlength{\tabcolsep}{0pt}
\begin{tabular}{l l l l}
  \JustifiedLabel{5em}{実験日} & 
  % 日付入力欄を 0.6cm に変更(これで数mm稼げます)
  \hspace{0.5em} 令和 \InputLine[0.6cm]{ExpYear} 年 \InputLine[0.6cm]{ExpMonth} 月 \InputLine[0.6cm]{ExpDay} 日 & & \\
  
  \JustifiedLabel{5em}{提出期限} & 
  \hspace{0.5em} 令和 \InputLine[0.6cm]{DueYear} 年 \InputLine[0.6cm]{DueMonth} 月 \InputLine[0.6cm]{DueDay} 日 & 
  % 矢印周りのスペースを 1em -> 0.5em に短縮
  \hspace{0.5em}$\Rightarrow$\hspace{0.5em} \JustifiedLabel{4em}{提出日} & 
  \hspace{0.5em} 令和 \InputLine[0.6cm]{SubYear} 年 \InputLine[0.6cm]{SubMonth} 月 \InputLine[0.6cm]{SubDay} 日 \\
  
  ( \JustifiedLabel{5em}{再提出期限} & 
  \hspace{0.5em} 令和 \InputLine[0.6cm]{ReDueYear} 年 \InputLine[0.6cm]{ReDueMonth} 月 \InputLine[0.6cm]{ReDueDay} 日 & 
  \hspace{0.5em}$\Rightarrow$\hspace{0.5em} \JustifiedLabel{4em}{再提出日} & 
  \hspace{0.5em} 令和 \InputLine[0.6cm]{ReSubYear} 年 \InputLine[0.6cm]{ReSubMonth} 月 \InputLine[0.6cm]{ReSubDay} 日 )
\end{tabular}

\vfill 

% --- 評価テーブル ---
\renewcommand{\arraystretch}{1.5}
\noindent
\begin{tabular}{|c|m{12.2cm}|c|}
\hline
\multicolumn{2}{|c|}{\JustifiedLabel{10em}{評価項目}} & \JustifiedLabel{3em}{評価} \\
\hline
\multirow{3}{*}{\shortstack{\large\bfseries 実 習\\[0.3em]\large\bfseries 評 価}} 
 & (1) 自ら積極的に実験に取り組めた & \InputBox{Score1} \\ \cline{2-3}
 & (2) 実験装置を適切に使用でき,正確に実験を行なえた & \InputBox{Score2} \\ \cline{2-3}
 & (3) グループ内で協力的に実験が行なえた & \InputBox{Score3} \\
\hline
\multirow{4}{*}{\shortstack{\large\bfseries 報告書\\[0.3em]\large\bfseries 評 価}} 
 & (1) 結果のまとめかた(図表を含む) & \InputBox{Score4} \\ \cline{2-3}
 & (2) 結果に対する考察 & \InputBox{Score5} \\ \cline{2-3}
 & (3) 報告事項/課題(正しい解答や適切な引用など) & \InputBox{Score6} \\ \cline{2-3}
 & (4) 報告書としての体裁が整っているか & \InputBox{Score7} \\
\hline
\end{tabular}

\end{Form}
\end{document}