\documentclass[a4paper,11pt]{ltjsarticle}

% ===== 基本設定(余白・日本語フォント) =====
\usepackage[top=25mm, bottom=30mm, left=25mm, right=25mm]{geometry} % フッター用に下の余白を少し確保
\usepackage{array}      % 表の微調整用
\usepackage{multirow}   % 表の縦結合用
\usepackage{fancyhdr}   % フッターを定位置に置くためのパッケージ

% ===== PDFフォーム機能 =====
\usepackage[hidelinks]{hyperref}

% ===== 入力欄コマンドの定義 =====
% 下線付き入力欄: \InputLine[幅cm]{識別ID}
\newcommand{\InputLine}[2][3cm]{%
  \underline{%
    \makebox[#1]{%
      \TextField[name=#2, width=#1, height=1em, borderwidth=0, backgroundcolor={1 1 1}, align=1]{}%
    }%
  }%
}
% 表の中用(下線なし): \InputBox[幅cm]{識別ID}
\newcommand{\InputBox}[2][1.8cm]{%
  \TextField[name=#2, width=#1, height=1em, borderwidth=0, backgroundcolor={1 1 1}, align=1]{}%
}

% ===== フッター設定(学校名をページ最下部に固定) =====
\pagestyle{fancy}
\fancyhf{} % 既存のヘッダー・フッターをクリア
\renewcommand{\headrulewidth}{0pt} % ヘッダーの線を消す
\renewcommand{\footrulewidth}{0pt} % フッターの線を消す
\cfoot{\large 国立長野高専 電気電子工学科} % フッター中央に配置

\begin{document}

% フォーム環境の開始
\begin{Form}

% --- タイトル ---
\begin{center}
    \vspace*{5mm}
    {\huge \bfseries 電気電子工学実験報告書}
    \vspace*{15mm}
\end{center}

% --- テーマ名 ---
\noindent
\begin{tabular}{@{}ll}
  \large テーマ名 & \InputLine[12cm]{Theme} \\[2em]
\end{tabular}

% --- 報告者情報 ---
\noindent
\textbf{報告者} \hspace{0.5em}
\InputLine[1.2cm]{Year} 年 \hspace{0.2em}
(\InputLine[1.2cm]{Class} 組) \hspace{0.2em}
番号 \InputLine[2.0cm]{Number} \hspace{0.5em}
\InputLine[1.2cm]{Group} 班 \hspace{1em}
\InputLine[4.0cm]{Name}
\vspace{2.5em}

% --- 実験場所・指導担当 ---
\noindent
\begin{tabular}{@{}p{0.48\textwidth} p{0.48\textwidth}}
  \textbf{実験場所} \hspace{1em} \InputLine[5cm]{Location} & 
  \textbf{指導担当} \hspace{1em} \InputLine[5cm]{Instructor}
\end{tabular}
\vspace{2.5em}

% --- 共同実験者 ---
% 微調整: 幅を11.9cmにしてOverfullを解消
\noindent
\textbf{共同実験者} \hspace{1em} \InputLine[11.9cm]{Partners}
\vspace{3.5em}

% --- 日付 ---
\noindent
\renewcommand{\arraystretch}{2}
\setlength{\tabcolsep}{3pt}
\begin{tabular}{@{}l l l l}
  \textbf{実 験 日} & 
  令和 \InputLine[0.8cm]{ExpYear} 年 \InputLine[0.8cm]{ExpMonth} 月 \InputLine[0.8cm]{ExpDay} 日 & & \\
  
  \textbf{提 出 期 限} & 
  令和 \InputLine[0.8cm]{DueYear} 年 \InputLine[0.8cm]{DueMonth} 月 \InputLine[0.8cm]{DueDay} 日 & 
  \hspace{0.2em}$\Rightarrow$\hspace{0.2em} \textbf{提 出 日} & 
  令和 \InputLine[0.8cm]{SubYear} 年 \InputLine[0.8cm]{SubMonth} 月 \InputLine[0.8cm]{SubDay} 日 \\
  
  ( \textbf{再提出期限} & 
  令和 \InputLine[0.8cm]{ReDueYear} 年 \InputLine[0.8cm]{ReDueMonth} 月 \InputLine[0.8cm]{ReDueDay} 日 & 
  \hspace{0.2em}$\Rightarrow$\hspace{0.2em} \textbf{再 提 出 日} & 
  令和 \InputLine[0.8cm]{ReSubYear} 年 \InputLine[0.8cm]{ReSubMonth} 月 \InputLine[0.8cm]{ReSubDay} 日 )
\end{tabular}
\vspace{3em}

% --- 評価テーブル ---
\renewcommand{\arraystretch}{1.5}
\noindent
% 幅調整: 全幅が余白内に収まるように設定
\begin{tabular}{|c|m{10.8cm}|c|}
\hline
\multicolumn{2}{|c|}{\textbf{評 価 項 目}} & \textbf{評 価} \\
\hline
\multirow{3}{*}{\shortstack{実 習\\[0.2em]評 価}} 
 & (1) 自ら積極的に実験に取り組めた & \InputBox{Score1} \\ \cline{2-3}
 & (2) 実験装置を適切に使用でき,正確に実験を行なえた & \InputBox{Score2} \\ \cline{2-3}
 & (3) グループ内で協力的に実験が行なえた & \InputBox{Score3} \\
\hline
\multirow{4}{*}{\shortstack{報告書\\[0.2em]評 価}} 
 & (1) 結果のまとめかた(図表を含む) & \InputBox{Score4} \\ \cline{2-3}
 & (2) 結果に対する考察 & \InputBox{Score5} \\ \cline{2-3}
 & (3) 報告事項/課題(正しい解答や適切な引用など) & \InputBox{Score6} \\ \cline{2-3}
 & (4) 報告書としての体裁が整っているか & \InputBox{Score7} \\
\hline
\end{tabular}

\end{Form}
\end{document}