\documentclass[a4paper,11pt]{ltjsarticle}

% =============================================
% 1. パッケージ設定 (統合版)
% =============================================
\usepackage[T1]{fontenc}
\usepackage{newtxtext}
\usepackage[varbb]{newtxmath} % 数式フォント (amssymbの代わり)
\usepackage{bm}      % ベクトル太字
\usepackage{mathtools}

% レイアウト・図表関連
% デフォルトは本文用の設定 (margin=25mm)
\usepackage[margin=25mm]{geometry}
\usepackage{array}      
\usepackage{multirow}   
\usepackage{fancyhdr}   
\usepackage{graphicx}
\usepackage{float}
\usepackage{booktabs}
\usepackage{subcaption}

% SI単位・数式処理
\usepackage{siunitx}
\sisetup{
  detect-all,
  inter-unit-product=\ensuremath{{}\cdot{}},
  separate-uncertainty=true
}

% グラフ描画
\usepackage{tikz}
\usepackage{pgfplots}
\pgfplotsset{compat=newest}
\usetikzlibrary{arrows.meta, positioning, calc}

% リンク・参照
\usepackage[superscript]{cite}
\usepackage[hidelinks]{hyperref}
\usepackage[nameinlink,noabbrev]{cleveref}
\crefname{figure}{図}{図}
\crefname{table}{表}{表}
\crefname{equation}{式}{式}

% キャプション
\usepackage{caption}
\captionsetup{
  format=hang,
  labelsep=quad,
  font={small},
  labelfont={bf},
  justification=centering
}
\captionsetup[figure]{justification=centerlast}

% =============================================
% 2. カスタムコマンド定義
% =============================================

% --- 表紙用コマンド ---
\newcommand{\UnderlineBox}[2][3cm]{%
  \underline{\makebox[#1][c]{\vphantom{lp}\large #2}}%
}
\newcommand{\JustifiedLabel}[2]{%
  \makebox[#1][s]{\large\bfseries #2}%
}
\newcommand{\BoldLabel}[1]{%
  {\large\bfseries #1}%
}

% --- 本文用コマンド (微分など) ---
\newcommand{\diff}[2]{\frac{\mathrm{d}#1}{\mathrm{d}#2}}
\newcommand{\pdiff}[2]{\frac{\partial #1}{\partial #2}}

% 画像パス設定 (必要に応じて変更してください)
\graphicspath{{image/}}

% =============================================
% 3. 表紙専用のページスタイル定義
% =============================================
\fancypagestyle{coverpage}{
  \fancyhf{} 
  \renewcommand{\headrulewidth}{0pt} 
  \renewcommand{\footrulewidth}{0pt} 
  % ページ番号は表示せず、学校名のみ表示
  \cfoot{\vspace{5mm}\Large \bfseries 国立長野高専 電気電子工学科}
}

% =============================================
% ドキュメント開始
% =============================================
\begin{document}

% /////////////////////////////////////////////
% ここから表紙 (Cover Page)
% /////////////////////////////////////////////

% 表紙用に余白を切り替え (レイアウト崩れ防止)
\newgeometry{top=25mm, bottom=20mm, left=18mm, right=18mm}
\thispagestyle{coverpage} % 表紙用スタイル適用

% --- タイトル ---
\begin{center}
    \vspace*{0mm} 
    {\Huge \bfseries 電気電子工学実験報告書}
    \vspace{10mm} 
\end{center}

% --- テーマ名 ---
\noindent
\begin{tabular}{@{}ll}
  \BoldLabel{テーマ名} & \UnderlineBox[13.5cm]{4. PIDによる温度制御} \\[2.0em] 
\end{tabular}

% --- 報告者情報 ---
\noindent
\BoldLabel{報告者} \hspace{0.5em}
\UnderlineBox[1.5cm]{5} {\large \textbf{年}} \hspace{0.2em}     
(\UnderlineBox[1.5cm]{E} {\large \textbf{組}}) \hspace{0.2em} 
{\large \textbf{番号}} \UnderlineBox[2.0cm]{234} \hspace{0.5em}   
\UnderlineBox[1.5cm]{B} {\large \textbf{班}} \hspace{1em}       
\UnderlineBox[4.5cm]{栁原魁人}                                         
\vspace{2.0em} 

% --- 実験場所・指導担当 ---
\noindent
\begin{tabular}{@{}p{0.48\textwidth} p{0.48\textwidth}}
  \BoldLabel{実験場所} \hspace{1em} \UnderlineBox[5.5cm]{} & 
  \BoldLabel{指導担当} \hspace{1em} \UnderlineBox[5.5cm]{}   
\end{tabular}
\vspace{2.0em} 

% --- 共同実験者 ---
\noindent
\BoldLabel{共同実験者} \hspace{1em} \UnderlineBox[12.5cm]{石坂知尋,倉科純太郎,中井智大,中澤耕平} 
\vspace{2.5em} 

% --- 日付セクション ---
\noindent
\renewcommand{\arraystretch}{2.0}
\setlength{\tabcolsep}{0pt}
\begin{tabular}{l l l l}
  % 1行目
  \JustifiedLabel{5em}{実験日} & 
  \hspace{0.3em} 令和 \UnderlineBox[0.65cm]{} 年 \UnderlineBox[0.65cm]{} 月 \UnderlineBox[0.65cm]{} 日 & & \\
  
  % 2行目
  \JustifiedLabel{5em}{提出期限} & 
  \hspace{0.3em} 令和 \UnderlineBox[0.65cm]{} 年 \UnderlineBox[0.65cm]{} 月 \UnderlineBox[0.65cm]{} 日 & 
  \hspace{0.3em}$\Rightarrow$\hspace{0.3em} \JustifiedLabel{4em}{提出日} & 
  \hspace{0.3em} 令和 \UnderlineBox[0.65cm]{} 年 \UnderlineBox[0.65cm]{} 月 \UnderlineBox[0.65cm]{} 日 \\
  
  % 3行目
  ( \JustifiedLabel{6em}{再提出期限} & 
  \hspace{0.3em} 令和 \UnderlineBox[0.65cm]{} 年 \UnderlineBox[0.65cm]{} 月 \UnderlineBox[0.65cm]{} 日 & 
  \hspace{0.3em}$\Rightarrow$\hspace{0.3em} \JustifiedLabel{5em}{再提出日} & 
  \hspace{0.3em} 令和 \UnderlineBox[0.65cm]{} 年 \UnderlineBox[0.65cm]{} 月 \UnderlineBox[0.65cm]{} 日 )
\end{tabular}

\vfill 

% --- 評価テーブル ---
\renewcommand{\arraystretch}{1.5}
\begin{center}
\begin{tabular}{|>{\centering\arraybackslash}m{2.4cm}|>{\raggedright\arraybackslash}m{12.1cm}|>{\centering\arraybackslash}m{2.4cm}|}
\hline
\multicolumn{2}{|c|}{\JustifiedLabel{11em}{評 価 項 目}} & \JustifiedLabel{4em}{評 価} \\
\hline
\multirow{3}{*}{\parbox[c][4.5em][c]{2.4cm}{\centering\shortstack{\large\bfseries 実 習\\[0.3em]\large\bfseries 評 価}}} 
 & (1) 自ら積極的に実験に取り組めた &  \\ \cline{2-3}
 & (2) 実験装置を適切に使用でき,正確に実験を行なえた &  \\ \cline{2-3}
 & (3) グループ内で協力的に実験が行なえた &  \\
\hline
\multirow{4}{*}{\parbox[c][6.0em][c]{2.4cm}{\centering\shortstack{\large\bfseries 報告書\\[0.3em]\large\bfseries 評 価}}} 
 & (1) 結果のまとめかた(図表を含む) &  \\ \cline{2-3}
 & (2) 結果に対する考察 &  \\ \cline{2-3}
 & (3) 報告事項/課題(正しい解答や適切な引用など) &  \\ \cline{2-3}
 & (4) 報告書としての体裁が整っているか &  \\
\hline
\end{tabular}
\end{center}

\clearpage % 改ページ

% /////////////////////////////////////////////
% ここから本文 (Main Body)
% /////////////////////////////////////////////

% 余白を本文用に復帰
\restoregeometry 
% ページ番号を1にリセットし、スタイルを標準に戻す
\setcounter{page}{1}
\pagestyle{plain} 

\section{目的}
本実験では,ワンボードマイコンを用いたヒーターの温度制御系を対象として,PID制御の実装およびパラメータ調整を行った.各制御要素(比例・積分・微分)が応答特性に与える影響を定量的に把握するとともに,ステップ応答法による最適なPIDパラメータの設計手法を習得することを目的とした.

\section{原理}
PID制御は,フィードバック制御において最も一般的な手法の一つであり,偏差に対する比例(Proportional),積分(Integral),微分(Derivative)の3要素を用いて操作量を決定する.

\subsection{PID制御の伝達関数}
制御対象への操作量 $u(t)$ は,目標値と出力値の偏差 $e(t)$ を用いて式(\ref{eq:pid_time})のように表される.
\begin{equation}
  u(t) = K_P e(t) + K_I \int_{0}^{t} e(\tau) d\tau + K_D \frac{de(t)}{dt}
  \label{eq:pid_time}
\end{equation}
これをラプラス変換し,伝達関数 $K_C(s)$ として表すと式(\ref{eq:pid_tf})となる.
\begin{equation}
  K_C(s) = K_P + \frac{K_I}{s} + K_D s = K_P \left( 1 + \frac{1}{T_I s} + T_D s \right)
  \label{eq:pid_tf}
\end{equation}
ここで,$K_P$ は比例ゲイン,$K_I$ は積分ゲイン,$K_D$ は微分ゲインである.また,$T_I (= K_P/K_I)$ は積分時間,$T_D (= K_D/K_P)$ は微分時間と定義される.この制御系のブロック図を図\ref{fig:pid_block1}および図\ref{fig:pid_block2}に示す.

\clearpage

% --- 図1 PID制御ブロック図(並列型) ---
\begin{figure}[tb]
  \centering
  \begin{tikzpicture}[auto, node distance=2.0cm, >=Latex]
    % Styles
    \tikzstyle{block} = [draw, rectangle, minimum height=3em, minimum width=4em]
    \tikzstyle{sum} = [draw, circle, inner sep=1pt]
    \tikzstyle{input} = [coordinate]
    \tikzstyle{output} = [coordinate]
    \tikzstyle{branch} = [circle,inner sep=0pt,minimum size=1mm,fill=black,draw=black]

    % Nodes definition
    \node [input] (input) {};
    \node [sum, right=1cm of input] (sum1) {};
    \node [left=0.1cm of sum1] {$+$};
    \node [branch, right=1cm of sum1] (b1) {};
    
    % PID Blocks
    \node [block, right=1.5cm of b1] (kp) {$K_P$};
    \node [block, above=1.8cm of kp] (ki) {$\displaystyle \frac{K_I}{s}$}; 
    \node [block, below=1.8cm of kp] (kd) {$K_D s$}; 

    \node [sum, right=2cm of kp] (sum2) {};
    
    % Dashed Frame
    \draw[dashed] ($(ki.north west)+(-0.8,0.6)$) rectangle ($(kd.south east)+(0.5,-0.6)$);
    \node at ($(ki.north)+(0,0.4)$) {\small PID制御器 $K_C$};

    % Connections
    \draw [->] (input) -- node {\small 目標値} (sum1);
    \draw [-] (sum1) -- (b1);
    \draw [->] (b1) -- (kp);
    \draw [->] (b1) |- (ki);
    \draw [->] (b1) |- (kd);
    \draw [->] (kp) -- (sum2);
    \draw [->] (ki) -| (sum2);
    \draw [->] (kd) -| (sum2);

    \node [block, right=1.5cm of sum2] (plant) {制御対象};
    \node [output, right=1.5cm of plant] (output) {};

    \draw [->] (sum2) -- (plant);
    \draw [->] (plant) -- node [name=y] {\small 出力} (output);
    \draw [->] (y) -- ++(0,-3.5) -| node[pos=0.99] {$-$} (sum1);
  \end{tikzpicture}
  \caption{PID制御のブロック図(並列型)}
  \label{fig:pid_block1}
\end{figure}

% --- 図2 標準形 ---
\begin{figure}[tb]
  \centering
  \begin{tikzpicture}[auto, node distance=1.5cm, >=Latex]
    \tikzstyle{block} = [draw, rectangle, minimum height=2.5em, minimum width=3.5em]
    \tikzstyle{sum} = [draw, circle, inner sep=1pt]
    \tikzstyle{input} = [coordinate]
    \tikzstyle{output} = [coordinate]
    \tikzstyle{branch} = [circle,inner sep=0pt,minimum size=1mm,fill=black,draw=black]

    \draw[dashed] (1.2,-2.5) rectangle (7.0, 2.5);
    \node at (4.1, 2.7) {\small PID制御器(標準形)};

    \node [input] (input) {};
    \node [sum, right=1cm of input] (sum1) {};
    \node [left=0.1cm of sum1] {$+$};
    \node [block, right=0.8cm of sum1] (kp) {$K_P$};
    \node [branch, right=0.6cm of kp] (b2) {};
    \node [block, right=1.8cm of kp] (one) {$1$};
    \node [block, above=1.2cm of one] (ti) {$\displaystyle \frac{1}{T_I s}$};
    \node [block, below=1.2cm of one] (td) {$T_D s$};
    \node [sum, right=1.5cm of one] (sum2) {};
    \node [block, right=1.2cm of sum2] (plant) {制御対象};
    \node [output, right=1.5cm of plant] (output) {};

    \draw [->] (input) -- node {\small 目標値} (sum1);
    \draw [->] (sum1) -- (kp);
    \draw [-] (kp) -- (b2);
    \draw [->] (b2) -- (one);
    \draw [->] (b2) |- (ti);
    \draw [->] (b2) |- (td);
    \draw [->] (one) -- (sum2);
    \draw [->] (ti) -| (sum2);
    \draw [->] (td) -| (sum2);
    \draw [->] (sum2) -- (plant);
    \draw [->] (plant) -- node [name=y] {\small 出力} (output);
    \draw [->] (y) -- ++(0,-3.5) -| node[pos=0.99] {$-$} (sum1);
  \end{tikzpicture}
  \caption{PID制御のブロック図(標準型)}
  \label{fig:pid_block2}
\end{figure}

各パラメータが制御系に及ぼす影響は以下の通りである\cite{toyohashi}.
\begin{description}
    \item[比例動作 ($K_P$)] 偏差に比例した操作量を出力する.$K_P$ の増加に伴い応答速度は向上し定常偏差は減少するが,過大なゲインは振動的な挙動を招く.
    \item[積分動作 ($K_I$ または $T_I$)] 偏差の積分値に応じた操作量を出力する.低周波成分のゲインが無限大となるため,定常偏差を完全に除去できる.
    \item[微分動作 ($K_D$ または $T_D$)] 偏差の変化率に応じた操作量を出力する.未来の偏差を予測して補償するため,速応性の向上と振動抑制(ダンピング)に寄与する.
\end{description}

\subsection{パラメータ調整法(ステップ応答法)}
制御対象が開ループにおいて安定であり,式(\ref{eq:plant})のような1次遅れ系とむだ時間要素で近似できる場合,ステップ応答法によるパラメータ調整が可能である.
\begin{equation}
  G_P(s) = \frac{K}{1+sT}e^{-Ls}
  \label{eq:plant}
\end{equation}
図\ref{fig:step_response_main}に示すステップ応答波形から,むだ時間 $L$ および時定数 $T$(または変曲点の接線から求めた傾き $R$)を読み取り,表\ref{tab:zn_parameter}に基づき各パラメータを設定する.

% --- 図3 ステップ応答波形 ---
\begin{figure}[tb]
  \centering
  \begin{tikzpicture}[>=Latex, xscale=5, yscale=2.5]
    \draw[->] (-0.1,0) -- (1.2,0) node[below] {$t$};
    \draw[->] (0,-0.1) -- (0,1.2);
    \draw[thick] (0,0) -- (0,1) -- (1.1,1);
    \node at (0.3, 1.3) {\small ステップ入力};
    \node at (-0.05, 1) {\small 1};
    \draw[thick, domain=0:1.1, smooth, samples=100] plot (\x, {1 / (1 + exp(-20*(\x-0.3))) * 0.6});
    \node at (0.6, 0.4) {\small ステップ応答};
    \draw[dashed] (0,0.6) node[left] {$K$} -- (1.1,0.6);
    \draw[thin] (0.2, 0) -- (0.45, 0.75);
    \node at (0.38, 0.62) {\small 接線};
    \draw[dotted] (0.2, 0) -- (0.2, -0.2);
    \draw[dotted] (0.4, 0) -- (0.4, -0.2);
    \draw[<->] (0, -0.1) -- node[fill=white] {$L$} (0.2, -0.1);
    \draw[<->] (0.2, -0.1) -- node[fill=white] {$T$} (0.4, -0.1);
    \node at (0.7, 0.25) {$\displaystyle R = \frac{K}{T}$};
    \node at (0.3, 0.3) [circle,fill,inner sep=1pt]{};
    \node at (0.25, 0.35) {\footnotesize 変曲点};
  \end{tikzpicture}
  \caption{ステップ応答法におけるパラメータの読み取り}
  \label{fig:step_response_main}
\end{figure}

\begin{table}[tb]
\centering
\caption{ステップ応答法によるPIDパラメータ調整則}
\label{tab:zn_parameter}
  \begin{tabular}{lccc}
    \toprule
    パラメータ & $K_P$ & $T_I$ & $T_D$ \\
    \midrule
    P制御 & $\displaystyle \frac{1}{RL}$ & --- & --- \\
    PI制御 & $\displaystyle \frac{0.9}{RL}$ & $3.3L$ & --- \\
    PID制御 & $\displaystyle \frac{1.2}{RL}$ & $2L$ & $0.5L$ \\
    \bottomrule
  \end{tabular}
\end{table}

\section{実験方法}

\subsection{使用機器}
本実験で使用した主要機器を表\ref{tab:equipment}に示す.

\begin{table}[H]
\centering
\caption{使用機器一覧}
\label{tab:equipment}
\begin{tabular}{lcc}
\toprule
品名 & 型番 & 備考 \\
\midrule
温度制御実習装置 & KENTAC3522S & ヒーター,温度センサ内蔵 \\
制御用PC & --- & Windows OS, PidMonitorインストール済 \\
\bottomrule
\end{tabular}
\end{table}

\subsection{実験手順}
ヒーターの温度を制御量とし,目標温度を \SI{100}{\degreeCelsius} に設定して以下の実験を行った.

\subsubsection{ON・OFF制御}
制御用ソフトウェア(PidMonitor)のAutoモードを解除し,手動によるON・OFF制御を行った.
MV(操作量)を 50.0 \% に設定し,PV(現在温度)が \SI{100}{\degreeCelsius} に到達するまで加熱した.\SI{100}{\degreeCelsius} を超過した時点でMVを 0.0 \% とし,下回った時点で再度 50.0 \% に戻す操作を繰り返し,その際の温度挙動(ハンチング)を確認した.
    \begin{figure}[H]
      \centering
      \includegraphics[width=0.65\linewidth]{手動ONOFF.png}
      \caption{手動ON/OFF制御時の温度挙動}
      \label{fig:onoff}
    \end{figure}
\subsubsection{P制御}
積分時間 $T_I = 0$(実験装置仕様),微分時間 $T_D = 0$ とし,比例制御のみを行った.
    比例ゲイン $K_P$ を 1,5,および 100 に設定した場合の温度変化を測定した.以下にそれぞれの応答波形を示す.
    \begin{figure}[H]
      \centering
      \begin{subfigure}[b]{0.8\textwidth}
        \centering
        \includegraphics[width=0.8\textwidth]{P_1.png}
        \caption{$K_P=1$}
        \label{fig:P1}
      \end{subfigure}
      \par\medskip
      \begin{subfigure}[b]{0.8\textwidth}
        \centering
        \includegraphics[width=0.8\textwidth]{P_5.png}
        \caption{$K_P=5$}
        \label{fig:P5}
      \end{subfigure}
      \par\medskip
      \begin{subfigure}[b]{0.8\textwidth}
        \centering
        \includegraphics[width=0.8\textwidth]{P_100.png}
        \caption{$K_P=100$}
        \label{fig:P100}
      \end{subfigure}
      \caption{P制御の比較($K_P$の違い)}
      \label{fig:P_control}
    \end{figure}

      その後,3分以内に目標温度 $\pm \SI{10}{\degreeCelsius}$ の範囲に収束する最適な $K_P=5$ を試行錯誤により決定した.

\subsubsection{PI制御}
    $K_P$ を $5$ に固定し,$T_D = 0$ としてPI制御を行った.
    積分時間 $T_I$ を 1,48,および 100 に設定した場合の応答を確認した.以下に代表的な波形を示す.
    \begin{figure}[H]
      \centering
      \begin{subfigure}[b]{0.8\textwidth}
        \centering
        \includegraphics[width=0.8\textwidth]{P_5-Ti_1.png}
        \caption{$T_I=1$}
        \label{fig:PI_Ti1}
      \end{subfigure}
      \par\medskip
      \begin{subfigure}[b]{0.8\textwidth}
        \centering
        \includegraphics[width=0.8\textwidth]{P_5-Ti_48.png}
        \caption{$T_I=48$}
        \label{fig:PI_Ti48}
      \end{subfigure}
      \par\medskip
      \begin{subfigure}[b]{0.8\textwidth}
        \centering
        \includegraphics[width=0.8\textwidth]{P_5-Ti_100.png}
        \caption{$T_I=100$}
        \label{fig:PI_Ti100}
      \end{subfigure}
      \caption{PI制御における積分時間 $T_I$ の影響($K_P=5$固定)}
      \label{fig:PI_control}
    \end{figure}

\subsubsection{PID制御}
    $K_P$ および $T_I$ をそれぞれ $5$ および $48$ に固定し,PID制御を行った.
    微分時間 $T_D$ を 1,8.7,および 100 に設定した場合の応答を確認した.以下に代表的な波形を示す.
    \begin{figure}[H]
      \centering
      \begin{subfigure}[b]{0.8\textwidth}
        \centering
        \includegraphics[width=0.8\textwidth]{P_5-Ti_48-Td1.png}
        \caption{$T_D=1$}
        \label{fig:PID_Td1}
      \end{subfigure}
      \par\medskip
      \begin{subfigure}[b]{0.8\textwidth}
        \centering
        \includegraphics[width=0.8\textwidth]{P_5-Ti_48-Td_8.7.png}
        \caption{$T_D=8.7$}
        \label{fig:PID_Td87}
      \end{subfigure}
      \par\medskip
      \begin{subfigure}[b]{0.8\textwidth}
        \centering
        \includegraphics[width=0.8\textwidth]{P_5-Ti_48-Td_100.png}
        \caption{$T_D=100$}
        \label{fig:PID_Td100}
      \end{subfigure}
      \caption{PID制御における微分時間 $T_D$ の影響($K_P=5$, $T_I=48$固定)}
      \label{fig:PID_control}
    \end{figure}

    その後,2分以内に目標温度 $\pm \SI{0.5}{\degreeCelsius}$ の範囲に収束する最適な $T_D=8.7$ を決定した.

\subsubsection{ステップ応答法による同定}
ステップ応答法を用いて制御対象のパラメータ同定を行った.
\begin{enumerate}
    \item ヒーター温度が室温で安定している状態から,MVを 10 \% $\sim$ 20 \% の範囲でステップ状に入力した.
    \item 温度が定常状態に達した際の入力変化幅 $U$ [\%] および温度変化幅 $Y$ [\%] を式(\ref{eq:Y}), (\ref{eq:U})により求めた.
    \item 定常ゲイン $K$ および,図\ref{fig:step_response_main}に基づく $L, T, R$ を算出した.
\end{enumerate}
\begin{align}
  U &= (\text{最終入力} - \text{初期入力}) \label{eq:U} \\
  Y &= \frac{(\text{最終温度} - \text{初期温度})}{200} \times 100 \label{eq:Y} \\
  K &= \frac{Y}{U} \label{eq:K} \\
  R &= \frac{K}{T} \label{eq:R}
\end{align}
得られたパラメータを基に,表\ref{tab:zn_parameter}を用いて $K_P, T_I, T_D$ を算出し,これを用いた制御実験を行った.以下にステップ応答法で算出したパラメータを示す.
\begin{table}[H]
  \centering
  \caption{ステップ応答法で算出したPIDパラメータ}
  \label{tab:step_response_params}
  \begin{tabular}{lccc}
    \toprule
    制御タイプ & $K_P$ & $T_I$ & $T_D$ \\
    \midrule
    P制御 & 8.4 & --- & --- \\
    PI制御 & 7.6 & 16 & --- \\
    PID制御 & 10.1 & 9.7 & 2.42 \\
    \bottomrule
  \end{tabular}
\end{table}
これらのパラメータを用いた制御実験の結果を図に示す.
\begin{figure}[H]
  \centering
  \begin{subfigure}[b]{0.8\textwidth}
    \centering
    \includegraphics[width=0.8\textwidth]{P_8.4.png}
    \caption{P制御 ($K_P=8.4$)}
    \label{fig:step_P}
  \end{subfigure}
  \par\medskip
  \begin{subfigure}[b]{0.8\textwidth}
    \centering
    \includegraphics[width=0.8\textwidth]{P_7.6-Ti_16.png}
    \caption{PI制御 ($K_P=7.6$, $T_I=16$)}
    \label{fig:step_PI}
  \end{subfigure}
  \par\medskip
  \begin{subfigure}[b]{0.8\textwidth}
    \centering
    \includegraphics[width=0.8\textwidth]{P_10.1-Ti_9.7-Td_2.42.png}
    \caption{PID制御 ($K_P=10.1$, $T_I=9.7$, $T_D=2.42$)}
    \label{fig:step_PID}
  \end{subfigure}
  \caption{ステップ応答法で算出したパラメータを用いた制御実験結果}
  \label{fig:step_response_experiments}
\end{figure}

\section{結果および考察}

\subsection{ON・OFF制御の挙動}
図\ref{fig:onoff}に示すように,ON・OFF制御では操作量が0 \% または 50 \% の2値しかとらないため,温度が目標値の \SI{100}{\degreeCelsius} に留まることができなかった.温度が上がれば切り,下がれば入れるという操作を繰り返した結果,永続的な温度の波打ち(ハンチング)が発生した.このことから,精密な温度制御にはON・OFF制御は不向きであり,操作量を連続的に変化させる制御が必要であることが確認できた.

\subsection{P制御(比例制御)の特性}
P制御における各ゲインでの応答(図\ref{fig:P_control})について考察する.
\begin{itemize}
    \item \textbf{$K_P=1$ の場合}: 操作量が不足しているため立ち上がりが遅く,時間が経過しても目標温度である \SI{100}{\degreeCelsius} に到達しなかった.このように,P制御だけでは解消できない目標値とのズレを「残留偏差(オフセット)」という.
    \item \textbf{$K_P=100$ の場合}: ゲインを大きくしすぎたため,少しの偏差に対して操作量が過剰に反応してしまった.その結果,激しい振動が発生し,制御不能となった.
    \item \textbf{$K_P=5$ の場合}: 立ち上がりの速さと安定性のバランスが比較的良かったが,それでも目標値に対して数度の残留偏差が残った.
\end{itemize}
以上の結果から,P制御においてゲイン $K_P$ を上げれば立ち上がりは速くなるが,上げすぎると振動すること,また適切な値に設定しても残留偏差は消せないことがわかった.

\subsection{PI制御(比例・積分制御)の特性}
PI制御(図\ref{fig:PI_control})では,積分動作を追加することで,P制御で問題となった残留偏差を解消することができた.積分時間 $T_I$ の違いによる影響は以下の通りである.
\begin{itemize}
    \item \textbf{$T_I=1$ の場合}: 積分時間が短すぎる(積分作用が強すぎる)ため,過去の偏差が急速に積み重なり,操作量が過大となった.その結果,目標温度を大きく超える「オーバーシュート」が発生し,収束までに時間がかかった.
    \item \textbf{$T_I=100$ の場合}: 積分時間が長すぎる(積分作用が弱い)ため,残留偏差を修正する力が弱く,目標温度である \SI{100}{\degreeCelsius} にピタリと合うまでに時間がかかった.
    \item \textbf{$T_I=48$ の場合}: オーバーシュートを抑えつつ,速やかに残留偏差を解消できており,最も良好な波形となった.
\end{itemize}

\subsection{PID制御(比例・積分・微分制御)の特性}
PID制御(図\ref{fig:PID_control})では,微分動作を加えることで,急激な温度変化に対してブレーキをかける効果が確認された.
\begin{itemize}
    \item \textbf{$T_D=100$ の場合}: 微分時間が長すぎるため,わずかな温度変化やセンサのノイズ(雑音)に対しても過敏に反応してしまった.図\ref{fig:PID_Td100}の青線(操作量MV)を見ると,激しくギザギザに変動していることがわかる.これは装置への負担が大きく,制御としても不安定である.
    \item \textbf{$T_D=8.7$ の場合}: 適度な微分作用により,立ち上がり時のオーバーシュートが抑制され,最も速く安定して目標温度に到達した.図\ref{fig:PID_Td87}を見ると,温度が滑らかに目標値に収束している.
\end{itemize}

\subsection{理論値(ステップ応答法)と実験値の比較}
ステップ応答法で算出したパラメータ(表\ref{tab:step_response_params})と,実際に試行錯誤で決定した最適値を比較すると,大きな差が見られた.
具体的には,計算で求めた比例ゲイン $K_P=10.1$ は,実験で最適とした $K_P=5$ よりもかなり大きい値であった.計算値をそのまま用いた図\ref{fig:step_PID}では波形が振動的になっており,安定しているとは言い難い.

この不一致の原因として,以下の点が考えられる.
\begin{itemize}
    \item \textbf{モデル化の誤差}: 計算式では「理想的な加熱」を仮定しているが,実際の実験装置には熱の逃げや,ヒーター自体の反応の遅れなど,単純な式では表せない複雑な要素が含まれている.
    \item \textbf{操作量の限界}: 計算上では操作量は無限に出せる前提だが,実際には 0 \% から 100 \% の範囲でしか出力できない(飽和という).これにより,理論通りの強力な制御ができなかった.
\end{itemize}
したがって,ステップ応答法などの計算で求めた値はあくまで「目安」として利用し,最終的には実際の波形を見ながら,少し弱めの設定(ゲインを下げるなど)から微調整を行うことが,現場では重要であると結論付けられる.

\section{報告事項}
課題として与えられた制御対象 $G(s)$ に対し,前述のステップ応答法(表\ref{tab:zn_parameter})を用いてPIDパラメータの設計を行った.
対象の伝達関数は次式で与えられる.
\begin{equation}
    G(s) = \frac{5}{2s+4}e^{-3s}
\end{equation}
これを1次遅れ系とむだ時間の標準形 $G_P(s) = \frac{K}{1+sT}e^{-Ls}$ に帰着させるため,分母の定数項が1となるよう分母・分子を4で除算し,式変形を行う.
\begin{equation}
    G(s) = \frac{5/4}{(2/4)s + 1}e^{-3s} = \frac{1.25}{1+0.5s}e^{-3s}
\end{equation}
上式と標準形の係数比較により,以下のシステムパラメータが得られる.
\begin{itemize}
    \item システムゲイン $K = 1.25$
    \item 時定数 $T = 0.5$
    \item むだ時間 $L = 3.0$
\end{itemize}
また,ステップ応答の最大傾斜 $R$ は次式で算出される.
\begin{equation}
    R = \frac{K}{T} = \frac{1.25}{0.5} = 2.5
\end{equation}
これらの値をPID制御の調整則($K_P=1.2/RL$, $T_I=2L$, $T_D=0.5L$)に代入し,各パラメータを算出した.結果を表\ref{tab:calc_result}に示す.

\begin{table}[H]
  \centering
  \caption{算出されたPIDパラメータ}
  \label{tab:calc_result}
  \begin{tabular}{lcc}
    \toprule
    パラメータ & 計算式 & 算出値 \\
    \midrule
    比例ゲイン $K_P$ & $\displaystyle \frac{1.2}{2.5 \times 3.0}$ & \textbf{0.16} \\
    積分時間 $T_I$ & $2 \times 3.0$ & \textbf{6.0} \\
    微分時間 $T_D$ & $0.5 \times 3.0$ & \textbf{1.5} \\
    \bottomrule
  \end{tabular}
\end{table}

\begin{thebibliography}{9}
\bibitem{toyohashi} 豊橋技術科学大学・高等専門学校制御工学教育連携プロジェクト 編: 『制御工学』, 実教出版, pp.146-147, 2012.
\end{thebibliography}

\end{document}