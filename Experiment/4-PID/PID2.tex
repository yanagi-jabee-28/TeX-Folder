% ===== ドキュメントクラス =====
\documentclass[
  a4paper,
  11pt,
  oneside
]{ltjsarticle}

% ----- フォント・数式関連 -----
\usepackage[T1]{fontenc}
\usepackage{newtxtext}
\usepackage{newtxmath}
\usepackage{amsmath}
\let\Bbbk\relax
\usepackage{amssymb}
\usepackage{bm}

% ----- 画像・単位・レイアウト関連 -----
\usepackage{graphicx} 
\usepackage[export]{adjustbox}
\usepackage{siunitx}
\usepackage{float}
\usepackage{placeins}
\usepackage[margin=25mm]{geometry}
\usepackage{booktabs}
\usepackage{url}
\usepackage{caption}
\usepackage{subcaption}
% 画像パスを相対指定(この TeX ファイルから見た image ディレクトリ)
\graphicspath{{image/}}

% ----- グラフ描画機能 -----
\usepackage{tikz}
\usepackage{pgfplots}
\pgfplotsset{compat=1.18}
\usetikzlibrary{arrows.meta, positioning, calc, shapes.geometric}

% 画像の高さ制限設定
\setkeys{Gin}{keepaspectratio}

% キャプション設定
\captionsetup[table]{name=表, labelsep=quad, justification=raggedright, singlelinecheck=false}
\captionsetup[figure]{name=図, labelsep=quad, justification=centerlast}

% 参考文献の上付き表示設定
\makeatletter
\def\@cite#1#2{\textsuperscript{[{#1\if@tempswa , #2\fi}]}}
\makeatother

\usepackage[colorlinks=true,linkcolor=black,citecolor=black,urlcolor=black,anchorcolor=black]{hyperref}

% ===== ドキュメント情報 =====
\title{\textbf{4. PIDによる温度制御}}
\author{電気電子工学科 X年 X組 番号 氏名} 
\date{2025年12月4日}

% ===== ここからドキュメント本体 =====
\begin{document}

\maketitle

\section{目的}
本実験では,ワンボードマイコンを用いたヒーターの温度制御系を対象として,PID制御の実装およびパラメータ調整を行った.各制御要素(比例・積分・微分)が応答特性に与える影響を定量的に把握するとともに,ステップ応答法による最適なPIDパラメータの設計手法を習得することを目的とした.

\section{原理}
PID制御は,フィードバック制御において最も一般的な手法の一つであり,偏差に対する比例(Proportional),積分(Integral),微分(Derivative)の3要素を用いて操作量を決定する.

\subsection{PID制御の伝達関数}
制御対象への操作量 $u(t)$ は,目標値と出力値の偏差 $e(t)$ を用いて式(\ref{eq:pid_time})のように表される.
\begin{equation}
  u(t) = K_P e(t) + K_I \int_{0}^{t} e(\tau) d\tau + K_D \frac{de(t)}{dt}
  \label{eq:pid_time}
\end{equation}
これをラプラス変換し,伝達関数 $K_C(s)$ として表すと式(\ref{eq:pid_tf})となる.
\begin{equation}
  K_C(s) = K_P + \frac{K_I}{s} + K_D s = K_P \left( 1 + \frac{1}{T_I s} + T_D s \right)
  \label{eq:pid_tf}
\end{equation}
ここで,$K_P$ は比例ゲイン,$K_I$ は積分ゲイン,$K_D$ は微分ゲインである.また,$T_I (= K_P/K_I)$ は積分時間,$T_D (= K_D/K_P)$ は微分時間と定義される.この制御系のブロック図を図\ref{fig:pid_block1}および図\ref{fig:pid_block2}に示す.

% --- 図1 PID制御ブロック図(並列型) ---
\begin{figure}[tb]
  \centering
  \begin{tikzpicture}[auto, node distance=2.0cm, >=Latex]
    % Styles
    \tikzstyle{block} = [draw, rectangle, minimum height=3em, minimum width=4em]
    \tikzstyle{sum} = [draw, circle, inner sep=1pt]
    \tikzstyle{input} = [coordinate]
    \tikzstyle{output} = [coordinate]
    \tikzstyle{branch} = [circle,inner sep=0pt,minimum size=1mm,fill=black,draw=black]

    % Nodes definition
    \node [input] (input) {};
    \node [sum, right=1cm of input] (sum1) {};
    \node [left=0.1cm of sum1] {$+$};
    \node [branch, right=1cm of sum1] (b1) {};
    
    % PID Blocks
    \node [block, right=1.5cm of b1] (kp) {$K_P$};
    \node [block, above=1.8cm of kp] (ki) {$\displaystyle \frac{K_I}{s}$}; 
    \node [block, below=1.8cm of kp] (kd) {$K_D s$}; 

    \node [sum, right=2cm of kp] (sum2) {};
    
    % Dashed Frame
    \draw[dashed] ($(ki.north west)+(-0.8,0.6)$) rectangle ($(kd.south east)+(0.5,-0.6)$);
    \node at ($(ki.north)+(0,0.4)$) {\small PID制御器 $K_C$};

    % Connections
    \draw [->] (input) -- node {\small 目標値} (sum1);
    \draw [-] (sum1) -- (b1);
    \draw [->] (b1) -- (kp);
    \draw [->] (b1) |- (ki);
    \draw [->] (b1) |- (kd);
    \draw [->] (kp) -- (sum2);
    \draw [->] (ki) -| (sum2);
    \draw [->] (kd) -| (sum2);

    \node [block, right=1.5cm of sum2] (plant) {制御対象};
    \node [output, right=1.5cm of plant] (output) {};

    \draw [->] (sum2) -- (plant);
    \draw [->] (plant) -- node [name=y] {\small 出力} (output);
    \draw [->] (y) -- ++(0,-3.5) -| node[pos=0.99] {$-$} (sum1);
  \end{tikzpicture}
  \caption{PID制御のブロック図(並列型)}
  \label{fig:pid_block1}
\end{figure}

% --- 図2 標準形 ---
\begin{figure}[tb]
  \centering
  \begin{tikzpicture}[auto, node distance=1.5cm, >=Latex]
    \tikzstyle{block} = [draw, rectangle, minimum height=2.5em, minimum width=3.5em]
    \tikzstyle{sum} = [draw, circle, inner sep=1pt]
    \tikzstyle{input} = [coordinate]
    \tikzstyle{output} = [coordinate]
    \tikzstyle{branch} = [circle,inner sep=0pt,minimum size=1mm,fill=black,draw=black]

    \draw[dashed] (1.2,-2.5) rectangle (7.0, 2.5);
    \node at (4.1, 2.7) {\small PID制御器(標準形)};

    \node [input] (input) {};
    \node [sum, right=1cm of input] (sum1) {};
    \node [left=0.1cm of sum1] {$+$};
    \node [block, right=0.8cm of sum1] (kp) {$K_P$};
    \node [branch, right=0.6cm of kp] (b2) {};
    \node [block, right=1.8cm of kp] (one) {$1$};
    \node [block, above=1.2cm of one] (ti) {$\displaystyle \frac{1}{T_I s}$};
    \node [block, below=1.2cm of one] (td) {$T_D s$};
    \node [sum, right=1.5cm of one] (sum2) {};
    \node [block, right=1.2cm of sum2] (plant) {制御対象};
    \node [output, right=1.5cm of plant] (output) {};

    \draw [->] (input) -- node {\small 目標値} (sum1);
    \draw [->] (sum1) -- (kp);
    \draw [-] (kp) -- (b2);
    \draw [->] (b2) -- (one);
    \draw [->] (b2) |- (ti);
    \draw [->] (b2) |- (td);
    \draw [->] (one) -- (sum2);
    \draw [->] (ti) -| (sum2);
    \draw [->] (td) -| (sum2);
    \draw [->] (sum2) -- (plant);
    \draw [->] (plant) -- node [name=y] {\small 出力} (output);
    \draw [->] (y) -- ++(0,-3.5) -| node[pos=0.99] {$-$} (sum1);
  \end{tikzpicture}
  \caption{PID制御のブロック図(標準型)}
  \label{fig:pid_block2}
\end{figure}

各パラメータが制御系に及ぼす影響は以下の通りである\cite{toyohashi}.
\begin{description}
    \item[比例動作 ($K_P$)] 偏差に比例した操作量を出力する.$K_P$ の増加に伴い応答速度は向上し定常偏差は減少するが,過大なゲインは振動的な挙動を招く.
    \item[積分動作 ($K_I$ または $T_I$)] 偏差の積分値に応じた操作量を出力する.低周波成分のゲインが無限大となるため,定常偏差を完全に除去できる.
    \item[微分動作 ($K_D$ または $T_D$)] 偏差の変化率に応じた操作量を出力する.未来の偏差を予測して補償するため,速応性の向上と振動抑制(ダンピング)に寄与する.
\end{description}

\subsection{パラメータ調整法(ステップ応答法)}
制御対象が開ループにおいて安定であり,式(\ref{eq:plant})のような1次遅れ系とむだ時間要素で近似できる場合,ステップ応答法によるパラメータ調整が可能である.
\begin{equation}
  G_P(s) = \frac{K}{1+sT}e^{-Ls}
  \label{eq:plant}
\end{equation}
図\ref{fig:step_response_main}に示すステップ応答波形から,むだ時間 $L$ および時定数 $T$(または変曲点の接線から求めた傾き $R$)を読み取り,表\ref{tab:zn_parameter}に基づき各パラメータを設定する.

% --- 図3 ステップ応答波形 ---
\begin{figure}[tb]
  \centering
  \begin{tikzpicture}[>=Latex, xscale=5, yscale=2.5]
    \draw[->] (-0.1,0) -- (1.2,0) node[below] {$t$};
    \draw[->] (0,-0.1) -- (0,1.2);
    \draw[thick] (0,0) -- (0,1) -- (1.1,1);
    \node at (0.3, 1.3) {\small ステップ入力};
    \node at (-0.05, 1) {\small 1};
    \draw[thick, domain=0:1.1, smooth, samples=100] plot (\x, {1 / (1 + exp(-20*(\x-0.3))) * 0.6});
    \node at (0.6, 0.4) {\small ステップ応答};
    \draw[dashed] (0,0.6) node[left] {$K$} -- (1.1,0.6);
    % 接線: 変曲点 (0.3, 0.3) における導関数は 3.0 (logistic のパラメタから求める)
    % よって接線は y = 3 (x - 0.3) + 0.3
    % ここでは x=0.2 ~ 0.45 の区間に描画して示す
    \draw[thin] (0.2, 0) -- (0.45, 0.75);
    \node at (0.38, 0.62) {\small 接線};
    \draw[dotted] (0.2, 0) -- (0.2, -0.2);
    \draw[dotted] (0.4, 0) -- (0.4, -0.2);
    \draw[<->] (0, -0.1) -- node[fill=white] {$L$} (0.2, -0.1);
    \draw[<->] (0.2, -0.1) -- node[fill=white] {$T$} (0.4, -0.1);
    \node at (0.7, 0.25) {$\displaystyle R = \frac{K}{T}$};
    \node at (0.3, 0.3) [circle,fill,inner sep=1pt]{};
    \node at (0.25, 0.35) {\footnotesize 変曲点};
  \end{tikzpicture}
  \caption{ステップ応答法におけるパラメータの読み取り}
  \label{fig:step_response_main}
\end{figure}

\begin{table}[tb]
\centering
\caption{ステップ応答法によるPIDパラメータ調整則}
\label{tab:zn_parameter}
\begin{tabular}{lccc}
\toprule
制御則 & $K_P$ & $T_I$ & $T_D$ \\
\midrule
P制御 & $\displaystyle \frac{1}{RL}$ & --- & --- \\[1.5ex]
PI制御 & $\displaystyle \frac{0.9}{RL}$ & $3.3L$ & --- \\[1.5ex]
PID制御 & $\displaystyle \frac{1.2}{RL}$ & $2L$ & $0.5L$ \\
\bottomrule
\end{tabular}
\end{table}

\section{実験方法}

\subsection{使用機器}
本実験で使用した主要機器を表\ref{tab:equipment}に示す.

\begin{table}[H]
\centering
\caption{使用機器一覧}
\label{tab:equipment}
\begin{tabular}{lcc}
\toprule
品名 & 型番 & 備考 \\
\midrule
温度制御実習装置 & KENTAC3522S & ヒーター,温度センサ内蔵 \\
制御用PC & --- & Windows OS, PidMonitorインストール済 \\
\bottomrule
\end{tabular}
\end{table}

\subsection{実験手順}
ヒーターの温度を制御量とし,目標温度を \SI{100}{\degreeCelsius} に設定して以下の実験を行った.

\subsubsection{ON・OFF制御}
制御用ソフトウェア(PidMonitor)のAutoモードを解除し,手動によるON・OFF制御を行った.
MV(操作量)を 50.0 \% に設定し,PV(現在温度)が \SI{100}{\degreeCelsius} に到達するまで加熱した.\SI{100}{\degreeCelsius} を超過した時点でMVを 0.0 \% とし,下回った時点で再度 50.0 \% に戻す操作を繰り返し,その際の温度挙動(ハンチング)を確認した.
    \begin{figure}[H]
      \centering
      \includegraphics[width=0.7\linewidth]{手動ONOFF.png}
      \caption{手動ON/OFF制御時の温度挙動}
      \label{fig:onoff}
    \end{figure}
\subsubsection{P制御}
積分時間 $T_I = 0$(実験装置仕様),微分時間 $T_D = 0$ とし,比例制御のみを行った.
    比例ゲイン $K_P$ を 1,5,および 100 に設定した場合の温度変化を測定した.以下にそれぞれの応答波形を示す.
    \begin{figure}[H]
      \centering
      \begin{subfigure}[b]{0.32\textwidth}
        \centering
        \includegraphics[width=\textwidth]{P_1.png}
        \caption{$K_P=1$}
        \label{fig:P1}
      \end{subfigure}\hfill
      \begin{subfigure}[b]{0.32\textwidth}
        \centering
        \includegraphics[width=\textwidth]{P_5.png}
        \caption{$K_P=5$}
        \label{fig:P5}
      \end{subfigure}\hfill
      \begin{subfigure}[b]{0.32\textwidth}
        \centering
        \includegraphics[width=\textwidth]{P_100.png}
        \caption{$K_P=100$}
        \label{fig:P100}
      \end{subfigure}
      \caption{P制御の比較($K_P$の違い)}
      \label{fig:P_control}
    \end{figure}

\subsubsection{PI制御}
$K_P$ を前項で決定した値に固定し,$T_D = 0$ としてPI制御を行った.
    積分時間 $T_I$ を 1,48,および 100 に設定した場合の応答を確認した.以下に代表的な波形を示す.
    \begin{figure}[H]
      \centering
      \begin{subfigure}[b]{0.32\textwidth}
        \centering
        \includegraphics[width=\textwidth]{P_5-Ti_1.png}
        \caption{$T_I=1$}
        \label{fig:PI_Ti1}
      \end{subfigure}\hfill
      \begin{subfigure}[b]{0.32\textwidth}
        \centering
        \includegraphics[width=\textwidth]{P_5-Ti_48.png}
        \caption{$T_I=48$}
        \label{fig:PI_Ti48}
      \end{subfigure}\hfill
      \begin{subfigure}[b]{0.32\textwidth}
        \centering
        \includegraphics[width=\textwidth]{P_5-Ti_100.png}
        \caption{$T_I=100$}
        \label{fig:PI_Ti100}
      \end{subfigure}
      \caption{PI制御における積分時間 $T_I$ の影響($K_P=5$固定)}
      \label{fig:PI_control}
    \end{figure}

\subsubsection{PID制御}
$K_P, T_I$ を前項までに決定した値とし,PID制御を行った.
    微分時間 $T_D$ を 1,8.7,および 100 に設定した場合の応答を確認した.以下に代表的な波形を示す.
    \begin{figure}[H]
      \centering
      \begin{subfigure}[b]{0.32\textwidth}
        \centering
        \includegraphics[width=\textwidth]{P_5-Ti_48-Td1.png}
        \caption{$T_D=1$}
        \label{fig:PID_Td1}
      \end{subfigure}\hfill
      \begin{subfigure}[b]{0.32\textwidth}
        \centering
        \includegraphics[width=\textwidth]{P_5-Ti_48-Td_8.7.png}
        \caption{$T_D=8.7$}
        \label{fig:PID_Td87}
      \end{subfigure}\hfill
      \begin{subfigure}[b]{0.32\textwidth}
        \centering
        \includegraphics[width=\textwidth]{P_5-Ti_48-Td_100.png}
        \caption{$T_D=100$}
        \label{fig:PID_Td100}
      \end{subfigure}
      \caption{PID制御における微分時間 $T_D$ の影響($K_P=5$, $T_I=48$固定)}
      \label{fig:PID_control}
    \end{figure}

\subsubsection{ステップ応答法による同定}
ステップ応答法を用いて制御対象のパラメータ同定を行った.
\begin{enumerate}
    \item ヒーター温度が室温で安定している状態から,MVを 10 \% $\sim$ 20 \% の範囲でステップ状に入力した.
    \item 温度が定常状態に達した際の入力変化幅 $U$ [\%] および温度変化幅 $Y$ [\%] を式(\ref{eq:Y}), (\ref{eq:U})により求めた.
    \item 定常ゲイン $K$ および,図\ref{fig:step_response_main}に基づく $L, T, R$ を算出した.
\end{enumerate}
\begin{align}
  U &= (\text{最終入力} - \text{初期入力}) \label{eq:U} \\
  Y &= \frac{(\text{最終温度} - \text{初期温度})}{200} \times 100 \label{eq:Y} \\
  K &= \frac{Y}{U} \label{eq:K} \\
  R &= \frac{K}{T} \label{eq:R}
\end{align}
得られたパラメータを基に,表\ref{tab:zn_parameter}を用いて $K_P, T_I, T_D$ を算出し,これを用いた制御実験を行った.

\section{結果および考察}
% ※ 以下の各サブセクションに対し、得られた実験データ(図表)を提示し、論理的な考察を記述すること。

\subsection{各制御手法における応答特性の比較}
本実験で観測されたP制御,PI制御,PID制御の応答波形を図\ref{fig:P_control},図\ref{fig:PI_control},図\ref{fig:PID_control}に示す.
P制御では,ゲイン $K_P$ を大きくすることで立ち上がり時間が短縮されたが,残留偏差(オフセット)が確認された.これは原理上,積分要素がないため定常偏差を除去しきれないことに起因する.
一方,PI制御では $T_I$ の導入により残留偏差が解消された.しかし... [ここに実験結果に基づく考察を記述]

\subsection{パラメータ設定値による挙動の変化}
$K_P, T_I, T_D$ の各値を極端に変化させた際(1 および 100)の挙動について考察する.
微分時間 $T_D$ を過大に設定したケースでは,測定ノイズの増幅により操作量が激しく変動する現象が確認された.これは... [ここに物理的・数学的な考察を記述]

\subsection{ステップ応答法による設計値の妥当性}
ステップ応答法により算出したパラメータを用いた制御結果は,試行錯誤法で得られた結果と比較して... [ここに定量的な比較考察を記述]

\section{報告事項}
以下の制御対象 $G(s)$ に対し,ステップ応答法(表\ref{tab:zn_parameter})を用いてPIDパラメータを設計した結果を示す.
\begin{equation}
    G(s) = \frac{5}{2s+4}e^{-3s}
\end{equation}
まず,式を標準形 $K/(1+sT) \cdot e^{-Ls}$ に変形すると,
\begin{equation}
    G(s) = \frac{1.25}{1 + 0.5s}e^{-3s}
\end{equation}
となる.これより $K=1.25, T=0.5, L=3$ が得られる.
したがって,$R = K/T = 2.5$ となり,PID制御の各パラメータは以下のように算出される.
\begin{align*}
    K_P &= \frac{1.2}{RL} = \frac{1.2}{2.5 \times 3} = 0.16 \\
    T_I &= 2L = 6.0 \\
    T_D &= 0.5L = 1.5
\end{align*}

\begin{thebibliography}{9}
\bibitem{toyohashi} 豊橋技術科学大学・高等専門学校制御工学教育連携プロジェクト 編: 『制御工学』, 実教出版, pp.146-147, 2012.
\end{thebibliography}

\end{document}