\documentclass[
  a4paper,  % 用紙サイズ
  11pt,     % フォントサイズ
]{ltjsarticle}% \usepackage{luatexja-fontspec} % lualatex用日本語フォント設定
\usepackage{newtxtext, newtxmath} % (推奨) Times系のフォント・数式パッケージ
\usepackage{amsmath,amssymb}   % 数式
\usepackage{graphicx}          % 画像の挿入
\usepackage{siunitx}           % 国際単位系(SI)
\usepackage{float}             % 図表の位置調整
\usepackage{anyfontsize}       % フォントサイズの警告抑制
\usepackage[margin=2.5cm]{geometry} % 余白の設定
\usepackage{tikz}
\usepackage{circuitikz}
\usepackage{pgfplots}          % 高機能なグラフ描画
\usepackage{pgfplotstable}     % グラフ描画のための表データ操作
\pgfplotsset{compat=1.18}      % pgfplotsのバージョン互換性設定
\usepgfplotslibrary{statistics} % 統計計算(回帰分析など)ライブラリ
\usetikzlibrary{positioning}   % ノードの相対配置
\usepackage{booktabs}          % 見栄えの良い表
\usepackage{listings}          % ソースコードの表示
\usepackage{wrapfig}           % 図の回り込み
\usepackage{algorithm}         % アルゴリズム記述の環境
\usepackage{algpseudocode}     % algorithm環境内で使う疑似コード
\usepackage{cancel}            % 数式の打ち消し線
\usepackage[hidelinks,colorlinks=true,linkcolor=blue,citecolor=green!60!black,urlcolor=blue]{hyperref} 

\title{パルス符号変調の包括的分析:理論、回路レベル実装、および現代的応用}
\author{氏名}
\date{\today}

\begin{document}

\maketitle
\tableofcontents
\clearpage

\section{序論}

\subsection{通信におけるデジタル革命}

アナログ信号処理が主流であった時代からデジタル信号処理への移行は、現代の通信技術における最も根源的なパラダイムシフトの一つである。アナログシステムは、その連続的な性質上、伝送路で加わる雑音の累積や、信号処理の柔軟性の欠如といった本質的な限界を抱えていた。これに対し、デジタル通信は、情報を離散的な数値(ビット)の系列として表現することで、これらの課題を克服する画期的な解決策を提示した。

\subsection{パルス符号変調(PCM)の定義}

パルス符号変調(Pulse Code Modulation, PCM)は、このデジタル革命の中核をなす技術であり、アナログ波形をデジタルビットストリームに変換するための最も基本的な手法である\cite{ref1}。PCMのプロセスは、三つの連続した基本操作、すなわち「標本化(Sampling)」「量子化(Quantization)」「符号化(Encoding)」によって構成される\cite{ref1}。標本化は連続的な時間軸を離散的な点に分割し、量子化は連続的な振幅値を有限個の離散レベルに近似し、最後に符号化はそれらのレベルを二進数の符号語に変換する。この一連のプロセスにより、いかなるアナログ信号も、デジタルデータとして表現、伝送、処理することが可能となる。

\subsection{PCMの核となる利点と重要性}

PCMが現代技術の礎石となった背景には、その卓越した利点が存在する。最も重要なのは、雑音や歪みに対する優れた耐性である\cite{ref1}。アナログ信号とは異なり、デジタル信号は「0」か「1」かの判別さえできれば、伝送中に加わった雑音を完全に除去し、元の信号を寸分違わず再生することが可能である。この特性は、信号を中継するごとに再生・整形を行う「再生中継」を可能にし、長距離伝送においても品質の劣化がほとんどない通信を実現した\cite{ref1}。さらに、音声、データ、映像といった異なる種類の情報を統一的なデジタル形式で扱えるため、これらの信号を多重化して単一の伝送路で効率的に伝送することにも適している\cite{ref1}。

\subsection{本レポートの構成と範囲}

本レポートは、PCMの包括的な理解を目的とし、その理論的基盤から具体的な回路実装、そして現代社会における応用までを体系的に解説する。まず第1章では、アナログ-デジタル変換の根幹をなす標本化定理、量子化理論、符号化の原理といった理論的枠組みを詳述する。続く第2章では、提供された実験報告書をケーススタディとし、PCM送受信機を構成する各機能ブロックの回路レベルでの動作を、シミュレーション結果を交えながら具体的に分析する。第3章では、電話網や音楽CDといった実世界の応用例を取り上げ、PCMがどのように標準化され、社会に浸透していったかを探る。最後に第4章では、PCMの原理を拡張した、より高度な波形符号化方式への進化の道筋を概観し、デジタル信号処理技術の発展におけるPCMの歴史的意義を結論づける。

\section{第1章 アナログ-デジタル変換の理論的枠組み}

PCMによるアナログ-デジタル変換を理解するためには、その根底にある数学的・概念的な原理を把握することが不可欠である。本章では、PCMを構成する三つの基本プロセス、すなわち標本化、量子化、符号化の理論的基盤を詳細に解説する。

\subsection{1.1 標本化プロセス:時間領域における信号の捕捉}

\subsubsection{1.1.1 ナイキスト-シャノンの標本化定理}

標本化定理(サンプリング定理)は、アナログ信号をデジタル化する上での最も重要な指導原理である。この定理は、「最大周波数$f_{\text{max}}$を持つ帯域制限されたアナログ信号は、その2倍を超える周波数$f_s$(すなわち$f_s > 2f_{\text{max}}$)で標本化(サンプリング)すれば、標本値から元の信号を完全に再構成できる」と定めている\cite{ref1}。この再構成に必要な最小サンプリング周波数$2f_{\text{max}}$の半分の周波数、すなわち$f_s/2$はナイキスト周波数(Nyquist Frequency)と呼ばれる\cite{ref5}。

数学的には、元の連続信号を$g(t)$とし、標本化周期を$T=1/f_s$とすると、標本化された信号列は$g(nT)$で与えられる。この標本値列から元の信号$g(t)$を復元する関係は、以下の染谷-シャノンの公式で示される\cite{ref1}。
\begin{equation}
  g(t) = \sum_{n=-\infty}^{\infty} g(nT) \frac{\sin(\pi f_s(t-nT))}{\pi f_s(t-nT)}
\end{equation}
この式は、各標本点$g(nT)$が、sinc関数($\frac{\sin(x)}{x}$)の形をした補間関数の振幅を決定し、それらを無限に足し合わせることで元の連続波形が復元されることを意味している\cite{ref1}。

\subsubsection{1.1.2 エイリアシング:不十分な標本化がもたらす歪み}

標本化定理の条件($f_s > 2f_{\text{max}}$)が満たされない場合、エイリアシング(折り返し雑音)と呼ばれる回復不可能な歪みが発生する\cite{ref8}。これは、ナイキスト周波数を超える周波数成分が、標本化プロセスを通じてそれよりも低い周波数成分として誤って現れる現象である。例えば、サンプリング周波数が10 kHz(ナイキスト周波数5 kHz)のシステムに7 kHzの信号が入力されると、この信号は折り返されて3 kHz(10 kHz−7 kHz)の偽の信号として現れてしまう。一度発生したエイリアシングは、後段の処理で元の信号と区別して取り除くことは不可能である。

\subsubsection{1.1.3 実践的実装:アンチエイリアシングフィルタの役割}

エイリアシングを未然に防ぐため、実際のシステムでは標本化を行う前に、アナログ信号をアンチエイリアシングフィルタと呼ばれるローパスフィルタ(LPF)に通すことが必須となる。このフィルタの役割は、入力信号の周波数帯域をナイキスト周波数以下に厳密に制限し、$f_{\text{max}}$を超える可能性のあるすべての周波数成分を予め除去することである。

ここで、理論と現実の間の重要な差異を認識する必要がある。標本化定理は、信号が数学的に厳密に帯域制限されていることを前提とする。しかし、現実のフィルタは、ある周波数で完全に信号を遮断する理想的な「レンガの壁」のような特性は持たず、なだらかな減衰特性(ロールオフ)を持つ。したがって、ナイキスト周波数近辺の不要な周波数成分を十分に減衰させるためには、サンプリング周波数$f_s$を理論的最小値である$2f_{\text{max}}$よりも高く設定する必要がある。例えば、人間の可聴域の上限である約20 kHzの音声を扱う音楽CDでは、サンプリング周波数が40 kHzではなく44.1 kHzに設定されている\cite{ref9}。この4.1 kHzの差分は、アンチエイリアシングフィルタのロールオフ特性を考慮した「ガードバンド」として機能し、実用的なフィルタ設計を可能にしている。このように、実際のサンプリング周波数の選定は、理論的要求、部品の物理的制約、そして経済的コストの間のトレードオフの結果なのである。

\subsection{1.2 振幅量子化:信号の値を離散化する}

\subsubsection{1.2.1 量子化の原理}

標本化によって時間軸が離散化された後、次のステップは振幅軸の離散化、すなわち量子化である。量子化とは、連続的な値を取りうる各標本値の振幅を、予め定められた有限個の離散的なレベルのいずれかに割り当てる(近似する)プロセスである\cite{ref1}。Nビットの量子化器を用いる場合、表現可能なレベル数は$2^N$個となる。例えば、本レポートの基となる実験で使用されている8ビットシステムでは、$2^8=256$段階のレベルで信号の振幅を表現する\cite{ref1}。

\subsubsection{1.2.2 量子化誤差と信号対量子化雑音比(SQNR)}

量子化は本質的に近似プロセスであるため、元の標本値と量子化後のレベルとの間には必ず微小な差が生じる。この差は量子化誤差または量子化雑音と呼ばれ、A/D変換における主要なノイズ源となる\cite{ref10}。この量子化誤差の大きさは、システムの性能を測る重要な指標である信号対量子化雑音比(Signal-to-Quantization Noise Ratio, SQNR)によって評価される。

均一量子化(すべての量子化ステップ幅が等しい)を行う線形PCMシステムにおいて、理論的なSQNRは以下の式で与えられることが知られている\cite{ref10}。
\begin{equation}
  \text{SQNR} \approx 20\log_{10}(2^N) + 10\log_{10}(1.5) \approx 6.02N + 1.76
\end{equation}
この式は、量子化ビット数を1ビット増やすごとに、SQNRが約6 dB改善されるという極めて重要な関係を示している。これは「6 dB/bitルール」として知られ、デジタルオーディオや通信システムの設計における基本原則となっている。

\subsubsection{1.2.3 非均一量子化とコンパンディング}

SQNRの式は均一量子化を前提としているが、音声信号のように振幅のダイナミックレンジが広く、かつ信号エネルギーが低振幅領域に集中している信号に対しては、均一量子化は非効率である。なぜなら、大きな振幅を持つ信号も小さな振幅を持つ信号も同じ大きさの量子化誤差を持つため、信号レベルが低いときに雑音が相対的に目立ちやすくなるからである。

この問題は、人間の聴覚が静かな音の中の雑音には敏感である一方、大きな音の中の雑音はマスキング効果によって感じにくいという知覚特性とも密接に関連している。この課題を解決するのが、非均一量子化、特にコンパンディング(Companding)と呼ばれる技術である。コンパンディングは、送信側で信号を非線形に圧縮(Compress)し、受信側で逆の特性で伸張(Expand)するプロセスを指す。具体的には、振幅が小さい信号領域には細かい量子化ステップを割り当てて精度を高く保ち、振幅が大きい信号領域には粗いステップを割り当てる。これにより、信号レベルの大小にかかわらず、知覚的なSN比をほぼ一定に保つことができる。

このコンパンディングを実現するための国際標準として、電話網で広く用いられている二つの方式がある。一つは北米や日本で採用されている\textbf{μ-law(ミュー則)}、もう一つはヨーロッパなどで採用されている\textbf{A-law(エー則)}であり、これらはITU-T勧告G.711として標準化されている\cite{ref11}。これらの対数的な圧縮特性は、人間の聴覚特性に巧みに適合させた音響心理学的なエンジニアリングの好例であり、8ビットという低いビットレートで、12ビットあるいは13ビットの線形PCMに匹敵する知覚品質を実現している。

\subsection{1.3 2進符号化}

\subsubsection{1.3.1 レベルからビットへ}

A/D変換の最終段階は符号化である。このプロセスでは、量子化によって決定された$2^N$個の離散レベルのそれぞれに、一意のNビットの2進符号語(コードワード)を割り当てる。これにより、アナログ信号の標本値は、デジタル回路で処理・伝送可能なビットの系列へと完全に変換される。

\subsubsection{1.3.2 ラインコーディング}

生成されたビットストリームを物理的な伝送路(ケーブルなど)で送る際には、さらにラインコーディング(伝送路符号化)という処理が施されることがある。これは、ビットの「0」と「1」を具体的な電圧レベルやパルス形状(例:NRZ、マンチェスタ符号など)に対応付けるプロセスであり、直流成分の抑制やタイミング情報の抽出といった、伝送に適した信号特性を確保する役割を担う。

\section{第2章 システムアーキテクチャと機能ブロック分析:実践的ケーススタディ}

本章では、前章で述べた理論的枠組みが、実際の電子回路としてどのように具現化されるかを明らかにする。そのために、提供された実験報告書\cite{ref1}に記載されているPCM通信システムの回路をケーススタディとして取り上げ、各機能ブロックの動作原理と役割を、シミュレーションソフトウェア「TINA-TI」\cite{ref14}による解析結果を基に詳細に分析する。

\subsection{2.1 PCM送受信システムの概要}

まず、システム全体の構成を把握する。図1に示されるように、PCMシステムは大きく変調器(送信側)と復調器(受信側)から構成される\cite{ref1}。
\begin{itemize}
    \item \textbf{変調器(送信側)}: 複数のアナログ入力信号(CH1, CH2)を受け、切替回路による多重化、標本化、量子化、符号化を経て、直列のデジタルパルス信号(PCM信号)を生成し、送出する。
    \item \textbf{復調器(受信側)}: 受信したPCM信号を、同期信号を基に分離・符号変換し、D/A変換器によってアナログの階段状波形に戻す。最終的にLPFを通して滑らかな元のアナログ信号を復元する。
\end{itemize}

\subsection{2.2 タイミングおよび多重化サブシステム:システムの心臓部}

\subsubsection{2.2.1 タイミングパルス発生回路}

いかなるデジタルシステムにおいても、各部の動作を正確に同期させるためのクロック信号は不可欠である。本システムでは、図2の回路がその役割を担う\cite{ref1}。125 kHzのマスタークロックを源とし、カウンタIC(SN74LS162)とJKフリップフロップ、論理ゲートを組み合わせることで、システム全体で必要とされる複数のタイミングパルス(T1, T2, T3, T4)を生成する。図11のシミュレーション波形は、カウンタの出力(QA-QD)とロジックに基づいて、各タイミングパルスが設計通りの論理関係で生成されていることを示しており、回路の正常な動作を裏付けている\cite{ref1}。

\subsubsection{2.2.2 時分割多重化(TDM)回路}

PCMの大きな利点の一つは、単一の伝送路で複数の通信チャネルを共有できることである。これを実現するのが時分割多重化(Time-Division Multiplexing, TDM)である。図3の切替回路は、TDMの基本原理を実装している\cite{ref1}。クロック信号(Vclock)の状態に応じて二つのアナログスイッチ(SW3, SW4)を交互にオン・オフさせることで、二つの異なる入力信号源(50 Hzと100 Hzの正弦波)からの信号を、時間的に重ならないように一つの出力ライン(Vout)に合成する。図12のシミュレーション結果は、Voutの波形がVclockの周期で50 Hzの信号と100 Hzの信号に切り替わっている様子を明瞭に示しており、TDMが成功していることを証明している\cite{ref1}。

\subsection{2.3 サンプル&ホールド(S&H)回路:時間を凍結する}

\subsubsection{2.3.1 動作原理}

標本化プロセスを物理的に実現するのが、サンプル&ホールド(S&H)回路である。図4の回路は、その典型的な構成を示している\cite{ref1}。
\begin{enumerate}
    \item \textbf{サンプル・モード}: クロックパルスがHighになりスイッチ(SW1)がオンになると、入力信号がコンデンサ(C1)に接続される。コンデンサは急速に充電され、その両端の電圧は入力信号の瞬間的な電圧値に追従する。
    \item \textbf{ホールド・モード}: クロックパルスがLowになりスイッチがオフになると、コンデンサは回路から切り離される。後段のオペアンプは入力インピーダンスが極めて高いため、コンデンサに蓄えられた電荷はほとんど放電されず、その電圧は一定に保たれる。
\end{enumerate}
この「ホールド」された安定した電圧が、後続のA/D変換器に供給される。A/D変換には有限の時間を要するため、変換中に電圧が変動しないように値を保持するS&H回路の役割は極めて重要である。

\subsubsection{2.3.2 シミュレーションによる動作検証}

図13のシミュレーション波形は、S&H回路の動作を見事に可視化している\cite{ref1}。入力信号(VG1)が連続的に変化するのに対し、出力信号(Vout)はクロック(Vclock)の立ち上がりエッジのタイミングでVG1の値をサンプリングし、次のサンプリングタイミングまでその値を保持することで、特徴的な「階段状」の波形を形成している。また、図5のシミュレーションは、抵抗R1が回路の安定性に関わるものであり、S&Hの基本機能そのものには影響しないことを示している\cite{ref1}。

\subsection{2.4 データ変換とフォーマット処理}

\subsubsection{2.4.1 A/DおよびD/A変換}

図9の回路ブロックは、システムの核となるA/D変換とD/A変換のプロセスを示している\cite{ref1}。注目すべきは、実用上の配慮であるDCオフセットの付加である。多くのA/D変換器は正の電圧しか扱えないため、-5Vから+5Vの交流入力信号(VG1)に+6Vの直流電圧を加えて、全体の信号範囲を+1Vから+11Vにシフトアップしている(Vin)。これにより、信号の負の部分が失われることなくデジタル化される。受信側では、D/A変換後にこの+6Vを差し引くことで、元の交流信号を復元する。図16のシミュレーション結果は、この一連の電圧シフトと、D/A変換器の出力(VDA)が階段状の波形となる様子、そして最終的に復調された信号(V復調)が元の波形を再現していることを示している\cite{ref1}。

\subsubsection{2.4.2 パラレル-シリアル変換(シフトレジスタ)}

A/D変換器の出力は、通常、Nビットのデータが並列(パラレル)に出力される。これを一本の線で伝送するためには、直列(シリアル)のビットストリームに変換する必要がある。この役割を担うのが、図6のシフトレジスタ(U25 74199)である\cite{ref1}。シフトレジスタは、パラレルに入力されたNビットのデータを内部にラッチ(保持)し、クロック信号に同期して1ビットずつ順番に出力する。図14のシミュレーションでは、スイッチで設定された8ビットのパラレル入力(High, Lowの組み合わせ)が、T3パルスのタイミングで取り込まれ、その後クロックに同期して1ビットずつシリアルデータとして出力される様子が確認できる\cite{ref1}。

\subsection{2.5 受信側での信号再構成}

\subsubsection{2.5.1 再構成フィルタ}

受信側でのD/A変換によって得られる信号は、S&H回路の出力と同様に、不自然な階段状の波形である。この波形には、元の信号成分に加えて、標本化プロセスに起因する高周波成分(高調波)が多数含まれている。元の滑らかなアナログ信号を復元するためには、これらの高周波成分を除去するローパスフィルタ(LPF)が不可欠である。本システムでは、図9に示されるサレン・キー(Sallen-Key)型のアクティブLPFがこの目的で使用されている\cite{ref1}。

\subsubsection{2.5.2 遮断周波数の導出}

フィルタの性能を決定づける最も重要なパラメータは、遮断周波数(カットオフ周波数)$f_c$である。これは、フィルタが通過させる周波数帯域と遮断する周波数帯域の境界を示す。報告書では、サレン・キー型LPFの伝達関数が詳細に導出されている\cite{ref1}。

伝達関数$G(j\omega)$は、
\begin{equation}
  G(j\omega) = \frac{\frac{1}{R_1 R_2 C_1 C_2}}{s^2 + s(\frac{1}{R_1 C_1} + \frac{1}{R_2 C_1}) + \frac{1}{R_1 R_2 C_1 C_2}}
\end{equation}
と表され、ここから遮断周波数$f_c$は次式で求められる。
\begin{equation}
  f_c = \frac{1}{2\pi\sqrt{R_1 R_2 C_1 C_2}}
\end{equation}
図9の回路定数($R_1=R_2=50 \text{k}\Omega$, $C_1=2.3 \text{nF}$, $C_2=1.1 \text{nF}$)を代入すると、遮断周波数は以下のように計算される\cite{ref1}。
\begin{equation}
  f_c = \frac{1}{2\pi\sqrt{50 \times 10^3 \times 50 \times 10^3 \times 2.3 \times 10^{-9} \times 1.1 \times 10^{-9}}} \approx 2.0 \times 10^3 \text{ [Hz]}
\end{equation}
すなわち、このLPFの遮断周波数は2.0 kHzである。
この具体的な数値は、システム設計における重要な関連性を浮き彫りにする。このA/D-D/A変換ブロックのシミュレーションでは、入力信号周波数は500 Hzである\cite{ref1}。一方、サンプリングを制御するタイミングパルスT2の周期は、図11から約32 μsと読み取れ、これはサンプリング周波数$f_s \approx 31.25 \text{kHz}$に相当する。標本化定理によれば、このシステムで扱える最大信号周波数は$f_s/2 \approx 15.6 \text{kHz}$である。LPFの設計は、通過させたい信号周波数(500 Hz)と、除去したい標本化に起因する高周波成分($f_s$近傍)の間に遮断周波数が位置するように行われる。計算された2.0 kHzという遮断周波数は、500 Hzを十分に通過させつつ、$f_s/2$よりもはるかに低いため、この設計要件($f_{\text{signal}} < f_c < f_s/2$)を適切に満たしている。これは、時間領域のパラメータ(サンプリング周波数)と周波数領域のパラメータ(フィルタの遮断周波数)が、システム全体として整合性を保つように密接に連携して設計されなければならないことを示す好例である。

\section{第3章 PCMの応用と標準化}

前章で詳述したPCMの原理は、実験室レベルの回路に留まらず、現代社会を支える数多くの通信・音響技術の基盤となっている。本章では、その中でも特に影響力の大きい二つの応用分野、すなわち公衆電話網と音楽CDを取り上げ、それぞれの規格におけるPCMの仕様と、それがもたらした技術的・社会的インパクトを分析する。

\subsection{3.1 電話:デジタルネットワークの声(G.711)}

\subsubsection{3.1.1 背景}

PCM技術が世界で初めて大規模に商用化されたのは、電話の分野であった。1962年に米Bell研究所が開発したT1方式は、24チャネルの音声通話を1本のメタルケーブルで伝送する画期的なシステムであり、その中核を担ったのがPCMであった\cite{ref1}。今日においても、従来の公衆交換電話網(PSTN)からIP網上で音声を伝送するVoIP(Voice over IP)に至るまで、デジタル音声通信の基本としてPCMは広く利用され続けている\cite{ref11}。

\subsubsection{3.1.2 G.711規格の仕様}

電話音声のデジタル化に関する国際標準は、ITU-T勧告G.711として定められている。その主な仕様は以下の通りである\cite{ref11}。
\begin{itemize}
    \item \textbf{サンプリング周波数}: 8 kHz。これは、人間の音声の主要な周波数成分が含まれる帯域(約300 Hzから3.4 kHz)をカバーするために、ナイキスト定理($2 \times 3.4 \text{ kHz} = 6.8 \text{ kHz}$)を十分に満たす値として設定された。
    \item \textbf{量子化ビット数}: 8ビット。
    \item \textbf{データレート}: 8,000 サンプル/秒 $\times$ 8 ビット/サンプル = 64 kbps。この64 kbpsという値は、デジタル通信における基本単位(DS0)として長らく用いられてきた。
\end{itemize}

\subsubsection{3.1.3 μ-lawとA-lawコンパンディング}

G.711の最大の特徴は、第1章で述べたコンパンディング技術を採用している点である。これにより、8ビットという限られたビット数で、音声信号の広いダイナミックレンジを効率的に表現している。
\begin{itemize}
    \item \textbf{μ-law(ミュー則)}: 主に北米や日本で用いられる方式。低レベル信号に対するSN比がA-lawよりもわずかに優れている特性を持つ\cite{ref11}。
    \item \textbf{A-law(エー則)}: 主にヨーロッパやその他の地域で用いられる方式。μ-lawと比較して、計算機での処理がやや簡素であるという利点がある\cite{ref11}。
\end{itemize}
これらの対数的な圧縮伸張アルゴリズムは、音声の明瞭度を保ちながら伝送帯域を節約するという、電話網における経済性と品質の両立という要求に応えるための重要な技術である。

\subsection{3.2 高忠実度オーディオ:CD-DA「レッドブック」規格}

\subsubsection{3.2.1 背景}

PCM技術が一般消費者に広く認知されるきっかけとなったのが、1982年に登場したコンパクトディスク(CD)である。CDは、高品位なデジタルオーディオを家庭で手軽に楽しめる初のマスマーケット製品であり、音楽鑑賞の体験を一変させた。その技術仕様は「レッドブック(Red Book)」と呼ばれる規格書にまとめられている。

\subsubsection{3.2.2 レッドブック規格の仕様}

音楽CDの音質を決定づけるPCMのパラメータは、電話音声とは大きく異なる\cite{ref9}。
\begin{itemize}
    \item \textbf{サンプリング周波数}: 44.1 kHz。これは、健康な若者の可聴域の上限とされる約20 kHzの周波数までを忠実に記録・再生するために、ナイキスト定理($2 \times 20 \text{ kHz} = 40 \text{ kHz}$)にアンチエイリアシングフィルタのためのガードバンドを加味して決定された。
    \item \textbf{量子化ビット数}: 16ビット(線形)。コンパンディングを用いない直線的な量子化であり、$2^{16}=65,536$段階の極めて細かいレベルで音の強弱を表現する。これにより、理論的には約98 dB($6.02 \times 16 + 1.76$)という広大なダイナミックレンジ(最も静かな音と最も大きな音の比)を実現している。
    \item \textbf{チャネル数}: 2チャネル(ステレオ)。
\end{itemize}

\subsubsection{3.2.3 データレートの計算}

CDの非圧縮データレートは、これらの仕様から次のように計算できる\cite{ref9}。
\begin{equation}
  2 \text{ channels} \times 16 \text{ bits/sample} \times 44,100 \text{ samples/s} = 1,411,200 \text{ bps} = 1.4112 \text{ Mbps}
\end{equation}
この1.4 Mbpsというデータレートは、G.711の64 kbpsと比較して約22倍にも達する。この膨大な情報量が、CDの持つ高い忠実度の源泉となっている。

\subsection{PCM主要規格の比較分析}

G.711とCD-DAという二つの代表的な規格を比較することで、同じPCMという技術が、応用分野の要求に応じていかに異なる形で最適化されているかが明確になる。以下の表は、その主要な特徴をまとめたものである。

\begin{table}[H]
  \centering
  \caption{PCM主要規格の比較分析}
  \begin{tabular}{lll}
    \toprule
    \textbf{特徴} & \textbf{G.711 (電話)} & \textbf{CD-DA (レッドブックオーディオ)} \\
    \midrule
    \textbf{主たる応用} & 音声通信 (PSTN, VoIP) & 高忠実度音楽 \\
    \textbf{対象帯域幅} & 300–3400 Hz & 20 Hz–20 kHz \\
    \textbf{サンプリング周波数 ($f_s$)} & 8 kHz & 44.1 kHz \\
    \textbf{量子化方式} & 非均一 (対数的) & 均一 (線形) \\
    \textbf{コンパンディング法則} & μ-law または A-law & なし \\
    \textbf{量子化ビット数 (N)} & 8 ビット & 16 ビット \\
    \textbf{理論的ダイナミックレンジ} & 約72 dB (線形換算) & 約98 dB (6.02×16+1.76) \\
    \textbf{ビットレート (チャネルあたり)} & 64 kbps & 705.6 kbps \\
    \textbf{総ビットレート (ステレオ)} & 128 kbps & 1.4112 Mbps \\
    \bottomrule
  \end{tabular}
\end{table}

この比較表は、技術仕様の選択が常にトレードオフであることを示している。G.711は、音声の「明瞭度」を維持しつつ「帯域効率」を最大化することに最適化されている。一方、CD-DAは、帯域を惜しみなく使用することで、音楽の「忠実度」を極限まで追求することに最適化されている。この対比は、PCMの柔軟性と、その設計思想が応用の目的に深く根差していることを浮き彫りにしている。

\section{第4章 発展的な波形符号化と将来展望}

PCMはデジタル信号処理の礎であるが、その基本的な形式は必ずしも効率的ではない。特に、音声や音楽のような自然信号には、時間的な相関、すなわち隣接するサンプル値が似通っているという強い冗長性が存在する\cite{ref22}。基本PCMは各サンプル値を独立に符号化するため、この冗長性を全く利用していない。本章では、この冗長性を削減し、より低いビットレートで同等の品質を実現するためにPCMから派生した、より高度な波形符号化技術の進化を概観する。

\subsection{4.1 効率化の必要性:信号の冗長性の活用}

PCMの効率を改善するための基本的な考え方は、信号の絶対値をそのまま符号化するのではなく、信号の予測可能な部分を取り除き、予測できなかった「差分(予測誤差)」のみを符号化することにある。自然信号は相関が高いため、過去のサンプル値から現在のサンプル値をある程度予測することが可能である。予測が正確であれば、予測誤差信号の振幅(分散)は元の信号よりもはるかに小さくなり、同じ品質を保ちながら、より少ないビット数で量子化できる。この「予測符号化」という哲学の転換が、PCM以降の波形符号化技術の発展を駆動した。

\subsection{4.2 差分パルス符号変調(DPCM):変化を符号化する}

\subsubsection{4.2.1 原理}

差分パルス符号変調(Differential Pulse Code Modulation, DPCM)は、予測符号化の概念を直接的に実装した最初のステップである\cite{ref2}。DPCMの送信機は、内部に「予測器」を持つ。予測器は、過去に符号化・復号されたサンプル値に基づいて現在のサンプル値を予測する。実際の入力サンプル値とこの予測値との差分が計算され、この差分信号が量子化・符号化されて伝送される。受信機も送信機と全く同じ予測器を持っており、受信した差分信号を予測値に加算することで、元のサンプル値を復元する。

\subsubsection{4.2.2 ブロック図}

DPCMのエンコーダは、減算器、量子化器、加算器、そしてフィードバックループ内に配置された予測器から構成される\cite{ref24}。このフィードバック構造が重要であり、送信機側で量子化誤差を含んだ状態で予測を行うことで、受信機側との予測値の同期を保ち、誤差の累積を防いでいる。

\subsection{4.3 デルタ変調(DM):最もシンプルな予測器}

\subsubsection{4.3.1 原理}

デルタ変調(Delta Modulation, DM)は、DPCMを極限まで単純化した1ビット版のDPCMと見なすことができる\cite{ref26}。DMでは、現在の入力信号が直前の再構成値(階段状の近似波形)よりも大きいか小さいかを比較し、その結果を「+Δ」(1)または「-Δ」(0)の1ビットの情報として送信する。受信機は、この1ビットのストリームを単純に積分(累積加算)することで、元の信号の階段状近似波形を再構成する。その単純さから、ハードウェア実装が非常に容易であるという利点がある。

\subsubsection{4.3.2 限界:傾斜過負荷と粒状雑音}

しかし、DMの単純さは二つの特徴的な歪みという代償を伴う\cite{ref27}。
\begin{itemize}
    \item \textbf{傾斜過負荷(Slope Overload)}: 入力信号の変化が急峻で、固定されたステップ幅Δが追従できない場合に発生する。近似波形が元の信号から大きく遅れてしまい、大きな歪みとなる。
    \item \textbf{粒状雑音(Granular Noise)}: 逆に入力信号が平坦か、変化が緩やかな場合に発生する。近似波形が元の信号の周りをステップ幅Δで上下に振動し続け、ざらざらとした雑音として知覚される。
\end{itemize}
この二つの歪みはトレードオフの関係にあり、ステップ幅Δを大きくすれば傾斜過負荷は改善されるが粒状雑音が増加し、小さくすればその逆となる。

\subsection{4.4 適応差分パルス符号変調(ADPCM):スマートな符号化器}

\subsubsection{4.4.1 原理}

DPCMの効率とDMの限界を踏まえ、さらなる性能向上を目指して開発されたのが適応差分パルス符号変調(Adaptive Differential Pulse Code Modulation, ADPCM)である\cite{ref25}。ADPCMの「適応」という言葉が示す通り、この方式は信号の統計的性質の変化に応じて、符号化パラメータを動的に調整する。主に二つの適応メカニズムがある。
\begin{itemize}
    \item \textbf{適応量子化}: 量子化のステップサイズを信号の振幅に応じて変化させる。信号の振幅が大きい(変化が激しい)区間ではステップサイズを大きくして傾斜過負荷を防ぎ、振幅が小さい(変化が緩やか)区間ではステップサイズを小さくして粒状雑音を抑制する。
    \item \textbf{適応予測}: 予測器の係数を、信号のスペクトル特性の変化に追従するように更新する。これにより、常に最適な予測を行い、予測誤差を最小化しようと試みる。
\end{itemize}
これらの適応技術により、ADPCMは様々な特性を持つ信号に対して頑健に動作し、DPCMよりも低いビットレート(例:32 kbps)で高品質な音声を実現できる。

\subsection{4.5 結論}

\subsubsection{4.5.1 PCMの技術的遺産}

本レポートで詳述してきたように、パルス符号変調(PCM)は、単なる一つの変調方式ではなく、アナログの世界からデジタルの世界への扉を開いた根源的な技術である。その基本原理である標本化、量子化、符号化は、今日のあらゆるデジタルオーディオ、音声通信、画像処理技術の根底に流れている。PCMが確立した、雑音耐性が高く、再生中継が可能で、多様な情報を統一的に扱えるというデジタル伝送の利点は、現代の情報化社会の基盤を築いた。

\subsubsection{4.5.2 中核となるトレードオフの再確認}

PCMとその派生技術の進化の歴史は、常に「忠実度(ビットレート)」「帯域幅」「複雑性」という三つの要素間のトレードオフを最適化する試みの連続であった。高忠実度を求めれば高いビットレートが必要となり(CD-DA)、帯域を節約しようとすれば品質が制限される(G.711)。そして、より低いビットレートで高い品質を達成しようとすれば、予測や適応といった高度なアルゴリズムが必要となり、システムの複雑性が増大する(ADPCM)。このトレードオフの理解は、通信システム設計における普遍的な課題である。

\subsubsection{4.5.3 進化の道筋}

PCMからDPCM、DM、そしてADPCMへと至る技術的系譜は、信号符号化の哲学がどのように進化してきたかを明確に示している。
\begin{itemize}
    \item \textbf{PCM}: 信号の「絶対値」を忠実に表現する。
    \item \textbf{DPCM/DM}: 信号の「差分(変化)」を捉えることで冗長性を削減する。
    \item \textbf{ADPCM}: 信号の「統計的性質の変化に適応」することで、さらなる効率化を達成する。
\end{itemize}
この流れは、信号そのものをナイーブに符号化する段階から、信号の背後にある構造やモデルを推定し、そのモデルからの逸脱(予測誤差)のみを情報として伝えるという、より洗練されたアプローチへの移行を意味する。この「モデルベース符号化」という考え方は、線形予測符号化(LPC)や、現代のMP3、AACといった高効率な音響圧縮アルゴリズムへと直接つながっていく。したがって、PCMを理解することは、デジタル通信の原点を学ぶだけでなく、現代に至る信号圧縮技術の壮大な進化の物語の序章を読むことに他ならない。

\begin{thebibliography}{99}
\bibitem{ref1} PCM通信テキスト(配布用).pdf
\bibitem{ref2} Pulse Code Modulation and Demodulation : Block Diagram & Its Working - ElProCus, 9月 26, 2025にアクセス、 \url{https://www.elprocus.com/pulse-code-modulation-and-demodulation/}
\bibitem{ref3} 標本化定理とは何? わかりやすく解説 Weblio辞書, 9月 26, 2025にアクセス、 \url{https://www.weblio.jp/content/%E6%A8%99%E6%9C%AC%E5%8C%96%E5%AE%9A%E7%90%86}
\bibitem{ref4} サンプリング・レート:これだけは知っておきたいアナログ用語 ..., 9月 26, 2025にアクセス、 \url{https://edn.itmedia.co.jp/edn/articles/1311/11/news008.html}
\bibitem{ref5} www.irisoele.com, 9月 26, 2025にアクセス、 \url{https://www.irisoele.com/jp/technology/knowledge/nyquist-frequency/#:~:text=%E6%9C%AC%E6%9D%A5%E3%81%AF%E3%82%A2%E3%83%8A%E3%83%AD%E3%82%B0%E4%BF%A1%E5%8F%B7%E3%81%AE,%E3%82%82%E4%BD%BF%E3%81%84%E3%81%A6%E3%81%84%E3%81%BE%E3%81%99%E3%80%82}
\bibitem{ref6} ナイキスト周波数(Nyquist Frequency) - Connectors - IRISO Electronics co.,ltd., 9月 26, 2025にアクセス、 \url{https://www.irisoele.com/jp/technology/knowledge/nyquist-frequency/}
\bibitem{ref7} 周波数とデータレートfrequency-and-datarate - Connectors - IRISO Electronics co.,ltd., 9月 26, 2025にアクセス、 \url{https://www.irisoele.com/jp/technology/column/frequency-and-datarate/}
\bibitem{ref8} 音楽に隠し音声!?『ホタルイカ音声暗号』の方法解説!|中井三十一 - note, 9月 26, 2025にアクセス、 \url{https://note.com/harai_tama/n/n280a1474672e}
\bibitem{ref9} Compact Disc Digital Audio - Wikipedia, 9月 26, 2025にアクセス、 \url{https://en.wikipedia.org/wiki/Compact_Disc_Digital_Audio}
\bibitem{ref10} 高いSNRのAD変換 システムを実現する ... - Analog Devices, 9月 26, 2025にアクセス、 \url{https://www.analog.com/media/jp/landing-pages/ATS/ats2020/ats2020_a1.pdf}
\bibitem{ref11} g711 ulaw codec - Ozeki VoIP SIP SDK, 9月 26, 2025にアクセス、 \url{https://voip-sip-sdk.com/p_7220-g711-ulaw-codec.html}
\bibitem{ref12} g711 alaw codec - Ozeki VoIP SIP SDK, 9月 26, 2025にアクセス、 \url{https://voip-sip-sdk.com/p_7214-g711-alaw-codec.html}
\bibitem{ref13} Programming Exercise: Mu-Law Encoding in C++ | PDF | Sampling (Signal Processing) | Theoretical Computer Science - Scribd, 9月 26, 2025にアクセス、 \url{https://www.scribd.com/document/466966210/Mu-Law-Encoding-in-C-pdf}
\bibitem{ref14} TINA-TI™ 操作入門, 9月 26, 2025にアクセス、 \url{https://www.ti.com/jp/lit/ug/jaju146/jaju146.pdf}
\bibitem{ref15} ドライバ回路の設計を容易化するΣΔ型/SAR型の最新ADC - Analog Devices, 9月 26, 2025にアクセス、 \url{https://www.analog.com/jp/resources/analog-dialogue/articles/maximize-the-performance-of-your-sigma-delta-adc-driver.html}
\bibitem{ref16} ベーシック・リニア・デザイン第 6 章: コンバータ - Analog Devices, 9月 26, 2025にアクセス、 \url{https://www.analog.com/media/jp/training-seminars/design-handbooks/basic-linear-design/EDCh_6_Converter_jp.pdf}
\bibitem{ref17} 高速A-D変換のしくみとIC活用術(前編) - Yokogawa Test & Measurement, 9月 26, 2025にアクセス、 \url{https://tmi.yokogawa.com/jp/library/resources/measurement-tips/high_speed_a_d_conversion_mechanism_and_ic_utilization_technique_part_one/}
\bibitem{ref18} DPCM(Differential Pulse Code Modulation)とは?意味をわかりやすく簡単に解説 - xexeq.jp, 9月 26, 2025にアクセス、 \url{https://xexeq.jp/blogs/media/it-glossary278}
\bibitem{ref19} Implementation of Speech Companding Technique in Blackfin Digital Signal Processor for Digital Telephone Systems - ResearchGate, 9月 26, 2025にアクセス、 \url{https://www.researchgate.net/publication/264418038_Implementation_of_Speech_Companding_Technique_in_Blackfin_Digital_Signal_Processor_for_Digital_Telephone_Systems}
\bibitem{ref20} Multimedia Full Notes | PDF | Compact Disc | Cd Rom - Scribd, 9月 26, 2025にアクセス、 \url{https://www.scribd.com/document/725178791/Multimedia-Full-Notes}
\bibitem{ref21} Compact Disc Digital Audio - Wikiwand, 9月 26, 2025にアクセス、 \url{https://www.wikiwand.com/en/articles/Compact_Disc_Digital_Audio}
\bibitem{ref22} Fractal Speech Processing - National Academic Digital Library of Ethiopia, 9月 26, 2025にアクセス、 \url{https://ndl.ethernet.edu.et/bitstream/123456789/45147/1/244.pdf}
\bibitem{ref23} Mobile Wireless Communications - Tablero al Parque, 9月 26, 2025にアクセス、 \url{https://tableroalparque.weebly.com/uploads/5/1/6/9/51696511/schwartz.pdf}
\bibitem{ref24} DESIGN IMPLEMETATION AND STUDY OF A CODE (DPC) MODULATOR AND DE-MODULATOR USING MATLAB AND SIMULINK NITIABALAN KARUPPAYAH Univer - EPrints USM, 9月 26, 2025にアクセス、 \url{http://eprints.usm.my/57608/1/Design%20Implemetation%20And%20Study%20Of%20A%20Code%20%28DPC%29%20Modulator%20And%20De-Modulator%20Using%20Matlab%20And%20Simulink_Nitiabalan%20Karuppayah.pdf}
\bibitem{ref25} Digital Television Systems - CONSULTEC - Beto Samaniego, 9月 26, 2025にアクセス、 \url{https://betosamaniego.files.wordpress.com/2012/03/digitaltelevisionsystemsalencar.pdf}
\bibitem{ref26} Physics Delta Modulation - SATHEE, 9月 26, 2025にアクセス、 \url{https://sathee.iitk.ac.in/article/physics/physics-delta-modulation/}
\bibitem{ref27} Delta Modulation: Working, Diagram, Benefits & Exam Tips - Vedantu, 9月 26, 2025にアクセス、 \url{https://www.vedantu.com/physics/delta-modulation}
\bibitem{ref28} Slide 1 - moodle, 9月 26, 2025にアクセス、 \url{https://moodle.najah.edu/mod/resource/view.php?id=256708}
\bibitem{ref29} Delta modulation - Wikipedia, 9月 26, 2025にアクセス、 \url{https://en.wikipedia.org/wiki/Delta_modulation}
\bibitem{ref30} DOCTOR OF PHILOSOPHY High fidelity music coding Smyth, Stephen - Queen's University Belfast, 9月 26, 2025にアクセス、 \url{https://pure.qub.ac.uk/files/463369586/High_fidelity_music_coding.pdf}
\bibitem{ref31} Digital communication unit II | PPTX - Slideshare, 9月 26, 2025にアクセス、 \url{https://www.slideshare.net/slideshow/digital-communication-unit-ii/111306118}
\bibitem{ref32} Comparison of Adaptive Linear Prediction Algorithms in ADPCM - ResearchGate, 9月 26, 2025にアクセス、 \url{https://www.researchgate.net/publication/224733079_Comparison_of_Adaptive_Linear_Prediction_Algorithms_in_ADPCM}
\end{thebibliography}

\end{document}