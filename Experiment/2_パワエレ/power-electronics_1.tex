\documentclass[a4paper,11pt]{ltjsarticle}

% =============================================
% 1. パッケージ設定 (SARP v3.0 NNCT-EE準拠)
% =============================================
\usepackage[T1]{fontenc}
\usepackage{newtxtext}
\usepackage[varbb]{newtxmath} % 数式フォント Times系
\usepackage{bm}      % ベクトル太字
\usepackage{mathtools}

% レイアウト・図表関連
\usepackage[margin=25mm]{geometry}
\usepackage{array}      
\usepackage{multirow}   
\usepackage{fancyhdr}   
\usepackage{graphicx}
% 画像検索パス(カレントディレクトリとサブフォルダ)
\graphicspath{{./}{1205/}{image/}}
\usepackage{float}
\usepackage{booktabs}
\usepackage{subcaption}
\usepackage[export]{adjustbox}

% 回路図・グラフ描画
\usepackage{circuitikz}
\usepackage{tikz}
\usepackage{pgfplots}
\pgfplotsset{compat=newest}
\usepackage{pgfplotstable}
\usetikzlibrary{arrows.meta, positioning, calc}

% SI単位・数式処理
\usepackage{siunitx}
\sisetup{
  detect-all,
  inter-unit-product=\ensuremath{{}\cdot{}},
  separate-uncertainty=true,
  number-unit-product = \hspace{0.5em} % 単位前の半角スペース強制
}

% リンク・参照
\usepackage{cite}
\usepackage{xurl}
\usepackage[hidelinks]{hyperref}
\usepackage[nameinlink,noabbrev]{cleveref}
\usepackage{needspace}
\Urlmuskip=0mu plus 1mu
\usepackage{titlesec}
\titlespacing*{\section}{0pt}{3.5ex plus 1ex minus .2ex}{0pt}
\titlespacing*{\subsection}{0pt}{2.5ex plus .5ex minus .2ex}{0pt}
\titlespacing*{\subsubsection}{0pt}{1.5ex plus .3ex minus .2ex}{0pt}
\usepackage{indentfirst}

% 参考文献の上付き表示設定 [1]形式
\makeatletter
\def\@cite#1#2{$^{\mbox{\scriptsize[#1\if@tempswa , #2\fi]}}$}
\def\@biblabel#1{[#1]}
\makeatother

\crefname{figure}{図}{図}
\crefname{table}{表}{表}
\crefname{equation}{式}{式}

% キャプション設定
\usepackage{caption}
\captionsetup{
  format=hang,
  labelsep=quad,
  font={small},
  labelfont={bf},
  justification=centering
}
\captionsetup[figure]{justification=centerlast}

% フロート・キャプション間隔の調整(全体の縦余白を詰める)
\setlength{\textfloatsep}{8pt plus 2pt minus 2pt}   % テキスト上・下のフロート間隔
\setlength{\floatsep}{6pt plus 2pt minus 2pt}       % フロート同士の間隔
\setlength{\intextsep}{6pt plus 2pt minus 2pt}      % インラインフロートの上下余白
\setlength{\abovecaptionskip}{6pt}   % キャプション上の余白
\setlength{\belowcaptionskip}{6pt}   % キャプション下の余白
\captionsetup{skip=4pt}

% =============================================
% 2. カスタムコマンド定義
% =============================================
\newcommand{\UnderlineBox}[2][3cm]{\underline{\makebox[#1][c]{\vphantom{lp}\large #2}}}
\newcommand{\JustifiedLabel}[2]{\makebox[#1][s]{\large\bfseries #2}}
\newcommand{\BoldLabel}[1]{{\large\bfseries #1}}

% 微分記号(ローマン体 d)
\newcommand{\diff}[2]{\frac{\mathrm{d}#1}{\mathrm{d}#2}}
\newcommand{\pdiff}[2]{\frac{\partial #1}{\partial #2}}

% 単位記号・ローマン体コマンド
\providecommand{\unit}[1]{\,\mathrm{#1}}
\newcommand{\rom}[1]{\mathrm{#1}}

% =============================================
% 3. 表紙専用のページスタイル定義
% =============================================
\fancypagestyle{coverpage}{
  \fancyhf{} 
  \renewcommand{\headrulewidth}{0pt} 
  \renewcommand{\footrulewidth}{0pt} 
  \cfoot{\vspace{2mm}\footnotesize \bfseries 国立長野高専 電気電子工学科}
}

% =============================================
% ドキュメント開始
% =============================================
\begin{document}

% /////////////////////////////////////////////
% 表紙 (Cover Page)
% /////////////////////////////////////////////

\newgeometry{top=25mm, bottom=15mm, left=15mm, right=15mm}
\thispagestyle{coverpage}

\begin{center}
    \vspace*{0mm} 
    {\Huge \bfseries 電気電子工学実験報告書}
    \vspace{10mm} 
\end{center}

\noindent
\begin{tabular}{@{}ll}
  \BoldLabel{テーマ名} & \UnderlineBox[13.5cm]{パワーエレクトロニクス実験} \\[2.0em] 
\end{tabular}

\noindent
\BoldLabel{報告者} \hspace{0.5em}
\UnderlineBox[1.5cm]{5} {\large \textbf{年}} \hspace{0.2em}      
(\UnderlineBox[1.5cm]{E} {\large \textbf{組}}) \hspace{0.2em} 
{\large \textbf{番号}} \UnderlineBox[2.0cm]{234} \hspace{0.5em}    
\UnderlineBox[1.5cm]{B} {\large \textbf{班}} \hspace{1em}        
\UnderlineBox[4.5cm]{栁原魁人}                                   
\vspace{2.0em} 

\noindent
\begin{tabular}{@{}p{0.48\textwidth} p{0.48\textwidth}}
  \BoldLabel{実験場所} \hspace{1em} \UnderlineBox[5.5cm]{エレクトロニクス工房} & 
  \BoldLabel{指導担当} \hspace{1em} \UnderlineBox[5.5cm]{鈴木 宏}    
\end{tabular}
\vspace{2.0em} 

\noindent
\BoldLabel{共同実験者} \hspace{1em} \UnderlineBox[12.5cm]{石坂知尋,倉科純太郎,中井智大,中澤耕平} 
\vspace{2.5em} 

\noindent
\renewcommand{\arraystretch}{2.0}
\setlength{\tabcolsep}{0pt}
\begin{tabular}{l l l l}
    \JustifiedLabel{5em}{実験日} & 
    \hspace{0.3em} 令和 \UnderlineBox[0.65cm]{7} 年 \UnderlineBox[0.65cm]{11} 月 \UnderlineBox[0.65cm]{28} 日 & & \\
    \JustifiedLabel{5em}{提出期限} & 
    \hspace{0.3em} 令和 \UnderlineBox[0.65cm]{7} 年 \UnderlineBox[0.65cm]{12} 月 \UnderlineBox[0.65cm]{31} 日 & 
    \hspace{0.3em}$\Rightarrow$\hspace{0.3em} \JustifiedLabel{4em}{提出日} & 
    \hspace{0.3em} 令和 \UnderlineBox[0.65cm]{7} 年 \UnderlineBox[0.65cm]{12} 月 \UnderlineBox[0.65cm]{16} 日 \\
    ( \JustifiedLabel{6em}{再提出期限} & 
    \hspace{0.3em} 令和 \UnderlineBox[0.65cm]{} 年 \UnderlineBox[0.65cm]{} 月 \UnderlineBox[0.65cm]{} 日 & 
    \hspace{0.3em}$\Rightarrow$\hspace{0.3em} \JustifiedLabel{5em}{再提出日} & 
    \hspace{0.3em} 令和 \UnderlineBox[0.65cm]{} 年 \UnderlineBox[0.65cm]{} 月 \UnderlineBox[0.65cm]{} 日 ) \\
\end{tabular}
\vfill 

\renewcommand{\arraystretch}{1.5}
\begin{center}
\begin{tabular}{|>{\centering\arraybackslash}m{2.4cm}|>{\raggedright\arraybackslash}m{12.1cm}|>{\centering\arraybackslash}m{2.4cm}|}
\hline
\multicolumn{2}{|c|}{\JustifiedLabel{11em}{評 価 項 目}} & \JustifiedLabel{4em}{評 価} \\
\hline
\multirow{3}{*}{\parbox[c][4.5em][c]{2.4cm}{\centering\shortstack{\large\bfseries 実 習\\[0.3em]\large\bfseries 評 価}}} 
 & (1) 自ら積極的に実験に取り組めた &  \\ \cline{2-3}
 & (2) 実験装置を適切に使用でき,正確に実験を行なえた &  \\ \cline{2-3}
 & (3) グループ内で協力的に実験が行なえた &  \\
\hline
\multirow{4}{*}{\parbox[c][6.0em][c]{2.4cm}{\centering\shortstack{\large\bfseries 報告書\\[0.3em]\large\bfseries 評 価}}} 
 & (1) 結果のまとめかた(図表を含む) &  \\ \cline{2-3}
 & (2) 結果に対する考察 &  \\ \cline{2-3}
 & (3) 報告事項/課題(正しい解答や適切な引用など) &  \\ \cline{2-3}
 & (4) 報告書としての体裁が整っているか &  \\
\hline
\end{tabular}
\end{center}
\clearpage

% /////////////////////////////////////////////
% 本文 (Main Body)
% /////////////////////////////////////////////

\restoregeometry 
\setcounter{page}{1}
\pagestyle{plain} 

\section{目的}
インバータを利用した電動機制御技術と,インバータの応用例として電気自動車の動作原理について理解する。さらには,電力の有効活用として回生について学ぶ。

\section{原理}

\subsection{三相インバータ}
\cref{fig:inverter}に三相インバータの動作原理図を示した。インテリジェントパワーモジュール(IPM,三相\unit{600V},\unit{30A},6素子内蔵)とマイクロコンピュータ(ルネサスエレクトロニクス,SH2 7085)とを組み合わせて,三相交流を発生する。
三相インバータの出力はACサーボモータ(ブラシレスDCモータ,同期電動機の一種)に接続され,速度制御などが行われる。

\begin{figure}[H]
  \centering
  % 図1はinverter.pngを使用
  \includegraphics[width=0.7\linewidth]{inverter.png}
  \caption{三相インバータの動作原理図}
  \label{fig:inverter}
\end{figure}

\section{実験方法}
詳細は取扱説明書(昭和電業社作成)にしたがって実験を行います。

\subsection{ACブラシレスモータのトルク制御}
\begin{enumerate}
    \item 電源装置の電圧を\unit{10V}に設定した後に,計測ソフトウェアの設定数値画面で「制御方法:電流制御(トルク制御)」,「出力変動方法:自動」に設定して,サンプルパターンをロードしてから「開始」ボタンを押してモータのトルク-回転数特性を測定する。
    \item 「トルク制御した場合のトルク-回転数特性」を作成する。
\end{enumerate}

\subsection{ACブラシレスモータの速度制御}
\begin{enumerate}
    \item 電源装置の電圧を\unit{15V}に設定した後に,計測ソフトウェアの設定数値画面で「制御方法:速度制御」,「出力変動方法:自動」に設定して,サンプルパターンをロードしてから「開始」ボタンを押してモータのトルク-回転数特性を測定する。
    \item 「速度制御した場合のトルク-回転数特性」を作成する。「トルク制御した場合のトルク-回転数特性」と「速度制御した場合のトルク-回転数特性」とを重ねたグラフを作成して,比較検討を行う。
\end{enumerate}

\subsection{ACブラシレスモータの回生実験}
\begin{enumerate}
    \item 電源装置の電圧を\unit{24V}に設定して,計測ソフトウェアの設定数値画面で「制御方法:速度制御」,「出力変動方法:加減速」,「加速時間:20秒」,「一定制御時間:20秒」,「減速時間:20秒」,「停止後計測時間:10秒」,「速度の設定:\unit{2000rpm}」に設定して(出力パターンが台形になっていることを確認),「開始」ボタンを押す。測定開始後20秒間で\unit{2000rpm}に達していることを確認する。
    \item 測定開始してから30秒(\unit{2000rpm}一定速度で回転している状態)経過したら電源装置の出力スイッチをOFFにする。
    \item 測定開始してから40秒後から減速を開始して,60秒後にモータが停止する。このとき,モータはフライホイールの慣性力により発電動作を行い,インバータを経由して電気二重層コンデンサに回生エネルギーとして戻される。
    \item 「回転数-時間特性」,「電源電力-時間特性」,「インバータ出力電力-時間特性」,「電源電圧-時間特性」を作成する。「インバータ出力電力-時間特性」,「電源電圧-時間特性」から回生動作時のエネルギーの流れについて検討する。
\end{enumerate}

\clearpage

\section{実験結果}
\subsection{電流制御実験の結果}
\begin{figure}[htbp]
    \centering
    \begin{subfigure}[b]{0.95\textwidth}
        \centering
        \includegraphics[width=0.86\linewidth]{data/torque_control_torque_vs_rpm.png}
        \caption{トルク-回転数特性(電流制御)}
    \end{subfigure}
    \caption{電流制御の測定結果}
    \label{fig:current-control-results}
\end{figure}

\clearpage

\subsection{速度制御実験の結果}
\begin{figure}[htbp]
    \centering
    \begin{subfigure}[b]{0.95\textwidth}
        \centering
        \includegraphics[width=0.84\linewidth]{data/speed_control_torque_vs_rpm.png}
        \caption{トルク-回転数特性(速度制御)}
    \end{subfigure}
    \begin{subfigure}[b]{0.95\textwidth}
        \centering
        \includegraphics[width=0.84\linewidth]{data/torque_speed_control_comparison.png}
        \caption{トルク制御と速度制御の比較}
    \end{subfigure}
    \caption{速度制御の測定結果}
    \label{fig:speed-control-results}
\end{figure}

\clearpage

\subsection{回生実験の結果}
\begin{figure}[htbp]
    \centering
    \begin{subfigure}[b]{0.95\textwidth}
        \centering
        \includegraphics[width=0.86\linewidth]{data/regen_rotational_speed_time.png}
        \caption{回転数-時間特性}
    \end{subfigure}
    \begin{subfigure}[b]{0.95\textwidth}
        \centering
        \includegraphics[width=0.86\linewidth]{data/regen_power_supply_voltage_time.png}
        \caption{電源電圧-時間特性}
    \end{subfigure}
    \caption{回生実験の測定結果(1/2)}
    \label{fig:regeneration-results-part1}
\end{figure}

\clearpage

\begin{figure}[htbp]
    \centering
    \begin{subfigure}[b]{0.95\textwidth}
        \centering
        \includegraphics[width=0.86\linewidth]{data/regen_inverter_output_power_time.png}
        \caption{インバータ出力電力-時間特性}
    \end{subfigure}
    \begin{subfigure}[b]{0.95\textwidth}
        \centering
        \includegraphics[width=0.86\linewidth]{data/regen_power_supply_power_time.png}
        \caption{電源電力-時間特性}
    \end{subfigure}
    \caption{回生実験の測定結果(2/2)}
    \label{fig:regeneration-results-part2}
\end{figure}
\section{報告事項}
(1) 回生実験において得られた「電源電圧-時間特性」から,減速開始時の電圧の変化量を求め,理論値と比較して,なぜ値が異なるのか検討せよ。コンデンサの電圧の理論値は以下の式より $V_1 = \unit{25.61}{V}$ となり,コンデンサの初期電圧(= 電源電圧)$V_0$ と比較して \unit{1.61}{V} 増加することになる。

\vspace{1em}
\noindent
\textbf{【理論値計算の参考】}

フライホイールの慣性モーメント:
\begin{equation}
    J = \frac{1}{8} m D^2 = \frac{1}{8} \times 17.65 \times 0.179^2 = \unit{0.07069}{kg \cdot m^2}
\end{equation}

フライホイールの運動エネルギー:
\begin{equation}
    E = \frac{1}{2} J \omega^2 = \frac{1}{2} \times 0.07069 \times \left( \frac{2000}{60} \times 2\pi \right)^2 = \unit{1549.8}{J}
\end{equation}

コンデンサの蓄積エネルギー(100\% 変換と仮定):
\begin{align}
    E &= \frac{1}{2} C (V_1^2 - V_0^2) \notag \\
    &= \frac{1}{2} \times 38.6 \times (V_1^2 - 24^2) = \unit{1549.8}{J}
\end{align}

ここで,
\begin{itemize}
    \item $m$: 質量 [\unit{kg}]
    \item $D$: 直径 [\unit{m}]
    \item $J$: 慣性モーメント [\unit{kg \cdot m^2}]
    \item $\omega$: 回転速度 [\unit{rad/s}]
    \item $C$: 静電容量 [\unit{F}]
    \item $V_0$: コンデンサの初期電圧 [\unit{V}]
\end{itemize}

\vspace{2em}
\noindent

\subsection*{【解答】}

\subsubsection*{1. 実測値の導出}
回生実験の結果(\cref{fig:regeneration-results-part1})より,減速開始直前($t=40\unit{s}$付近)の電源電圧 $V_0$ および,減速動作中に観測された最大電圧 $V_1$ は以下の通りである。

\begin{itemize}
    \item 減速開始直前電圧: $V_0 = 23.40 \unit{V}$
    \item 最大到達電圧: $V_1 = 24.60 \unit{V}$
\end{itemize}

したがって,実測による電圧の変化量 $\Delta V_{\mathrm{exp}}$ は次のように求められる。
\begin{equation}
    \Delta V_{\mathrm{exp}} = V_1 - V_0 = 24.60 - 23.40 = 1.20 \unit{V}
\end{equation}

\subsubsection*{2. 理論値との比較}
課題で与えられた理論値 $\Delta V_{\mathrm{th}} = 1.61 \unit{V}$ と実測値 $\Delta V_{\mathrm{exp}} = 1.20 \unit{V}$ を比較すると,実測値のほうが $0.41 \unit{V}$ 小さい値となった。これは理論値に対して約 25\% の減少に相当する。

\subsubsection*{3. 誤差要因の検討}
理論計算式では,フライホイールの運動エネルギー $E$ が損失なく $100\%$ コンデンサの静電エネルギーに変換されると仮定している(エネルギー保存則)。しかし,実際の実験系ではエネルギー変換および伝送の過程で複数の損失が発生する。
これら損失により,コンデンサに回収されるエネルギーが減少し,電圧上昇量が理論値よりも小さくなったと考えられる。具体的な損失要因を以下に挙げる。

\noindent\textbf{(1) モータ内部における損失}\par
\begin{itemize}
    \item \textbf{機械損}: 減速時においてもロータは回転しているため,軸受の摩擦やフライホイールの空気抵抗(風損)によって運動エネルギーの一部が熱として消費される。これは発電機としての入力エネルギーそのものを減少させる要因となる。
    \item \textbf{鉄損}: モータコア内の磁束変化により,ヒステリシス損および渦電流損が発生する。これらは回転数(周波数)に依存してエネルギーを消費する。
    \item \textbf{銅損}: 発電された回生電流がモータの巻線を流れる際,巻線抵抗 $R_{\mathrm{motor}}$ によりジュール熱($P_{\mathrm{cu}} = 3 R_{\mathrm{motor}} I^2$)が発生し,電力が消費される。
\end{itemize}

\noindent\textbf{(2) インバータおよび回路における損失}\par
\begin{itemize}
    \item \textbf{パワーデバイスの損失}: インバータを構成するスイッチング素子(IGBT等)や還流ダイオードにおいて以下の損失が発生する。
    \begin{itemize}
        \item \textbf{オン損失(導通損失)}: 素子に電流が流れる際のオン電圧($V_{\mathrm{CE(sat)}}$ や $V_{\mathrm{F}}$)による電力損失。
        \item \textbf{スイッチング損失}: 素子がオン・オフする遷移期間における電圧・電流の重なりによる電力損失。
    \end{itemize}
    \item \textbf{配線の損失}: モータからインバータ,およびインバータからコンデンサまでの配線抵抗によるジュール熱損失。
    \item \textbf{コンデンサの内部損失}: 電気二重層コンデンサの等価直列抵抗(ESR)に回生電流が流れることで電力損失が発生する。
\end{itemize}

以上の要因により,フライホイールが持っていた運動エネルギーの一部は熱エネルギーとして散逸し,最終的にコンデンサに蓄積されたエネルギーは理論値よりも小さくなったと結論付けられる。

\end{document}