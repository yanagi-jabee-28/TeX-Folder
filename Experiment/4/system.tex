\documentclass[a4paper,11pt]{ltjsarticle}

% =============================================
% 1. パッケージ設定 (SARP v2.0準拠)
% =============================================
\usepackage[T1]{fontenc}
\usepackage{newtxtext}
\usepackage[varbb]{newtxmath} % 数式フォント
\usepackage{bm}      % ベクトル太字
\usepackage{mathtools}

% レイアウト・図表関連
\usepackage[margin=25mm]{geometry}
\usepackage{array}      
\usepackage{multirow}   
\usepackage{fancyhdr}   
\usepackage{graphicx}
\usepackage{float}
\usepackage{booktabs}
\usepackage{subcaption}

% 回路図・グラフ描画
\usepackage{circuitikz}
\usepackage{tikz}
\usepackage{pgfplots}
\pgfplotsset{compat=newest}
\usetikzlibrary{arrows.meta, positioning, calc}

% SI単位・数式処理
\usepackage{siunitx}
\sisetup{
  detect-all,
  inter-unit-product=\ensuremath{{}\cdot{}},
  separate-uncertainty=true
}

% リンク・参照
\usepackage{cite}
\usepackage[hidelinks]{hyperref}
\usepackage[nameinlink,noabbrev]{cleveref}

% 参考文献の上付き表示設定
\makeatletter
\def\@cite#1#2{$^{\mbox{\scriptsize[#1\if@tempswa , #2\fi]}}$}
\def\@biblabel#1{[#1]}
\makeatother

\crefname{figure}{図}{図}
\crefname{table}{表}{表}
\crefname{equation}{式}{式}

% キャプション設定
\usepackage{caption}
\captionsetup{
  format=hang,
  labelsep=quad,
  font={small},
  labelfont={bf},
  justification=centering
}
\captionsetup[figure]{justification=centerlast}

% =============================================
% 2. カスタムコマンド定義
% =============================================
\newcommand{\UnderlineBox}[2][3cm]{\underline{\makebox[#1][c]{\vphantom{lp}\large #2}}}
\newcommand{\JustifiedLabel}[2]{\makebox[#1][s]{\large\bfseries #2}}
\newcommand{\BoldLabel}[1]{{\large\bfseries #1}}

% 数式用コマンド
\newcommand{\diff}[2]{\frac{\mathrm{d}#1}{\mathrm{d}#2}}
\newcommand{\ze}{\zeta}      % 減衰係数
\newcommand{\wn}{\omega_n}   % 固有角周波数

% =============================================
% 3. 表紙専用のページスタイル定義
% =============================================
\fancypagestyle{coverpage}{
  \fancyhf{} 
  \renewcommand{\headrulewidth}{0pt} 
  \renewcommand{\footrulewidth}{0pt} 
  \cfoot{\vspace{5mm}\Large \bfseries 国立長野高専 電気電子工学科}
}

% =============================================
% ドキュメント開始
% =============================================
\begin{document}

% /////////////////////////////////////////////
% 表紙 (Cover Page)
% /////////////////////////////////////////////

\newgeometry{top=25mm, bottom=20mm, left=18mm, right=18mm}
\thispagestyle{coverpage}

\begin{center}
    \vspace*{0mm} 
    {\Huge \bfseries 電気電子工学実験報告書}
    \vspace{10mm} 
\end{center}

\noindent
\begin{tabular}{@{}ll}
  \BoldLabel{テーマ名} & \UnderlineBox[13.5cm]{自動制御実験(2次遅れ系の周波数応答)} \\[2.0em] 
\end{tabular}

\noindent
\BoldLabel{報告者} \hspace{0.5em}
\UnderlineBox[1.5cm]{5} {\large \textbf{年}} \hspace{0.2em}      
(\UnderlineBox[1.5cm]{E} {\large \textbf{組}}) \hspace{0.2em} 
{\large \textbf{番号}} \UnderlineBox[2.0cm]{234} \hspace{0.5em}    
\UnderlineBox[1.5cm]{B} {\large \textbf{班}} \hspace{1em}        
\UnderlineBox[4.5cm]{栁原魁人}                                   
\vspace{2.0em} 

\noindent
\begin{tabular}{@{}p{0.48\textwidth} p{0.48\textwidth}}
  \BoldLabel{実験場所} \hspace{1em} \UnderlineBox[5.5cm]{情報工学実験室} & 
  \BoldLabel{指導担当} \hspace{1em} \UnderlineBox[5.5cm]{鈴木 宏}    
\end{tabular}
\vspace{2.0em} 

\noindent
\BoldLabel{共同実験者} \hspace{1em} \UnderlineBox[12.5cm]{石坂知尋,倉科純太郎,中井智大,中澤耕平} 
\vspace{2.5em} 

\noindent
\renewcommand{\arraystretch}{2.0}
\setlength{\tabcolsep}{0pt}
\begin{tabular}{l l l l}
    \JustifiedLabel{5em}{実験日} & 
    \hspace{0.3em} 令和 \UnderlineBox[0.65cm]{7} 年 \UnderlineBox[0.65cm]{11} 月 \UnderlineBox[0.65cm]{28} 日 & & \\
    \JustifiedLabel{5em}{提出期限} & 
    \hspace{0.3em} 令和 \UnderlineBox[0.65cm]{7} 年 \UnderlineBox[0.65cm]{12} 月 \UnderlineBox[0.65cm]{12} 日 & 
    \hspace{0.3em}$\Rightarrow$\hspace{0.3em} \JustifiedLabel{4em}{提出日} & 
    \hspace{0.3em} 令和 \UnderlineBox[0.65cm]{7} 年 \UnderlineBox[0.65cm]{} 月 \UnderlineBox[0.65cm]{} 日 \\
    ( \JustifiedLabel{6em}{再提出期限} & 
    \hspace{0.3em} 令和 \UnderlineBox[0.65cm]{} 年 \UnderlineBox[0.65cm]{} 月 \UnderlineBox[0.65cm]{} 日 & 
    \hspace{0.3em}$\Rightarrow$\hspace{0.3em} \JustifiedLabel{5em}{再提出日} & 
    \hspace{0.3em} 令和 \UnderlineBox[0.65cm]{} 年 \UnderlineBox[0.65cm]{} 月 \UnderlineBox[0.65cm]{} 日 ) \\
\end{tabular}
\vfill 

\renewcommand{\arraystretch}{1.5}
\begin{center}
\begin{tabular}{|>{\centering\arraybackslash}m{2.4cm}|>{\raggedright\arraybackslash}m{12.1cm}|>{\centering\arraybackslash}m{2.4cm}|}
\hline
\multicolumn{2}{|c|}{\JustifiedLabel{11em}{評 価 項 目}} & \JustifiedLabel{4em}{評 価} \\
\hline
\multirow{3}{*}{\parbox[c][4.5em][c]{2.4cm}{\centering\shortstack{\large\bfseries 実 習\\[0.3em]\large\bfseries 評 価}}} 
 & (1) 自ら積極的に実験に取り組めた &  \\ \cline{2-3}
 & (2) 実験装置を適切に使用でき,正確に実験を行なえた &  \\ \cline{2-3}
 & (3) グループ内で協力的に実験が行なえた &  \\
\hline
\multirow{4}{*}{\parbox[c][6.0em][c]{2.4cm}{\centering\shortstack{\large\bfseries 報告書\\[0.3em]\large\bfseries 評 価}}} 
 & (1) 結果のまとめかた(図表を含む) &  \\ \cline{2-3}
 & (2) 結果に対する考察 &  \\ \cline{2-3}
 & (3) 報告事項/課題(正しい解答や適切な引用など) &  \\ \cline{2-3}
 & (4) 報告書としての体裁が整っているか &  \\
\hline
\end{tabular}
\end{center}
\clearpage

% /////////////////////////////////////////////
% 本文 (Main Body)
% /////////////////////////////////////////////

\restoregeometry 
\setcounter{page}{1}
\pagestyle{plain} 

\section{目的}
2次遅れ系となる回路(Sallen-Key型ローパスフィルタ)を作成し,その周波数応答を測定した。測定結果から伝達関数を算出し,理論値と比較することで,周波数応答と伝達関数の関係を理解することを目的とする。

\section{原理および理論計算}
本実験では,オペアンプを用いたアクティブフィルタを使用して2次遅れ系の回路を作成した。

\subsection{伝達関数の導出}
実験で使用した回路図を\cref{fig:circuit}に示す。
\begin{figure}[H]
    \centering
    \includegraphics[width=0.65\linewidth]{原理図.png}
    \caption{アクティブフィルタの回路構成 (Sallen-Key Topology)}
    \label{fig:circuit}
\end{figure}
この回路の伝達関数 $G(s)$ は,キルヒホッフの法則より次のように導出できる。
\begin{align}
    G(s) = \frac{V_o(s)}{V_i(s)} &= \frac{\left(\frac{1}{CR}\right)^2}{s^2 + \frac{2}{\alpha CR}s + \left(\frac{1}{CR}\right)^2}
\end{align}
標準的な2次遅れ系の伝達関数 $G(s) = \frac{\wn^2}{s^2 + 2\ze\wn s + \wn^2}$ と係数を比較すると,各パラメータは以下のようになる。
\begin{equation}
    \ze = \frac{1}{\alpha}, \quad \wn = \frac{1}{CR}
\end{equation}

\subsection{理論値の算出}
使用した素子の定数は $\alpha C = \SI{10}{\micro F}$, $C/\alpha = \SI{0.5}{\micro F}$,$R=\SI{100}{\ohm}$ である。これらよりキャパシタンス $C$ および係数 $\alpha$ を計算すると,
\begin{equation}
    C^2 = (\alpha C) \cdot (C/\alpha) = 10 \cdot 0.5 = 5.0 \implies C = \sqrt{5}\,\si{\micro F} \approx \SI{2.236}{\micro F}
\end{equation}
\begin{equation}
    \alpha^2 = \frac{\alpha C}{C/\alpha} = \frac{10}{0.5} = 20 \implies \alpha = \sqrt{20} \approx 4.472
\end{equation}
となった。したがって,理論的な減衰係数 $\ze$ および固有角周波数 $\wn$ は以下の通りである。
\begin{equation}
    \ze = \frac{1}{\alpha} = \frac{1}{\sqrt{20}} \approx \mathbf{0.2236}, \quad \wn = \frac{1}{CR} \approx \mathbf{\SI{4472}{rad/s}}
\end{equation}
このとき,共振ピーク値 $M_p$ および共振角周波数 $\omega_p$ は次のように予測される。
\begin{equation}
    M_p = \frac{1}{2\ze\sqrt{1-\ze^2}} \approx \mathbf{2.29} \quad (\approx \mathbf{7.21\,\si{dB}})
\end{equation}
\begin{equation}
    \omega_p = \wn \sqrt{1-2\ze^2} \approx \SI{4472}{} \cdot \sqrt{1 - 2(0.05)} \approx \mathbf{\SI{4243}{rad/s}} \quad (\approx \SI{675}{Hz})
\end{equation}

\section{実験方法および使用機器}

\subsection{使用機器}
本実験で使用した主要機器を\cref{tab:equip}に示す。

\begin{table}[H]
    \centering
    \caption{使用機器一覧}
    \label{tab:equip}
    % --- 修正: Overfull対策 ---
    % \small化、列幅の調整、余白の圧縮を実行
    \begingroup
    \small
    \setlength{\tabcolsep}{2pt}
    \renewcommand{\arraystretch}{1.2}
    \begin{tabular}{@{} >{\centering\arraybackslash}m{0.8cm} >{\raggedright\arraybackslash}m{2.8cm} >{\raggedright\arraybackslash}m{4.0cm} >{\raggedright\arraybackslash}m{3.6cm}@{\hspace{2pt}} >{\raggedright\arraybackslash}m{3.0cm} @{} }
        \toprule
        No. & 機器名 & メーカー / 型番 & 定格・仕様 & 管理番号 \\
        \midrule
        1 & 発振器 & KENWOOD / AG-203D & $10\,\si{Hz} \sim 1\,\si{MHz}$ & 60800025 \\
        2 & オシロスコープ & Tektronix / TBS1052B & $50\,\si{MHz}, 1\,\si{GS/s}$ & W5022S10030760 \\
        3 & 直流安定化電源 & KENWOOD / PD18-10AD & $0 \sim 18\,\si{V}, 10\,\si{A}$ & 005153, 005156 \\
        \bottomrule
    \end{tabular}
    \endgroup
\end{table}

\subsection{実験方法}
回路に入力電圧 $V_i \approx 1.5\,\mathrm{V_{p-p}}$ の正弦波を入力し,周波数を $\SI{100}{Hz}$ から $\SI{7}{kHz}$ まで変化させた。各周波数における入出力電圧の振幅および位相差をオシロスコープで測定した。共振点付近では周波数間隔を細かく設定し,ピークの値を正確に測定できるようにした。

\section{実験結果}
測定結果の一部を\cref{tab:result}に,周波数応答のグラフを\cref{fig:bode_images}に示す。共振周波数付近($\SI{707.3}{Hz}$)で,ゲインが最大になるピークが確認できた。

また,MATLABによるシミュレーション結果(ボード線図)と,実測値のゲイン・位相プロットを比較のために並べて示した。図\ref{fig:bode_images}の(a)はMATLABシミュレーション,(b)は実測のゲイン,(c)は実測の位相である。

\begin{table}[H]
    \centering
    \caption{周波数応答 測定結果(抜粋)}
    \label{tab:result}
    \begingroup
    \small 
    \setlength{\tabcolsep}{2pt}
    \renewcommand{\arraystretch}{1.1}
    \begin{tabular}{
        S[table-format=4.1] 
        S[table-format=5.1] 
        >{\centering\arraybackslash}m{0.9cm} 
        >{\centering\arraybackslash}m{0.9cm} 
        >{\centering\arraybackslash}m{1.1cm} 
        S[table-format=-2.2] 
        >{\centering\arraybackslash}m{1.0cm} 
        S[table-format=-3.2]
        }
        \toprule
        {周波数 $f$} & {角周波数 $\omega$} & {$V_i$} & {$V_o$} & {電圧比} & {ゲイン $G$} & {遅れ $t$} & {位相 $\theta$} \\
        {[\si{Hz}]} & {[\si{rad/s}]} & {[\si{V}]} & {[\si{V}]} & {$V_o/V_i$} & {[\si{dB}]} & {[\si{ms}]} & {[$^\circ$]} \\ 
        \midrule
        103.7  & 651.57   & 1.50 & 1.52 & 1.013 & 0.12  & 0.01 & -0.37  \\
        151.7  & 953.16   & 1.62 & 1.66 & 1.025 & 0.21  & 0.06 & -3.28  \\
        201.7  & 1267.32  & 1.50 & 1.60 & 1.067 & 0.56  & 0.10 & -7.26  \\
        303.4  & 1906.32  & 1.50 & 1.72 & 1.147 & 1.19  & 0.10 & -11.36 \\
        400.0  & 2513.27  & 1.58 & 2.00 & 1.266 & 2.05  &      &        \\
        499.7  & 3139.71  & 1.54 & 2.26 & 1.468 & 3.33  & 0.14 & -25.90 \\
        600.0  & 3769.91  & 1.54 & 2.50 & 1.623 & 4.21  &      &        \\
        650.0  & 4084.07  & 1.50 & 2.78 & 1.853 & 5.36  &      &        \\
        677.0  & 4253.72  & 1.50 & 2.80 & 1.867 & 5.42  &      &        \\
        707.3  & 4444.10  & 1.50 & 2.96 & 1.973 & \bfseries 5.90 & 0.27 & -68.24 \\
        720.0  & 4523.89  & 1.50 & 2.76 & 1.840 & 5.30  &      &        \\
        746.0  & 4687.26  & 1.50 & 2.72 & 1.813 & 5.17  &      &        \\
        800.0  & 5026.55  & 1.50 & 2.26 & 1.507 & 3.56  &      &        \\
        900.0  & 5654.87  & 1.58 & 1.90 & 1.203 & 1.60  &      &        \\
        1013.0 & 6364.87  & 1.50 & 1.52 & 1.013 & 0.12  & 0.36 & -132.74 \\
        1120.0 & 7037.17  & 1.50 & 1.00 & 0.667 & -3.52 &      &        \\
        1200.0 & 7539.82  & 1.56 & 0.90 & 0.577 & -4.78 &      &        \\
        1559.0 & 9795.49  & 1.60 & 0.512 & 0.320 & -9.90 & 0.28 & -154.90 \\
        2040.0 & 12817.70 & 1.50 & 0.26 & 0.173 & -15.22 & 0.21 & -155.69 \\
        3050.0 & 19163.72 & 1.50 & 0.122 & 0.081 & -21.79 & 0.14 & -151.52 \\
        5001.0 & 31422.21 & 1.50 & 0.078 & 0.052 & -25.68 & 0.07 & -126.03 \\
        6936.0 & 43580.17 & 1.50 & 0.042 & 0.028 & -31.06 & 0.05 & -114.86 \\
        \bottomrule
    \end{tabular}%
    \endgroup
\end{table}

% --------------- MATLAB and Measurement Figures -----------------
\begin{figure}[H]
    \centering
    \begin{subfigure}[b]{0.95\linewidth}
        \centering
        \includegraphics[width=0.95\linewidth]{B1.jpg}
        \caption{MATLAB による理論・シミュレーション(ボード線図)}
        \label{fig:bode_sim}
    \end{subfigure}
    \\[1em]
    \begin{subfigure}[b]{0.48\linewidth}
        \centering
        \includegraphics[width=0.95\linewidth]{bode_gain_plot.png}
        \caption{実測:ゲイン}
        \label{fig:bode_gain_meas}
    \end{subfigure}
    \hfill
    \begin{subfigure}[b]{0.48\linewidth}
        \centering
        \includegraphics[width=0.95\linewidth]{bode_phase_plot.png}
        \caption{実測:位相}
        \label{fig:bode_phase_meas}
    \end{subfigure}
    \caption{比較図:(a) MATLABシミュレーション, (b) 実測ゲイン, (c) 実測位相}
    \label{fig:bode_images}
\end{figure}

\section{考察}

\subsection{実験結果からの伝達関数の導出と検証}
実験で得られたゲインのピーク値から,実際の伝達関数のパラメータ($\ze, \wn$)を逆算する。

\subsubsection{$\zeta$ と $\omega_n$ の導出}
実験データのピーク値 $M_p$ とその周波数 $f_p$ は以下の値が得られた。
\begin{equation}
    M_p (\text{実験値}) = 1.973 \quad (5.90\,\si{dB}), \qquad f_p = \SI{707.3}{Hz}
\end{equation}
理論的に共振ピーク値 $M_p$ は減衰係数 $\ze$ を用いて次式で表される。
\begin{equation}
    M_p = \frac{1}{2\ze\sqrt{1-\ze^2}}
\end{equation}
この式を $\ze$ について解くために両辺を2乗して整理する。
\begin{equation}
    M_p^2 = \frac{1}{4\ze^2(1-\ze^2)} \iff 4M_p^2 \ze^4 - 4M_p^2 \ze^2 + 1 = 0
\end{equation}
$\ze^2$ についての2次方程式として解の公式を用いると,
\begin{align}
    \ze^2 &= \frac{4M_p^2 \pm \sqrt{(4M_p^2)^2 - 16M_p^2}}{8M_p^2} \notag \\
          &= \frac{4M_p^2 \pm 4M_p^2 \sqrt{1 - \frac{1}{M_p^2}}}{8M_p^2} \notag \\
          &= \frac{1 \pm \sqrt{1 - \frac{1}{M_p^2}}}{2}
\end{align}
ここで,共振が起こる条件は $\ze < 1/\sqrt{2}$ (つまり $\ze^2 < 1/2$)である。$\sqrt{1 - 1/M_p^2}$ は正の値なので,$+$ を選ぶと $\ze^2 > 1/2$ となり条件を満たさない。よって $-$ を選択する。
したがって,$\ze$ は次の式で計算できる。
\begin{equation}
    \ze = \sqrt{\frac{1 - \sqrt{1 - \frac{1}{M_p^2}}}{2}}
\end{equation}
ここに実験値 $M_p = 1.973$ を代入して計算すると,以下の値になった。
\begin{equation}
    \ze_{\text{exp}} = \sqrt{\frac{1 - \sqrt{1 - \frac{1}{1.973^2}}}{2}}
    \approx \mathbf{0.262679} \quad (\text{計算のため桁数を保持})
\end{equation}
次に,共振角周波数 $\omega_p = 2\pi f_p$ の関係式 $\omega_p = \wn\sqrt{1-2\ze^2}$ を変形して,固有角周波数 $\wn$ を求めた。
\begin{equation}
    \omega_{n,\text{exp}} = \frac{2\pi \cdot 707.3}{\sqrt{1 - 2\ze_{\text{exp}}^2}} 
    \approx \frac{2\pi \cdot 707.3}{\sqrt{1 - 2(0.262679)^2}} 
    \approx \mathbf{\SI{4786.92}{rad/s}}
\end{equation}
以上より,実験結果から求めた伝達関数 $G_{\text{exp}}(s)$ は以下のようになる。
\begin{align}
  G_{\text{exp}}(s) &= \frac{\omega_{n,\text{exp}}^2}{s^2 + 2\ze_{\text{exp}}\omega_{n,\text{exp}}\,s + \omega_{n,\text{exp}}^2} \notag \\
  &= \frac{(4786.92)^2}{s^2 + 2(0.262679)(4786.92)\,s + (4786.92)^2} \notag \\
  &= \frac{2.2914603\times 10^{7}}{s^2 + 2.514879\times 10^{3}\,s + 2.2914603\times 10^{7}}
\end{align}

\subsection{理論値と実験値の比較検討}
理論値と実験値を比較し,その誤差率を\cref{tab:error}にまとめた。

\begin{table}[H]
    \centering
    \caption{理論値と実験値の比較}
    \label{tab:error}
    \begin{tabular}{lccc}
        \toprule
        パラメータ & 理論値 & 実験値 & 誤差率 [\%] \\
        \midrule
        減衰係数 $\ze$ & 0.224 & 0.263 & $+17.4$ \\
        固有角周波数 $\wn$ [\si{rad/s}] & 4472 & 4787 & $+7.0$ \\
        \bottomrule
    \end{tabular}
\end{table}

\noindent
\textbf{誤差要因の考察:}
\begin{enumerate}
    \item \textbf{静電容量の公差}: セラミックコンデンサやフィルムコンデンサには通常 $\pm 5\% \sim 10\%$ 程度の公差がある。$\wn$ は $C$ に反比例するため,実際の容量が理論値より小さければ $\wn$ は大きくなる。本実験の $+7\%$ という誤差は,コンデンサの個体差による影響だと考えられる。
    \item \textbf{オペアンプの特性}: 理論式では理想的なオペアンプを仮定しているが,実際には周波数特性の限界がある。特に $\wn$ 付近の高周波域では位相遅れが生じやすく,これが減衰係数 $\ze$ の値に影響を与えた可能性がある。
\end{enumerate}

\subsubsection{高周波領域における位相特性の上昇について(オペアンプの影響)}
図\ref{fig:bode_phase_meas}では,約 $\SI{2}{kHz}$ を超える領域で位相が理論値の $-180^\circ$ へ収束せず,逆に戻るように見える。

主な原因として考えられるのは,使用したオペアンプの高周波特性の限界である。高周波ではオペアンプの開ループ利得が下がり,位相余裕が減少するため,閉ループでの理想的な応答を保てなくなる。結果として位相が理論値からずれ,元の予定より早く位相の戻りが生じることがある。

\subsection{2次遅れ系としての検証}
図\ref{fig:bode_images}のグラフを確認すると,共振周波数より高い周波数では,周波数が10倍になるごとにゲインが約$-40\,\si{dB}$低下している($-40\,\si{dB}/\text{dec}$ の傾きに近い)。また位相も,低周波の $0^\circ$ から高周波の $-180^\circ$ へ変化しており,共振点付近でちょうど $-90^\circ$ を通過している。これらの特徴は2次遅れ系の理論的な挙動と一致しているため,作成した回路は2次遅れ系として正しく動作していると言える。

\section{参考文献}
\begin{enumerate}
    \item 鈴木 宏: 自動制御実験テキスト, 国立長野高専 電気電子工学科.
    \item 制御工学 教科書 P127 $\sim$ P128.
\end{enumerate}

\end{document}