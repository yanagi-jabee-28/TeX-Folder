\documentclass[a4paper,11pt]{ltjsarticle}

% =============================================
% 1. パッケージ設定 (統合版)
% =============================================
\usepackage[T1]{fontenc}
\usepackage{newtxtext}
\usepackage[varbb]{newtxmath} % 数式フォント
\usepackage{bm}      % ベクトル太字
\usepackage{mathtools}

% レイアウト・図表関連
\usepackage[margin=25mm]{geometry}
\usepackage{array}      
\usepackage{multirow}   
\usepackage{fancyhdr}   
\usepackage{graphicx}
\usepackage{float}
\usepackage{booktabs}
\usepackage{subcaption}

% --- 追加・修正パッケージ ---
\usepackage{circuitikz} % 回路図描画用
\usepackage{listings}   % ソースコード掲載用

% Listingsの設定(ソースコードの見た目を整える)
\lstset{
  basicstyle=\ttfamily\small,
  frame=trBL, % 枠線
  numbers=left,
  stepnumber=1,
  numberstyle=\scriptsize,
  breaklines=true,
  captionpos=b,
  xleftmargin=2em,
  xrightmargin=1em
}

% SI単位・数式処理
\usepackage{siunitx}
\sisetup{
  detect-all,
  inter-unit-product=\ensuremath{{}\cdot{}},
  separate-uncertainty=true
}

% グラフ描画
\usepackage{tikz}
\usepackage{pgfplots}
\pgfplotsset{compat=newest}
\usetikzlibrary{arrows.meta, positioning, calc}

% リンク・参照
\usepackage{cite}
\usepackage[hidelinks]{hyperref}
\usepackage[nameinlink,noabbrev]{cleveref}

% 参考文献の上付き表示設定
\makeatletter
\def\@cite#1#2{$^{\mbox{\scriptsize[#1\if@tempswa , #2\fi]}}$}
\def\@biblabel#1{[#1]}
\makeatother

\crefname{figure}{図}{図}
\crefname{table}{表}{表}
\crefname{equation}{式}{式}

% キャプション設定
\usepackage{caption}
\captionsetup{
  format=hang,
  labelsep=quad,
  font={small},
  labelfont={bf},
  justification=centering
}
\captionsetup[figure]{justification=centerlast}

% =============================================
% 2. カスタムコマンド定義
% =============================================

% --- 表紙用コマンド ---
\newcommand{\UnderlineBox}[2][3cm]{%
  \underline{\makebox[#1][c]{\vphantom{lp}\large #2}}%
}
\newcommand{\JustifiedLabel}[2]{%
  \makebox[#1][s]{\large\bfseries #2}%
}
\newcommand{\BoldLabel}[1]{%
  {\large\bfseries #1}%
}

% --- 数式用コマンド (修正・追加) ---
\newcommand{\diff}[2]{\frac{\mathrm{d}#1}{\mathrm{d}#2}}
\newcommand{\pdiff}[2]{\frac{\partial #1}{\partial #2}}

% ★エラー原因となっていたマクロを定義★
\newcommand{\ze}{\zeta}      % 減衰係数
\newcommand{\wn}{\omega_n}   % 固有角周波数

% 画像パス設定
\graphicspath{{image/}}

% =============================================
% 3. 表紙専用のページスタイル定義
% =============================================
\fancypagestyle{coverpage}{
  \fancyhf{} 
  \renewcommand{\headrulewidth}{0pt} 
  \renewcommand{\footrulewidth}{0pt} 
  \cfoot{\vspace{5mm}\Large \bfseries 国立長野高専 電気電子工学科}
}

% =============================================
% ドキュメント開始
% =============================================
\begin{document}

% /////////////////////////////////////////////
% ここから表紙 (Cover Page)
% /////////////////////////////////////////////

\newgeometry{top=25mm, bottom=20mm, left=18mm, right=18mm}
\thispagestyle{coverpage}

% --- タイトル ---
\begin{center}
    \vspace*{0mm} 
    {\Huge \bfseries 電気電子工学実験報告書}
    \vspace{10mm} 
\end{center}

% --- テーマ名 ---
\noindent
\begin{tabular}{@{}ll}
  \BoldLabel{テーマ名} & \UnderlineBox[13.5cm]{4. PIDによる温度制御} \\[2.0em] 
\end{tabular}

% --- 報告者情報 ---
\noindent
\BoldLabel{報告者} \hspace{0.5em}
\UnderlineBox[1.5cm]{5} {\large \textbf{年}} \hspace{0.2em}     
(\UnderlineBox[1.5cm]{E} {\large \textbf{組}}) \hspace{0.2em} 
{\large \textbf{番号}} \UnderlineBox[2.0cm]{234} \hspace{0.5em}   
\UnderlineBox[1.5cm]{B} {\large \textbf{班}} \hspace{1em}       
\UnderlineBox[4.5cm]{栁原魁人}                                         
\vspace{2.0em} 

% --- 実験場所・指導担当 ---
\noindent
\begin{tabular}{@{}p{0.48\textwidth} p{0.48\textwidth}}
  \BoldLabel{実験場所} \hspace{1em} \UnderlineBox[5.5cm]{} & 
  \BoldLabel{指導担当} \hspace{1em} \UnderlineBox[5.5cm]{}   
\end{tabular}
\vspace{2.0em} 

% --- 共同実験者 ---
\noindent
\BoldLabel{共同実験者} \hspace{1em} \UnderlineBox[12.5cm]{石坂知尋,倉科純太郎,中井智大,中澤耕平} 
\vspace{2.5em} 

% --- 日付セクション ---
\noindent
\renewcommand{\arraystretch}{2.0}
\setlength{\tabcolsep}{0pt}
\begin{tabular}{l l l l}
  \JustifiedLabel{5em}{実験日} & 
  \hspace{0.3em} 令和 \UnderlineBox[0.65cm]{} 年 \UnderlineBox[0.65cm]{} 月 \UnderlineBox[0.65cm]{} 日 & & \\
  
  \JustifiedLabel{5em}{提出期限} & 
  \hspace{0.3em} 令和 \UnderlineBox[0.65cm]{} 年 \UnderlineBox[0.65cm]{} 月 \UnderlineBox[0.65cm]{} 日 & 
  \hspace{0.3em}$\Rightarrow$\hspace{0.3em} \JustifiedLabel{4em}{提出日} & 
  \hspace{0.3em} 令和 \UnderlineBox[0.65cm]{} 年 \UnderlineBox[0.65cm]{} 月 \UnderlineBox[0.65cm]{} 日 \\
  
  ( \JustifiedLabel{6em}{再提出期限} & 
  \hspace{0.3em} 令和 \UnderlineBox[0.65cm]{} 年 \UnderlineBox[0.65cm]{} 月 \UnderlineBox[0.65cm]{} 日 & 
  \hspace{0.3em}$\Rightarrow$\hspace{0.3em} \JustifiedLabel{5em}{再提出日} & 
  \hspace{0.3em} 令和 \UnderlineBox[0.65cm]{} 年 \UnderlineBox[0.65cm]{} 月 \UnderlineBox[0.65cm]{} 日 )
\end{tabular}

\vfill 

% --- 評価テーブル ---
\renewcommand{\arraystretch}{1.5}
\begin{center}
\begin{tabular}{|>{\centering\arraybackslash}m{2.4cm}|>{\raggedright\arraybackslash}m{12.1cm}|>{\centering\arraybackslash}m{2.4cm}|}
\hline
\multicolumn{2}{|c|}{\JustifiedLabel{11em}{評 価 項 目}} & \JustifiedLabel{4em}{評 価} \\
\hline
\multirow{3}{*}{\parbox[c][4.5em][c]{2.4cm}{\centering\shortstack{\large\bfseries 実 習\\[0.3em]\large\bfseries 評 価}}} 
 & (1) 自ら積極的に実験に取り組めた &  \\ \cline{2-3}
 & (2) 実験装置を適切に使用でき,正確に実験を行なえた &  \\ \cline{2-3}
 & (3) グループ内で協力的に実験が行なえた &  \\
\hline
\multirow{4}{*}{\parbox[c][6.0em][c]{2.4cm}{\centering\shortstack{\large\bfseries 報告書\\[0.3em]\large\bfseries 評 価}}} 
 & (1) 結果のまとめかた(図表を含む) &  \\ \cline{2-3}
 & (2) 結果に対する考察 &  \\ \cline{2-3}
 & (3) 報告事項/課題(正しい解答や適切な引用など) &  \\ \cline{2-3}
 & (4) 報告書としての体裁が整っているか &  \\
\hline
\end{tabular}
\end{center}

\clearpage

% /////////////////////////////////////////////
% ここから本文 (Main Body)
% /////////////////////////////////////////////

\restoregeometry 
\setcounter{page}{1}
\pagestyle{plain} 

\section{目的}
2次遅れ系の制御系を電子回路(アクティブフィルタ)で実現し,その周波数応答を実験により測定する。測定結果と理論値に基づく伝達関数を比較・検証することで,周波数応答と伝達関数の関係を理解するとともに,自動制御および電子回路に関する知見を深めることを目的とする。

\section{実験報告}

本節では,課題として課された「使用機器」「理論値の算出」「実験結果」について報告する。

\subsection{使用機器}
本実験で使用した主要機器を\cref{tab:equip}に示す。

\begin{table}[H]
    \centering
    \caption{使用機器一覧}
    \label{tab:equip}
    \renewcommand{\arraystretch}{1.2}
    \begin{tabular}{@{}cclcl@{}}
        \toprule
        No. & 機器名 & メーカー / 型番 & 定格・仕様 & 管理番号 \\
        \midrule
        1 & 発振器 &  &  &  \\
        2 & オシロスコープ &  &  &  \\
        3 & 直流安定化電源 &  &  &  \\
        4 & デジタルマルチメータ &  &  &  \\
        \bottomrule
    \end{tabular}
\end{table}

\subsection{理論値の算出}

\subsubsection{伝達関数の導出}
本実験で用いるアクティブフィルタの回路図を\cref{fig:circuit}に示す。

\begin{figure}[H]
    \centering
    \begin{circuitikz}[american, scale=1.0, transform shape]
        % オペアンプ
        \draw (5,2) node[op amp] (opamp) {};
        
        % opamp.+ は非反転入力端子
        \draw (0,2) node[left] {$V_i$} to[R, l=$R$, a=\SI{100}{\ohm}] (2,2) coordinate(node1);
        \draw (node1) to[R, l=$R$, a=\SI{100}{\ohm}] (4,2) -- (opamp.+);
        
        % Voltage Follower Feedback
        \draw (opamp.-) -- ++(0,1) -| (opamp.out); 
        \draw (opamp.out) to[short, -o] (7,2) node[right] {$V_o$};
        
        % Feedback Capacitor (alpha*C)
        \draw (opamp.out) -- ++(0,1.5) coordinate(top) to[C, l=$\alpha C\,(\SI{10}{\micro F})$] (2,3.5) -- (node1);
        
        % Ground Capacitor (C/alpha)
        \draw (4,2) to[C, l=$C/\alpha\,(\SI{0.5}{\micro F})$] (4,0) node[ground]{};
    \end{circuitikz}
    \caption{アクティブフィルタの回路図(Sallen-Key LPF構成)}
    \label{fig:circuit}
\end{figure}

本回路の入力電圧を $V_i$,出力電圧を $V_o$ とするとき,伝達関数 $G(s)$ は以下のように導出される。
キルヒホッフの法則およびオペアンプの理想特性(イマジナリショート等)を用いると,
\begin{align}
    G(s) = \frac{V_o(s)}{V_i(s)} &= \frac{1}{1 + \frac{2}{\alpha}sCR + s^2 C^2 R^2} \\
         &= \frac{\left(\frac{1}{CR}\right)^2}{s^2 + 2\cdot\frac{1}{\alpha}\cdot\frac{1}{CR}s + \left(\frac{1}{CR}\right)^2} \label{eq:circuit_tf}
\end{align}
となる。一方,標準的な2次遅れ系の伝達関数は,減衰係数 $\ze$,固有角周波数 $\wn$ を用いて次式で表される。
\begin{equation}
    G(s) = \frac{\wn^2}{s^2 + 2\ze\wn s + \wn^2} \label{eq:standard_tf}
\end{equation}
\cref{eq:circuit_tf} と \cref{eq:standard_tf} の係数を比較することにより,以下の関係が得られる。
\begin{equation}
    \ze = \frac{1}{\alpha}, \quad \wn = \frac{1}{CR}
\end{equation}

\subsubsection{パラメータの計算}
与えられた条件 $\alpha C = \SI{10}{\micro F}$, $C/\alpha = \SI{0.5}{\micro F}$ より,まずキャパシタンス $C$ および係数 $\alpha$ を求める。
\begin{align}
    (\alpha C) \times (C/\alpha) = C^2 &= 10 \times 0.5 = 5 \quad [\si{\micro F}^2] \notag \\
    \therefore C &= \sqrt{5}\,\si{\micro F} \approx \SI{2.236}{\micro F}
\end{align}
これより $\alpha$ は,
\begin{equation}
    \alpha = \frac{\alpha C}{C} = \frac{10}{\sqrt{5}} = 2\sqrt{5} \approx 4.472
\end{equation}
となる。したがって,減衰係数 $\ze$ は次のように求まる。
\begin{equation}
    \ze = \frac{1}{\alpha} = \frac{1}{2\sqrt{5}} \approx \mathbf{0.224}
\end{equation}
次に,抵抗 $R = \SI{100}{\ohm}$ としたときの固有角周波数 $\wn$ を求める。
\begin{equation}
    \wn = \frac{1}{CR} = \frac{1}{\sqrt{5}\times 10^{-6} \cdot 100} = \frac{10^4}{\sqrt{5}} \approx \mathbf{\SI{4472}{rad/s}}
\end{equation}

\subsubsection{周波数応答特性値の計算}
共振ピーク値 $M_p$ および共振角周波数 $\omega_p$ を理論式より算出する。
\begin{align}
    M_p &= \frac{1}{2\ze\sqrt{1-\ze^2}} \notag \\
        &= \frac{1}{2 \cdot 0.224 \cdot \sqrt{1 - 0.224^2}} \approx \mathbf{2.29} \quad (20\log_{10} 2.29 \approx \mathbf{7.2\,\si{dB}}) \\[1.5em]
    \omega_p &= \wn\sqrt{1 - 2\ze^2} \notag \\
             &= 4472 \times \sqrt{1 - 2 \cdot 0.224^2} \approx \mathbf{\SI{4242}{rad/s}}
\end{align}
また,共振周波数 $f_p$ は以下のようになる。
\begin{equation}
    f_p = \frac{\omega_p}{2\pi} \approx \frac{4242}{2\pi} \approx \mathbf{\SI{675}{Hz}}
\end{equation}

\subsection{実験結果}
入力電圧 $V_i$ を約 $1.5\,\mathrm{V_{p-p}}$ に設定し,周波数を $\SI{100}{Hz}$ から $\SI{7}{kHz}$ まで変化させた際の測定結果を\cref{tab:result}に示す。特に共振点付近(約 $\SI{675}{Hz}$)では細かく測定を行った。

\begin{table}[H]
    \centering
    \caption{周波数応答 測定結果集計表(例)}
    \label{tab:result}
    \small
    \begin{tabular}{@{}cccccccc@{}}
        \toprule
        \multirow{2}{*}{\shortstack{周波数\\$f\,[\si{Hz}]$}} & 
        \multirow{2}{*}{\shortstack{角周波数\\$\omega\,[\si{rad/s}]$}} & 
        入力電圧 & 出力電圧 & 電圧比 & ゲイン & 遅れ時間 & 位相 \\
         & & $V_i\,[\si{V}]$ & $V_o\,[\si{V}]$ & $V_o/V_i$ & $G\,[\si{dB}]$ & $t\,[\si{sec}]$ & $\theta\,[^\circ]$ \\
        \midrule
        99.6 & 625.8 & 1.304 & 1.304 & 1.00 & 0.00 & \SI{0.11}{m} & -3.94 \\
        \vdots & \vdots & \vdots & \vdots & \vdots & \vdots & \vdots & \vdots \\
        (共振点) & \dots & \dots & \dots & \dots & \dots & \dots & \dots \\
        \vdots & \vdots & \vdots & \vdots & \vdots & \vdots & \vdots & \vdots \\
        7000 & \dots & \dots & \dots & \dots & \dots & \dots & \dots \\
        \bottomrule
    \end{tabular}
\end{table}

\noindent
※ 実験データに基づくゲイン特性および位相特性のグラフ(ボード線図)を別紙として添付する。

\section{考察}

\subsection{実験結果からの伝達関数の導出}
実験で得られたゲイン特性のグラフより読み取った共振ピーク値を $M_{p}'$ とする。理論式 $M_p = 1 / (2\ze\sqrt{1-\ze^2})$ を $\ze$ について解くことで,実験値に基づく減衰係数を逆算できる。
式を変形すると $4M_{p}^2 \ze^4 - 4M_{p}^2 \ze^2 + 1 = 0$ となり,これを $\ze$ について解くと次式が得られる。
\begin{equation}
    \ze = \sqrt{\frac{1}{2}\left( 1 - \sqrt{1 - \frac{1}{M_{p}^2}} \right)}
\end{equation}
なお,数学的には複号($\pm$)により4つの解が存在するが,物理的な意味($\ze$ は実数かつ正,また共振が生じる範囲 $0 < \ze < 1/\sqrt{2}$)を考慮し,上記の通り解を選定した。
求めた $\ze$ を用い,ピーク周波数 $\omega_p'$ から固有角周波数 $\wn$ を次式で決定する。
\begin{equation}
    \wn = \frac{\omega_p'}{\sqrt{1 - 2\ze^2}}
\end{equation}
これにより,実験結果に基づく伝達関数が同定された。

\subsection{理論値と実験値の比較}
(ここに,算出した理論値 $\ze, \wn$ と,実験結果から逆算した値の比較を記述する。誤差率を計算し,素子の許容差や測定器の内部インピーダンスの影響などを考察に含める。)

\subsection{MATLABによる検証}
理論値および実験値の妥当性を検証するため,MATLABを用いてボード線図を作成した。使用したコマンドおよび伝達関数の設定を以下に示す。なお,下記コード内の数値は本実験パラメータに基づく一例である。

\begin{lstlisting}[caption=MATLABによるボード線図描画コード, language=Matlab]
% 伝達関数の定義
% 分子: wn^2 = 41269009 (approx 6424^2 ??? Note: check values)
% ※ここでは実験パラメータ(wn=4472)に基づき再計算した値を使用すべきだが、
%   参考資料の数値を記載する。
%   wn = 4472 -> wn^2 = 19998784
%   分母: s^2 + 2*zeta*wn*s + wn^2
%   2*zeta*wn = 2 * 0.224 * 4472 = 2003.5
%
% 資料の例:
sys2 = tf([41269009], [1 4625.35 41269009]); 

disp('System Transfer Function:');
disp(sys2);

% ボード線図のプロット
figure;
bode(sys2);
grid on;
title('Bode Plot of 2nd Order System');
\end{lstlisting}

MATLABにより得られたボード線図と,実験結果のグラフを比較した結果,(ここに一致度や波形の特徴についての記述を追加。例:低域でのフラットな特性およびカットオフ周波数以降の $-40\,\si{dB/dec}$ の傾きが確認でき,本回路が2次遅れ系として正常に動作していることを確認した,等)。

\section{参考文献}
\begin{enumerate}
    \item 鈴木 宏: 自動制御実験テキスト, 国立長野高専 電気電子工学科.
    \item 制御工学 教科書 P127 $\sim$ P128.
\end{enumerate}

\end{document}