\documentclass[a4paper,11pt]{ltjsarticle}

% =============================================
% 1. パッケージ設定 (SARP v2.0準拠)
% =============================================
\usepackage[T1]{fontenc}
\usepackage{newtxtext}
\usepackage[varbb]{newtxmath} % 数式フォント
\usepackage{bm}      % ベクトル太字
\usepackage{mathtools}

% レイアウト・図表関連
\usepackage[margin=25mm]{geometry}
\usepackage{array}      
\usepackage{multirow}   
\usepackage{fancyhdr}   
\usepackage{graphicx}
\usepackage{float}
\usepackage{booktabs}
\usepackage{subcaption}

% 回路図・コード・グラフ描画
\usepackage{circuitikz}
\usepackage{listings}
\usepackage{tikz}
\usepackage{pgfplots}
\pgfplotsset{compat=newest}
\usetikzlibrary{arrows.meta, positioning, calc}

% Listingsの設定
\lstset{
  basicstyle=\ttfamily\small,
  frame=trBL,
  numbers=left,
  stepnumber=1,
  numberstyle=\scriptsize,
  breaklines=true,
  captionpos=b,
  xleftmargin=2em,
  xrightmargin=1em
}

% SI単位・数式処理
\usepackage{siunitx}
\sisetup{
  detect-all,
  inter-unit-product=\ensuremath{{}\cdot{}},
  separate-uncertainty=true
}

% リンク・参照
\usepackage{cite}
\usepackage[hidelinks]{hyperref}
\usepackage[nameinlink,noabbrev]{cleveref}

% 参考文献の上付き表示設定
\makeatletter
\def\@cite#1#2{$^{\mbox{\scriptsize[#1\if@tempswa , #2\fi]}}$}
\def\@biblabel#1{[#1]}
\makeatother

\crefname{figure}{図}{図}
\crefname{table}{表}{表}
\crefname{equation}{式}{式}

% キャプション設定
\usepackage{caption}
\captionsetup{
  format=hang,
  labelsep=quad,
  font={small},
  labelfont={bf},
  justification=centering
}
\captionsetup[figure]{justification=centerlast}

% =============================================
% 2. カスタムコマンド定義
% =============================================
\newcommand{\UnderlineBox}[2][3cm]{\underline{\makebox[#1][c]{\vphantom{lp}\large #2}}}
\newcommand{\JustifiedLabel}[2]{\makebox[#1][s]{\large\bfseries #2}}
\newcommand{\BoldLabel}[1]{{\large\bfseries #1}}

% 数式用コマンド
\newcommand{\diff}[2]{\frac{\mathrm{d}#1}{\mathrm{d}#2}}
\newcommand{\pdiff}[2]{\frac{\partial #1}{\partial #2}}
\newcommand{\ze}{\zeta}      % 減衰係数
\newcommand{\wn}{\omega_n}   % 固有角周波数

% 画像パス設定
\graphicspath{{image/}}

% =============================================
% 3. 表紙専用のページスタイル定義
% =============================================
\fancypagestyle{coverpage}{
  \fancyhf{} 
  \renewcommand{\headrulewidth}{0pt} 
  \renewcommand{\footrulewidth}{0pt} 
  \cfoot{\vspace{5mm}\Large \bfseries 国立長野高専 電気電子工学科}
}

% =============================================
% ドキュメント開始
% =============================================
\begin{document}

% /////////////////////////////////////////////
% 表紙 (Cover Page)
% /////////////////////////////////////////////

\newgeometry{top=25mm, bottom=20mm, left=18mm, right=18mm}
\thispagestyle{coverpage}

\begin{center}
    \vspace*{0mm} 
    {\Huge \bfseries 電気電子工学実験報告書}
    \vspace{10mm} 
\end{center}

\noindent
\begin{tabular}{@{}ll}
  \BoldLabel{テーマ名} & \UnderlineBox[13.5cm]{4. PIDによる温度制御} \\[2.0em] 
\end{tabular}
% ※ 注: 実験内容は「2次遅れ系のアクティブフィルタ」であるが,指定されたテンプレートのテーマ名を維持する。

\noindent
\BoldLabel{報告者} \hspace{0.5em}
\UnderlineBox[1.5cm]{5} {\large \textbf{年}} \hspace{0.2em}     
(\UnderlineBox[1.5cm]{E} {\large \textbf{組}}) \hspace{0.2em} 
{\large \textbf{番号}} \UnderlineBox[2.0cm]{234} \hspace{0.5em}   
\UnderlineBox[1.5cm]{B} {\large \textbf{班}} \hspace{1em}       
\UnderlineBox[4.5cm]{栁原魁人}                                         
\vspace{2.0em} 

\noindent
\begin{tabular}{@{}p{0.48\textwidth} p{0.48\textwidth}}
  \BoldLabel{実験場所} \hspace{1em} \UnderlineBox[5.5cm]{} & 
  \BoldLabel{指導担当} \hspace{1em} \UnderlineBox[5.5cm]{}   
\end{tabular}
\vspace{2.0em} 

\noindent
\BoldLabel{共同実験者} \hspace{1em} \UnderlineBox[12.5cm]{石坂知尋,倉科純太郎,中井智大,中澤耕平} 
\vspace{2.5em} 

\noindent
\renewcommand{\arraystretch}{2.0}
\setlength{\tabcolsep}{0pt}
\begin{tabular}{l l l l}
  \JustifiedLabel{5em}{実験日} & 
  \hspace{0.3em} 令和 \UnderlineBox[0.65cm]{} 年 \UnderlineBox[0.65cm]{} 月 \UnderlineBox[0.65cm]{} 日 & & \\
  \JustifiedLabel{5em}{提出期限} & 
  \hspace{0.3em} 令和 \UnderlineBox[0.65cm]{} 年 \UnderlineBox[0.65cm]{} 月 \UnderlineBox[0.65cm]{} 日 & 
  \hspace{0.3em}$\Rightarrow$\hspace{0.3em} \JustifiedLabel{4em}{提出日} & 
  \hspace{0.3em} 令和 \UnderlineBox[0.65cm]{} 年 \UnderlineBox[0.65cm]{} 月 \UnderlineBox[0.65cm]{} 日 \\
  ( \JustifiedLabel{6em}{再提出期限} & 
  \hspace{0.3em} 令和 \UnderlineBox[0.65cm]{} 年 \UnderlineBox[0.65cm]{} 月 \UnderlineBox[0.65cm]{} 日 & 
  \hspace{0.3em}$\Rightarrow$\hspace{0.3em} \JustifiedLabel{5em}{再提出日} & 
  \hspace{0.3em} 令和 \UnderlineBox[0.65cm]{} 年 \UnderlineBox[0.65cm]{} 月 \UnderlineBox[0.65cm]{} 日 )
\end{tabular}
\vfill 

\renewcommand{\arraystretch}{1.5}
\begin{center}
\begin{tabular}{|>{\centering\arraybackslash}m{2.4cm}|>{\raggedright\arraybackslash}m{12.1cm}|>{\centering\arraybackslash}m{2.4cm}|}
\hline
\multicolumn{2}{|c|}{\JustifiedLabel{11em}{評 価 項 目}} & \JustifiedLabel{4em}{評 価} \\
\hline
\multirow{3}{*}{\parbox[c][4.5em][c]{2.4cm}{\centering\shortstack{\large\bfseries 実 習\\[0.3em]\large\bfseries 評 価}}} 
 & (1) 自ら積極的に実験に取り組めた &  \\ \cline{2-3}
 & (2) 実験装置を適切に使用でき,正確に実験を行なえた &  \\ \cline{2-3}
 & (3) グループ内で協力的に実験が行なえた &  \\
\hline
\multirow{4}{*}{\parbox[c][6.0em][c]{2.4cm}{\centering\shortstack{\large\bfseries 報告書\\[0.3em]\large\bfseries 評 価}}} 
 & (1) 結果のまとめかた(図表を含む) &  \\ \cline{2-3}
 & (2) 結果に対する考察 &  \\ \cline{2-3}
 & (3) 報告事項/課題(正しい解答や適切な引用など) &  \\ \cline{2-3}
 & (4) 報告書としての体裁が整っているか &  \\
\hline
\end{tabular}
\end{center}
\clearpage

% /////////////////////////////////////////////
% 本文 (Main Body)
% /////////////////////////////////////////////

\restoregeometry 
\setcounter{page}{1}
\pagestyle{plain} 

\section{目的}
2次遅れ系の制御系を電子回路(Sallen-Key型ローパスフィルタ)で実現し,その周波数応答を実験により測定した。測定結果と理論値に基づく伝達関数を比較・検証することで,周波数応答と伝達関数の関係を理解するとともに,自動制御および電子回路に関する知見を深めることを目的とする。

\section{原理および理論計算}
本実験では2次遅れ系の挙動を模倣するため,オペアンプを用いたアクティブフィルタを使用する。

\subsection{伝達関数の導出}
実験で使用した回路図を\cref{fig:circuit}に示す。

\begin{figure}[H]
    \centering
    \begin{circuitikz}[american, scale=1.0, transform shape]
        \draw (5,2) node[op amp] (opamp) {};
        \draw (0,2) node[left] {$V_i$} to[R, l=$R$, a=\SI{100}{\ohm}] (2,2) coordinate(node1);
        \draw (node1) to[R, l=$R$, a=\SI{100}{\ohm}] (4,2) -- (opamp.+);
        \draw (opamp.-) -- ++(0,1) -| (opamp.out); 
        \draw (opamp.out) to[short, -o] (7,2) node[right] {$V_o$};
        \draw (opamp.out) -- ++(0,1.5) coordinate(top) to[C, l=$\alpha C\,(\SI{10}{\micro F})$] (2,3.5) -- (node1);
        \draw (4,2) to[C, l=$C/\alpha\,(\SI{0.5}{\micro F})$] (4,0) node[ground]{};
    \end{circuitikz}
    \caption{アクティブフィルタの回路図(Sallen-Key LPF)}
    \label{fig:circuit}
\end{figure}

本回路の伝達関数 $G(s)$ は,キルヒホッフの法則より以下の通り導出される。
\begin{align}
    G(s) = \frac{V_o(s)}{V_i(s)} &= \frac{\left(\frac{1}{CR}\right)^2}{s^2 + \frac{2}{\alpha CR}s + \left(\frac{1}{CR}\right)^2} \label{eq:circuit_tf}
\end{align}
標準的な2次遅れ系の伝達関数 $G(s) = \frac{\wn^2}{s^2 + 2\ze\wn s + \wn^2}$ と係数比較を行うことで,パラメータは次のように対応付けられる。
\begin{equation}
    \ze = \frac{1}{\alpha}, \quad \wn = \frac{1}{CR}
\end{equation}

\subsection{理論値の算出}
使用素子の定数は $\alpha C = \SI{10}{\micro F}$, $C/\alpha = \SI{0.5}{\micro F}$,$R=\SI{100}{\ohm}$ である。これより,$C = \sqrt{5}\,\si{\micro F}$,$\alpha = 2\sqrt{5} \approx 4.472$ となる。
したがって,理論的な減衰係数 $\ze$ および固有角周波数 $\wn$ は以下の通りとなる。
\begin{equation}
    \ze = \frac{1}{\alpha} \approx \mathbf{0.224}, \quad \wn = \frac{1}{CR} \approx \mathbf{\SI{4472}{rad/s}}
\end{equation}
このとき,共振ピーク値 $M_p$ は次式で予測される。
\begin{equation}
    M_p = \frac{1}{2\ze\sqrt{1-\ze^2}} \approx \mathbf{2.29} \quad (\approx \mathbf{7.2\,\si{dB}})
\end{equation}

\section{実験方法および使用機器}

\subsection{使用機器}
本実験で使用した主要機器を\cref{tab:equip}に示す。

\begin{table}[H]
    \centering
    \caption{使用機器一覧}
    \label{tab:equip}
    \renewcommand{\arraystretch}{1.2}
    \begin{tabular}{@{}cclcl@{}}
        \toprule
        No. & 機器名 & メーカー / 型番 & 定格・仕様 & 管理番号 \\
        \midrule
        1 & 発振器 & KENWOOD / AG-203D & $10\,\si{Hz} \sim 1\,\si{MHz}$ & - \\
        2 & オシロスコープ & Tektronix / TBS1052B & $50\,\si{MHz}, 1\,\si{GS/s}$ & - \\
        3 & 直流安定化電源 & KENWOOD / PD18-10AD & $0 \sim 18\,\si{V}, 10\,\si{A}$ & - \\
        4 & デジタルマルチメータ & FLUKE / 179 & True RMS & - \\
        \bottomrule
    \end{tabular}
\end{table}

\subsection{実験方法}
回路に入力電圧 $V_i \approx 1.5\,\mathrm{V_{p-p}}$ の正弦波を印加し,周波数を $\SI{100}{Hz}$ から $\SI{7}{kHz}$ まで変化させた。各周波数における入出力電圧の振幅および位相差をオシロスコープにて測定した。

\section{実験結果}
測定結果の一部を\cref{tab:result}に示す。共振点付近(約 $\SI{675}{Hz}$)では詳細な測定を行った。

\begin{table}[H]
    \centering
    \caption{周波数応答 測定結果(抜粋)}
    \label{tab:result}
    \small
    \begin{tabular}{@{}cccccccc@{}}
        \toprule
        周波数 & 角周波数 & 入力電圧 & 出力電圧 & 電圧比 & ゲイン & 遅れ時間 & 位相 \\
        $f\,[\si{Hz}]$ & $\omega\,[\si{rad/s}]$ & $V_i\,[\si{V}]$ & $V_o\,[\si{V}]$ & $V_o/V_i$ & $G\,[\si{dB}]$ & $t\,[\si{sec}]$ & $\theta\,[^\circ]$ \\
        \midrule
        100 & 628 & 1.30 & 1.30 & 1.00 & 0.00 & 0.11m & -4.0 \\
        675 & 4241 & 1.30 & 2.95 & 2.27 & 7.12 & 0.37m & -90.0 \\
        7000 & 43982 & 1.30 & 0.01 & 0.008 & -41.9 & 0.07m & -175.0 \\
        \bottomrule
    \end{tabular}
\end{table}

\section{考察}

\subsection{実験結果からの減衰係数 $\zeta$ の厳密な導出}
実験より得られた共振ピーク値を $M_p$ とする。この $M_p$ から,系の減衰係数 $\ze$ を逆算する過程を以下に示す。理論式におけるピーク値の関係式は次式で与えられる。
\begin{equation}
    M_p = \frac{1}{2\ze\sqrt{1-\ze^2}} \label{eq:Mp_def}
\end{equation}
\cref{eq:Mp_def}の両辺を2乗し,分母を払うと次式となる。
\begin{align}
    M_p^2 &= \frac{1}{4\ze^2(1-\ze^2)} \\
    4M_p^2 \ze^2 (1 - \ze^2) &= 1 \\
    4M_p^2 \ze^2 - 4M_p^2 \ze^4 - 1 &= 0
\end{align}
全体に $-1$ を乗じて整理すると,$\ze$ に関する複2次方程式が得られる。
\begin{equation}
    4M_p^2 \ze^4 - 4M_p^2 \ze^2 + 1 = 0 \label{eq:biquad}
\end{equation}
ここで $X = \ze^2$ とおくと,\cref{eq:biquad}は $X$ に関する2次方程式 $4M_p^2 X^2 - 4M_p^2 X + 1 = 0$ となる。解の公式を用いると,
\begin{align}
    X &= \frac{4M_p^2 \pm \sqrt{(4M_p^2)^2 - 4 \cdot 4M_p^2 \cdot 1}}{2 \cdot 4M_p^2} \notag \\
      &= \frac{4M_p^2 \pm \sqrt{16M_p^4 - 16M_p^2}}{8M_p^2} \notag \\
      &= \frac{4M_p^2 \pm 4M_p\sqrt{M_p^2 - 1}}{8M_p^2} \notag \\
      &= \frac{1 \pm \sqrt{1 - \frac{1}{M_p^2}}}{2}
\end{align}
したがって,$\ze = \pm \sqrt{X}$ であるから,数学的には以下の4つの解が存在する。
\begin{equation}
    \ze = \pm \sqrt{\frac{1 + \sqrt{1 - \frac{1}{M_p^2}}}{2}}, \quad \pm \sqrt{\frac{1 - \sqrt{1 - \frac{1}{M_p^2}}}{2}} \label{eq:four_solutions}
\end{equation}

\noindent
\textbf{物理的妥当性による解の選定:}
これら4つの解から,物理的条件に基づき真の解を選定する。
\begin{enumerate}
    \item \textbf{安定性の条件 ($\ze > 0$)}:
    系が安定である(発振せず減衰する)ためには $\ze > 0$ でなければならない。これにより,負号を持つ2つの解は不適として除外される。
    
    \item \textbf{共振の存在条件 ($0 < \ze < 1/\sqrt{2}$)}:
    周波数応答においてピーク値 $M_p > 1$ が観測されるための条件は,共振角周波数 $\omega_p = \wn\sqrt{1-2\ze^2}$ が実数であること,すなわち $1-2\ze^2 > 0 \iff \ze^2 < 0.5$ である。
    
    ここで,正の解の候補である2つの式の中身($\ze^2$の値)を比較する。
    \begin{itemize}
        \item 候補1 ($\ze^2 = \frac{1 + D}{2}$): ここで $D = \sqrt{1 - 1/M_p^2} > 0$ であるため,$\frac{1+D}{2} > 0.5$ となる。これは共振条件 $\ze^2 < 0.5$ に矛盾するため不適である。
        \item 候補2 ($\ze^2 = \frac{1 - D}{2}$): $D < 1$ であるため,$\frac{1-D}{2} < 0.5$ となり,共振条件を満たす。
    \end{itemize}
\end{enumerate}

以上の考察により,実験値 $M_p$ から導かれる物理的に意味のある唯一の解は次式となる。
\begin{equation}
    \ze = \sqrt{\frac{1}{2}\left( 1 - \sqrt{1 - \frac{1}{M_{p}^2}} \right)}
\end{equation}
本実験結果 $M_p \approx 2.27$ を代入すると,
\begin{equation}
    \ze \approx \sqrt{\frac{1}{2}\left( 1 - \sqrt{1 - \frac{1}{2.27^2}} \right)} \approx \mathbf{0.226}
\end{equation}
となり,理論値 $\ze = 0.224$ と極めて良好に一致した。

\subsection{理論値と実験値の比較}
算出した $\ze$ およびピーク周波数から逆算した $\wn$ を用いてMATLABによるシミュレーションを行った結果,実験データのプロットとボード線図は誤差範囲内で一致した。誤差の主要因としては,コンデンサの静電容量の公差(通常$\pm 10\%$程度)およびオペアンプの有限ゲイン帯域幅積(GB積)の影響が考えられる。特に高周波域における位相遅れの増大は,オペアンプの理想特性からの乖離を示唆している。

\section{参考文献}
\begin{enumerate}
    \item 鈴木 宏: 自動制御実験テキスト, 国立長野高専 電気電子工学科.
    \item 制御工学 教科書 P127 $\sim$ P128.
\end{enumerate}

\end{document}