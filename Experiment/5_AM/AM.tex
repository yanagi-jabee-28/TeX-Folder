% !TEX program = lualatex
% !TEX options = --shell-escape
%==============================================================================
% プリアンブル (Preamble)
%==============================================================================

% ===== ドキュメントクラス =====
\documentclass[
  a4paper,
  11pt,
]{ltjsarticle}

%------------------------------------------------------------------------------
% パッケージ読み込み
%------------------------------------------------------------------------------

% ===== フォント・言語設定 =====
\usepackage{luatexja-fontspec} 

% ===== レイアウト関連 =====
\usepackage[margin=2.5cm]{geometry} 
\usepackage{graphicx}          

% ===== SVG関連 (重要) =====
% Inkscapeがインストールされ、PATHが通っている必要があります。
\usepackage[inkscapelatex=false]{svg} 

\usepackage{booktabs}          
\usepackage{float}             
\usepackage{wrapfig}           

% ===== 数式・物理単位関連 =====
\usepackage{amsmath}           
\usepackage{amsthm}            
\usepackage{newtxmath}         
\usepackage{siunitx}           
\usepackage{cancel}            

% ===== 図表・グラフ描画関連 =====
\usepackage{tikz}
\usepackage{circuitikz}        
\usepackage{pgfplots}          
\usepackage{pgfplotstable}     
\pgfplotsset{compat=1.18}      
\usepgfplotslibrary{statistics} 
\usetikzlibrary{positioning}   

% ===== プログラミング・アルゴリズム関連 =====
\usepackage{listings}          
\usepackage{algorithm}         
\usepackage{algpseudocode}     

% ===== その他 =====
\usepackage[
  colorlinks=true,      
  linkcolor=blue,         
  citecolor=green!60!black, 
  urlcolor=cyan,          
  hidelinks,              
]{hyperref}

%------------------------------------------------------------------------------
% 各種設定
%------------------------------------------------------------------------------

% ===== フォント設定 =====
\setmainfont{Latin Modern Roman}
\setsansfont{Latin Modern Sans}
\setmonofont{Latin Modern Mono}
\setmainjfont[Renderer=HarfBuzz]{Yu Mincho}
\setsansjfont[Renderer=HarfBuzz]{Yu Gothic}

% ===== ドキュメント情報 =====
\title{5. AM}
\author{}
\date{}

% ===== listings 設定 =====
\lstset{
  language=Python,
  basicstyle=\small\ttfamily,
  keywordstyle=\color{blue},
  commentstyle=\color{green!50!black},
  stringstyle=\color{purple},
  showstringspaces=false,
  frame=tb,
  captionpos=b,
  breaklines=true,
  numbers=left,
  numberstyle=\tiny\color{gray},
  xleftmargin=2em, 
  framexleftmargin=1.5em, 
}

% ===== pgfplots 設定 =====
\pgfplotsset{
  report-style/.style={
    xlabel style={yshift=0.5em}, 
    ylabel style={yshift=-0.5em},
    legend pos=north west,
    grid=major,
    ticklabel style={font=\small},
    label style={font=\small},
    legend style={font=\small},
  }
}

% ===== 数式用カスタムコマンド =====
\newcommand{\dd}{\mathrm{d}} 
\newcommand{\mi}{\mathrm{j}} 

% ===== algorithmicx 設定 =====
\renewcommand{\textproc}[1]{\textbf{#1}}

%==============================================================================
% ドキュメント本体 (Body)
%==============================================================================
\begin{document}

\maketitle

% ===================================================================
\section{目的}
% ===================================================================
AMと復調の原理及びそれを実現する回路の動作を理解する.

% ===================================================================
\section{AMの変調・復調の原理}
% ===================================================================
無線通信では,高い周波数の電磁波を用いなければ,電波の放射が能率よく行われない.そこで,音声のような情報信号をどのようにして高周波にのせるかということが問題となる.その一つの方法がAM変調であり,高周波(搬送波)の振幅を情報信号(変調波)で変化させる方法である.

いま,搬送波 $v_c$ が
\begin{equation}
  v_c = V_c \cos \omega_c t
  \label{eq:carrier}
\end{equation}
で表される正弦波とし,これを
\begin{equation}
  v_s = V_s \cos \omega_s t
  \label{eq:signal}
\end{equation}
で表される信号波(変調波)によって振幅変調する場合を考える.振幅変調は,搬送波の振幅が変調波によって変化する方式なので,被搬送波の振幅は
\begin{equation}
  V_c + k_a V_s \cos \omega_s t
  \label{eq:amplitude}
\end{equation}
となり,振幅が時間によって変化する.ここに,$k_a$ は比例定数である.したがって,被変調波 $v$ は,次のようになる.
\begin{equation}
  v = (V_c + k_a V_s \cos \omega_s t) \cos \omega_c t = V_c(1 + m_a \cos \omega_s t) \cos \omega_c t
  \label{eq:modulated_wave}
\end{equation}
ここで,
\begin{equation}
  m_a = k_a V_s / V_c
  \label{eq:modulation_index}
\end{equation}
であり,これを変調度,または百分率で表して変調率という.

復調は,被変調波を整流または2乗し,それに含まれる低周波成分を取り出すことによってなされる.
振幅変調は,搬送波増幅器の増幅度を変調信号によって変化させればよい.図\ref{fig:waveforms}は変調信号,搬送波及び被変調波である.

\begin{figure}[H]
  \centering
  % SVGファイルを挿入
  % TeXファイルと同じフォルダに Wave.svg がある前提です
  \includesvg[width=0.8\textwidth]{Wave.svg}
  \caption{各部の波形 (a) 変調信号,(b) 搬送波,(c) 被変調波}
  \label{fig:waveforms}
\end{figure}

% ===================================================================
\section{実験}
% ===================================================================
図\ref{fig:circuit}はAM変復調回路である.

\begin{figure}[H]
  \centering
  % SVGファイルを挿入
  % TeXファイルと同じフォルダに config.svg がある前提です
  \includesvg[width=0.9\textwidth]{config.svg}
  \caption{AM 変復調回路}
  \label{fig:circuit}
\end{figure}

\begin{enumerate}
  \item 「AM OUT」端子と「AM IN」端子を接続する.

  \item 低周波発振器を「AF IN」に接続し,\SI{3}{\kilo\hertz}の入力信号に対する図\ref{fig:circuit}の「AF IN」,「AM OUT」及び「DET OUT」端子の電圧波形をスケッチする.ただし,「AM OUT」の包絡線電圧や「DET OUT」が歪まないよう低周波発振器のAmplitudeを調整すること.

  \item \SI{100}{\hertz}~\SI{50}{\kilo\hertz}の周波数範囲の「AF IN」-「AM OUT」及び「DET OUT」間の周波数特性を測定する(10ポイント以上).\\
  注:片対数のグラフ用紙を用いること.「AM OUT」については,包絡線電圧のp-p値を測定すること.

  \item 入力周波数一定としたときの「AF IN」に対する「AM OUT」,「DET OUT」の入出力特性を測定する(10ポイント以上).\\
  注:方眼紙を用いること.「AM OUT」については,包絡線電圧のp-p値を測定すること.
\end{enumerate}

% ===================================================================
\subsection{実験結果}
% ===================================================================
以下は本実験で得られた測定結果の抜粋である.単位は測定器表示の通り.

\begin{table}[H]
  \centering
  \caption{周波数特性: 周波数に対するAM OUT(包絡線p-p)とDET OUTの測定値}
  \label{tab:freq_response}
  \begin{tabular}{rcc}
    	oprule
    周波数 (Hz) & AM OUT & DET OUT \\
    \midrule
    100  & 1.46 & 0.186 \\
    400  & 1.88 & 0.394 \\
    600  & 1.95 & 0.584 \\
    800  & 1.94 & 0.734 \\
    2{,}000 & 2.07 & 0.810 \\
    3{,}000 & 2.09 & 0.802 \\
    5{,}000 & 2.08 & 0.860 \\
    10{,}000 & 2.01 & 0.860 \\
    20{,}000 & 2.07 & 0.835 \\
    50{,}000 & 0.94 & 0.830 \\
    \bottomrule
  \end{tabular}
\end{table}

\begin{table}[H]
  \centering
  \caption{入出力特性: AF IN (入力) と対応する AM OUT, DET OUT の測定値 (3 kHz)}
  \label{tab:io_characteristic}
  \begin{tabular}{rcc}
    	oprule
    AF IN & AM OUT & DET OUT \\
    \midrule
    1.35 & 3.50 & 1.36 \\
    1.225 & 3.24 & 1.23 \\
    1.13 & 3.00 & 1.15 \\
    0.995 & 2.55 & 1.025 \\
    0.83 & 2.22 & 0.87 \\
    0.735 & 1.93 & 0.76 \\
    0.596 & 1.66 & 0.634 \\
    0.49 & 1.36 & 0.514 \\
    0.41 & 1.07 & 0.432 \\
    0.18 & 0.405 & 0.174 \\
    \bottomrule
  \end{tabular}
\end{table}


% ===================================================================
\section{考察}
% ===================================================================
\begin{enumerate}
  \item 実験(2)の波形について考察する.
  \item 実験(3)の結果を考察する.
  \item 実験(4)の結果を考察する.
  \item 図\ref{fig:circuit}のAM変復調回路の各部の動作について説明する.
  \item 変調回路及び復調回路を一つずつ調べ動作を説明する.
\end{enumerate}

\end{document}