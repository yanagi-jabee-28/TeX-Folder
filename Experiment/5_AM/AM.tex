% !TEX program = lualatex
% !TEX options = --shell-escape
%==============================================================================
% プリアンブル (Preamble)
%==============================================================================

% ===== ドキュメントクラス =====
\documentclass[
  a4paper,
  11pt,
]{ltjsarticle}

%------------------------------------------------------------------------------
% パッケージ読み込み
%------------------------------------------------------------------------------

% ===== フォント・言語設定 =====
\usepackage{luatexja-fontspec} 

% ===== レイアウト関連 =====
\usepackage[margin=2.5cm]{geometry} 
\usepackage{graphicx}          
% Set graphics path to image folder
\graphicspath{{image/}}

% ===== SVG関連 (重要) =====
% Inkscapeがインストールされ、PATHが通っている必要があります。
\usepackage[inkscapelatex=false]{svg} 

\usepackage{booktabs}          
\usepackage{float}             
\usepackage{wrapfig}           

% ===== 数式・物理単位関連 =====
\usepackage{amsmath}           
\usepackage{amsthm}            
\usepackage{newtxmath}         
\usepackage{siunitx}           
\usepackage{cancel}            

% ===== 図表・グラフ描画関連 =====
\usepackage{tikz}
\usepackage{circuitikz}        
\usepackage{pgfplots}          
\usepackage{pgfplotstable}     
\pgfplotsset{compat=1.18}      
\usepgfplotslibrary{statistics} 
\usetikzlibrary{positioning}   

% ===== プログラミング・アルゴリズム関連 =====
\usepackage{listings}          
\usepackage{algorithm}         
\usepackage{algpseudocode}     
% Enumerate style
\usepackage{enumitem}
% Use parenthesized numbers like (1), (2) for top-level enumerate
\setlist[enumerate,1]{label=(\arabic*)}
% Subfigure package for arranging multiple images
\usepackage{subcaption}

% ===== その他 =====
\usepackage[
  colorlinks=true,      
  linkcolor=blue,         
  citecolor=green!60!black, 
  urlcolor=cyan,          
  hidelinks,              
]{hyperref}

%------------------------------------------------------------------------------
% 各種設定
%------------------------------------------------------------------------------

% ===== フォント設定 =====
\setmainfont{Latin Modern Roman}
\setsansfont{Latin Modern Sans}
\setmonofont{Latin Modern Mono}
\setmainjfont[Renderer=HarfBuzz]{Yu Mincho}
\setsansjfont[Renderer=HarfBuzz]{Yu Gothic}

% ===== ドキュメント情報 =====
\title{5. AM}
\author{}
\date{}

% ===== listings 設定 =====
\lstset{
  language=Python,
  basicstyle=\small\ttfamily,
  keywordstyle=\color{blue},
  commentstyle=\color{green!50!black},
  stringstyle=\color{purple},
  showstringspaces=false,
  frame=tb,
  captionpos=b,
  breaklines=true,
  numbers=left,
  numberstyle=\tiny\color{gray},
  xleftmargin=2em, 
  framexleftmargin=1.5em, 
}

% ===== pgfplots 設定 =====
\pgfplotsset{
  report-style/.style={
    xlabel style={yshift=0.5em}, 
    ylabel style={yshift=-0.5em},
    legend pos=north west,
    grid=major,
    ticklabel style={font=\small},
    label style={font=\small},
    legend style={font=\small},
  }
}

% ===== 数式用カスタムコマンド =====
\newcommand{\dd}{\mathrm{d}} 
\newcommand{\mi}{\mathrm{j}} 

% ===== algorithmicx 設定 =====
\renewcommand{\textproc}[1]{\textbf{#1}}

%==============================================================================
% ドキュメント本体 (Body)
%==============================================================================
\begin{document}

\maketitle

% ===================================================================
\section{目的}
% ===================================================================
AMと復調の原理及びそれを実現する回路の動作を理解する.

% ===================================================================
\section{AMの変調・復調の原理}
% ===================================================================
無線通信では,高い周波数の電磁波を用いなければ,電波の放射が能率よく行われない.そこで,音声のような情報信号をどのようにして高周波にのせるかということが問題となる.その一つの方法がAM変調であり,高周波(搬送波)の振幅を情報信号(変調波)で変化させる方法である.

いま,搬送波 $v_c$ が
\begin{equation}
  v_c = V_c \cos \omega_c t
  \label{eq:carrier}
\end{equation}
で表される正弦波とし,これを
\begin{equation}
  v_s = V_s \cos \omega_s t
  \label{eq:signal}
\end{equation}
で表される信号波(変調波)によって振幅変調する場合を考える.振幅変調は,搬送波の振幅が変調波によって変化する方式なので,被搬送波の振幅は
\begin{equation}
  V_c + k_a V_s \cos \omega_s t
  \label{eq:amplitude}
\end{equation}
となり,振幅が時間によって変化する.ここに,$k_a$ は比例定数である.したがって,被変調波 $v$ は,次のようになる.
\begin{equation}
  v = (V_c + k_a V_s \cos \omega_s t) \cos \omega_c t = V_c(1 + m_a \cos \omega_s t) \cos \omega_c t
  \label{eq:modulated_wave}
\end{equation}
ここで,
\begin{equation}
  m_a = k_a V_s / V_c
  \label{eq:modulation_index}
\end{equation}
であり,これを変調度,または百分率で表して変調率という.

復調は,被変調波を整流または2乗し,それに含まれる低周波成分を取り出すことによってなされる.
振幅変調は,搬送波増幅器の増幅度を変調信号によって変化させればよい.図\ref{fig:waveforms}は変調信号,搬送波及び被変調波である.

\begin{figure}[H]
  \centering
  % SVGファイルを挿入
  \includesvg[width=0.8\textwidth]{Wave.svg}
  \caption{各部の波形 (a) 変調信号,(b) 搬送波,(c) 被変調波}
  \label{fig:waveforms}
\end{figure}

% ===================================================================
\section{実験}
% ===================================================================
図\ref{fig:circuit}はAM変復調回路である.

\begin{figure}[H]
  \centering
  % SVGファイルを挿入
  \includesvg[width=0.9\textwidth]{config.svg}
  \caption{AM 変復調回路}
  \label{fig:circuit}
\end{figure}

\begin{enumerate}
  \item \label{exp:1} 「AM OUT」端子と「AM IN」端子を接続した.

  \item \label{exp:2} 低周波発振器を「AF IN」に接続し,\SI{3}{\kilo\hertz}の入力信号に対する図\ref{fig:circuit}の「AF IN」,「AM OUT」及び「DET OUT」端子の電圧波形をスケッチした.ただし,「AM OUT」の包絡線電圧や「DET OUT」が歪まないよう低周波発振器のAmplitudeを調整した.

  \item \label{exp:3} \SI{100}{\hertz}~\SI{50}{\kilo\hertz}の周波数範囲の「AF IN」-「AM OUT」及び「DET OUT」間の周波数特性を測定した(10ポイント以上).\\
  注:片対数のグラフ用紙を用いた.「AM OUT」については,包絡線電圧のp-p値を測定した.

  \item \label{exp:4} 入力周波数を一定(\SI{3}{\kilo\hertz})としたときの「AF IN」に対する「AM OUT」,「DET OUT」の入出力特性を測定した(10ポイント以上).\\
  注:方眼紙を用いた.「AM OUT」については,包絡線電圧のp-p値を測定した.
\end{enumerate}

% ===================================================================
\subsection{実験結果}
% ===================================================================
以下は本実験で得られた測定結果の抜粋である.単位は測定器表示の通り.

% 注意事項: プローブ補正
% -----------------------------------------------------------------------------
\paragraph{注意}
今回の測定はオシロスコープにプローブ\(\times10\)を使用して行った.そのため,表示された値はプローブの分圧により10倍(または表示がそのままの場合は実際値が表示の1/10)になっている場合がある.本稿では既存の測定結果はそのまま残し,後に補正(測定値を10で割る)した補正済みの表を示す.

以下,実験項目ごとに結果をまとめる.
\begin{itemize}
  \item 実験\ref{exp:2}: 波形のスケッチ(AF IN, AM OUT, DET OUT)
  \item 実験\ref{exp:3}: 周波数特性(AF IN - AM OUT, DET OUT)
  \item 実験\ref{exp:4}: 入出力特性(AF IN 対 AM OUT, DET OUT)
\end{itemize}

% ------------------
% 波形の図(挿入)
% ------------------
\subsubsection*{実験(2) の波形スケッチ}
\label{sec:result_exp2}
\noindent 実験\ref{exp:2} により観測した波形を以下に示す.\par
\begin{figure}[H]
  \centering
  \begin{subfigure}[b]{0.48\textwidth}
    \centering
    \includegraphics[width=\textwidth]{AFIN_pk-pk.jpg}
    \caption{AF IN (p-p)}
    \label{fig:afin_pkpk}
  \end{subfigure}
  \hfill
  \begin{subfigure}[b]{0.48\textwidth}
    \centering
    \includegraphics[width=\textwidth]{AFIN_T.jpg}
    \caption{AF IN (周期T)}
    \label{fig:afin_T}
  \end{subfigure}
  \caption{AF IN の波形(p-p と周期)}
  \label{fig:AFIN_waveforms}
\end{figure}

\begin{figure}[H]
  \centering
  \begin{subfigure}[b]{0.32\textwidth}
    \centering
    \includegraphics[width=\textwidth]{AMOUT_山ー山.jpg}
    \caption{AM OUT 山–山}
    \label{fig:amout_peakpeak}
  \end{subfigure}
  \hfill
  \begin{subfigure}[b]{0.32\textwidth}
    \centering
    \includegraphics[width=\textwidth]{AMOUT_山ー谷.jpg}
    \caption{AM OUT 山–谷}
    \label{fig:amout_peaktrough}
  \end{subfigure}
  \hfill
  \begin{subfigure}[b]{0.32\textwidth}
    \centering
    \includegraphics[width=\textwidth]{AMOUT_谷-谷.jpg}
    \caption{AM OUT 谷–谷}
    \label{fig:amout_troughtrough}
  \end{subfigure}
  \caption{AM OUT の包絡線波形(各代表測定)}
  \label{fig:AMOUT_envelope}
\end{figure}

\begin{figure}[H]
  \centering
  \begin{subfigure}[b]{0.48\textwidth}
    \centering
    \includegraphics[width=\textwidth]{DETOUT_pk-pk.jpg}
    \caption{DET OUT (p-p)}
    \label{fig:detout_pkpk}
  \end{subfigure}
  \hfill
  \begin{subfigure}[b]{0.48\textwidth}
    \centering
    \includegraphics[width=\textwidth]{DETOUT_T.jpg}
    \caption{DET OUT (周期T)}
    \label{fig:detout_T}
  \end{subfigure}
  \caption{DET OUT の波形(p-p と周期)}
  \label{fig:DETOUT_waveforms}
\end{figure}

\subsubsection*{実験(3) の周波数特性}
\label{sec:result_exp3}
\noindent 実験\ref{exp:3} により得られた周波数特性の測定結果は表および図の通りである.\par
\begin{table}[H]
  \centering
  \caption{周波数特性: 周波数に対するAM OUT(包絡線p-p)とDET OUTの測定値}
  \label{tab:freq_response}
  \begin{tabular}{rcc}
    \toprule
    周波数 (Hz) & AM OUT & DET OUT \\
    \midrule
    100  & 1.46 & 0.186 \\
    400  & 1.88 & 0.394 \\
    600  & 1.95 & 0.584 \\
    800  & 1.94 & 0.734 \\
    2{,}000 & 2.07 & 0.810 \\
    3{,}000 & 2.09 & 0.802 \\
    5{,}000 & 2.08 & 0.860 \\
    10{,}000 & 2.01 & 0.860 \\
    20{,}000 & 2.07 & 0.835 \\
    50{,}000 & 0.94 & 0.830 \\
    \bottomrule
  \end{tabular}
\end{table}

% 補正後の表: プローブ倍率 ×10 の補正 (測定値を10で割る)
\begin{table}[H]
  \centering
  \caption{周波数特性: 測定値を10で割った実際のAM OUT(包絡線p-p)とDET OUT}
  \label{tab:freq_response_corrected}
  \begin{tabular}{rcc}
    \toprule
    周波数 (Hz) & AM OUT (補正後) & DET OUT (補正後) \\
    \midrule
    100  & 0.146 & 0.0186 \\
    400  & 0.188 & 0.0394 \\
    600  & 0.195 & 0.0584 \\
    800  & 0.194 & 0.0734 \\
    2{,}000 & 0.207 & 0.0810 \\
    3{,}000 & 0.209 & 0.0802 \\
    5{,}000 & 0.208 & 0.0860 \\
    10{,}000 & 0.201 & 0.0860 \\
    20{,}000 & 0.207 & 0.0835 \\
    50{,}000 & 0.094 & 0.0830 \\
    \bottomrule
  \end{tabular}
\end{table}

% 周波数特性グラフ(片対数)
\begin{figure}[H]
  \centering
  \includesvg[width=0.9\textwidth]{freq_response.svg}
  \caption{周波数特性(補正後)のグラフ(片対数表示)}
  \label{fig:freq_response_corrected}
\end{figure}

\subsubsection*{実験(4) の入出力特性}
\label{sec:result_exp4}
\noindent 実験\ref{exp:4} により測定した入出力特性を表と図に示す.\par
\begin{table}[H]
  \centering
  \caption{入出力特性: AF INと対応する AM OUT, DET OUT の測定値 (3 kHz)}
  \label{tab:io_characteristic}
  \begin{tabular}{rcc}
    \toprule
    AF IN & AM OUT & DET OUT \\
    \midrule
    1.35 & 3.50 & 1.36 \\
    1.225 & 3.24 & 1.23 \\
    1.13 & 3.00 & 1.15 \\
    0.995 & 2.55 & 1.025 \\
    0.83 & 2.22 & 0.87 \\
    0.735 & 1.93 & 0.76 \\
    0.596 & 1.66 & 0.634 \\
    0.49 & 1.36 & 0.514 \\
    0.41 & 1.07 & 0.432 \\
    0.18 & 0.405 & 0.174 \\
    \bottomrule
  \end{tabular}
\end{table}

% 補正後の入出力特性 (測定値を10で割る)
\begin{table}[H]
  \centering
  \caption{入出力特性: AF INと対応する AM OUT, DET OUT の補正後測定値 (3 kHz)}
  \label{tab:io_characteristic_corrected}
  \begin{tabular}{rcc}
    \toprule
    AF IN & AM OUT (補正後) & DET OUT (補正後) \\
    \midrule
    0.135 & 0.350 & 0.136 \\
    0.1225 & 0.324 & 0.123 \\
    0.113 & 0.300 & 0.115 \\
    0.0995 & 0.255 & 0.1025 \\
    0.083 & 0.222 & 0.087 \\
    0.0735 & 0.193 & 0.076 \\
    0.0596 & 0.166 & 0.0634 \\
    0.049 & 0.136 & 0.0514 \\
    0.041 & 0.107 & 0.0432 \\
    0.018 & 0.0405 & 0.0174 \\
    \bottomrule
  \end{tabular}
\end{table}

% 入出力特性グラフ(通常の軸)
\begin{figure}[H]
  \centering
  \includesvg[width=0.9\textwidth]{io_characteristics.svg}
  \caption{入出力特性(補正後)のグラフ(3 kHz)}
  \label{fig:io_characteristic_corrected}
\end{figure}


% ===================================================================
\section{考察}
% ===================================================================
\begin{enumerate}
  \item 実験(2)の波形について考察する.
  \item 実験(3)の結果を考察する.
  \item 実験(4)の結果を考察する.
  \item 図\ref{fig:circuit}のAM変復調回路の各部の動作について説明する.
  \item 変調回路及び復調回路を一つずつ調べ動作を説明する.
\end{enumerate}

\subsection*{1. 実験(2) 波形の考察}
実験(2)で観測された波形(図\ref{fig:AFIN_waveforms}~図\ref{fig:DETOUT_waveforms})および測定値に基づき考察を行う.
まず,変調信号(AF IN)と復調信号(DET OUT)の周期と周波数を比較する.オシロスコープによる測定値より,
\begin{itemize}
    \item AF IN: $T = \SI{335}{\micro\second} \implies f \simeq \SI{2.99}{\kilo\hertz}$
    \item DET OUT: $T = \SI{332}{\micro\second} \implies f \simeq \SI{3.01}{\kilo\hertz}$
\end{itemize}
となり,両者の周波数は測定誤差の範囲内で一致している.また,復調波形に著しい歪みは見られず,正常に変調・復調が行われていることが確認できる.

次に,被変調波(AM OUT)の包絡線波形から変調度 $m$ を算出する.包絡線の最大振幅(山--山)を $A$,最小振幅(谷--谷)を $B$ とすると,測定値はそれぞれ $A = \SI{4.26}{\volt}$,$B = \SI{1.85}{\volt}$ であった.これより変調度 $m$ は次式で求められる.
\[
m = \frac{A - B}{A + B} = \frac{4.26 - 1.85}{4.26 + 1.85} \simeq 0.395
\]
したがって,本実験における変調度は約 39.5 \% である.$m < 1$ であるため,過変調によるクロスオーバー歪み等は発生していないことが示された.

\subsection*{2. 実験(3) 周波数特性の考察}
実験(3)で得られた周波数特性(図\ref{fig:freq_response_corrected})について考察する.
グラフより,AM OUTおよびDET OUTともに,\SI{2}{\kilo\hertz}付近から\SI{20}{\kilo\hertz}付近までは出力がほぼ一定(平坦)な特性を示している.
\begin{itemize}
    \item \textbf{低域の減衰}: \SI{1}{\kilo\hertz}以下での出力低下は,回路内の結合コンデンサ($C_1, C_{16}$等)やバイパスコンデンサによるハイパスフィルタ効果,およびトランス $T_2$ の低域特性によるものと考えられる.
    \item \textbf{高域の減衰}: \SI{20}{\kilo\hertz}以上での出力低下は,同調回路($L_2, C_{18}$)の帯域幅(Q値による選択度)による制限や,トランス $T_2$ の高域特性が影響していると考えられる.
\end{itemize}
この帯域特性は,音声信号(\SI{300}{\hertz}~\SI{3.4}{\kilo\hertz}程度)を伝送するには十分な性能を持っていると言える.

\subsection*{3. 実験(4) 入出力特性の考察}
実験(4)で得られた入出力特性(図\ref{fig:io_characteristic_corrected})について考察する.
グラフより,入力電圧(AF IN)に対して,AM OUT(包絡線振幅)およびDET OUT(復調出力)は良好な直線性を示しており,原点を通る直線に近似できる.
この結果は以下のことを示唆している.
\begin{itemize}
    \item 変調回路が入力信号の振幅に対してリニアに動作していること.
    \item 復調回路(検波ダイオード)が,二乗検波領域ではなく直線検波領域(入力振幅がダイオードの順方向電圧に対して十分大きい領域)で動作していること.
\end{itemize}
本実験の範囲内では飽和や歪みは見られず,適切なレベル配分で回路が動作していることが確認できた.

\subsection*{4. 回路動作および変復調方式の特定}
図\ref{fig:circuit}に示したAM変復調回路の各ブロックの役割,および採用されている変調・復調方式について考察する\cite{tanno1988}.

\paragraph{変調回路(送信部)}
回路図左上および下段は送信機(変調部)を構成している.
\begin{itemize}
    \item \textbf{低周波増幅 (Q3)}: 下段のトランジスタQ3は,AF INに入力された音声信号を増幅する役割を担う.増幅された信号は変調用トランスT2を介して出力される.
    \item \textbf{搬送波発振 (Q1)}: 左上のトランジスタQ1は,コイルL1やコンデンサC6, C7, C9からなる共振回路により高周波の搬送波を生成している(変形コルピッツ発振回路).
    \item \textbf{変調方式の特定(コレクタ変調)}: 
    トランスT2の2次側は,電源ライン(+12V)と発振トランジスタQ1のコレクタ回路の間に直列に挿入されている.これにより,音声信号の振幅に応じてQ1の\textbf{コレクタ電圧}が変動する.コレクタ電圧の変化は発振振幅の変化となって現れるため,結果として振幅変調(AM)が行われる.
    したがって,本回路の変調方式は\textbf{コレクタ変調}であると判断できる.ベース変調と比較して,コレクタ変調は深い変調をかけても歪みが少なく,比較的大きな電力を扱える特徴がある.
\end{itemize}

\paragraph{緩衝増幅部}
\begin{itemize}
    \item \textbf{バッファアンプ (Q2)}: 中央のFET Q2はバッファアンプ(ソースフォロワ等)として動作している.発振回路(Q1)と後段の負荷を電気的に分離することで,負荷変動が発振周波数に影響を与えることを防いでいる.
\end{itemize}

\paragraph{復調回路(受信部)}
回路図右上は受信機(復調部)を構成している.
\begin{itemize}
    \item \textbf{検波回路 (D3)}: AM OUTからの信号はダイオードD3に入力される.D3は交流信号を整流(片方向のみ通電)する役割を持つ.
    \item \textbf{平滑回路}: R20, C11, C12等により構成される回路は,整流された波形から高周波成分(搬送波)を除去し,信号波の包絡線(エンベロープ)のみを取り出すローパスフィルタの役割を果たしている.
    \item \textbf{復調方式の特定(直線復調)}: 
    この構成は,ダイオードの整流作用とCRフィルタを用いて包絡線を取り出すことから,包絡線検波回路である.ダイオードに十分な振幅の信号を入力した場合,その動作は直線領域で行われるため,方式としては\textbf{直線復調}(直線検波)に分類される\cite{tanno1988}.
    二乗復調は微小信号入力時にダイオードの二乗特性領域を利用するものであるが,本実験のような信号レベルおよび回路構成(一般的なAMラジオ検波段)においては,歪みの少ない直線復調として動作させることを意図している.
\end{itemize}

% ===================================================================
\section*{参考文献}
% ===================================================================
\begin{thebibliography}{9}
\bibitem{tanno1988}
丹野頼元『森北電気工学シリーズ2 電子回路(第2版)』森北出版株式会社, 1988年, pp.253--265.
\end{thebibliography}

\end{document}