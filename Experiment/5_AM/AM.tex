% !TEX program = lualatex
% !TEX options = --shell-escape
%==============================================================================
% プリアンブル (Preamble)
%==============================================================================

% ===== ドキュメントクラス =====
\documentclass[
  a4paper,
  11pt,
]{ltjsarticle}

%------------------------------------------------------------------------------
% パッケージ読み込み
%------------------------------------------------------------------------------

% ===== フォント・言語設定 =====
\usepackage{luatexja-fontspec} 

% ===== レイアウト関連 =====
\usepackage[margin=2.5cm]{geometry} 
\usepackage{graphicx}          
% Set graphics path to image folder
\graphicspath{{image/}}

% ===== SVG関連 (重要) =====
% Inkscapeがインストールされ、PATHが通っている必要があります。
\usepackage[inkscapelatex=false]{svg} 

\usepackage{booktabs}          
\usepackage{float}             
\usepackage{wrapfig}           

% ===== 数式・物理単位関連 =====
\usepackage{amsmath}           
\usepackage{amsthm}            
\usepackage{newtxmath}         
\usepackage{siunitx}           
\usepackage{cancel}            

% ===== 図表・グラフ描画関連 =====
\usepackage{tikz}
\usepackage{circuitikz}        
\usepackage{pgfplots}          
\usepackage{pgfplotstable}     
\pgfplotsset{compat=1.18}      
\usepgfplotslibrary{statistics} 
\usetikzlibrary{positioning}   

% ===== プログラミング・アルゴリズム関連 =====
\usepackage{listings}          
\usepackage{algorithm}         
\usepackage{algpseudocode}     
% Enumerate style
\usepackage{enumitem}
% Use parenthesized numbers like (1), (2) for top-level enumerate
\setlist[enumerate,1]{label=(\arabic*)}
% Subfigure package for arranging multiple images
\usepackage{subcaption}

% ===== その他 =====
\usepackage[
  colorlinks=true,      
  linkcolor=blue,         
  citecolor=green!60!black, 
  urlcolor=cyan,          
  hidelinks,              
]{hyperref}
% 文献番号を本文中で上付きの[数字]形式にする
\usepackage{cite}
\makeatletter
% 本文中引用: ^[1] 形式(複数: ^[1--3,5])
\def\@cite#1#2{$^{[#1\if@tempswa , #2\fi]}$}
% 参考文献一覧ラベル: [1](上付きにしない)
\def\@biblabel#1{[#1]}
\makeatother

%------------------------------------------------------------------------------
% 各種設定
%------------------------------------------------------------------------------

% ===== フォント設定 =====
\setmainfont{Latin Modern Roman}
\setsansfont{Latin Modern Sans}
\setmonofont{Latin Modern Mono}
\setmainjfont[Renderer=HarfBuzz]{Yu Mincho}
\setsansjfont[Renderer=HarfBuzz]{Yu Gothic}

% ===== ドキュメント情報 =====
\title{5. AM}
\author{}
\date{}

% ===== listings 設定 =====
\lstset{
  language=Python,
  basicstyle=\small\ttfamily,
  keywordstyle=\color{blue},
  commentstyle=\color{green!50!black},
  stringstyle=\color{purple},
  showstringspaces=false,
  frame=tb,
  captionpos=b,
  breaklines=true,
  numbers=left,
  numberstyle=\tiny\color{gray},
  xleftmargin=2em, 
  framexleftmargin=1.5em, 
}

% ===== pgfplots 設定 =====
\pgfplotsset{
  report-style/.style={
    xlabel style={yshift=0.5em}, 
    ylabel style={yshift=-0.5em},
    legend pos=north west,
    grid=major,
    ticklabel style={font=\small},
    label style={font=\small},
    legend style={font=\small},
  }
}

% ===== 数式用カスタムコマンド =====
\newcommand{\dd}{\mathrm{d}} 
\newcommand{\mi}{\mathrm{j}} 

% ===== algorithmicx 設定 =====
\renewcommand{\textproc}[1]{\textbf{#1}}

%==============================================================================
% ドキュメント本体 (Body)
%==============================================================================
\begin{document}

\maketitle

% ===================================================================
\section{目的}
% ===================================================================
AMと復調の原理及びそれを実現する回路の動作を理解する.

% ===================================================================
\section{AMの変調・復調の原理}
% ===================================================================
無線通信では,高い周波数の電磁波を用いなければ,電波の放射が能率よく行われない.そこで,音声のような情報信号をどのようにして高周波にのせるかということが問題となる.その一つの方法がAM変調であり,高周波(搬送波)の振幅を情報信号(変調波)で変化させる方法である.

いま,搬送波 $v_c$ が
\begin{equation}
  v_c = V_c \cos \omega_c t
  \label{eq:carrier}
\end{equation}
で表される正弦波とし,これを
\begin{equation}
  v_s = V_s \cos \omega_s t
  \label{eq:signal}
\end{equation}
で表される信号波(変調波)によって振幅変調する場合を考える.振幅変調は,搬送波の振幅が変調波によって変化する方式なので,被搬送波の振幅は
\begin{equation}
  V_c + k_a V_s \cos \omega_s t
  \label{eq:amplitude}
\end{equation}
となり,振幅が時間によって変化する.ここに,$k_a$ は比例定数である.したがって,被変調波 $v$ は,次のようになる.
\begin{equation}
  v = (V_c + k_a V_s \cos \omega_s t) \cos \omega_c t = V_c(1 + m_a \cos \omega_s t) \cos \omega_c t
  \label{eq:modulated_wave}
\end{equation}
ここで,
\begin{equation}
  m_a = k_a V_s / V_c
  \label{eq:modulation_index}
\end{equation}
であり,これを変調度,または百分率で表して変調率という.

復調は,被変調波を整流または2乗し,それに含まれる低周波成分を取り出すことによってなされる.
振幅変調は,搬送波増幅器の増幅度を変調信号によって変化させればよい.図\ref{fig:waveforms}は変調信号,搬送波及び被変調波である.

\begin{figure}[H]
  \centering
  % SVGファイルを挿入
  \includesvg[width=0.8\textwidth]{Wave.svg}
  \caption{各部の波形 (a) 変調信号,(b) 搬送波,(c) 被変調波}
  \label{fig:waveforms}
\end{figure}

% ===================================================================
\section{実験}
% ===================================================================
図\ref{fig:circuit}はAM変復調回路である.

\begin{figure}[H]
  \centering
  % SVGファイルを挿入
  \includesvg[width=0.9\textwidth]{config.svg}
  \caption{AM 変復調回路}
  \label{fig:circuit}
\end{figure}

\begin{enumerate}
  \item \label{exp:1} 「AM OUT」端子と「AM IN」端子を接続した.

  \item \label{exp:2} 低周波発振器を「AF IN」に接続し,\SI{3}{\kilo\hertz}の入力信号に対する図\ref{fig:circuit}の「AF IN」,「AM OUT」及び「DET OUT」端子の電圧波形をスケッチした.ただし,「AM OUT」の包絡線電圧や「DET OUT」が歪まないよう低周波発振器のAmplitudeを調整した.

  \item \label{exp:3} \SI{100}{\hertz}~\SI{50}{\kilo\hertz}の周波数範囲の「AF IN」-「AM OUT」及び「DET OUT」間の周波数特性を測定した(10ポイント以上).\\
  注:片対数のグラフ用紙を用いた.「AM OUT」については,包絡線電圧のp-p値を測定した.

  \item \label{exp:4} 入力周波数を一定(\SI{3}{\kilo\hertz})としたときの「AF IN」に対する「AM OUT」,「DET OUT」の入出力特性を測定した(10ポイント以上).\\
  注:方眼紙を用いた.「AM OUT」については,包絡線電圧のp-p値を測定した.
\end{enumerate}

% ===================================================================
\subsection{実験結果}
% ===================================================================
以下は本実験で得られた測定結果の抜粋である.単位は測定器表示の通り.

% 注意事項: プローブ補正
% -----------------------------------------------------------------------------
\paragraph{注意}
今回の測定はオシロスコープにプローブ\(\times10\)を使用して行った.そのため,表示された値はプローブの分圧により10倍(または表示がそのままの場合は実際値が表示の1/10)になっている場合がある.本稿では既存の測定結果はそのまま残し,後に補正(測定値を10で割る)した補正済みの表を示す.

以下,実験項目ごとに結果をまとめる.
\begin{itemize}
  \item 実験\ref{exp:2}: 波形のスケッチ(AF IN, AM OUT, DET OUT)
  \item 実験\ref{exp:3}: 周波数特性(AF IN - AM OUT, DET OUT)
  \item 実験\ref{exp:4}: 入出力特性(AF IN 対 AM OUT, DET OUT)
\end{itemize}

% ------------------
% 波形の図(挿入)
% ------------------
\subsubsection*{実験(2) の波形スケッチ}
\label{sec:result_exp2}
\noindent 実験\ref{exp:2} により観測した波形を以下に示す.\par
\begin{figure}[H]
  \centering
  \begin{subfigure}[b]{0.48\textwidth}
    \centering
    \includegraphics[width=\textwidth]{AFIN_pk-pk.jpg}
    \caption{AF IN (p-p)}
    \label{fig:afin_pkpk}
  \end{subfigure}
  \hfill
  \begin{subfigure}[b]{0.48\textwidth}
    \centering
    \includegraphics[width=\textwidth]{AFIN_T.jpg}
    \caption{AF IN (周期T)}
    \label{fig:afin_T}
  \end{subfigure}
  \caption{AF IN の波形(p-p と周期)}
  \label{fig:AFIN_waveforms}
\end{figure}

\begin{figure}[H]
  \centering
  \begin{subfigure}[b]{0.32\textwidth}
    \centering
    \includegraphics[width=\textwidth]{AMOUT_山ー山.jpg}
    \caption{AM OUT 山–山}
    \label{fig:amout_peakpeak}
  \end{subfigure}
  \hfill
  \begin{subfigure}[b]{0.32\textwidth}
    \centering
    \includegraphics[width=\textwidth]{AMOUT_山ー谷.jpg}
    \caption{AM OUT 山–谷}
    \label{fig:amout_peaktrough}
  \end{subfigure}
  \hfill
  \begin{subfigure}[b]{0.32\textwidth}
    \centering
    \includegraphics[width=\textwidth]{AMOUT_谷-谷.jpg}
    \caption{AM OUT 谷–谷}
    \label{fig:amout_troughtrough}
  \end{subfigure}
  \caption{AM OUT の包絡線波形(各代表測定)}
  \label{fig:AMOUT_envelope}
\end{figure}

\begin{figure}[H]
  \centering
  \begin{subfigure}[b]{0.48\textwidth}
    \centering
    \includegraphics[width=\textwidth]{DETOUT_pk-pk.jpg}
    \caption{DET OUT (p-p)}
    \label{fig:detout_pkpk}
  \end{subfigure}
  \hfill
  \begin{subfigure}[b]{0.48\textwidth}
    \centering
    \includegraphics[width=\textwidth]{DETOUT_T.jpg}
    \caption{DET OUT (周期T)}
    \label{fig:detout_T}
  \end{subfigure}
  \caption{DET OUT の波形(p-p と周期)}
  \label{fig:DETOUT_waveforms}
\end{figure}

\subsubsection*{実験(3) の周波数特性}
\label{sec:result_exp3}
\noindent 実験\ref{exp:3} により得られた周波数特性の測定結果は,表\ref{tab:freq_response}および表\ref{tab:freq_response_corrected},並びに図\ref{fig:freq_response_corrected}に示す.\par
\begin{table}[H]
  \centering
  \caption{周波数特性: 周波数に対するAM OUT(包絡線p-p)とDET OUTの測定値}
  \label{tab:freq_response}
  \begin{tabular}{rcc}
    \toprule
    周波数 (Hz) & AM OUT & DET OUT \\
    \midrule
    100  & 1.46 & 0.186 \\
    400  & 1.88 & 0.394 \\
    600  & 1.95 & 0.584 \\
    800  & 1.94 & 0.734 \\
    2{,}000 & 2.07 & 0.810 \\
    3{,}000 & 2.09 & 0.802 \\
    5{,}000 & 2.08 & 0.860 \\
    10{,}000 & 2.01 & 0.860 \\
    20{,}000 & 2.07 & 0.835 \\
    50{,}000 & 0.94 & 0.830 \\
    \bottomrule
  \end{tabular}
\end{table}

% 補正後の表: プローブ倍率 ×10 の補正 (測定値を10で割る)
\begin{table}[H]
  \centering
  \caption{周波数特性: 測定値を10で割った実際のAM OUT(包絡線p-p)とDET OUT}
  \label{tab:freq_response_corrected}
  \begin{tabular}{rcc}
    \toprule
    周波数 (Hz) & AM OUT (補正後) & DET OUT (補正後) \\
    \midrule
    100  & 0.146 & 0.0186 \\
    400  & 0.188 & 0.0394 \\
    600  & 0.195 & 0.0584 \\
    800  & 0.194 & 0.0734 \\
    2{,}000 & 0.207 & 0.0810 \\
    3{,}000 & 0.209 & 0.0802 \\
    5{,}000 & 0.208 & 0.0860 \\
    10{,}000 & 0.201 & 0.0860 \\
    20{,}000 & 0.207 & 0.0835 \\
    50{,}000 & 0.094 & 0.0830 \\
    \bottomrule
  \end{tabular}
\end{table}

% 周波数特性グラフ(片対数)
\begin{figure}[H]
  \centering
  \includesvg[width=0.9\textwidth]{freq_response.svg}
  \caption{周波数特性(補正後)のグラフ(片対数表示)}
  \label{fig:freq_response_corrected}
\end{figure}

\subsubsection*{実験(4) の入出力特性}
\label{sec:result_exp4}
\noindent 実験\ref{exp:4} により測定した入出力特性は,表\ref{tab:io_characteristic}および表\ref{tab:io_characteristic_corrected},並びに図\ref{fig:io_characteristic_corrected}に示す.\par
\begin{table}[H]
  \centering
  \caption{入出力特性: AF INと対応する AM OUT, DET OUT の測定値 (3 kHz)}
  \label{tab:io_characteristic}
  \begin{tabular}{rcc}
    \toprule
    AF IN & AM OUT & DET OUT \\
    \midrule
    1.35 & 3.50 & 1.36 \\
    1.225 & 3.24 & 1.23 \\
    1.13 & 3.00 & 1.15 \\
    0.995 & 2.55 & 1.025 \\
    0.83 & 2.22 & 0.87 \\
    0.735 & 1.93 & 0.76 \\
    0.596 & 1.66 & 0.634 \\
    0.49 & 1.36 & 0.514 \\
    0.41 & 1.07 & 0.432 \\
    0.18 & 0.405 & 0.174 \\
    \bottomrule
  \end{tabular}
\end{table}

% 補正後の入出力特性 (測定値を10で割る)
\begin{table}[H]
  \centering
  \caption{入出力特性: AF INと対応する AM OUT, DET OUT の補正後測定値 (3 kHz)}
  \label{tab:io_characteristic_corrected}
  \begin{tabular}{rcc}
    \toprule
    AF IN & AM OUT (補正後) & DET OUT (補正後) \\
    \midrule
    0.135 & 0.350 & 0.136 \\
    0.1225 & 0.324 & 0.123 \\
    0.113 & 0.300 & 0.115 \\
    0.0995 & 0.255 & 0.1025 \\
    0.083 & 0.222 & 0.087 \\
    0.0735 & 0.193 & 0.076 \\
    0.0596 & 0.166 & 0.0634 \\
    0.049 & 0.136 & 0.0514 \\
    0.041 & 0.107 & 0.0432 \\
    0.018 & 0.0405 & 0.0174 \\
    \bottomrule
  \end{tabular}
\end{table}

% 入出力特性グラフ(通常の軸)
\begin{figure}[H]
  \centering
  \includesvg[width=0.9\textwidth]{io_characteristics.svg}
  \caption{入出力特性(補正後)のグラフ(3 kHz)}
  \label{fig:io_characteristic_corrected}
\end{figure}


% ===================================================================
\section{考察}
% ===================================================================
\begin{enumerate}
  \item 実験(2)の波形について考察する.
  \item 実験(3)の結果を考察する.
  \item 実験(4)の結果を考察する.
  \item 図\ref{fig:circuit}のAM変復調回路の各部の動作について説明する.
  \item 変調回路及び復調回路を一つずつ調べ動作を説明する.
\end{enumerate}

\subsection*{1. 実験(2) 波形の考察}
実験(2)の波形(図\ref{fig:AFIN_waveforms}~図\ref{fig:DETOUT_waveforms})を簡潔にまとめる.AF IN と DET OUT の周期はそれぞれ $T=\SI{335}{\micro\second}$, $T=\SI{332}{\micro\second}$ である.周波数は
\[
 f=\frac{1}{T}
\]
より,AF IN で $f\simeq\SI{2.99}{\kilo\hertz}$,DET OUT で $f\simeq\SI{3.01}{\kilo\hertz}$ と求まる.復調波形に顕著な歪みはない.なお,DETOUT の振幅がやや大きめに観測される傾向があったが,その原因は特定できなかった.包絡線の最大値 $A=\SI{4.26}{\volt}$,最小値 $B=\SI{1.85}{\volt}$ から
また,DETOUT には搬送波(高周波)の成分がわずかに乗っている様子が観測された.
\[
 m=\frac{A-B}{A+B}=\frac{4.26-1.85}{4.26+1.85}\simeq 0.395
\]
変調度は約 40\% で,過変調は生じていない.\cite{tanno1988}

\subsection*{2. 実験(3) 周波数特性の考察}
図\ref{fig:freq_response_corrected} では \SI{2}{\kilo\hertz}~\SI{20}{\kilo\hertz} でほぼ一定出力を示した。低域の減少は結合/バイパス用コンデンサで低周波が通りにくいため,高域低下は同調回路とトランスの限界による。音声帯域 (約 \SI{300}{\hertz}~\SI{3.4}{\kilo\hertz}) を扱うには十分平坦である。なお,理論的には高周波帯で減衰が期待されるが,\SI{50}{\kilo\hertz} においても DET OUT がほぼ一定に保たれた点は測定誤差の影響と考えられる。\cite{tanno1988}

\subsection*{3. 実験(4) 入出力特性の考察}
図\ref{fig:io_characteristic_corrected} では AF IN に対し AM OUT と DET OUT がほぼ比例し原点近くを通る。この線形性は,変調部が入力振幅に比例して搬送波の振幅を変化させる線形変調器として動作し,復調側のダイオードも十分な振幅領域で大信号動作(包絡線検波)を行っているためである。飽和や顕著な歪みは観測されず,設計通りの線形入出力特性が確認された。\cite{tanno1988}

\subsection*{4. 回路動作および変復調方式の特定}
図\ref{fig:circuit} の各部の主な働きと方式を簡潔に示す\cite{tanno1988}。

\paragraph{変調部}
\begin{itemize}
  \item Q3: 音声信号(AF IN)を増幅しトランス T2 へ送る。
  \item Q1: LC 回路で搬送波を発振。
  \item 方式: T2 が電源と Q1 コレクタの間に入りコレクタ電圧が音声で揺れ振幅が変化するため AM (コレクタ変調)。深い変調でも歪みが少ない。
\end{itemize}

\paragraph{緩衝部}
\begin{itemize}
  \item Q2: バッファ。発振部を負荷変動から守る。
\end{itemize}

\paragraph{復調部}
\begin{itemize}
  \item D3: 整流して包絡線を得る第一段。
  \item R20, C11, C12 など: 平滑し高周波成分を除き包絡線のみ取り出す。
  \item 方式: 十分な振幅で動作する包絡線検波で直線復調。微小信号用の二乗復調ではない。
\end{itemize}

% ===================================================================

% ===================================================================
\begin{thebibliography}{9}
\bibitem{tanno1988}
丹野頼元『森北電気工学シリーズ2 電子回路(第2版)』森北出版株式会社, 1988年, pp.253--265.
\end{thebibliography}

\end{document}