% ===== ドキュメントクラスと基本的なパッケージ =====
\documentclass[
  a4paper,  % 用紙サイズ
  11pt,     % フォントサイズ
]{ltjsarticle}
\usepackage{newtxtext, newtxmath} % Times系のフォント・数式
\usepackage{amsmath,amssymb}   % 数式
\usepackage{graphicx}          % 画像の挿入
\usepackage{siunitx}           % 国際単位系(SI)
\usepackage{float}             % 図表の位置調整
\usepackage[margin=2.5cm]{geometry} % 余白の設定
\usepackage{hyperref}          % PDF内リンク

% ===== ドキュメント情報 =====
\title{高周波線路の電流分布特性}
\author{氏名}
\date{\today}

\begin{document}

\maketitle
\tableofcontents
\clearpage

% ===================================================================
\section{目的}
% ===================================================================
高周波帯における信号伝送特性を明確にするため,プリント線路上の磁界測定から電流分布を推定し,定在波の分布とその影響を考察する.

% ===================================================================
\section{高周波帯における線路上の電流分布}
% ===================================================================

\subsection{分布定数回路における電圧と電流}
高周波を扱う回路では,信号波長の長さに応じて集中定数回路と分布定数回路に分ける.信号の周波数を $f$,伝搬速度を $v$ とすると,波長 $\lambda$ は次式で与えられる.
\begin{equation}
  \lambda = \frac{v}{f}
\end{equation}
波長 $\lambda$ より十分に短い線路は集中定数回路,波長 $\lambda$ と同程度かそれより短い線路は分布定数回路として扱う.

伝送線路には平行銅線や同軸ケーブルなどがあり,導体には抵抗やインダクタ成分といった寄生成分が存在する.導体間にはキャパシタ成分や微小な漏れ電流も考えられる.単位長さ当たりの抵抗 $R$,インダクタンス $L$,線間のキャパシタ $C$,コンダクタンス $G$ を用いて微小区間ごとの寄生成分とそれに伴う電圧変化を考察する.図\ref{fig:distributed_constant_circuit}に分布定数線路の等価回路を示す.ここで $v_s$ と $\SI{50}{\ohm}$ は入力電圧および入力抵抗,$Z_L$ は終端負荷である.

\begin{figure}[H]
  \centering
  % ここに分布定数線路の等価回路図を挿入 (例: circuitikz)
  \caption{分布定数線路の等価回路}
  \label{fig:distributed_constant_circuit}
\end{figure}

伝送線路の伝搬定数 $\gamma$ は
\begin{equation}
  \gamma = \sqrt{ZY} = \sqrt{(R+j\omega L)(G+j\omega C)}
\end{equation}
で与えられる.一方,特性インピーダンス $Z_0$ は
\begin{equation}
  Z_0 = \sqrt{\frac{R+j\omega L}{G+j\omega C}}
\end{equation}
で求められ,無損失線路では $L$ と $C$ のみで近似される.

伝送線路を伝搬する信号は進行波と反射波の重ね合わせで表現できる.距離 $x$ における電圧と電流は次式の通りである.
\begin{align}
  V(x) &= \frac{V_1 - Z_0 I_1}{2}e^{+\gamma x} + \frac{V_1 + Z_0 I_1}{2}e^{-\gamma x} \\
  I(x) &= \frac{V_1 - Z_0 I_1}{2Z_0}e^{+\gamma x} - \frac{V_1 + Z_0 I_1}{2Z_0}e^{-\gamma x}
\end{align}
ここで $e^{+\gamma x}$ の項が進行波,$e^{-\gamma x}$ の項が反射波に対応する.双曲線関数の定義
\begin{equation}
  \cosh(\gamma x) = \frac{e^{\gamma x} + e^{-\gamma x}}{2}, \quad \sinh(\gamma x) = \frac{e^{\gamma x} - e^{-\gamma x}}{2}
\end{equation}
を用いると,基礎伝送行列は次のように表される.
\begin{equation}
  \begin{pmatrix} V(x) \\ I(x) \end{pmatrix} =
  \begin{pmatrix} \cosh(\gamma x) & -Z_0 \sinh(\gamma x) \\ -\frac{1}{Z_0}\sinh(\gamma x) & \cosh(\gamma x) \end{pmatrix}
  \begin{pmatrix} V_1 \\ I_1 \end{pmatrix}
\end{equation}

線路終端で特性インピーダンス $Z_0$ と終端負荷 $Z_L$ が不整合の場合,進行波の一部が反射される.反射係数 $\Gamma_L$ は
\begin{equation}
  \Gamma_L = \frac{v_0^-}{v_0^+} = \frac{Z_L - Z_0}{Z_L + Z_0}
\end{equation}
で与えられる.これを用いると,距離 $x$ における電圧および電流の振幅は次式となる.
\begin{align}
  |V(x)| &= |v_i^+| \left|1 + |\Gamma_L| e^{-j(\phi+2\beta x)}\right| \\
  |I(x)| &= \frac{|v_i^+|}{|Z_0|} \left|1 - |\Gamma_L| e^{-j(\phi+2\beta x)}\right|
\end{align}
終端条件により反射波の位相と大きさが決まり,開放終端では $\Gamma_L = +1$(同相反射),短絡終端では $\Gamma_L = -1$(逆相反射)となる.進行波と反射波の重ね合わせにより定在波が形成される.

特性インピーダンスとの不整合は負荷へのエネルギー供給効率低下を招き,定在波はコモンモードによる放射ノイズを発生させる場合があるため,MHz〜GHz 帯での対策が重要である.

\subsection{マイクロストリップ線路}
プリント基板を用いた高周波回路では,裏面が全面銅箔(GND)で表面に信号線が配置された構造が一般的であり,これをマイクロストリップ線路と呼ぶ.マイクロストリップ線路の特性インピーダンス $Z_0$ は近似式
\begin{equation}
  Z_0 = \frac{87.0}{\sqrt{\epsilon_r + 1.41}} \ln\left(\frac{5.98H}{0.8W + T}\right)
\end{equation}
で求められる [1].ここで $W$ は線路幅,$H$ は基板厚さ,$T$ は銅箔厚さ,$\epsilon_r$ は基板の比誘電率(公称 4.4)である.

\subsection{近磁界プローブ}
プリント基板上の電流分布を直接測定することは困難であるため,近磁界プローブを用いて線路上の磁界を測定し,電流分布を推定する手法が用いられる.

磁界によりプローブ端に誘起される起電力をスペクトラムアナライザで検出し,得られた検出電圧 $V$ をアンテナ係数 $AF$ により磁界 $H$ に換算する.
\begin{equation}
  H = AF \cdot V \quad [\mathrm{A/m}]
\end{equation}
測定結果をデシベル表記で扱う場合は,次のように表す.
\begin{equation}
  H_{\mathrm{dB}} = AF + V_{\mathrm{dB}} \quad [\mathrm{dB\mu A/m}]
\end{equation}
検出電圧のデシベル換算は次式で与えられる.
\begin{equation}
  \mathrm{dB\mu V} = 20 \log_{10}\left(\frac{V}{1\ \mu\mathrm{V}}\right)
\end{equation}

% ===================================================================
\section{実験方法}
% ===================================================================
\begin{enumerate}
  \item 使用機器を準備する.\
    スペクトラムアナライザ,磁界プローブ(NEC: CP-2S),3軸ステージ,プリント基板,同軸ケーブル(×2)
  \item 3軸ステージの同軸ケーブルに磁界プローブを取り付け,同軸ケーブルのもう一端をスペクトラムアナライザのRFIN に接続する.
  \item 3軸ステージ上にプリント基板(開放)を設置し,プリント基板の SMA コネクタとスペクトラムアナライザの TG を同軸ケーブルで接続する.
  \item スペクトラムアナライザの設定を以下とする.\
    Center Frequency: 1 GHz, Span: 10 MHz, BW: 10 kHz, Sweep: auto\
    Reference: (初期確認)107 dBμV、(測定時)87 dBμV、Pre amp. ON、Attenuation: 0 dB、TG 出力: 107 dBμV.
  \item 磁界プローブをプリント線路の終端位置に合わせ,プローブ先端が基板上 1 mm となるようセットする.3軸ステージの x 軸は,測定範囲確保のためあらかじめ 100 mm だけ移動させておく.プリント線路の両端は可能な限り同一高さ(水平)に揃え,ずれは 0.1 mm 以下とする.
  \item 5 mm 間隔で磁界データを取得する.定在波の節付近はより細かくサンプリングする.
  \item プリント基板を短絡,整合,負荷の各条件に交換し,同様の測定を実施する.
  \item 測定値にアンテナ係数 AF を乗じて磁界に変換する.
\end{enumerate}
ここで校正係数は次式で与えられる [2].
\begin{equation}
  AF = -8.1 \ln f + 106.2
\end{equation}
(注)f は MHz 単位で計算する.例:1 GHz は f = 1000 として計算する.

% ===================================================================
\section{報告事項}
% ===================================================================
\begin{enumerate}
  \item 測定結果から波長 $\lambda$ および伝搬速度 $v$ を算出せよ.伝搬速度が光速より小さくなる理由を述べよ.
  \item 負荷条件の測定結果から定在波比(SWR)を求めよ.
  \item 反射係数を算出し,接続した負荷抵抗値を求めよ.
  \item 終端を開放および短絡した線路の入力インピーダンスを導出せよ.両者の積により線路の特性インピーダンスが得られることを示せ.
\end{enumerate}

% ===================================================================
\begin{thebibliography}{9}
% ===================================================================
\bibitem{Montrose1999} Mark I. Montrose 著, 出口・田上 共訳, プリント基板の EMC 設計, p.95, オーム社.
\bibitem{NEC2000} 磁界プローブ CP-2S 取扱説明書, NEC.
\end{thebibliography}

\end{document}
