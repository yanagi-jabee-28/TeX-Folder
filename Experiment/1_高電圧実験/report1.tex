\documentclass[
  a4paper,
  11pt,
]{ltjsarticle}

% ===== 基本パッケージ =====
\usepackage[margin=2.5cm]{geometry} % 余白の設定
\usepackage{luatexja-fontspec} % lualatex用日本語フォント設定
\usepackage{newtxtext, newtxmath} % Times系のフォント・数式パッケージ
\usepackage{amsmath,amssymb}   % 数式
\usepackage{graphicx}          % 画像の挿入
\usepackage{tikz}             % 図作成
\usepackage{pgfplots}         % グラフ作成
\pgfplotsset{compat=1.18}
\usepackage{siunitx}           % 国際単位系(SI)
\usepackage{float}             % 図表の位置調整
\usepackage{booktabs}          % 見栄えの良い表
\usepackage[hidelinks]{hyperref} % PDF内リンク

% ===== ドキュメント情報 =====
\title{高電圧実験レポート}
\author{}
\date{}

\begin{document}

\maketitle

\section{目的}

気体,液体,固体の三態の絶縁体について破壊試験を行い,高電圧工学についての理解を深める.気体絶縁体として空気を用い,絶縁距離と破壊電圧の関係を調べる.液体絶縁体としては絶縁油,固体絶縁体としてはベークライトを用いて絶縁破壊試験を行い,三態の絶縁体の破壊電圧を比較する.更に,碍子に雷インパルス電圧を印加して碍子表面の空気の破壊電圧を調べる.

\section{実験結果と報告事項}

本実験は,気中直流火花放電特性,液体および固体の絶縁破壊試験,碍子のインパルス電圧特性の3つの課題から構成される.気中直流火花放電特性と液体の絶縁破壊試験は直流電圧を印加し,固体の絶縁破壊試験では交流電圧を印加した.実験を通じて共通項目として,使用機器を表\ref{tab:equipment}に,気象条件を表\ref{tab:weather}に示す.表\ref{tab:weather}には実験式で用いる相対空気密度も示した.3つの課題の結果と報告事項は課題毎に後述する.

\subsection{気中直流火花放電特性}

本節では,気体絶縁体である空気の絶縁破壊試験の結果を報告する.形状が異なる電極を用いてギャップ長と破壊電圧の関係と,針-平板電極を用いた極性効果についての結果を確認する.

\begin{table}[H]
  \centering
  \caption{使用機器}
  \label{tab:equipment}
  \begin{tabular}{llll}
    \toprule
    機器名 & メーカ名 & 型番 & シリアル \\
    \midrule
    高電圧試験装置 & パルス電子技術株式会社 & CR-200F-32 & 5477 \\
    試験用変圧器 & 同上 & TT-50K6U & 5477 \\
    直流接地装置 & 同上 & DDC-50K & 5477 \\
    標準球ギャップ & 同上 & G-100H & 5477 \\
    気中/油中試験装置 & 同上 & ZSK-50W & 5477 \\
    インパルス電圧発生装置 & 同上 & IVG-200AS & 5477 \\
    懸垂がいし & 同上 & A-S-1527AB JIS 記号: 180EP & \\
    高電圧プローブ & 同上 & EP-100K & \\
    オシロスコープ & Tektronix & TDS 3032 & B012774 \\
    デジタルテスタ & オーム電機 & TST-KJ380 &  \\
    \bottomrule
  \end{tabular}
\end{table}

\begin{table}[H]
  \centering
  \caption{気象条件}
  \label{tab:weather}
  \begin{tabular}{cccc}
    \toprule
    気温 [\si{\celsius}] & 気圧 [\si{\hecto\pascal}] & 相対空気密度 & 相対空気密度(針-針) \\
    \midrule
    24 & 986 & 0.959 & 0.976 \\
    \bottomrule
  \end{tabular}
\end{table}

\subsubsection{形状が異なる電極のギャップ長と破壊電圧}

\paragraph{a) 実験結果}
平板-平板電極の結果を表\ref{tab:plate}に,球-球電極(球半径 \SI{5}{\centi\meter})の結果を表\ref{tab:sphere}に,針-針電極の結果を表\ref{tab:needle}に示す.更に,表\ref{tab:plate}〜\ref{tab:needle}を比較するため1つにまとめたグラフを図\ref{fig:gap-comparison}に示す.

% --- 図1: 3種類の電極の実験式と実測値を重ねたグラフ (pgfplots) ---
\begin{figure}[H]
  \centering
  \begin{tikzpicture}
    \begin{axis}[
      width=0.85\linewidth,
      height=7cm,
      xlabel={ギャップ長 [cm]},
      ylabel={破壊電圧 [kV]},
      legend pos=south east,
      legend style={font=\small},
      grid=major,
      mark size=3pt,
      xmin=0, xmax=5.5,
      ymin=0
    ]

    % --- データテーブル定義 (x y) ---
    % --- 平板-平板: 計算値 (点) ---
    \addplot[only marks, mark=o, mark options={fill=black,draw=black}] coordinates { (0.4,14.0) (0.7,22.4) (1.0,30.6) (1.3,38.5) (1.6,46.3) };
    \addlegendentry{平板-平板 実験式}
    % 平板-平板 回帰直線: y = 26.9 x + 3.46
    \addplot[black, thick, mark=none, domain=0.2:1.8,samples=2] {26.9*x + 3.46};

    % --- 平板-平板: 実測値 (点) ---
    \addplot[only marks, mark=o, mark options={fill=white,draw=black}] coordinates { (0.4,13.3) (0.7,21.8) (1.0,31.0) (1.3,38.1) (1.6,46.4) };
    \addlegendentry{平板-平板 実測値}
    % 平板-平板 実測 回帰直線: y = 27.5 x + 2.62
    \addplot[black, thick, mark=none, domain=0.2:1.8,samples=2] {27.5*x + 2.62};

    % --- 球-球: 計算値 (点) ---
    \addplot[only marks, mark=square*, mark options={fill=black,draw=black}] coordinates { (0.4,13.0) (0.7,22.2) (1.0,31.2) (1.3,39.8) (1.6,48.0) };
    \addlegendentry{球-球 実験式}
    % 球-球 回帰直線: y = 29.2 x + 1.64
    \addplot[black, thick, mark=none, domain=0.2:1.8,samples=2] {29.2*x + 1.64};

    % --- 球-球: 実測値 (点) ---
    \addplot[only marks, mark=square, mark options={fill=white,draw=black}] coordinates { (0.4,13.8) (0.7,22.6) (1.0,32.4) (1.3,41.0) (1.6,48.2) };
    \addlegendentry{球-球 実測値}
    % 球-球 実測 回帰直線: y = 29.0667 x + 2.5333
    \addplot[black, thick, mark=none, domain=0.2:1.8,samples=2] {29.0667*x + 2.5333};

    % --- 針-針: 計算値 (点) ---
    \addplot[only marks, mark=diamond*, mark options={fill=black,draw=black}] coordinates { (1.0,22.8) (2.0,27.7) (3.0,32.6) (4.0,37.5) (5.0,42.4) };
    \addlegendentry{針-針 実験式}
    % 針-針 回帰直線: y = 4.9 x + 17.9
    \addplot[black, thick, mark=none, domain=0.8:5.2,samples=2] {4.9*x + 17.9};

    % --- 針-針: 実測値 (点) ---
    \addplot[only marks, mark=diamond, mark options={fill=white,draw=black}] coordinates { (1.0,8.7) (2.0,21.8) (3.0,31.4) (4.0,40.9) (5.0,48.4) };
    \addlegendentry{針-針 実測値}
    % 針-針 実測 回帰直線: y = 9.85 x + 0.69
    \addplot[black, thick, mark=none, domain=0.8:5.2,samples=2] {9.85*x + 0.69};

    \end{axis}
  \end{tikzpicture}
  \caption{平板-平板,球-球,針-針電極の破壊電圧}
  \label{fig:gap-comparison}
\end{figure}

% 【修正箇所】
\paragraph{b) 考察および報告事項}
図\ref{fig:gap-comparison}に示される電極形状と破壊電圧の関係について考察する.
平等電界を形成する平板-平板電極および球-球電極では,実測値と実験式による計算値が極めて良好に一致し,破壊電圧とギャップ長が比例関係にあることが確認できる.これは,電界がギャップ全体で均一であるため,放電現象が安定し,予測性の高い絶縁破壊が生じたためと考察される.

一方,不平等電界を形成する針-針電極では,ギャップ長が長くなるにつれて破壊電圧が上昇する傾向は共通しているものの,実測値と実験式の値には大きな乖離が見られる.この乖離の主な原因として,実験式が\SI{30}{\centi\meter}以上の長ギャップを想定しているのに対し,本実験が短ギャップ領域 (\SI{1}{\centi\meter}〜\SI{5}{\centi\meter}) で行われたことが挙げられる.短ギャップ領域では,コロナ放電の発生・進展メカニズムや,それによって形成される空間電荷効果の影響が長ギャップの場合と異なり,実験式の前提条件から外れるため,このようなズレが生じたと示唆される.

全体を比較すると,同一ギャップ長において,平等電界ギャップ(平板-平板,球-球)の破壊電圧は,不平等電界ギャップ(針-針)のそれを数倍上回っている.これは不平等電界では,針電極の先端に電界が集中し,低い印加電圧でも局所的に絶縁破壊が開始するためである.対照的に,平等電界ではギャップ全体が均等に電圧を分担するため,ギャップ全体としての絶縁耐力が高くなる.以上の結果は,高電圧工学における電界緩和の重要性を示すものであり,実験は原理に即した妥当な結果が得られたと判断できる.

\begin{table}[H]
  \centering
  \caption{平板-平板電極におけるギャップ長と破壊電圧}
  \label{tab:plate}
  \begin{tabular}{ccccc}
    \toprule
    ギャップ長 & \multicolumn{4}{c}{破壊電圧 [\si{\kilo\volt}]} \\
    \cmidrule(lr){2-5}
    $[\si{\centi\meter}]$ & 実験式 & \multicolumn{3}{c}{実測値} \\
    \cmidrule(lr){3-5}
     & & 1回目 & 2回目 & 平均値 \\
    \midrule
    0.4 & 14.0 & 13.3 & 13.3 & 13.3 \\
    0.7 & 22.4 & 21.7 & 21.9 & 21.8 \\
    1.0 & 30.6 & 30.9 & 31.0 & 31.0 \\
    1.3 & 38.5 & 38.3 & 37.9 & 38.1 \\
    1.6 & 46.3 & 46.7 & 46.1 & 46.4 \\
    \bottomrule
  \end{tabular}
\end{table}

\begin{table}[H]
  \centering
  \caption{球-球電極におけるギャップ長と破壊電圧}
  \label{tab:sphere}
  \begin{tabular}{cccccc}
    \toprule
    ギャップ長 & \multicolumn{4}{c}{破壊電圧 [\si{\kilo\volt}]} & 実験式における \\
    \cmidrule(lr){2-5}
    $[\si{\centi\meter}]$ & 実験式 & \multicolumn{3}{c}{実測値} & 補正係数 \\
    \cmidrule(lr){3-5}
     & & 1回目 & 2回目 & 平均値 & \\
    \midrule
    0.4 & 13.0 & 13.9 & 13.7 & 13.8 & 1.03 \\
    0.7 & 22.2 & 22.6 & 22.5 & 22.6 & 1.05 \\
    1.0 & 31.2 & 32.4 & 32.4 & 32.4 & 1.07 \\
    1.3 & 39.8 & 41.0 & 41.0 & 41.0 & 1.09 \\
    1.6 & 48.0 & 48.2 & 48.2 & 48.2 & 1.11 \\
    \bottomrule
  \end{tabular}
\end{table}

\begin{table}[H]
  \centering
  \caption{針-針電極におけるギャップ長と破壊電圧}
  \label{tab:needle}
  \begin{tabular}{ccccc}
    \toprule
    ギャップ長 & \multicolumn{4}{c}{破壊電圧 [\si{\kilo\volt}]} \\
    \cmidrule(lr){2-5}
    $[\si{\centi\meter}]$ & 実験式 & \multicolumn{3}{c}{実測値} \\
    \cmidrule(lr){3-5}
     & & 1回目 & 2回目 & 平均値 \\
    \midrule
    1.0 & 22.8 & 8.8 & 8.6 & 8.7 \\
    2.0 & 27.7 & 21.9 & 21.7 & 21.8 \\
    3.0 & 32.6 & 31.4 & 31.4 & 31.4 \\
    4.0 & 37.5 & 40.8 & 41.0 & 40.9 \\
    5.0 & 42.4 & 48.4 & 48.4 & 48.4 \\
    \bottomrule
  \end{tabular}
\end{table}

\subsubsection{針-平板電極を用いた極性効果}

\paragraph{a) 実験結果}
針-平板電極の結果を表\ref{tab:needle_plate}に示す.極性効果を確認するため,同表をグラフにしたものを図\ref{fig:needle-plate}に示す.

% 【修正箇所】
\paragraph{b) 考察および報告事項}
図\ref{fig:needle-plate}の結果は,不平等電界における「極性効果」を明瞭に示している.全てのギャップ長において,針電極が負極性(針(-))の場合の破壊電圧が,正極性(針(+))の場合を上回っている.
% 【追加箇所】
この現象は,放電の進展メカニズムに起因する.針が負極性の場合,針先端から放出された電子が負コロナを形成する.この負イオンの空間電荷が針先端の電界集中を緩和する効果を持つため,ギャップ全体の絶縁破壊に至るにはより高い電圧が必要となる.一方,針が正極性の場合,空間の電子が針先端に引き寄せられて正イオンが生成され,ストリーマ放電が進展しやすくなるため,比較的低い電圧で破壊に至る.

% 【追加箇所】
また,近似直線からは,ギャップ長が極めて短い領域(本実験では$d \approx \SI{0.25}{\centi\meter}$ 付近)で破壊電圧の関係が逆転する可能性が示唆される.これは,極短ギャップでは空間電荷形成の効果が薄れ,電極表面の状態や電子放出過程そのものが破壊電圧を決定する支配的な要因となるためと考えられる.
以上の考察から,本実験では原理通りの極性効果が確認できたといえる.

\begin{table}[H]
  \centering
  \caption{針-平板電極におけるギャップ長と破壊電圧}
  \label{tab:needle_plate}
  \begin{tabular}{ccccccc}
    \toprule
    ギャップ長 & \multicolumn{6}{c}{破壊電圧 [\si{\kilo\volt}]} \\
    \cmidrule(lr){2-7}
    $[\si{\centi\meter}]$ & \multicolumn{3}{c}{針(+)} & \multicolumn{3}{c}{針(-)} \\
    \cmidrule(lr){2-4} \cmidrule(lr){5-7}
     & 1回目 & 2回目 & 平均値 & 1回目 & 2回目 & 平均値 \\
    \midrule
    0.4 & 5.5 & 5.6 & 5.6 & 7.3 & 7.3 & 7.3 \\
    0.7 & 9.0 & 9.1 & 9.1 & 12.2 & 12.2 & 12.2 \\
    1.0 & 11.9 & 11.9 & 11.9 & 18.7 & 18.7 & 18.7 \\
    1.3 & 14.6 & 15.0 & 14.8 & 25.5 & 25.5 & 25.5 \\
    1.6 & 18.5 & 18.5 & 18.5 & 31.7 & 31.5 & 31.6 \\
    \bottomrule
  \end{tabular}
\end{table}

% --- 図: 針-平板電極の平均破壊電圧 (針(+) と 針(-) を重ねる) ---
\begin{figure}[H]
  \centering
  \begin{tikzpicture}
    \begin{axis}[
      width=0.75\linewidth,
      height=6cm,
      xlabel={ギャップ長 [cm]},
      ylabel={破壊電圧 [kV]},
      legend pos=south east,
      grid=major,
      mark size=3pt,
      xmin=0, xmax=1.8,
      ymin=0
    ]
  % 針(+) 平均値 (点のみ)
  \addplot[only marks, black, thick, mark=o, mark options={fill=black,draw=black}] coordinates { (0.4,5.6) (0.7,9.1) (1.0,11.9) (1.3,14.8) (1.6,18.5) };
    \addlegendentry{針(+) 平均値}
  % 針(+) 回帰直線 (最小二乗): y = 10.5 x + 1.48 (範囲を少しだけ拡張)
  \addplot[black, thick, mark=none, domain=0.35:1.65,samples=2] {10.5*x + 1.48};

  % 針(-) 平均値 (点のみ)
  \addplot[only marks, black, thick, mark=triangle, mark options={fill=white,draw=black}] coordinates { (0.4,7.3) (0.7,12.2) (1.0,18.7) (1.3,25.5) (1.6,31.6) };
    \addlegendentry{針(-) 平均値}
  % 針(-) 回帰直線 (最小二乗): y = 20.6333 x - 1.5733 (範囲を少しだけ拡張)
  \addplot[black, thick, mark=none, domain=0.35:1.65,samples=2] {20.6333*x - 1.5733};
    \end{axis}
  \end{tikzpicture}
  \caption{針-平板電極におけるギャップ長と破壊電圧 (平均値)}
  \label{fig:needle-plate}
\end{figure}

\subsection{液体および固体の絶縁破壊試験}

本節では,液体絶縁体および固体絶縁体の破壊試験の結果を報告する.更に,前節の気体絶縁体を含めて気体・液体・固体絶縁体の破壊電圧を比較する.

\subsubsection{液体絶縁体の破壊試験の結果}
液体絶縁体である絶縁油(JIS C 2320 1種2号)の破壊試験の結果を表\ref{tab:oil}に示す.同表の*の記載がある1回目と3回目は,実験装置の最大出力である \SI{50.1}{\kilo\volt} の印加時に絶縁破壊しなかった箇所となる.但し他の試行において \SI{50.1}{\kilo\volt} 以下で絶縁破壊していることから,1回目と3回目の破壊電圧は,\SI{50.1}{\kilo\volt} を僅かに上回った値となると考えられる.従って,2〜5回目の平均値も \SI{47.3}{\kilo\volt} を僅かに上回る値となる.JIS C2101 (1999) における平均値の判定は,合格基準である \SI{30}{\kilo\volt} 以上であることから合格とした.

\subsubsection{固体絶縁体の破壊試験の結果}
固体絶縁体である厚さ \SI{1}{\milli\meter} のベークライトの破壊試験の結果を表\ref{tab:bakelite}に示す.2回の測定を行いその平均値を求め,交流電圧を印加したことから平均値から波高値を求めて,その値を破壊電圧とした.試料の厚さが \SI{1}{\milli\meter} であることから,破壊電圧を絶縁耐力とした.絶縁耐力は \SI{61.2}{\kilo\volt\per\milli\meter} となる.

\subsubsection{気体・液体・固体絶縁体の破壊電圧の比較}
本実験で扱った三態の絶縁体について,同一ギャップ長における破壊電圧の比較を行う.絶縁油およびベークライトのギャップ長 \SI{1}{\centi\meter} の破壊電圧を計算した後,三態の絶縁体の破壊電圧を比較する.

\begin{table}[H]
  \centering
  \caption{絶縁油の破壊電圧 (ギャップ長 \SI{0.25}{\centi\meter},*の記載がある値は \SI{50.1}{\kilo\volt} 以上)}
  \label{tab:oil}
  \begin{tabular}{cccccc}
    \toprule
    1回目 [\si{\kilo\volt}] & 2回目 [\si{\kilo\volt}] & 3回目 [\si{\kilo\volt}] & 4回目 [\si{\kilo\volt}] & 5回目 [\si{\kilo\volt}] & 2〜5回目の平均値 [\si{\kilo\volt}] \\
    \midrule
    50.1* & 46.4 & 46.2 & 46.9 & 44.8 & 46.1 \\
    \bottomrule
  \end{tabular}
\end{table}

\begin{table}[H]
  \centering
  \caption{ベークライト板の破壊電圧 (ギャップ長 \SI{0.1}{\centi\meter})}
  \label{tab:bakelite}
  \begin{tabular}{cccc}
    \toprule
    1回目 [\si{\kilo\volt}] & 2回目 [\si{\kilo\volt}] & 平均値 [\si{\kilo\volt}] & 波高値 ($\sqrt{2}$×平均値) [\si{\kilo\volt}] \\
    \midrule
    42.8 & 45.4 & 44.1 & 62.4 \\
    \bottomrule
  \end{tabular}
\end{table}

\paragraph{a) 絶縁油のギャップ長1 cmの破壊電圧}
絶縁油のギャップ長 \SI{1}{\centi\meter} における破壊電圧 $V$ を以下のように計算した.表\ref{tab:oil}よりギャップ長 \SI{0.25}{\centi\meter} における破壊電圧 \SI{46.1}{\kilo\volt} から定数 $n$ を 0.5 として定数 $A$ を求めると
\begin{equation}
  A = \frac{V}{d^n} = \frac{46.1}{0.25^{0.5}} = 92.2
\end{equation}
である.この $A$ を用いて破壊電圧 $V$ を計算すると,以下となる.
\begin{equation}
  V = A \times d^n = 92.2 \times 1^{0.5} = 92.2 \, \text{[\si{\kilo\volt}]}
\end{equation}

\paragraph{b) ベークライトのギャップ長1 cmの破壊電圧}
ベークライトのギャップ長 \SI{1}{\centi\meter} における破壊電圧 $V$ を以下のように計算した.表\ref{tab:bakelite}のギャップ長 \SI{0.1}{\centi\meter} における破壊電圧 \SI{62.4}{\kilo\volt} から定数 $n$ を 0.6 として定数 $A$ を求めると
\begin{equation}
  A = \frac{V}{d^n} = \frac{62.4}{0.1^{0.6}} = 248.3
\end{equation}
である.この $A$ を用いて破壊電圧 $V$ を計算すると,以下となる.
\begin{equation}
  V = A \times d^n = 248.3 \times 1^{0.6} = 248.3 \, \text{[\si{\kilo\volt}]}
\end{equation}

\paragraph{c) 気体・液体・固体絶縁体の破壊電圧の比較}
ギャップ長 \SI{1}{\centi\meter} における気体・液体・固体絶縁体の破壊電圧の比較を表\ref{tab:comparison}に示す.各絶縁体の破壊電圧は次の値を用いた.気体絶縁体は,2.1.1 の空気の平板-平板電極ギャップ長 \SI{1.0}{\centi\meter} における平均値を用いた.液体絶縁体は,2.2.3 a)の絶縁油のギャップ長 \SI{1.0}{\centi\meter} の推定値を用いた.固体絶縁体は,2.2.3 b)のベークライトのギャップ長 \SI{1.0}{\centi\meter} の推定値を用いた.各絶縁体の破壊電圧を昇順に並べると,気体(空気): \SI{31.0}{\kilo\volt},液体(絶縁油): \SI{92.2}{\kilo\volt},固体(ベークライト): \SI{248.3}{\kilo\volt} の順となった.気体を1としたときのおおよその比は,1:2.97:8.01 となり,気体に対して液体が約2.97倍程度の絶縁耐力を有し,更には,液体に対して固体が 2.7 倍程度の絶縁耐力を有することを確認した.

\begin{table}[H]
  \centering
  \caption{ギャップ長\SI{1}{\centi\meter}における気体・液体・固体絶縁物の破壊電圧の比較}
  \label{tab:comparison}
  \begin{tabular}{lccc}
    \toprule
     & 気体 (空気) & 液体 (絶縁油) & 固体 (ベークライト) \\
    \midrule
    \textbf{破壊電圧 [\si{\kilo\volt}]} & 31.0 & 92.2 & 248.3 \\
    \textbf{気体に対する比} & 1.00 & 2.97 & 8.01 \\
    \bottomrule
  \end{tabular}
\end{table}

\subsection{碍子のインパルス電圧特性}

本節では,碍子の50\%フラッシオーバ電圧の実験結果と報告事項について述べる.

\paragraph{a) 実験結果}
耐雷インパルス電圧 \SI{75}{\kilo\volt}/個の懸垂碍子の放電率を表\ref{tab:insulator}に,碍子に印加した雷インパルス電圧の波形を図\ref{fig:waveform}に示す.同電圧は負極性であるため,下側のピークが波高点となる.更に,表\ref{tab:insulator}をもとに作成した放電率とインパルス電圧の関係を図4に示す.同図には補間法(内挿法)によって求めた50\%フラッシオーバ電圧も示した.

はじめに,図\ref{fig:waveform}の雷インパルス電圧の波形について確認する.同図(a)の波頭長の波形では,横方向1グリッドあたりの時間が \SI{400}{\nano\second} (\SI{0.4}{\micro\second})で,波形の原点から約3グリッドでピークに達しており,波頭長は約 \SI{1.2}{\micro\second} となる.同図(b)の波尾長の波形では,横方向1グリッドあたりの時間が \SI{10}{\micro\second} で,原点からピークに達した後ピークの50\%の電圧に低下するまで約5グリッドを要しており,波尾長は約 \SI{50}{\micro\second} となる.以上よりこの波形の表記は-1.2/50 \si{\micro\second} と表すことができ,電気規格調査会標準規格(JEC)のJEC0202 (1994)で規定されている雷インパルスの標準波形に則った波形を印加したことが確認できる.

次に,図4より碍子の 50\%フラッシオーバ電圧 $V_{50}$ を求める.表\ref{tab:insulator}より,フラッシオーバの確率が30\%となる破壊電圧 $V_m = \SI{92.7}{\kilo\volt}$,及び,60\%となる破壊電圧 $V_n = \SI{93.4}{\kilo\volt}$ を用いて,2点間を直線補間(内挿法)によって $V_{50}$ を求めたところ,$V_{50} = \SI{93.2}{\kilo\volt}$ であった.

\begin{table}[H]
  \centering
  \caption{碍子のインパルス電圧と放電率}
  \label{tab:insulator}
  \begin{tabular}{cccccccccccccc}
    \hline
    充電電圧 & インパルス電圧 & \multicolumn{10}{c}{フラッシオーバの有無 (有: ○, 無: ×)} & 放電率 p [\%] \\
    $V_i$ [\si{\kilo\volt}] & $V_o$ [\si{\kilo\volt}] & \\
    \cmidrule(lr){3-12}
    & & 1 & 2 & 3 & 4 & 5 & 6 & 7 & 8 & 9 & 10 & \\
    \midrule
    27.2 & 92.0 & × & × & × & × & × & × & × & × & × & × & 0 \\
    27.4 & 92.7 & × & × & 〇 & × & × & 〇 & × & × & × & 〇 & 30 \\
    27.6 & 93.4 & 〇 & × & × & 〇 & 〇 & 〇 & 〇 & × & × & 〇 & 60 \\
    \bottomrule
  \end{tabular}
\end{table}

\begin{figure}[H]
  \centering
  \includegraphics[width=0.8\linewidth]{"雷インパルス電圧の波形 (負極性),(a) 波頭長,(b) 波尾長.png"}
  \caption{雷インパルス電圧の波形(負極性),(a)波頭長,(b)波尾長}
  \label{fig:waveform}
\end{figure}

最後に,碍子が確実に絶縁できる最大電圧値を考察する.図4の放電率-インパルス電圧特性から,フラッシオーバの確率が0\%となる破壊電圧 $V_0$ を求める.$V_m, V_n$ の2点を通る直線の式を得るため,傾き $a$,及び,切片 $b$ を求めると,
\begin{equation}
  a = \frac{p_n - p_m}{V_n - V_m} = \frac{60 - 30}{93.4 - 92.7} \approx 42.86
\end{equation}
\begin{equation}
  b = p_m - a \times V_m = 30 - 42.86 \times 92.7 \approx -3943
\end{equation}
となる.従って,放電率-インパルス電圧特性の式は以下で表される.
\begin{equation}
  p = 42.86V_o - 3943
\end{equation}
ここで,フラッシオーバの確率が0\%となる破壊電圧 $V_0$ を求めると,以下となる.
\begin{equation}
  V_0 = \frac{0 + 3943}{42.86} \approx 92.0 \, \text{[\si{\kilo\volt}]}
\end{equation}
この破壊電圧 $V_0$ は,実験時の気象条件に依存した値のため,標準状態における破壊電圧 $V_{\mathrm{on}}$ を求めると,
\begin{equation}
  V_{\mathrm{on}} = \frac{V_0}{\delta} = \frac{92.0}{0.959} \approx 95.9 \, \text{[\si{\kilo\volt}]}
\end{equation}
が得られる.フラッシオーバの確率にはランダム性があることから,この電圧で確実にフラッシオーバの確率が0\%となるとは言えないが,概ねこの電圧付近で0\%となる可能性は高いと言える.標準状態において $V_{\mathrm{on}}$ と碍子の仕様上の耐雷インパルス電圧 \SI{75}{\kilo\volt} を比較すると,$V_{\mathrm{on}}$ は碍子の耐雷インパルス電圧より約28\%高い電圧となる.このことから,気温,気圧,湿度等の気象条件の変化や,軽度の汚損等による絶縁性能の低下に対応できるように,2割以上の余裕が設けられていることが確認できる.

\paragraph{b) 報告事項}
(1) 本実験の50\%フラッシオーバ電圧に感電した場合について
自身の抵抗値をデジタルテスタで測定したところ,$R=374 \, \text{\si{\kilo\ohm}}$ であった.ここで本実験で求めた50\%フラッシオーバ電圧 $V_{50} = \SI{93.2}{\kilo\volt}$ に感電した際に流れる電流 $I$ を計算すると,
\begin{equation}
  I = \frac{V_{50}}{R} = \frac{93.2 \times 10^3}{374 \times 10^3} \approx 0.249 \, \text{[\si{\ampere}]} = 249 \, \text{[\si{\milli\ampere}]}
\end{equation}
となる.
% 【修正箇所】
ここで,成人男性の離脱電流(自力で電路から離脱できる限界電流)が約\SI{22.70}{\milli\ampere}であることを考慮すると,この電流値はその10倍以上に達する.人体は高電圧に感電するほど内部抵抗が低下する傾向があるため,実際にはさらに大きな電流が流れる可能性が高い.
% 【追加箇所】
このような大電流が人体,特に心臓を通過した場合,心室細動を引き起こす危険性が極めて高く,即座に生命の危機に瀕する.従って,自力での離脱は不可能であり,極めて危険な状況であると結論付けられる.

% 【修正箇所】
(2) 雷は落ちているのかについて
文献\cite{hidaka2009}によると,雷放電は大きく分けて2つの段階で進展する.第一段階として,雷雲から地面に向かって「ステップトリーダ(段階的先行放電)」と呼ばれる微弱な放電が,進んでは止まるという動作を繰り返しながら進展し,放電路を形成する.第二段階として,このステップトリーダが地面に到達した瞬間,地面から雷雲に向かって,形成された放電路を逆に辿る形で「リターンストローク(主雷撃)」と呼ばれる非常に明るく強力な放電が発生する.我々が「稲妻」として目視しているのは,主にこのリターンストロークの強力な発光である.したがって,放電の通り道は上から下へと形成されるが,目に見える強い光(稲妻)は下から上へと進展しているため,「稲妻は下から上へ登っていく」と表現することができる.

\section*{参考文献}
\begin{thebibliography}{9}
\bibitem{hidaka2009}
日高邦彦: “高電圧工学(新・電気システム工学)”,数理工学社,pp.55-56,(2009).
\end{thebibliography}

\end{document}