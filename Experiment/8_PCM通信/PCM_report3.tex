% ===== ドキュメントクラス =====
% LuaLaTeXでのコンパイルを前提としたクラス設定
\documentclass[
  a4paper,
  11pt,
]{ltjsarticle}

% ----- フォント・数式関連 -----
\usepackage{newtxtext}
\usepackage{newtxmath}
\usepackage{amsmath}
% newtxmathとの競合回避
\let\Bbbk\relax
\usepackage{amssymb}
\usepackage{bm} % ベクトル太字用

% ----- 画像・単位・レイアウト関連 -----
% 【重要】コンパイラの自動判定設定
\usepackage{graphicx} 
\usepackage[export]{adjustbox}
\usepackage{siunitx}
\usepackage{float}
\usepackage{placeins}
\usepackage[margin=25mm]{geometry}

% 画像の高さ制限設定
\setkeys{Gin}{keepaspectratio}

% 便利マクロ: max size をデフォルトで使うラッパー
\newcommand{\includefig}[2][]{\includegraphics[max size={\textwidth}{0.45\textheight},#1]{#2}}

% ページ下部の余白調整
\raggedbottom

% ----- 参考文献の上付き表示設定 -----
\makeatletter
\def\@cite#1#2{\textsuperscript{[{#1\if@tempswa , #2\fi}]}}
\makeatother

% ----- ハイパーリンク -----
\usepackage[colorlinks=true,linkcolor=black,citecolor=black,urlcolor=black,anchorcolor=black]{hyperref}

% ===== ドキュメント情報 =====
\title{PCM通信 実験報告書}
\author{電気電子工学科 X年 X組 番号 氏名} % 氏名は適宜変更してください
\date{2025年12月3日}

% ===== ここからドキュメント本体 =====
\begin{document}

\section{目的}
実機デモおよび回路シミュレータを用いた回路設計・解析を通じて,アナログ信号をパルス信号で変調・復調するパルス符号変復調(PCM)回路の仕組みと,その動作原理を習得することを目的として本実験を実施した.

\section{原理}

\subsection{パルス符号変・復調}
図\ref{fig:1}にPCM変・復調回路の基本方式を示す.
PCM変調回路は,入力切換回路で選択された入力信号を標本化パルスにより標本化(サンプリング)する標本化回路,標本化された入力信号を量子化レベルに変換する量子化回路,量子化レベルを2進符号化信号に変換する符号化回路,および同期信号・チャンネル信号を挿入して送信する送信回路より構成される.

一方,PCM復調回路は,受信信号からチャンネル信号とPCM変調信号を分離する分離回路,分離されたチャンネル信号を解読するチャンネル判別回路,PCM変調信号を並列信号に変換する符号変換回路,並列信号に変換されたPCM変調信号をチャンネル別に復調するD/A変換器,および不要な高周波成分を除去するLPF(ローパスフィルタ)回路より構成される.

\begin{figure}[H]
  \centering
  \includegraphics[max size={0.6\textwidth}{0.45\textheight}]{image/img_000.png}
  \caption{PCM通信の基本方式}
  \label{fig:1}
\end{figure}

\clearpage

\subsection{タイミングパルス発生回路}
タイミングパルス発生回路は,1つの発振器を基準として,装置各部に必要なタイミングパルスを供給する回路である.図\ref{fig:2}にタイミングパルス発生回路の回路図およびその動作を表すカルノー図を示す.

カウンタICであるU3 (SN74LS162) のCLK端子には,U1より周波数 \SI{125}{\kilo\hertz} のクロックパルスが入力されている.U3はこのパルスをカウントし,所定のカウント数に達するとRCO端子からキャリーオーバー信号(CO)を出力する.図\ref{fig:2}において,このRCO信号は後段のJKフリップフロップ回路へ入力される.

JKフリップフロップ回路では,JおよびK入力の双方がHighレベルに固定されているため,クロック入力(1番ピン)に信号が立ち上がるたびに,出力Qの状態が反転する動作(トグル動作)となる.すなわち,U1のクロックパルスが一定数カウントされRCOから信号が出力されるたびに,Q出力であるT1信号が反転を繰り返す.

\begin{figure}[H]
  \centering
  \includegraphics[max size={0.75\textwidth}{0.45\textheight}]{image/img_001.png}
  \caption{タイミングパルス発生回路とカルノー図}
  \label{fig:2}
\end{figure}

\subsection{切換回路}
PCM通信を多重化するための入力切換回路を図\ref{fig:3}に示す.本回路は,上段と下段に配置された2チャンネルの入力信号源を切り替えて出力する機能を持つ.この出力信号が次段の標本化回路への入力となる.

U1のクロック信号によってアナログスイッチSW3およびSW4のオン・オフが制御される.クロックがLowレベルのときはSW3がオン,SW4がオフとなり,上段の信号源(\SI{50}{\hertz} 正弦波)がVoutに出力される.逆にクロックがHighレベルのときはSW3がオフ,SW4がオンとなり,下段の信号源(\SI{100}{\hertz} 正弦波)が出力される.このように,複数のチャンネル信号を時分割で切り替えて標本化回路へ送出することで,多重化を実現している.

\begin{figure}[H]
  \centering
  \includegraphics[max size={0.6\textwidth}{0.45\textheight}]{image/img_002.png}
  \caption{切換回路}
  \label{fig:3}
\end{figure}

\subsection{標本化回路(サンプル\&ホールド回路)}
標本化回路は,アナログ信号を一定の時間間隔でサンプリングし,次のサンプリングタイミングまでその電圧値を保持(ホールド)する回路である.回路図を図\ref{fig:4}に示す.

U1のクロックパルスがHighレベルとなる期間,SW1がオンとなり,入力信号(変調波)によってコンデンサC1が充電される.これによりC1の端子電圧は入力信号の瞬時値に追従する.その後,クロックパルスがLowレベルとなりSW1がオフになっても,後段のオペアンプ(ボルテージフォロワ構成)の入力インピーダンスが極めて高いため,C1に蓄えられた電荷は保持され,Voutには一定の電圧が出力され続ける.このホールド期間中に,Voutの電圧値はA/Dコンバータによってデジタル信号へ変換される.このサイクルを繰り返すことで,連続的なアナログ信号が離散的な階段状波形へと変換される.

なお,図中の抵抗R1 (\SI{1}{\mega\ohm}) は回路シミュレーション上の安定性を確保するために挿入されたものであり,理想的な動作原理上は開放(無限大)とみなして差し支えない.実際にR1を除去してシミュレーションを行った結果を図\ref{fig:5}に示すが,波形に変化は見られない.

\begin{figure}[H]
  \centering
  \includegraphics[max size={0.7\textwidth}{0.45\textheight}]{image/img_003.png}
  \caption{標本化回路(サンプル\&ホールド回路)}
  \label{fig:4}
\end{figure}

\begin{figure}[H]
  \centering
  \includegraphics[max size={0.75\textwidth}{0.45\textheight}]{image/img_004.png}
  \caption{サンプル\&ホールド回路におけるR1の影響検証}
  \label{fig:5}
\end{figure}

\subsection{シフトレジスタ(パラレルシリアル変換)}
8ビットの並列データを直列データに変換し,順次送信するための回路を図\ref{fig:6}に示す.

本来,U17~U24の入力端子にはA/D変換された8ビットのデジタル信号が入力されるが,本実験では動作確認のため,手動スイッチ(SW-HL1~8)を接続している.ロード信号T3が入力されるタイミングでスイッチの状態がシフトレジスタU25 (74199) に取り込まれ,その後Clock信号に同期してOUTPUT端子からデータが1ビットずつシリアル出力される.

\begin{figure}[H]
  \centering
  \includegraphics[max size={0.7\textwidth}{0.45\textheight}]{image/img_005.png}
  \caption{シフトレジスタ(パラレルシリアル変換回路)}
  \label{fig:6}
\end{figure}

\subsection{波形合成・分離回路}
実際の通信装置ではデータ信号とチャンネル信号を合成して伝送するが,本実験ではクロック信号T2およびT3を用いて,波形の合成と分離の動作原理を確認する.回路図を図\ref{fig:7}に示す.

まず,T3信号をIdeal inverterで反転させ,その出力とT2信号をIdeal adderで加算することで合成波形Vmixを生成する.

\begin{figure}[H]
  \centering
  \includegraphics[max size={0.7\textwidth}{0.45\textheight}]{image/img_006.png}
  \caption{波形合成・分離回路}
  \label{fig:7}
\end{figure}

受信側の分離回路では,理想ダイオードと反転増幅器を用いて元の信号を復元する.この動作原理を図\ref{fig:8}の等価回路を用いて説明する.
\begin{itemize}
    \item Vmixが \SI{0}{\volt} の場合:電源V1 (\SI{5}{\volt}) によりダイオードが導通し,VT2およびVT3端子は接地電位となるため,出力は \SI{0}{\volt} となる(図\ref{fig:8}(b), (c)).
    \item Vmixが \SI{5}{\volt} の場合:上側のダイオードは逆バイアスとなりオフ,下側のダイオードは反転増幅回路を介してオンとなる.その結果,VT3には \SI{-5}{\volt} が現れるが,最終段の反転増幅器により \SI{5}{\volt} として出力される(図\ref{fig:8}(d)).
    \item Vmixが \SI{-5}{\volt} の場合:上側のダイオードがオン,下側がオフとなり,VT2経由で信号が取り出され,同様に反転されて \SI{5}{\volt} が出力される(図\ref{fig:8}(f)).
\end{itemize}
このように,合成波形Vmixから,T2成分はVT3へ,T3成分はVT2へと分離して取り出すことが可能である.

\begin{figure}[H]
  \centering
  \includegraphics[max size={0.6\textwidth}{0.45\textheight}]{image/img_007.png}
  \caption{波形合成・分離回路の動作原理}
  \label{fig:8}
\end{figure}

\subsection{AD-DA変換回路,ローパスフィルタ}
A/D変換およびD/A変換を行う回路を図\ref{fig:9}に示す.
信号源VG1より交流信号 ($V_m = \SI{5}{\volt}$, $f = \SI{500}{\hertz}$) を供給する.A/DコンバータU10は負電圧の入力に対応していないため,前段のIdeal adderにてオフセット電圧 \SI{6}{\volt} を加算し,信号全体が正電圧となるようにレベルシフトを行う.

U1 (74199) はラッチ回路として機能し,クロックT2の立ち下がりエッジでデータを取り込み,High区間その値を保持する.

A/D変換されたデジタルデータは,D/AコンバータMV95308に入力され再度アナログ信号へ変換される.ここで,入力時に加算したオフセット分をキャンセルするため,Ideal subtracterを用いて出力から \SI{6}{\volt} を減算する.
D/A変換直後の出力波形VDAは階段状であるため,最終段のローパスフィルタ(LPF)を通して高調波成分を除去し,滑らかなアナログ波形に復調する.

\begin{figure}[H]
  \centering
  \includegraphics[max size={0.7\textwidth}{0.45\textheight}]{image/img_008.png}
  \caption{AD・DA変換回路とローパスフィルタ}
  \label{fig:9}
\end{figure}
\FloatBarrier

\section{実験方法}
本実験では,回路シミュレーションソフト「TINA-TI」を用いた.
まず,図\ref{fig:2}に示したタイミングパルス発生回路からクロック源と電圧ピンを取り外し,マクロピンを接続した回路を作成した(図\ref{fig:10}).この回路を「新規マクロ・ウィザード」機能を用いてマクロ化(サブサーキット化)し,以降の実験で部品として使用できるように保存した.

\begin{figure}[H]
  \centering
  \includegraphics[max size={0.7\textwidth}{0.45\textheight}]{image/img_009.png}
  \caption{タイミングパルス発生回路のマクロ化}
  \label{fig:10}
\end{figure}

続いて,図\ref{fig:2}から図\ref{fig:9}(図\ref{fig:5}, \ref{fig:8}を除く)に示した各機能ブロックの回路をシミュレータ上で構築した.解析メニューより「過渡解析」を実行し,各部の電圧波形を観測した.
また,図\ref{fig:6}のシフトレジスタについては,入力スイッチ(SW-HL)の設定値を変更し,OUTPUT端子からのシリアル出力波形が入力データに対応して変化することを確認した.

\section{使用機器}
本実験では,以下のソフトウェアを使用した.
\begin{itemize}
    \item シミュレーションソフト:TINA-TI (DesignSoft / Texas Instruments)
\end{itemize}
\FloatBarrier

\section{実験結果および考察}

\subsection{タイミングパルス発生回路}
タイミングパルス発生回路(図\ref{fig:2})の過渡解析結果を図\ref{fig:11}に示す.
観測された波形を確認すると,T1出力はRCO信号が出力されるタイミングで反転動作を行っている.これは図\ref{fig:2}に示したJKフリップフロップの真理値表およびタイムチャートの動作と一致しており,回路が設計通り正常に動作していることが確認できた.

\begin{figure}[H]
  \centering
  \includegraphics[max size={0.75\textwidth}{0.45\textheight}]{image/img_010.png}
  \caption{タイミングパルス発生回路の過渡特性}
  \label{fig:11}
\end{figure}

\subsection{切替回路}
切替回路(図\ref{fig:3})のシミュレーション結果を図\ref{fig:12}に示す.

出力波形Voutに着目すると,制御信号Vclockのレベル変化に同期して,\SI{100}{\hertz} の正弦波と \SI{50}{\hertz} の正弦波が交互に出力されている様子が明瞭に確認できる.したがって,アナログスイッチによる信号の切り替え動作が理論通りに行われているといえる.

\begin{figure}[H]
  \centering
  \includegraphics[max size={0.75\textwidth}{0.45\textheight}]{image/img_011.png}
  \caption{切替回路の過渡特性}
  \label{fig:12}
\end{figure}

\subsection{標本化回路(サンプル\&ホールド回路)}
標本化回路(図\ref{fig:4})のシミュレーション結果を図\ref{fig:13}に示す.

Voutの波形を確認すると,Vclockが立ち上がるタイミングで入力信号VG1の電圧値をサンプリングし,VclockがLowレベルとなりSW1がオフになった後も,その電圧値が一定期間維持(ホールド)されている.この挙動はサンプル\&ホールド回路の理論動作と完全に一致するものである.

\begin{figure}[H]
  \centering
  \includegraphics[max size={0.75\textwidth}{0.45\textheight}]{image/img_012.png}
  \caption{標本化回路の過渡特性}
  \label{fig:13}
\end{figure}

\subsection{シフトレジスタ(パラレルシリアル変換)}
シフトレジスタ(図\ref{fig:6})のシミュレーション結果を図\ref{fig:14}に示す.
ここではパラレル入力として,SW-HL1, 6, 7, 8をHigh,それ以外をLowに設定した.
OUTPUT波形を確認すると,ロード信号T3の入力後に,設定したスイッチの状態に対応して「High, High, High, Low, Low, Low, Low, High」の順(SW-HL8側から順)でデータが出力されている.これにより,パラレルデータが正しくシリアルデータに変換されていることが確認できた.

\begin{figure}[H]
  \centering
  \includegraphics[max size={0.75\textwidth}{0.45\textheight}]{image/img_013.png}
  \caption{シフトレジスタの過渡特性}
  \label{fig:14}
\end{figure}

\subsection{波形合成・分離回路}
波形合成・分離回路(図\ref{fig:7})のシミュレーション結果を図\ref{fig:15}に示す.

合成波形Vmixには,T2信号(正極性)とT3信号(負極性)が重畳されていることがわかる.分離後の出力端子VT2にはT3と同じ波形が,VT3にはT2と同じ波形が,それぞれ正の電圧(振幅 \SI{5.00}{\volt})として現れている.この結果から,合成された信号がダイオードと反転増幅器を用いた分離回路によって,適切に元の各信号成分へと分離・復元されていることが実証された.

\begin{figure}[H]
  \centering
  \includegraphics[max size={0.75\textwidth}{0.45\textheight}]{image/img_014.png}
  \caption{波形合成・分離回路の過渡特性}
  \label{fig:15}
\end{figure}

\subsection{AD-DA変換回路,ローパスフィルタ}
AD-DA変換回路およびLPF(図\ref{fig:9})のシミュレーション結果を図\ref{fig:16}に示す.

まずVin波形は,入力信号VG1が \SI{+6}{\volt} シフトされ,正電圧範囲に収まっていることが確認できる.
次にD/A変換後の出力VDAに着目すると,振幅値がクロックT2ごとに階段状に変化しており,デジタル信号に対応したアナログ電圧が出力されている.
また,VQAおよびVQHは,入力VG1の変化に応じて \SI{0}{\volt} または \SI{4}{\volt} のデジタル値をとっており,正常にA/D変換が行われていることを示している.

最終的な復調出力であるV復調の波形を見ると,LPFを通過することでVDAの階段状の成分が平滑化され,連続的な正弦波に近い波形が得られている.最大振幅は中心電圧 \SI{-0.07}{\volt} に対して約 \SI{5.00}{\volt} であり,入力信号が適切に復調されているといえる.

\begin{figure}[H]
  \centering
  \includegraphics[max size={0.75\textwidth}{0.45\textheight}]{image/img_015.png}
  \caption{AD-DA変換回路,ローパスフィルタの過渡特性}
  \label{fig:16}
\end{figure}

\section{報告事項}

\subsection{標本化定理(サンプリング定理)の概要}
標本化(sampling)とは,音声や画像などの連続的なアナログ信号を一定周期 $T$ で抽出する操作を指す.連続信号を $g(t)$ とすると,抽出された信号の振幅は標本点 $nT$ における値 $g(nT)$ となり,これを標本値と呼ぶ.標本値からなるパルス列 $\sum_{n=-\infty}^{\infty}g(nT)$ は,パルス振幅変調信号(PAM信号)を表す.

連続信号 $g(t)$ の周波数帯域が $0 \sim f_0/2$ に制限されているとし,これを標本化周期 $T = 1/f_0$ で標本化すると,PAMパルス列から元の $g(t)$ は以下の式(\ref{eq:1})のように再構成できる.
\begin{equation}
  g(t) = \sum_{n=-\infty}^{\infty} g(nT) \frac{\sin \pi f_0(t - nT)}{\pi f_0 (t - nT)} \label{eq:1}
\end{equation}
この関係は「染谷-Shannonの標本化定理」と呼ばれる.すなわち,アナログ信号に含まれる最高周波数を $f_{max}$ とするとき,その2倍以上の周波数(標本化周波数 $f_s \ge 2f_{max}$)で標本化を行えば,標本化された離散信号から元のアナログ信号を完全に再現可能であることを示している\texorpdfstring{\cite{hatori2012}}{}.

この定理の物理的意味について記述する.理想的な低域通過フィルタ(遮断周波数 $f_c=f_0/2$)のインパルス応答 $h_s(t)$ は,以下の式(\ref{eq:3})で示されるような Sinc関数($\sin x / x$)の形状をとる.
\begin{equation}
  h_s(t) = \frac{\sin \pi f_0 t}{\pi f_0 t} \label{eq:3}
\end{equation}

入力信号 $g(t)$ を無限小のパルス幅を持つインパルス列 $\delta(t-nT)$ で標本化すると,標本化信号 $g_T(t)$ は以下の式(\ref{eq:2})のように表される.
\begin{equation}
  g_T(t) = \sum_{n=-\infty}^{\infty} g(nT) \cdot \delta(t - nT) \label{eq:2}
\end{equation}

このインパルス列を上述の理想低域通過フィルタに通すと,出力 $y(t)$ は各インパルスに対するフィルタ応答(Sinc関数)の総和となる.すなわち畳み込み積分により以下のように計算される.
\begin{equation}
  y(t) = \int_{-\infty}^{\infty} g_T(\tau) h_s(t-\tau)d\tau = \sum_{n=-\infty}^{\infty} g(nT) \cdot h_s(t - nT) \label{eq:4}
\end{equation}
この式は,標本点ごとに中心を持つSinc関数($h_s(t-nT)$)を,その時点の標本値 $g(nT)$ で重み付けして重ね合わせる操作を意味する.
Sinc関数は,中心点 $t=0$ で値 $1$ をとり,それ以外のすべての標本点 $t=k/f_0 (k \neq 0)$ において値 $0$ をとる特性(直交性)を持つ.このため,ある標本点での信号の振幅値は,他の標本点からの干渉を受けずに復元される.これらすべてのSinc関数を足し合わせることで,標本点間が滑らかに補間され,元の連続信号 $g(t)$ が再現される\texorpdfstring{\cite{kaneko1977_1}}{}.

\subsection{本実験における量子化レベル}
量子化とは,標本化によって時間的に離散化された信号に対し,その振幅値を離散的な数値に置き換える操作である\texorpdfstring{\cite{kaneko1977_2}}{}.
本実験では,アナログ信号を8ビットの2進数に変換している.したがって,量子化ビット数は8であり,量子化レベル数(ステップ数)は $2^8 = 256$ ステップとなる.

\subsection{理想ダイオードの特性}
通常のシリコンダイオードには順方向降下電圧(約 \SIrange{0.6}{0.7}{\volt})が存在し,これを超えない電圧では電流が流れない.これに対し,理想ダイオードとは,順方向電圧印加時には抵抗ゼロで短絡(導通)し,逆方向電圧印加時には抵抗無限大で開放(遮断)するという,理想的な整流特性を持つ素子として定義される\texorpdfstring{\cite{takasaki2021}}{}.

\subsection{ローパスフィルタの遮断周波数の導出}
本実験で使用したローパスフィルタ(図\ref{fig:9}の末段)は,オペアンプ,抵抗,コンデンサで構成されるサレン・キー(Sallen-Key)型2次LPFである.回路の構成要素を以下のように定義する.
\begin{itemize}
    \item 入力側から順に,直列接続された抵抗: $R_1, R_2$
    \item フィードバック($R_1$と$R_2$の中点への帰還):コンデンサ $C_1$
    \item 非反転入力端子の接地:コンデンサ $C_2$
    \item オペアンプはユニティゲインバッファ(ボルテージフォロワ)として動作
\end{itemize}

オペアンプの入力インピーダンスが無限大であるという理想特性を仮定し,各ノードにおけるキルヒホッフの法則を適用して回路方程式を解くと,サレン・キー型LPFの伝達関数 $G(s) = V_{OUT}/V_{IN}$ は一般に次式で導かれる.
\begin{equation}
  G(s) = \frac{\frac{1}{R_1 R_2 C_1 C_2}}{s^2 + s(\frac{1}{R_1 C_1} + \frac{1}{R_2 C_1}) + \frac{1}{R_1 R_2 C_1 C_2}} \label{eq:15}
\end{equation}
これは標準的な2次遅れ系の伝達関数形式
\begin{equation}
  G(s) = \frac{\omega_n^2}{s^2 + 2\zeta\omega_n s + \omega_n^2} \label{eq:16}
\end{equation}
に対応する.ここで固有角周波数 $\omega_n$ は定数項の平方根として求まるため,
\begin{equation}
  \omega_n = \frac{1}{\sqrt{R_1 R_2 C_1 C_2}} \label{eq:18}
\end{equation}
となり,これがカットオフ角周波数 $\omega_c$ に相当する.したがって,カットオフ周波数 $f_c$ は次式で与えられる.
\begin{equation}
  f_c = \frac{1}{2\pi\sqrt{R_1 R_2 C_1 C_2}} \label{eq:19}
\end{equation}
本実験における各素子の値($R_1 = R_2 = \SI{50}{\kilo\ohm}$, $C_1 = \SI{2.3}{\nano\farad}$, $C_2 = \SI{1.1}{\nano\farad}$)を代入すると,遮断周波数 $f_c$ は以下のように計算される.
\begin{equation}
  f_c = \frac{1}{2\pi\sqrt{50 \times 10^3 \times 50 \times 10^3 \times 2.3 \times 10^{-9} \times 1.1 \times 10^{-9}}} \approx \SI{2.0e3}{\hertz} = \SI{2.0}{\kilo\hertz} \label{eq:20}
\end{equation}
以上より,本回路の遮断周波数は約 \SI{2.0}{\kilo\hertz} である\texorpdfstring{\cite{elec-info, terashima2019}}{}.

\subsection{PCM変調の優位性}
PCM変調が他のアナログ変調方式と比較して優れている主な点を以下に挙げる\texorpdfstring{\cite{kaneko1977_3}}{}.
\begin{itemize}
    \item \textbf{耐雑音性}: 再生時にパルスの有無(1か0か)を判別できればよいため,雑音や波形歪みの影響を受けにくい.
    \item \textbf{再生中継による品質維持}: リピータによる波形整形・再生中継が可能なため,伝送距離が長くなっても雑音や歪みが累積せず,一定の通信品質を維持できる.
    \item \textbf{多重化の容易さ}: 時分割多重(TDM)により,電話,データ,画像など異なる種類の信号を統一的なデジタルデータとして効率的に多重伝送できる.
    \item \textbf{伝送媒体の柔軟性}: 光ファイバ,同軸ケーブル,無線など,多様な伝送媒体に対して柔軟に適応可能である.
    \item \textbf{経済性}: LSI技術の進展により,かつては複雑であった変復調装置の小型化・低価格化が実現されている.
\end{itemize}

\subsection{PCM通信における多重化の原理}
多重化伝送とは,1つの伝送路を用いて複数の情報を同時に送る技術である.PCM通信では,時分割多重(TDM: Time Division Multiplexing)が用いられる.

標本化定理に基づき,アナログ信号は連続波形でなく,離散的なパルス列として表現できる.この際,各パルスの間には信号が存在しない空白時間が生じる.TDMでは,この空白時間に他のチャンネル(Ch.2,Ch.3...)のパルスを挿入(インターリーブ)することで多重化を実現する.
すなわち,標本化周期 $T$ 内において,各チャンネルの標本化タイミング(位相)を互いに重ならないようにわずかにずらす(位相オフセットを持たせる)ことで,時間軸上で複数の信号を単一のパルス列として混信なく合成することが可能となる.
受信側では,送信側と完全に同期したタイミングで信号を読み取る(ゲートを開く)ことで,特定のタイムスロットの信号のみを分離して取り出し,各チャンネルの信号を個別に復元する\texorpdfstring{\cite{kaneko1977_4}}{}.

\subsection{PCM通信の応用分野}
PCM通信技術は,1962年のT1方式(Bell研究所)の実用化以来,通信インフラの基盤技術として広く普及した.初期の電話回線網における局間伝送(24回線PCM方式等)から始まり,現在では光ファイバ通信,衛星通信,携帯電話網,ディジタル放送,さらにはインターネット上のデータ伝送に至るまで,あらゆるデジタル通信システムの根幹技術として利用されている\texorpdfstring{\cite{kaneko1977_5}}{}.

% ===== 参考文献 =====
\begin{thebibliography}{99}
    \bibitem{kaneko1977_1} 金子尚志:「PCM 通信の技術」,産報出版株式会社,pp.17-19 (1977).
    \bibitem{hatori2012} 羽鳥光俊:「わかりやすい通信工学」,コロナ社,p.21 (2012).
    \bibitem{kaneko1977_2} 金子尚志:「PCM 通信の技術」,産報出版株式会社,p.27 (1977).
    \bibitem{takasaki2021} 高崎和之:「基本からわかる電子回路」,株式会社ナツメ社,pp.36-37 (2021).
    \bibitem{elec-info} Electrical Information: “Sallen-Key Low-pass Filter”, \url{https://detail-infomation.com/sallen-key-low-pass-filter/} (参照 2023-12-14).
    \bibitem{terashima2019} 寺島一彦,兼重明宏:「制御工学 技術者のための,理論・設計から実装まで」,実教出版株式会社,pp.121-128 (2019).
    \bibitem{kaneko1977_3} 金子尚志:「PCM 通信の技術」,産報出版株式会社,pp.14-15 (1977).
    \bibitem{kaneko1977_4} 金子尚志:「PCM 通信の技術」,産報出版株式会社,pp.10-11 (1977).
    \bibitem{kaneko1977_5} 金子尚志:「PCM 通信の技術」,産報出版株式会社,p.10 (1977).
\end{thebibliography}

\end{document}