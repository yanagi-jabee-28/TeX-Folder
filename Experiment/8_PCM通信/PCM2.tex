\documentclass[a4paper,11pt]{ltjsarticle}

% =============================================
% 1. パッケージ設定 (SARP v3.0 NNCT-EE準拠)
% =============================================
\usepackage[T1]{fontenc}
\usepackage{newtxtext}
\usepackage[varbb]{newtxmath} % 数式フォント Times系
\usepackage{bm}      % ベクトル太字
\usepackage{mathtools}

% レイアウト・図表関連
\usepackage[margin=25mm]{geometry}
\usepackage{array}      
\usepackage{multirow}   
\usepackage{fancyhdr}   
\usepackage{graphicx}
% 画像検索パス
\graphicspath{{./}{image/}}
\usepackage{float}
\usepackage{booktabs}
\usepackage{subcaption}
\usepackage[export]{adjustbox}

% 回路図・グラフ描画
\usepackage{circuitikz}
\usepackage{tikz}
\usepackage{pgfplots}
\pgfplotsset{compat=newest}
\usepackage{pgfplotstable}
\usetikzlibrary{arrows.meta, positioning, calc}

% SI単位・数式処理
\usepackage{siunitx}
\sisetup{
  detect-all,
  inter-unit-product=\ensuremath{{}\cdot{}},
  separate-uncertainty=true,
  number-unit-product = \hspace{0.5em} % 単位前の半角スペース強制
}

% リンク・参照
\usepackage{cite}
\usepackage{xurl}
\usepackage[hidelinks]{hyperref}
\usepackage[nameinlink,noabbrev]{cleveref}
\usepackage{needspace}
\Urlmuskip=0mu plus 1mu
\usepackage{titlesec}
\titlespacing*{\section}{0pt}{3.5ex plus 1ex minus .2ex}{0pt}
\titlespacing*{\subsection}{0pt}{2.5ex plus .5ex minus .2ex}{0pt}
\titlespacing*{\subsubsection}{0pt}{1.5ex plus .3ex minus .2ex}{0pt}
\usepackage{indentfirst}

% 参考文献の上付き表示設定 [1]形式
\makeatletter
\def\@cite#1#2{$^{\mbox{\scriptsize[#1\if@tempswa , #2\fi]}}$}
\def\@biblabel#1{[#1]}
\makeatother

\crefname{figure}{図}{図}
\crefname{table}{表}{表}
\crefname{equation}{式}{式}

% キャプション設定
\usepackage{caption}
\captionsetup{
  format=hang,
  labelsep=quad,
  font={small},
  labelfont={bf},
  justification=centering
}
\captionsetup[figure]{justification=centerlast}

% =============================================
% 2. カスタムコマンド定義
% =============================================
\newcommand{\UnderlineBox}[2][3cm]{\underline{\makebox[#1][c]{\vphantom{lp}\large #2}}}
\newcommand{\JustifiedLabel}[2]{\makebox[#1][s]{\large\bfseries #2}}
\newcommand{\BoldLabel}[1]{{\large\bfseries #1}}

% 微分記号(ローマン体 d)
\newcommand{\diff}[2]{\frac{\mathrm{d}#1}{\mathrm{d}#2}}
\newcommand{\pdiff}[2]{\frac{\partial #1}{\partial #2}}

% 単位記号・ローマン体コマンド
\providecommand{\unit}[1]{\,\mathrm{#1}}
\newcommand{\rom}[1]{\mathrm{#1}}

% =============================================
% 3. 表紙専用のページスタイル定義
% =============================================
\fancypagestyle{coverpage}{
  \fancyhf{} 
  \renewcommand{\headrulewidth}{0pt} 
  \renewcommand{\footrulewidth}{0pt} 
  \cfoot{\vspace{2mm}\footnotesize \bfseries 国立長野高専 電気電子工学科}
}

% =============================================
% ドキュメント開始
% =============================================
\begin{document}

% /////////////////////////////////////////////
% 表紙 (Cover Page)
% /////////////////////////////////////////////

\newgeometry{top=30mm, bottom=18mm, left=18mm, right=18mm}
\thispagestyle{coverpage}

\begin{center}
    \vspace*{0mm} 
    {\Huge \bfseries 電気電子工学実験報告書}
    \vspace{6mm} 
\end{center}

\noindent
\begin{tabular}{@{}ll}
  \BoldLabel{テーマ名} & \UnderlineBox[13.5cm]{PCM通信} \\[2.0em] 
\end{tabular}

\noindent
\BoldLabel{報告者} \hspace{0.5em}
\UnderlineBox[1.5cm]{5} {\large \textbf{年}} \hspace{0.2em}      
(\UnderlineBox[1.5cm]{E} {\large \textbf{組}}) \hspace{0.2em} 
{\large \textbf{番号}} \UnderlineBox[2.0cm]{234} \hspace{0.5em}    
\UnderlineBox[1.5cm]{B} {\large \textbf{班}} \hspace{1em}        
\UnderlineBox[4.5cm]{栁原 魁人}                                   
\vspace{0.3em} 

\noindent
\begin{tabular}{@{}p{0.48\textwidth} p{0.48\textwidth}}
  \BoldLabel{実験場所} \hspace{1em} \UnderlineBox[5.5cm]{エレクトロニクス工房} & 
  \BoldLabel{指導担当} \hspace{1em} \UnderlineBox[5.5cm]{斎藤 栄輔}    
\end{tabular}
\vspace{0.8em} 

\noindent
\BoldLabel{共同実験者} \hspace{1em} \UnderlineBox[12.5cm]{石坂知尋,倉科純太郎,中井智大,中澤耕平} 
\vspace{1.5em} 

\noindent
\renewcommand{\arraystretch}{1.3}
\setlength{\tabcolsep}{0pt}
\begin{tabular}{l l l l}
    \JustifiedLabel{5em}{実験日} & 
    \hspace{0.3em} 令和 \UnderlineBox[0.65cm]{7} 年 \UnderlineBox[0.65cm]{9} 月 \UnderlineBox[0.65cm]{26} 日 & & \\
    \JustifiedLabel{5em}{提出期限} & 
    \hspace{0.3em} 令和 \UnderlineBox[0.65cm]{7} 年 \UnderlineBox[0.65cm]{12} 月 \UnderlineBox[0.65cm]{31} 日 & 
    \hspace{0.3em}$\Rightarrow$\hspace{0.3em} \JustifiedLabel{4em}{提出日} & 
    \hspace{0.3em} 令和 \UnderlineBox[0.65cm]{7} 年 \UnderlineBox[0.65cm]{12} 月 \UnderlineBox[0.65cm]{16} 日 \\
    ( \JustifiedLabel{6em}{再提出期限} & 
    \hspace{0.3em} 令和 \UnderlineBox[0.65cm]{} 年 \UnderlineBox[0.65cm]{} 月 \UnderlineBox[0.65cm]{} 日 & 
    \hspace{0.3em}$\Rightarrow$\hspace{0.3em} \JustifiedLabel{5em}{再提出日} & 
    \hspace{0.3em} 令和 \UnderlineBox[0.65cm]{} 年 \UnderlineBox[0.65cm]{} 月 \UnderlineBox[0.65cm]{} 日 ) \\
\end{tabular}
\vspace{40mm}

\renewcommand{\arraystretch}{1.6}
\begin{center}
\begin{tabular}{|>{\centering\arraybackslash}m{2.4cm}|>{\raggedright\arraybackslash}m{12.1cm}|>{\centering\arraybackslash}m{2.4cm}|}
\hline
\multicolumn{2}{|c|}{\JustifiedLabel{11em}{評 価 項 目}} & \JustifiedLabel{4em}{評 価} \\
\hline
\multirow{3}{*}{\parbox[c][5.0em][c]{2.4cm}{\centering\shortstack{\large\bfseries 実 習\\[0.3em]\large\bfseries 評 価}}} 
 & (1) 自ら積極的に実験に取り組めた &  \\ \cline{2-3}
 & (2) 実験装置を適切に使用でき,正確に実験を行なえた &  \\ \cline{2-3}
 & (3) グループ内で協力的に実験が行なえた &  \\
\hline
\multirow{4}{*}{\parbox[c][6.0em][c]{2.4cm}{\centering\shortstack{\large\bfseries 報告書\\[0.3em]\large\bfseries 評 価}}} 
 & (1) 結果のまとめかた(図表を含む) &  \\ \cline{2-3}
 & (2) 結果に対する考察 &  \\ \cline{2-3}
 & (3) 報告事項/課題(正しい解答や適切な引用など) &  \\ \cline{2-3}
 & (4) 報告書としての体裁が整っているか &  \\
\hline
\end{tabular}
\end{center}
\vspace*{0mm}
\enlargethispage{30pt}
\clearpage

% /////////////////////////////////////////////
% 本文 (Main Body)
% /////////////////////////////////////////////

\restoregeometry 
\setcounter{page}{1}
\pagestyle{plain} 

\section{目的}
現代のデジタル通信の基礎となるパルス符号変調(PCM: Pulse Code Modulation)方式について,その変復調プロセスの原理を理解することを目的とする.具体的には,回路シミュレータ TINA-TI を用いて,標本化(サンプリング),量子化,符号化,および復号化の各工程を担う機能ブロックを構成し,信号波形の観測を通じて標本化定理や時分割多重(TDM)の動作原理を検証する.

\section{原理}

\subsection{PCM 通信の基本構成}
PCM 通信方式は,アナログ信号をデジタル符号に変換して伝送する方式であり,\cref{fig:block_diagram}に示すように変調側と復調側で構成される.

変調側では,入力されたアナログ信号に対し,以下の 3 段階の処理を行う.
\begin{enumerate}
    \item \textbf{標本化(Sampling)}: 時間的に連続な信号を一定の周期 $T$(標本化周期)で離散的な値(標本値)として取り出す.
    \item \textbf{量子化(Quantization)}: 連続的な振幅値を持つ標本値を,あらかじめ定められた離散的なレベル(代表値)に近似する.
    \item \textbf{符号化(Encoding)}: 量子化された値を 2 進数のデジタル符号(パルス列)に変換する.
\end{enumerate}
さらに,同期信号やチャンネル識別信号を付加して伝送路へ送出する.

復調側では,受信したパルス列から同期信号等を分離し,D/A 変換(復号化)によって量子化レベルに対応した階段状波形を再生する.最後にローパスフィルタ(LPF)を通すことで,高調波成分を除去し,元のアナログ信号を復元する.

\begin{figure}[H]
    \centering
    \includegraphics[width=0.8\linewidth]{image/img_000.png}
    \caption{PCM 通信の基本ブロック図}
    \label{fig:block_diagram}
\end{figure}

\subsection{タイミングパルス発生回路}
PCM 装置内の各回路ブロックは,基準となるクロックに同期して動作する必要がある.\cref{fig:timing_circuit}にタイミングパルス発生回路を示す.
本回路は,\SI{125}{\kilo\hertz} の原発振クロックを分周・計数し,システムに必要な制御信号(T1, T2 等)を生成する.

U3 (SN74LS162) は同期 4 ビットカウンタであり,クロックを計数して所定の値(たとえば 10 進数の 9 等)に達するとリップルキャリー出力 (RCO) を High にする.後段の JK フリップフロップ(U5)は,$J=K=\rom{High}$ に設定されており,RCO の立ち下がりエッジ等をトリガとして出力 Q を反転(トグル)させる.これにより,原発振周波数を分周したタイミング信号 T1 が生成される.

\begin{figure}[H]
    \centering
    \includegraphics[width=0.9\linewidth]{image/img_001.png}
    \caption{タイミングパルス発生回路とカルノー図}
    \label{fig:timing_circuit}
\end{figure}

\subsection{切換回路(マルチプレクサ)}
複数の信号を 1 つの伝送路で送る時分割多重(TDM)を実現するため,入力信号を順次切り替える回路である(\cref{fig:switching_circuit}).
アナログスイッチ SW3 および SW4 は,制御クロックの論理レベルによって排他的に ON/OFF する.
\begin{itemize}
    \item Clock = Low のとき: SW3 が ON,SW4 が OFF $\to$ CH1(\SI{50}{\hertz})が出力
    \item Clock = High のとき: SW3 が OFF,SW4 が ON $\to$ CH2(\SI{100}{\hertz})が出力
\end{itemize}
これにより,出力 $V_{\rom{out}}$ には 2 つの信号が交互に現れ,後段の標本化回路へ送られる.

\begin{figure}[H]
    \centering
    \includegraphics[width=0.4\linewidth]{image/img_002.png}
    \caption{切換回路}
    \label{fig:switching_circuit}
\end{figure}

\subsection{標本化回路(サンプル \& ホールド回路)}
標本化回路は,アナログ信号の瞬時電圧をコンデンサに記憶(ホールド)し,A/D 変換に必要な時間を確保するための回路である(\cref{fig:sample_hold}).

動作原理は以下の通りである.
\begin{enumerate}
    \item \textbf{サンプルモード}: スイッチ SW1 が ON になると,入力電圧によってホールドコンデンサ $C_1$ が急速に充電され,その端子電圧は入力信号に追従する.
    \item \textbf{ホールドモード}: SW1 が OFF になると,$C_1$ は入力から切り離される.後段のオペアンプ(ボルテージフォロワ接続)は入力インピーダンスが極めて高いため,$C_1$ に蓄えられた電荷は放電されず,電圧が保持される.
\end{enumerate}
なお,図中の抵抗 $R_1$ (\SI{1}{\mega\ohm}) は,シミュレーション上の収束安定性や実機でのバイアス電流経路確保のために挿入される場合があるが,理想的には開放(無限大)とみなせるため,回路動作への影響は無視できる(\cref{fig:sample_hold_r1}).

\begin{figure}[H]
    \centering
    \includegraphics[width=0.6\linewidth]{image/img_003.png}
    \caption{標本化回路(サンプル\&ホールド回路)}
    \label{fig:sample_hold}
\end{figure}

\begin{figure}[H]
    \centering
    \includegraphics[width=0.9\linewidth]{image/img_004.png}
    \caption{サンプル\&ホールド回路における抵抗 $R_1$ の影響検証}
    \label{fig:sample_hold_r1}
\end{figure}

\subsection{パラレル-シリアル変換(P-S 変換)}
A/D 変換後の並列デジタルデータ(パラレルデータ)を,1 本の伝送路で送るために直列データ(シリアルデータ)に変換する回路である(\cref{fig:shift_register}).
シフトレジスタ(U25)は,Load 信号(実験では T3 に同期)がアクティブになると,入力端子 A-H のデータを内部フリップフロップに取り込む.その後,クロックが入るたびにデータを 1 ビットずつシフトし,出力端子から順次送出する.

\begin{figure}[H]
    \centering
    \includegraphics[width=0.6\linewidth]{image/img_005.png}
    \caption{シフトレジスタ(P-S 変換回路)}
    \label{fig:shift_register}
\end{figure}

\subsection{波形合成・分離回路}
伝送路上での多重化信号の合成と,受信側での分離を模した回路である(\cref{fig:synth_sep_circuit}).本実験では,異なるタイミングパルス T2,T3 を正負の極性を変えて合成し,それを分離することで TDM の原理を確認する.

\begin{itemize}
    \item \textbf{合成}: T2 そのものと,T3 を反転(Invert)したものを加算器で合成する.これにより,$V_{\rom{mix}}$ には正極性の T2 パルスと負極性の T3 パルスが混在した波形が現れる.
    \item \textbf{分離}: 理想ダイオードと反転増幅回路を用いて極性分離を行う.
    \begin{itemize}
        \item $V_{\rom{mix}} > 0$ のとき: 上側のダイオードが導通し,T2 成分が抽出される.
        \item $V_{\rom{mix}} < 0$ のとき: 下側のダイオード回路(反転アンプ含む)が動作し,負のパルスが正極性に反転されて出力される(T3 成分の抽出).
    \end{itemize}
\end{itemize}

\begin{figure}[H]
    \centering
    \includegraphics[width=0.6\linewidth]{image/img_006.png}
    \caption{波形合成・分離回路}
    \label{fig:synth_sep_circuit}
\end{figure}

\begin{figure}[H]
    \centering
    \includegraphics[width=0.5\linewidth]{image/img_007.png}
    \caption{波形合成・分離回路の等価回路と動作モード}
    \label{fig:sep_operation}
\end{figure}

\subsection{AD・DA 変換および信号再生}
アナログ信号をデジタル化し,再びアナログに戻す一連のプロセスである(\cref{fig:ad_da_circuit}).
実験回路では,A/D コンバータへの入力電圧範囲(単電源動作)に合わせるため,入力信号(交流)にオフセット電圧(\SI{+6}{\volt})を加算する.
D/A 変換後は,階段状の波形(PAM 波形に近いもの)が得られるため,再度オフセット(\SI{-6}{\volt})を除去した後,ローパスフィルタ(LPF)に通す.LPF は,標本化に伴って発生した高調波成分(折り返し雑音等)を除去し,元の滑らかな正弦波を復元する役割(平滑化)を担う.

\begin{figure}[H]
    \centering
    \includegraphics[width=0.6\linewidth]{image/img_008.png}
    \caption{AD・DA 変換回路および再構成フィルタ}
    \label{fig:ad_da_circuit}
\end{figure}

\section{実験方法}
シミュレーションソフト「TINA-TI」(Texas Instruments 社製)を用いて,以下の手順で実験を行った.

\begin{enumerate}
    \item \textbf{タイミング生成部のマクロ化}:
    \cref{fig:timing_circuit}に示す回路を作成し,入出力ピンを設定した上でマクロ(サブサーキット)化した(\cref{fig:macro_circuit}).これにより,以降の回路図作成を簡略化した.

    \item \textbf{各ブロックの特性解析}:
    切換回路,標本化回路,シフトレジスタ,波形合成・分離回路,AD-DA 変換回路をそれぞれ個別に作成し,「過渡解析(Transient Analysis)」を実行した.入力信号やクロックに対する出力波形の変化を観測し,各部の動作電圧やタイミングを確認した.

    \item \textbf{シフトレジスタの動作確認}:
    パラレル入力スイッチ(SW-HL1~8)の設定を任意に変更(例:特定のビットのみ High)し,シリアル出力波形がその設定に対応したパルス列となるかを確認した.
\end{enumerate}

\begin{figure}[H]
    \centering
    \includegraphics[width=0.6\linewidth]{image/img_009.png}
    \caption{マクロ化されたタイミングパルス発生回路}
    \label{fig:macro_circuit}
\end{figure}

\section{使用機器}
\begin{table}[H]
\centering
\caption{使用機器およびソフトウェア}
\begin{tabular}{l l l}
\toprule
名称 & メーカー・開発元 & 備考 \\
\midrule
回路シミュレータ TINA-TI & DesignSoft / Texas Instruments & Ver. 9 \\
PC 端末 & (実験室備品) & Windows OS \\
\bottomrule
\end{tabular}
\end{table}

\section{実験結果および考察}

\subsection{タイミングパルス発生回路}
過渡解析の結果を\cref{fig:res_timing}に示す.図より,カウンタの出力である RCO 信号が High になった直後のクロックエッジで,出力 T1 が反転している様子が観測された.これは回路設計通りの分周動作であり,正常に機能していると判断できる.

\begin{figure}[H]
    \centering
    \includegraphics[width=0.8\linewidth]{image/img_010.png}
    \caption{タイミングパルス発生回路の出力波形}
    \label{fig:res_timing}
\end{figure}

\subsection{切換回路}
シミュレーション結果を\cref{fig:res_switching}に示す.$V_{\rom{out}}$ の波形において,制御クロック $V_{\rom{clock}}$ の変化に同期して,\SI{50}{\hertz} 正弦波と \SI{100}{\hertz} 正弦波が交互に切り替わって出力されている.スイッチング時のノイズや電圧降下も見られず,理想的な TDM 動作が得られている.

\begin{figure}[H]
    \centering
    \includegraphics[width=0.7\linewidth]{image/img_011.png}
    \caption{切換回路の出力波形(時分割多重動作)}
    \label{fig:res_switching}
\end{figure}

\subsection{標本化回路}
結果を\cref{fig:res_sample}に示す.$V_{\rom{out}}$ は $V_{\rom{clock}}$ が High の期間(サンプルモード)では入力波形 $V_{\rom{G1}}$ に追従し,Low の期間(ホールドモード)では直前の電圧値を一定に維持している.ホールド期間中の電圧減衰(ドループ)は観測されず,理想的なサンプル&ホールド動作が確認された.これは,シミュレーション上のオペアンプの入力インピーダンスが無限大であるためである.

\begin{figure}[H]
    \centering
    \includegraphics[width=0.7\linewidth]{image/img_012.png}
    \caption{標本化回路の入出力波形}
    \label{fig:res_sample}
\end{figure}

\subsection{シフトレジスタ}
SW-HL1, 6, 7, 8 を High(データ「10000111」相当のパターン)とした場合の結果を\cref{fig:res_shift}に示す.OUTPUT 端子からは,ロード信号 T3 の後に設定したビットパターンに対応した High/Low のパルス列が順次出力されている.これにより,パラレルデータが正しくシリアルデータに変換されていることが確認できた.

\begin{figure}[H]
    \centering
    \includegraphics[width=0.7\linewidth]{image/img_013.png}
    \caption{シフトレジスタによる並列-直列変換波形}
    \label{fig:res_shift}
\end{figure}

\subsection{波形合成・分離回路}
結果を\cref{fig:res_synth}に示す.$V_{\rom{mix}}$ 波形は,正側に T2 パルス,負側に反転した T3 パルスが加算された合成波形となっている.分離後の出力 VT2,VT3 を確認すると,VT2 には正の T3 成分が,VT3 には正の T2 成分がそれぞれ分離・復元されている.
ここで,VT2 に T3 由来の信号が出ている点に注意が必要である.原理図(\cref{fig:sep_operation})の接続構成により,極性分離された信号がそれぞれの端子へ適切に振り分けられていることが検証された.

\begin{figure}[H]
    \centering
    \includegraphics[width=0.7\linewidth]{image/img_014.png}
    \caption{波形合成および分離回路の動作波形}
    \label{fig:res_synth}
\end{figure}

\subsection{AD-DA 変換回路,ローパスフィルタ}
結果を\cref{fig:res_ad_da}に示す.
\begin{itemize}
    \item $V_{\rom{in}}$: \SI{0}{\volt} 中心の正弦波が \SI{+6}{\volt} シフトされ,正の範囲に収まっている.
    \item $V_{\rom{DA}}$: A/D 変換および D/A 変換を経た信号であり,時間軸方向に離散的(階段状)な波形となっている.量子化による離散化が視覚的に確認できる.
    \item $V_{\rom{復調}}$: LPF 通過後の波形であり,$V_{\rom{DA}}$ の階段状の角が取れ,滑らかな正弦波に復元されている.振幅および周波数も入力信号と一致しており,PCM 伝送系の正常動作が実証された.
\end{itemize}

\begin{figure}[H]
    \centering
    \includegraphics[width=0.7\linewidth]{image/img_015.png}
    \caption{PCM 復調波形と LPF による平滑化}
    \label{fig:res_ad_da}
\end{figure}

\section{報告事項}

\begin{enumerate}
    \item \textbf{標本化定理(サンプリング定理)について}
    
    標本化定理とは,帯域制限された信号をデジタル化する際に,情報の欠落なく元の信号を復元するための条件を示した定理である.
    アナログ信号 $g(t)$ に含まれる最高周波数を $f_{\rom{max}}$ とするとき,標本化周波数 $f_s$(周期 $T_s = 1/f_s$)が以下の条件を満たせば,標本値の系列から元の信号 $g(t)$ を完全に再現できる.
    \begin{equation}
        f_s \ge 2 f_{\rom{max}}
    \end{equation}
    この $2 f_{\rom{max}}$ をナイキスト周波数と呼ぶ.
    
    数理的説明を行う.時間領域での標本化は,元の信号 $g(t)$ とインパルス列 $\delta_T(t) = \sum_{n=-\infty}^{\infty} \delta(t - nT_s)$ の積として表現される.
    \begin{equation}
        g_s(t) = g(t) \cdot \sum_{n=-\infty}^{\infty} \delta(t - nT_s)
    \end{equation}
    これをフーリエ変換すると,周波数領域では畳み込み積分となり,元の信号のスペクトル $G(f)$ が周波数 $f_s$ ごとに周期的に繰り返されるスペクトル $G_s(f)$ が得られる.
    \begin{equation}
        G_s(f) = \frac{1}{T_s} \sum_{n=-\infty}^{\infty} G(f - n f_s)
    \end{equation}
    ここで $f_s \ge 2 f_{\rom{max}}$ であれば,隣り合うスペクトル同士($G(f)$ と $G(f-f_s)$)が重なり合わない(エイリアシングが発生しない).この状態であれば,理想的なローパスフィルタ(遮断周波数 $f_c = f_s/2$)を通すことで,中心の $G(f)$ のみを取り出すことができ,元の信号を復元可能となる\cite{ref1,ref15}.
    \begin{figure}[H]
        \centering
        \includegraphics[max size={0.6\linewidth}{0.5\textheight},keepaspectratio]{image/img_016.png}
        \caption{標本化と復元の概念図\cite{ref1}}
        \label{fig:sampling_theory}
    \end{figure}

    \item \textbf{本実験における量子化レベル数}
    
    本実験で使用した A/D コンバータおよび D/A コンバータは 8 ビット(Binary Digit)の分解能を持つ.
    量子化レベル数 $L$ はビット数を $n$ とすると $L = 2^n$ で表されるため,
    \begin{equation}
        L = 2^8 = 256
    \end{equation}
    となり,入力信号の振幅範囲を 256 段階の離散値で表現している.

    \item \textbf{理想ダイオードの特性}
    
    理想ダイオード(Ideal Diode)とは,順方向電圧降下がゼロ($V_F = 0\unit{V}$),逆方向漏れ電流がゼロ,かつスイッチング時間がゼロである仮想的な素子である.
    \begin{itemize}
        \item 順方向バイアス時($V > 0$): 完全導体(抵抗 $0\unit{\Omega}$,短絡状態)として振る舞う.
        \item 逆方向バイアス時($V < 0$): 完全絶縁体(抵抗 $\infty\unit{\Omega}$,開放状態)として振る舞う.
    \end{itemize}
    実際のシリコンダイオードでは約 \SI{0.6}{\volt} ~ \SI{0.7}{\volt} の順方向降下電圧が存在するため,微小信号の整流では無視できない誤差となるが,オペアンプを用いた「理想ダイオード回路」を構成することで,この特性に極めて近い動作を実現できる\cite{ref4}.

    \item \textbf{\cref{fig:ad_da_circuit}のローパスフィルタの遮断周波数}
    
    図中の LPF はサレン・キー(Sallen-Key)型 2 次ローパスフィルタである.この回路のカットオフ周波数 $f_c$ は次式で与えられる\cite{ref14}.
    \begin{equation}
        f_c = \frac{1}{2\pi \sqrt{R_1 R_2 C_1 C_2}}
        \label{eq:lpf_cutoff}
    \end{equation}
    実験回路定数 $R_1 = R_2 = \SI{50}{\kilo\ohm}$,$C_1 = \SI{2.3}{\nano\farad}$,$C_2 = \SI{1.1}{\nano\farad}$ を代入して計算する.
    \begin{align}
        f_c &= \frac{1}{2\pi \sqrt{(50 \times 10^3)^2 \cdot (2.3 \times 10^{-9}) \cdot (1.1 \times 10^{-9})}} \nonumber \\
            &= \frac{1}{2\pi \cdot (50 \times 10^3) \cdot \sqrt{2.53} \times 10^{-9}} \nonumber \\
            &\approx \frac{1}{2\pi \cdot 50000 \cdot 1.5906 \cdot 10^{-9}} \nonumber \\
            &\approx \frac{1}{4.997 \times 10^{-4}} \approx 2001 \unit{Hz}
    \end{align}
    したがって,遮断周波数は約 \SI{2.0}{\kilo\hertz} である.
    \begin{figure}[H]
        \centering
        \includegraphics[max size={0.6\linewidth}{0.45\textheight},keepaspectratio]{image/img_017.png}
        \caption{サレン・キー型 LPF の回路構成}
        \label{fig:sallen_key}
    \end{figure}

    \item \textbf{PCM 変調の優位性}
    
    他のアナログ変調方式(AM, FM 等)と比較して,PCM は以下の点で優れている\cite{ref7}.
    \begin{enumerate}
        \item \textbf{耐雑音性}: 信号が「0」と「1」のパルスで構成されるため,伝送路で雑音が重畳しても,閾値判定により完全に元のパルスを再生(リジェネレーション)できる.これにより,多段中継を行っても雑音が累積しない.
        \item \textbf{伝送品質の均一化}: アナログ方式では距離と共に品質が劣化するが,PCM では符号誤りが発生しない限り,距離によらず一定の高品質を維持できる.
        \item \textbf{多重化の容易性}: 信号がデジタルデータとして扱われるため,コンピュータデータや画像など,種類の異なる情報を統一的に多重化して伝送できる.
    \end{enumerate}

    \item \textbf{PCM 通信における多重化の原理}
    
    PCM 通信では,時分割多重(TDM: Time Division Multiplexing)が可能である.これは,各チャンネルの信号を標本化パルスの隙間(タイムスロット)に順次割り当てることで実現される.
    標本化定理により,信号は常時接続されている必要はなく,離散的な瞬時値さえ送ればよい.あるチャンネルの信号を送信していない空き時間に,別のチャンネルの標本値を送信することで,1 本の伝送路を複数の通信で共有できる(\cref{fig:tdm})\cite{ref8}.

    \begin{figure}[H]
        \centering
        \includegraphics[max size={0.6\linewidth}{0.45\textheight},keepaspectratio]{image/img_018.png}
        \caption{時分割多重(TDM)の概念\cite{ref8}}
        \label{fig:tdm}
    \end{figure}

    \item \textbf{PCM 通信の応用分野}
    
    PCM 技術は現代の通信インフラの根幹をなしており,以下のような広範な分野で利用されている\cite{ref9}.
    \begin{itemize}
        \item \textbf{公衆回線網}: 固定電話網のバックボーン(基幹回線)や光ファイバ通信システム.
        \item \textbf{デジタルオーディオ}: CD(コンパクトディスク),WAV フォーマット,Blu-ray 等の音声記録.CD では 16bit, \SI{44.1}{\kilo\hertz} の PCM が採用されている.
        \item \textbf{宇宙通信}: 信号強度が微弱となる深宇宙探査機との通信において,高い耐雑音性を活かして採用されている.
    \end{itemize}
\end{enumerate}

\begin{thebibliography}{99}
\raggedright
\bibitem{ref1} 金子尚志:「PCM 通信の技術」,産報出版株式会社,pp.17-19(1977)
\bibitem{ref2} 羽鳥光俊:「わかりやすい通信工学」,コロナ社,p.21(2012)
\bibitem{ref3} 金子尚志:「PCM 通信の技術」,産報出版株式会社,p.27(1977)
\bibitem{ref4} 高崎和之:「基本からわかる電子回路」,株式会社ナツメ社,pp.36-37(2021)
\bibitem{ref5} 「Electrical Information」,\url{https://detail-infomation.com/sallen-key-low-pass-filter/}(2023 年 12 月 14 日参照)
\bibitem{ref6} 寺島一彦,兼重明宏:「制御工学 技術者のための,理論・設計から実装まで」,実教出版株式会社,pp.121-128(2019)
\bibitem{ref7} 金子尚志:「PCM 通信の技術」,産報出版株式会社,pp.14-15(1977)
\bibitem{ref8} 金子尚志:「PCM 通信の技術」,産報出版株式会社,pp.10-11(1977)
\bibitem{ref9} 金子尚志:「PCM 通信の技術」,産報出版株式会社,p.10(1977)
\bibitem{ref10} 石井 聡:TNJ-045:LTspiceでサレン・キー型フィルタ(第2回),Analog Devices Japan,\url{https://www.analog.com/jp/resources/technical-articles/tnj-045.html}(参照 2025 年 12 月 19 日)
\bibitem{ref11} 石井 聡:TNJ-044:LTspiceでサレン・キー型フィルタ(第1回),Analog Devices Japan,\url{https://www.analog.com/jp/resources/technical-articles/tnj-044.html}(参照 2025 年 12 月 19 日)
\bibitem{ref12} 石井 聡:TNJ-046:LTspiceでサレン・キー型フィルタ(第3回),Analog Devices Japan,\url{https://www.analog.com/jp/resources/technical-articles/tnj-046.html}(参照 2025 年 12 月 19 日)
\bibitem{ref13} 石井 聡:TNJ-047:LTspiceでサレン・キー型フィルタ(第4回),Analog Devices Japan,\url{https://www.analog.com/jp/resources/technical-articles/tnj-047.html}(参照 2025 年 12 月 19 日)
\bibitem{ref14} 石井 聡:TNJ-048:LTspiceでサレン・キー型フィルタ(第5回),Analog Devices Japan,\url{https://www.analog.com/jp/resources/technical-articles/tnj-048.html}(参照 2025 年 12 月 19 日)
\bibitem{ref15} 村田昇:「信号処理論 講義スライド(標本化)」,\url{https://noboru-murata.github.io/signal-processing/pdfs/slide10.pdf}(参照 2025 年 12 月 19 日)
\end{thebibliography}

\end{document}