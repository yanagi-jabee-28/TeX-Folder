\documentclass[a4paper,11pt]{ltjsarticle}

% =============================================
% 1. パッケージ設定 (SARP v3.0 NNCT-EE準拠)
% =============================================
\usepackage[T1]{fontenc}
\usepackage{newtxtext}
\usepackage[varbb]{newtxmath} % 数式フォント Times系
\usepackage{bm}      % ベクトル太字
\usepackage{mathtools}

% レイアウト・図表関連
\usepackage[margin=25mm]{geometry}
\usepackage{array}      
\usepackage{multirow}   
\usepackage{fancyhdr}   
\usepackage{graphicx}
% 画像検索パス
\graphicspath{{./}{image/}}
\usepackage{float}
\usepackage{booktabs}
\usepackage{subcaption}
\usepackage[export]{adjustbox}

% 回路図・グラフ描画
\usepackage{circuitikz}
\usepackage{tikz}
\usepackage{pgfplots}
\pgfplotsset{compat=newest}
\usepackage{pgfplotstable}
\usetikzlibrary{arrows.meta, positioning, calc}

% SI単位・数式処理
\usepackage{siunitx}
\sisetup{
  detect-all,
  inter-unit-product=\ensuremath{{}\cdot{}},
  separate-uncertainty=true,
  number-unit-product = \hspace{0.5em} % 単位前の半角スペース強制
}

% リンク・参照
\usepackage{cite}
\usepackage{xurl}
\usepackage[hidelinks]{hyperref}
\usepackage[nameinlink,noabbrev]{cleveref}
\usepackage{needspace}
\Urlmuskip=0mu plus 1mu

% 書式設定
\usepackage{titlesec}
\titlespacing*{\section}{0pt}{3.5ex plus 1ex minus .2ex}{0pt}
\titlespacing*{\subsection}{0pt}{2.5ex plus .5ex minus .2ex}{0pt}
\titlespacing*{\subsubsection}{0pt}{1.5ex plus .3ex minus .2ex}{0pt}
\usepackage{indentfirst}

% 参考文献の上付き表示設定 [1]形式
\makeatletter
\def\@cite#1#2{$^{\mbox{\scriptsize[#1\if@tempswa , #2\fi]}}$}
\def\@biblabel#1{[#1]}
\makeatother

\crefname{figure}{図}{図}
\crefname{table}{表}{表}
\crefname{equation}{式}{式}

% キャプション設定
\usepackage{caption}
\captionsetup{
  format=hang,
  labelsep=quad,
  font={small},
  labelfont={bf},
  justification=centering
}
\captionsetup[figure]{justification=centerlast}

% =============================================
% 2. カスタムコマンド定義
% =============================================
\newcommand{\UnderlineBox}[2][3cm]{\underline{\makebox[#1][c]{\vphantom{lp}\large #2}}}
\newcommand{\JustifiedLabel}[2]{\makebox[#1][s]{\large\bfseries #2}}
\newcommand{\BoldLabel}[1]{{\large\bfseries #1}}

% 微分記号(ローマン体 d)
\newcommand{\diff}[2]{\frac{\mathrm{d}#1}{\mathrm{d}#2}}
\newcommand{\pdiff}[2]{\frac{\partial #1}{\partial #2}}

% 単位記号・ローマン体コマンド
\providecommand{\unit}[1]{\,\mathrm{#1}}
% ローマン体をテキストモードで安全に出力するマクロ(\mathrm は数式専用なので \textup を使う)
\newcommand{\rom}[1]{\textup{#1}}

% =============================================
% 3. 表紙専用のページスタイル定義
% =============================================
\fancypagestyle{coverpage}{
  \fancyhf{} 
  \renewcommand{\headrulewidth}{0pt} 
  \renewcommand{\footrulewidth}{0pt} 
  \cfoot{\vspace{2mm}\footnotesize \bfseries 国立長野高専 電気電子工学科}
}

% =============================================
% ドキュメント開始
% =============================================
\begin{document}

% /////////////////////////////////////////////
% 表紙 (Cover Page)
% /////////////////////////////////////////////

\newgeometry{top=30mm, bottom=18mm, left=18mm, right=18mm}
\thispagestyle{coverpage}

\begin{center}
    \vspace*{0mm} 
    {\Huge \bfseries 電気電子工学実験報告書}
    \vspace{6mm} 
\end{center}

\noindent
\begin{tabular}{@{}ll}
  \BoldLabel{テーマ名} & \UnderlineBox[13.5cm]{PCM通信} \\[2.0em] 
\end{tabular}

\noindent
\BoldLabel{報告者} \hspace{0.5em}
\UnderlineBox[1.5cm]{5} {\large \textbf{年}} \hspace{0.2em}      
(\UnderlineBox[1.5cm]{E} {\large \textbf{組}}) \hspace{0.2em} 
{\large \textbf{番号}} \UnderlineBox[2.0cm]{234} \hspace{0.5em}    
\UnderlineBox[1.5cm]{B} {\large \textbf{班}} \hspace{1em}        
\UnderlineBox[4.5cm]{栁原 魁人}                                   
\vspace{0.3em} 

\noindent
\begin{tabular}{@{}p{0.48\textwidth} p{0.48\textwidth}}
  \BoldLabel{実験場所} \hspace{1em} \UnderlineBox[5.5cm]{エレクトロニクス工房} & 
  \BoldLabel{指導担当} \hspace{1em} \UnderlineBox[5.5cm]{斎藤 栄輔}    
\end{tabular}
\vspace{0.8em} 

\noindent
\BoldLabel{共同実験者} \hspace{1em} \UnderlineBox[12.5cm]{石坂知尋,倉科純太郎,中井智大,中澤耕平} 
\vspace{1.5em} 

\noindent
\renewcommand{\arraystretch}{1.3}
\setlength{\tabcolsep}{0pt}
\begin{tabular}{l l l l}
    \JustifiedLabel{5em}{実験日} & 
    \hspace{0.3em} 令和 \UnderlineBox[0.65cm]{7} 年 \UnderlineBox[0.65cm]{9} 月 \UnderlineBox[0.65cm]{26} 日 & & \\
    \JustifiedLabel{5em}{提出期限} & 
    \hspace{0.3em} 令和 \UnderlineBox[0.65cm]{7} 年 \UnderlineBox[0.65cm]{12} 月 \UnderlineBox[0.65cm]{31} 日 & 
    \hspace{0.3em}$\Rightarrow$\hspace{0.3em} \JustifiedLabel{4em}{提出日} & 
    \hspace{0.3em} 令和 \UnderlineBox[0.65cm]{7} 年 \UnderlineBox[0.65cm]{12} 月 \UnderlineBox[0.65cm]{16} 日 \\
    ( \JustifiedLabel{6em}{再提出期限} & 
    \hspace{0.3em} 令和 \UnderlineBox[0.65cm]{} 年 \UnderlineBox[0.65cm]{} 月 \UnderlineBox[0.65cm]{} 日 & 
    \hspace{0.3em}$\Rightarrow$\hspace{0.3em} \JustifiedLabel{5em}{再提出日} & 
    \hspace{0.3em} 令和 \UnderlineBox[0.65cm]{} 年 \UnderlineBox[0.65cm]{} 月 \UnderlineBox[0.65cm]{} 日 ) \\
\end{tabular}
\vspace{40mm}

\renewcommand{\arraystretch}{1.6}
\begin{center}
\begin{tabular}{|>{\centering\arraybackslash}m{2.4cm}|>{\raggedright\arraybackslash}m{12.1cm}|>{\centering\arraybackslash}m{2.4cm}|}
\hline
\multicolumn{2}{|c|}{\JustifiedLabel{11em}{評 価 項 目}} & \JustifiedLabel{4em}{評 価} \\
\hline
\multirow{3}{*}{\parbox[c][5.0em][c]{2.4cm}{\centering\shortstack{\large\bfseries 実 習\\[0.3em]\large\bfseries 評 価}}} 
 & (1) 自ら積極的に実験に取り組めた &  \\ \cline{2-3}
 & (2) 実験装置を適切に使用でき,正確に実験を行なえた &  \\ \cline{2-3}
 & (3) グループ内で協力的に実験が行なえた &  \\
\hline
\multirow{4}{*}{\parbox[c][6.0em][c]{2.4cm}{\centering\shortstack{\large\bfseries 報告書\\[0.3em]\large\bfseries 評 価}}} 
 & (1) 結果のまとめかた(図表を含む) &  \\ \cline{2-3}
 & (2) 結果に対する考察 &  \\ \cline{2-3}
 & (3) 報告事項/課題(正しい解答や適切な引用など) &  \\ \cline{2-3}
 & (4) 報告書としての体裁が整っているか &  \\
\hline
\end{tabular}
\end{center}
\vspace*{0mm}
\enlargethispage{30pt}
\clearpage

% /////////////////////////////////////////////
% 本文 (Main Body)
% /////////////////////////////////////////////

\restoregeometry 
\setcounter{page}{1}
\pagestyle{plain} 

\section{目的}
本実験は,アナログ信号をデジタル信号に変換して伝送し,再びアナログ信号へ復元する PCM(Pulse Code Modulation)通信方式の原理を理解することを目的とする.具体的には,回路シミュレータを用いた回路設計および過渡解析を通じて,標本化(サンプリング),量子化,符号化,および復号化といった一連の信号処理過程とその動作特性を習得した.

\section{原理}

\subsection{パルス符号変調・復調の基本構成}
PCM 通信方式の基本ブロック線図を\cref{fig:block_diagram}に示す.
送信側の PCM 変調回路は以下の要素で構成される.
\begin{enumerate}
    \item \textbf{標本化回路(Sample \& Hold)}: 入力された連続アナログ信号を,一定の時間間隔(標本化周期)でサンプリングし,その瞬時値を保持する.
    \item \textbf{量子化回路}: 標本化された連続的な振幅値を,離散的なレベル(量子化レベル)に近似・変換する.
    \item \textbf{符号化回路}: 量子化されたレベルを 2 進数のデジタル符号に変換する.
    \item \textbf{送信回路}: デジタル符号に,同期信号や多重化のためのチャンネル信号を付加して送出する.
\end{enumerate}

受信側の PCM 復調回路は,以下の逆処理を行う.
\begin{enumerate}
    \item \textbf{分離回路}: 受信信号から PCM データと同期信号・チャンネル信号を分離する.
    \item \textbf{符号変換回路}: シリアル信号として送られてきた PCM データをパラレル信号に変換する.
    \item \textbf{D/A 変換器}: パラレルデジタル信号を対応するアナログ電圧レベル(階段波)に戻す.
    \item \textbf{ローパスフィルタ(LPF)}: D/A 変換後の階段状波形に含まれる高調波成分を除去し,元の連続アナログ波形を平滑化・復元する.
\end{enumerate}

\begin{figure}[H]
    \centering
    \includegraphics[width=0.8\linewidth]{image/img_000.png}
    \caption{PCM 通信の基本方式構成図}
    \label{fig:block_diagram}
\end{figure}

\subsection{タイミングパルス発生回路}
本実験装置全体の動作基準となるタイミングパルス発生回路を\cref{fig:timing_circuit}に示す.
基本クロックには U1 による \SI{125}{\kilo\hertz} のパルスが用いられる.同期カウンタ U3(SN74LS162)はこのクロックを計数し,所定のカウント数に達すると RCO(Ripple Carry Output)端子よりパルスを出力する.
RCO 出力は後段の JK フリップフロップ(U5)のクロック入力となる.この JK フリップフロップは $J=K=\rom{High}$ に設定されているため,RCO パルスが入力されるたびに出力 $Q$ の状態が反転(トグル動作)する.この出力 $Q$ はシステム全体の制御信号 T1 として利用される.

\begin{figure}[H]
    \centering
    \includegraphics[width=0.9\linewidth]{image/img_001.png}
    \caption{タイミングパルス発生回路と論理動作}
    \label{fig:timing_circuit}
\end{figure}

\subsection{切換回路(時分割多重化)}
複数のアナログ信号を同一回線で伝送するための入力切換回路を\cref{fig:switching_circuit}に示す.これは時分割多重化(TDM)の基本となる回路である.
アナログスイッチ SW3 および SW4 は,制御クロック信号によって交互に ON/OFF を繰り返す.
\begin{itemize}
    \item クロックが \rom{Low} の区間: SW3 が ON,SW4 が OFF となり,Ch.1(\SI{50}{\hertz})の信号が出力される.
    \item クロックが \rom{High} の区間: SW3 が OFF,SW4 が ON となり,Ch.2(\SI{100}{\hertz})の信号が出力される.
\end{itemize}
この動作により,2 つの異なる信号源が時間的に切り分けられて後段の標本化回路へ送られる.

\begin{figure}[H]
    \centering
    \includegraphics[width=0.4\linewidth]{image/img_002.png}
    \caption{切換回路の構成}
    \label{fig:switching_circuit}
\end{figure}

\subsection{標本化回路(サンプル \& ホールド回路)}
標本化回路は,アナログ信号の瞬時値を捉え(サンプル),A/D 変換が完了するまでその値を保持(ホールド)する機能を持つ.回路図を\cref{fig:sample_hold}に示す.
動作原理は以下の通りである.
\begin{enumerate}
    \item クロックパルスが \rom{High} の期間,アナログスイッチ SW1 が ON となる.このとき,コンデンサ $C_1$ は入力電圧まで急速に充電され,端子電圧は入力信号の瞬時値に追従する(サンプルモード).
    \item クロックが \rom{Low} になると SW1 は OFF となり,入力から切り離される.オペアンプの入力インピーダンスは極めて高いため,$C_1$ に蓄えられた電荷は放電されず,電圧が一定に保たれる(ホールドモード).
    \item オペアンプはボルテージフォロワとして動作し,インピーダンス変換を行って $V_{\rom{out}}$ へ出力する.
\end{enumerate}
なお,図中の $R_1$ (\SI{1}{\mega\ohm})は回路シミュレーション上の安定動作等のために付加されているが,原理的には開放とみなして差し支えない.\cref{fig:sample_hold_r1}に示すように,$R_1$ の有無は出力波形に影響を与えない.

\begin{figure}[H]
    \centering
    \includegraphics[width=0.6\linewidth]{image/img_003.png}
    \caption{標本化回路(サンプル \& ホールド回路)}
    \label{fig:sample_hold}
\end{figure}

\begin{figure}[H]
    \centering
    \includegraphics[width=0.9\linewidth]{image/img_004.png}
    \caption{サンプル \& ホールド回路における抵抗 $R_1$ の影響確認}
    \label{fig:sample_hold_r1}
\end{figure}

\subsection{シフトレジスタ(パラレル - シリアル変換)}
並列(パラレル)データを直列(シリアル)データに変換して伝送路へ送り出すための回路を\cref{fig:shift_register}に示す.
A/D 変換後の 8 ビットパラレルデータは,ロード信号(本実験では T3)のタイミングでシフトレジスタ U25(74199)に取り込まれる.その後,システムクロックに従って 1 ビットずつ順次シフトされ,OUTPUT 端子よりシリアル信号として出力される.

\begin{figure}[H]
    \centering
    \includegraphics[width=0.6\linewidth]{image/img_005.png}
    \caption{シフトレジスタによる P-S 変換}
    \label{fig:shift_register}
\end{figure}

\subsection{波形合成・分離回路}
伝送路における信号の多重化と分離の原理を確認するための回路を\cref{fig:synth_sep_circuit}に示す.本実験では,実際のデータではなくクロック信号 T2,T3 を用いて擬似的に動作を確認する.
\begin{itemize}
    \item \textbf{合成}: T3 を反転させた信号と T2 を加算器(Ideal Adder)で合成し,$V_{\rom{mix}}$ を生成する.これにより,正極性と負極性に異なる情報を持たせた 3 値の伝送波形が模擬される.
    \item \textbf{分離}: 受信側では,ダイオードの整流作用と反転増幅器を利用して,正電圧部分と負電圧部分を分離する.
\end{itemize}
分離回路の等価回路を\cref{fig:sep_operation}に示す.$V_{\rom{mix}}$ の極性に応じて,上側または下側のダイオードが導通・遮断することで,特定の信号成分のみが VT2,VT3 端子に現れる仕組みである.

\begin{figure}[H]
    \centering
    \includegraphics[width=0.6\linewidth]{image/img_006.png}
    \caption{波形合成・分離回路}
    \label{fig:synth_sep_circuit}
\end{figure}

\begin{figure}[H]
    \centering
    \includegraphics[width=0.5\linewidth]{image/img_007.png}
    \caption{波形合成・分離回路の等価回路と動作状態}
    \label{fig:sep_operation}
\end{figure}

\subsection{A/D・D/A 変換回路およびローパスフィルタ}
全体の信号処理回路を\cref{fig:ad_da_circuit}に示す.
入力された交流信号 $V_{\rom{G1}}$ は,単電源動作の A/D コンバータ(U10)に入力するため,加算器により直流バイアス(オフセット電圧 \SI{6}{\volt})が重畳される.
D/A 変換器(MV95308)によりアナログ電圧に戻された信号 $V_{\rom{DA}}$ は,再び減算器を通してオフセット分が除去される.この時点での波形は階段状であるため,最終段のローパスフィルタ(LPF)により高調波成分を除去し,滑らかな正弦波 $V_{\rom{復調}}$ を得る.

\begin{figure}[H]
    \centering
    \includegraphics[width=0.6\linewidth]{image/img_008.png}
    \caption{A/D・D/A 変換回路と復調用 LPF}
    \label{fig:ad_da_circuit}
\end{figure}

\section{実験方法}
シミュレーションソフト「TINA-TI」を用いて以下の手順で実験を行った.
\begin{enumerate}
    \item \cref{fig:timing_circuit}に示したタイミングパルス発生回路を作成した.クロック源および電圧ピン部分は,マクロ化のために専用の入出力ピンに置換し,\cref{fig:macro_circuit}のような回路ブロックとして保存した.
    \item \cref{fig:timing_circuit}から\cref{fig:ad_da_circuit}に示す各機能ブロック(切換回路,標本化回路,シフトレジスタ,合成・分離回路,AD/DA 回路)を個別に回路図入力した.
    \item 各回路に対して「過渡解析(Transient Analysis)」を実行し,入出力波形および主要ノードの電圧変化を観測した.
    \item シフトレジスタ回路においては,入力のディップスイッチ(SW-HL)の設定を変化させ,パラレル入力とシリアル出力の対応関係を確認した.
\end{enumerate}

\begin{figure}[H]
    \centering
    \includegraphics[width=0.6\linewidth]{image/img_009.png}
    \caption{マクロ化を行ったタイミングパルス発生回路}
    \label{fig:macro_circuit}
\end{figure}

\section{使用機器}
本実験では,シミュレーションソフト「TINA-TI」を用いた.TINA は DesignSoft の製品であり,TEXAS INSTRUMENTS 社専用で,無償で提供されている.

\section{実験結果および考察}

\subsection{タイミングパルス発生回路}
タイミングパルス発生回路の過渡解析結果を\cref{fig:res_timing}に示す.
解析結果より,カウンタのキャリー出力 RCO が発生するタイミングで,T1 信号が \rom{Low} から \rom{High},あるいは \rom{High} から \rom{Low} へと反転していることが確認できる.これは JK フリップフロップがトグル動作を行っていることを示しており,設計通りの分周動作およびタイミング生成が行われていると判断できる.

\begin{figure}[H]
    \centering
    \includegraphics[width=0.8\linewidth]{image/img_010.png}
    \caption{タイミングパルス発生回路の動作波形}
    \label{fig:res_timing}
\end{figure}

\subsection{切換回路}
切換回路のシミュレーション結果を\cref{fig:res_switching}に示す.
出力 $V_{\rom{out}}$ の波形を確認すると,制御クロック $V_{\rom{clock}}$ のレベル変化に同期して,Ch.1(\SI{50}{\hertz})と Ch.2(\SI{100}{\hertz})の信号が交互に切り替わって出力されている.これにより,2 つのアナログ信号が時分割で多重化されていることが確認された.

\begin{figure}[H]
    \centering
    \includegraphics[width=0.7\linewidth]{image/img_011.png}
    \caption{切換回路による信号多重化の波形}
    \label{fig:res_switching}
\end{figure}

\subsection{標本化回路(サンプル \& ホールド回路)}
標本化回路の応答波形を\cref{fig:res_sample}に示す.
$V_{\rom{out}}$ は,$V_{\rom{clock}}$ が \rom{High} の区間では入力信号 $V_{\rom{G1}}$ に追従し,$V_{\rom{clock}}$ が \rom{Low} になるとその直前の電圧値を一定期間保持(ホールド)している.
ホールド期間中の電圧降下(ドループ)はほとんど観測されなかった.これは,次段のバッファアンプの入力インピーダンスが十分に高く,ホールドコンデンサからのリーク電流が無視できるほど小さいためであると考えられる.

\begin{figure}[H]
    \centering
    \includegraphics[width=0.7\linewidth]{image/img_012.png}
    \caption{標本化回路の入出力波形}
    \label{fig:res_sample}
\end{figure}

\subsection{シフトレジスタ}
パラレル - シリアル変換の動作確認結果を\cref{fig:res_shift}に示す.
入力スイッチ条件(1, 0, 0, 0, 0, 0, 1, 1)に対し,OUTPUT 端子からはロード信号 T3 の入力後,クロックに同期して「High, High, High, Low...」といった対応するシリアルパルス列が出力された.データの並び順が保持されたまま直列変換されており,正常に動作している.

\begin{figure}[H]
    \centering
    \includegraphics[width=0.7\linewidth]{image/img_013.png}
    \caption{シフトレジスタの出力波形}
    \label{fig:res_shift}
\end{figure}

\subsection{波形合成・分離回路}
波形合成および分離の様子を\cref{fig:res_synth}に示す.
合成波形 $V_{\rom{mix}}$ は,正側に T2 由来のパルス,負側に T3 由来のパルスを持つ 3 値波形となっている.
分離後の波形 VT2 および VT3 を確認すると,VT2 には T3 のタイミングの信号が,VT3 には T2 のタイミングの信号がそれぞれ正極性のパルスとして復元されている.ダイオードの整流作用とオペアンプによる極性反転・増幅が理論通り機能し,信号の分離に成功していることがわかる.

\begin{figure}[H]
    \centering
    \includegraphics[width=0.7\linewidth]{image/img_014.png}
    \caption{波形合成・分離回路の各部波形}
    \label{fig:res_synth}
\end{figure}

\subsection{A/D・D/A 変換および復調}
総合的な変復調動作の結果を\cref{fig:res_ad_da}に示す.
\begin{enumerate}
    \item $V_{\rom{in}}$ は入力信号 $V_{\rom{G1}}$ に対して \SI{+6}{\volt} のバイアスがかかっており,単極性の A/D 変換範囲に収まっている.
    \item D/A 変換直後の $V_{\rom{DA}}$ は,入力正弦波の形状に沿った離散的な階段波形となっている.ステップの切り替わりはサンプリング周期と一致している.
    \item LPF 通過後の $V_{\rom{復調}}$ は,階段状の段差(量子化ノイズおよびサンプリング周波数成分)が平滑化され,元の正弦波に近い連続波形が得られている.
\end{enumerate}
以上より,PCM 方式による信号のデジタル化,伝送,およびアナログ復元の一連のプロセスが正しく機能していると結論付けられる.

\begin{figure}[H]
    \centering
    \includegraphics[width=0.7\linewidth]{image/img_015.png}
    \caption{AD・DA 変換および LPF 通過後の復調波形}
    \label{fig:res_ad_da}
\end{figure}

\section{報告事項}

\subsection{標本化定理(サンプリング定理)について}
標本化定理とは,連続的なアナログ信号を離散的なデジタル信号に変換する際,元の信号を情報欠損なく再現するために必要な条件を示した定理である.
帯域が $0 \sim f_0$ に制限された信号 $g(t)$ を標本化する場合,標本化周波数 $f_s$ は信号に含まれる最高周波数成分 $f_0$ の 2 倍以上である必要がある.すなわち,
\begin{equation}
    f_s \ge 2 f_0
    \label{eq:sampling_condition}
\end{equation}
この条件を満たすとき,標本化されたパルス列から理想的なローパスフィルタ(遮断周波数 $f_0$)を通すことで,元の信号 $g(t)$ を完全に復元できる.これを染谷・シャノン(Shannon)の標本化定理という\cite{ref2,ref15}.

数式的には,元の信号 $g(t)$ は標本値 $g(nT)$ と sinc 関数(標本化関数)を用いた補間公式によって以下のように表される.
\begin{equation}
    g(t) = \sum_{n=-\infty}^{\infty} g(nT) \frac{\sin \pi f_s (t - nT)}{\pi f_s (t - nT)}
    \label{eq:reconstruction}
\end{equation}
ここで $T = 1/f_s$ は標本化周期である.物理的には,\cref{fig:sampling_theory}に示すように,各標本点におけるインパルス応答(sinc 関数)を重ね合わせることで,元の連続波形が滑らかに結ばれて再生されることを意味する\cite{ref1}.
なお,標本化周波数が $2f_0$ を下回ると,元の信号の高周波成分が低周波側に折り返されて雑音となる「エイリアシング(折り返し誤差)」が発生し,元の波形を正しく復元できなくなる.

\begin{figure}[H]
    \centering
    \includegraphics[max size={0.6\linewidth}{0.5\textheight},keepaspectratio]{image/img_016.png}
    \caption{標本化パルス列からの信号復元(sinc 関数による補間)\cite{ref1}}
    \label{fig:sampling_theory}
\end{figure}

\subsection{本実験における量子化レベル数}
量子化とは,標本化された信号の振幅値を離散的な数値に丸める操作である\cite{ref3}.
本実験で使用した A/D コンバータおよびシステムは 8 ビット(bit)処理を行っている.$n$ ビットの量子化における量子化レベル数(ステップ数)$L$ は $2^n$ で表されるため,本実験における量子化レベルは,
\begin{equation}
    L = 2^8 = 256
\end{equation}
となり,入力信号の振幅範囲を 256 段階で表現している.

\subsection{理想ダイオードの特性}
理想ダイオード(Ideal Diode)とは,整流作用において理想的な特性を持つ素子と定義される.具体的には以下の特性を有する\cite{ref4}.
\begin{itemize}
    \item \textbf{順方向バイアス時}: 抵抗値がゼロ(短絡状態)であり,順方向電圧降下が \SI{0}{\volt} である.
    \item \textbf{逆方向バイアス時}: 抵抗値が無限大(開放状態)であり,漏れ電流が一切流れない.
    \item \textbf{スイッチング特性}: ON/OFF の切り替わり時間がゼロであり,遅延がない.
\end{itemize}
実際のダイオードでは約 \SI{0.6}{\volt} ~ \SI{0.7}{\volt} の順方向電圧降下や逆方向漏れ電流が存在するが,シミュレーション上の理想素子やオペアンプを用いた理想ダイオード回路では,上記の理想特性を実現できる.

\subsection{ローパスフィルタの遮断周波数の導出}
\cref{fig:ad_da_circuit}で使用されたローパスフィルタは,オペアンプを用いた 2 次のサレン・キー(Sallen-Key)型 LPF である.回路図およびインピーダンス配置を\cref{fig:sallen_key}に示す.

\begin{figure}[H]
    \centering
    \includegraphics[max size={0.6\linewidth}{0.45\textheight},keepaspectratio]{image/img_017.png}
    \caption{サレン・キー型ローパスフィルタの回路構成}
    \label{fig:sallen_key}
\end{figure}

図中の各ノードにおけるキルヒホッフの電流則(KCL)より,以下の関係式が成り立つ.
\begin{equation}
    \frac{V_{\rom{in}} - V_{\rom{x}}}{R_1} + \frac{V_{\rom{out}} - V_{\rom{x}}}{Z_{C1}} + \frac{V_{\rom{out}} - V_{\rom{x}}}{R_2} = 0
\end{equation}
ただし,オペアンプの入力端子間はイマジナリショートにより同電位とみなせるため,非反転入力端子の電圧は $V_{\rom{out}}$ に等しい.
詳細な導出は割愛するが,この回路の伝達関数 $G(s) = V_{\rom{out}}/V_{\rom{in}}$ は一般に次式で与えられる\cite{ref10}.
\begin{equation}
    G(s) = \frac{\frac{1}{R_1 R_2 C_1 C_2}}{s^2 + s \left( \frac{1}{R_1 C_1} + \frac{1}{R_2 C_1} \right) + \frac{1}{R_1 R_2 C_1 C_2}}
    \label{eq:lpf_transfer}
\end{equation}
これは標準的な 2 次遅れ系の形式 $G(s) = \frac{\omega_n^2}{s^2 + 2\zeta\omega_n s + \omega_n^2}$ と対応する.ここで固有角周波数 $\omega_n$ が遮断角周波数 $\omega_c$ となる.係数比較により,
\begin{equation}
    \omega_c = \frac{1}{\sqrt{R_1 R_2 C_1 C_2}}
\end{equation}
したがって,遮断周波数 $f_c$ は次式となる.
\begin{equation}
    f_c = \frac{\omega_c}{2\pi} = \frac{1}{2\pi \sqrt{R_1 R_2 C_1 C_2}}
    \label{eq:fc_formula}
\end{equation}

実験回路の定数 $R_1 = R_2 = \SI{50}{\kilo\ohm}$,$C_1 = \SI{2.3}{\nano\farad}$,$C_2 = \SI{1.1}{\nano\farad}$ を\cref{eq:fc_formula}に代入して計算する.
\begin{align}
    f_c &= \frac{1}{2\pi \sqrt{(50 \times 10^3)^2 \times (2.3 \times 10^{-9}) \times (1.1 \times 10^{-9})}} \nonumber \\
        &= \frac{1}{2\pi \times (50 \times 10^3) \times \sqrt{2.53} \times 10^{-9}} \nonumber \\
        &\approx \frac{1}{2\pi \times 50 \times 10^3 \times 1.59 \times 10^{-9}} \nonumber \\
        &\approx \frac{1}{314.16 \times 50 \times 1.59 \times 10^{-6}} \nonumber \\
        &\approx \SI{2000}{\hertz}
\end{align}
以上より,本回路の遮断周波数は約 \SI{2.0}{\kilo\hertz} であると求められる.

\subsection{PCM 通信方式の利点}
PCM 通信方式が従来のアナログ変調方式と比較して優れている主な点を以下に挙げる\cite{ref7}.
\begin{enumerate}
    \item \textbf{耐雑音性}: 受信側では「0」か「1」かのパルスの有無さえ判別できればよいため,伝送路で重畳されたアナログ的な雑音や波形歪みの影響を受けにくい.
    \item \textbf{再生中継による品質維持}: リピータ(中継器)でパルスの整形・再タイミング(リシェイピング,リタイミング,リジェネレーション)を行うことで,雑音を累積させずに長距離伝送が可能である.
    \item \textbf{多重化の容易性}: 信号がデジタルデータとして扱われるため,時分割多重(TDM)による多重化が容易であり,コンピュータデータや画像など種類の異なる情報の統合伝送に適している.
    \item \textbf{伝送路の柔軟性}: 光ファイバや無線など,多様な伝送媒体に対して共通のデジタル変調技術を適用できる.
\end{enumerate}

\subsection{PCM 通信における多重化の原理}
PCM 通信で多重化が可能となる主な理由は,信号を「標本化」することにある.
連続的なアナログ信号を標本化すると,信号は時間軸上で離散的なパルス列(PAM 信号)となる.パルスとパルスの間には信号が存在しない空き時間が生じるため,\cref{fig:tdm}に示すように,この隙間に別のチャンネルのパルスを挿入することが可能となる.
これを時分割多重化(TDM: Time Division Multiplexing)という.送信側でタイミングを同期させて複数の信号を順次送り出し,受信側で同じタイミングで分離することで,1 つの伝送路で複数の通信を同時に行うことができる\cite{ref8}.

\begin{figure}[H]
    \centering
    \includegraphics[max size={0.6\linewidth}{0.45\textheight},keepaspectratio]{image/img_018.png}
    \caption{時分割多重化(TDM)の概念図\cite{ref8}}
    \label{fig:tdm}
\end{figure}

\begin{thebibliography}{99}
\raggedright
\bibitem{ref1} 金子尚志:「PCM 通信の技術」,産報出版株式会社,pp.17-19(1977)
\bibitem{ref2} 羽鳥光俊:「わかりやすい通信工学」,コロナ社,p.21(2012)
\bibitem{ref3} 金子尚志:「PCM 通信の技術」,産報出版株式会社,p.27(1977)
\bibitem{ref4} 高崎和之:「基本からわかる電子回路」,株式会社ナツメ社,pp.36-37(2021)
\bibitem{ref5} 「Electrical Information」,\url{https://detail-infomation.com/sallen-key-low-pass-filter/}(2023 年 12 月 14 日参照)
\bibitem{ref6} 寺島一彦,兼重明宏:「制御工学 技術者のための,理論・設計から実装まで」,実教出版株式会社,pp.121-128(2019)
\bibitem{ref7} 金子尚志:「PCM 通信の技術」,産報出版株式会社,pp.14-15(1977)
\bibitem{ref8} 金子尚志:「PCM 通信の技術」,産報出版株式会社,pp.10-11(1977)
\bibitem{ref9} 金子尚志:「PCM 通信の技術」,産報出版株式会社,p.10(1977)
\bibitem{ref10} 石井 聡:TNJ-045:LTspiceでサレン・キー型フィルタ(第2回)「教科書で見る理解不能な伝達関数の式と実際の回路との関係はどうなるのか(後編)」, Analog Devices Japan, 2019-02-04, \url{https://www.analog.com/jp/resources/technical-articles/tnj-045.html}(参照 2025 年 12 月 19 日)
\bibitem{ref11} 石井 聡:TNJ-044:LTspiceでサレン・キー型フィルタ(第1回)「教科書で見る理解不能な伝達関数の式と実際の回路との関係はどうなるのか(前編)」, Analog Devices Japan, 2018-10-18, \url{https://www.analog.com/jp/resources/technical-articles/tnj-044.html}(参照 2025 年 12 月 19 日)
\bibitem{ref12} 石井 聡:TNJ-046:LTspiceでサレン・キー型フィルタ(第3回)「理解不能な伝達関数多項式と実回路の関係をより深く考える(第3回)」, Analog Devices Japan, 2019, \url{https://www.analog.com/jp/resources/technical-articles/tnj-046.html}(参照 2025 年 12 月 19 日)
\bibitem{ref13} 石井 聡:TNJ-047:LTspiceでサレン・キー型フィルタ(第4回)「理解不能な伝達関数多項式と実回路の関係をより深く考える(第4回)」, Analog Devices Japan, 2019, \url{https://www.analog.com/jp/resources/technical-articles/tnj-047.html}(参照 2025 年 12 月 19 日)
\bibitem{ref14} 石井 聡:TNJ-048:LTspiceでサレン・キー型フィルタ(第5回)「サレン・キー型LPFの素子定数と伝達関数の関係を考える」, Analog Devices Japan, 2019-02-05, \url{https://www.analog.com/jp/resources/technical-articles/tnj-048.html}(参照 2025 年 12 月 19 日)
\bibitem{ref15} 村田昇:「信号処理論 講義スライド(標本化)」, \url{https://noboru-murata.github.io/signal-processing/pdfs/slide10.pdf}(参照 2025 年 12 月 19 日)
\end{thebibliography}

\end{document}