% !TEX program = lualatex
%==============================================================================
% 電子工学(5E) 後期中間到達度試験 統合対策資料 (完全版・改訂用語解説付)
%==============================================================================
% 制作:試験対策委員会 + AI特別顧問
% 内容:試験範囲ポイント①〜⑲の完全網羅解説 (深堀り版) + 2024/2023年度過去問 超詳細解答 + 記述/計算完全対策
%==============================================================================

\documentclass[a4paper,11pt]{ltjsarticle}

%------------------------------------------------------------------------------
% パッケージ設定
%------------------------------------------------------------------------------
\usepackage{lmodern}              % フォントサイズ警告を解消(スケーラブルフォント)
\usepackage[left=15mm,right=15mm,top=20mm,bottom=20mm]{geometry}
\usepackage{amsmath, amssymb, bm} % 数式

%--- siunitxとphysicsの競合対策 ---
\usepackage{physics}              % 微分などの記述簡略化
\usepackage{siunitx}              % 単位の統一

% siunitxがphysicsパッケージを検知した際の警告を抑制
\ExplSyntaxOn
\msg_redirect_name:nnn { siunitx } { physics-package } { none }
\ExplSyntaxOff

% \qtyコマンドの競合回避(siunitxの機能を優先)
\AtBeginDocument{\RenewCommandCopy\qty\SI \let\qty\SI}
%--------------------------------

\usepackage{graphicx}             % 画像
\usepackage{booktabs}             % 表
\usepackage{tcolorbox}            % 枠線・ボックス
\tcbuselibrary{skins, breakable, theorems}
\usepackage{enumitem}             % リスト設定
\usepackage{hyperref}             % リンク
\usepackage{cancel}               % 計算過程のキャンセル線

\hypersetup{colorlinks=true, linkcolor=blue, urlcolor=cyan}

%------------------------------------------------------------------------------
% 定数・コマンド定義
%------------------------------------------------------------------------------
\newcommand{\EF}{E_{\mathrm{F}}} % フェルミ準位
\newcommand{\kB}{k}              % ボルツマン定数

% \unitコマンドの定義(単位を角括弧で囲むスタイル)
\AtBeginDocument{%
    \ifcsname unit\endcsname
        \expandafter\let\csname siunitxUnit\endcsname\unit
    \else
        \providecommand{\siunitxUnit}[1]{\,[\mathrm{#1}]}%
    \fi
    \renewcommand{\unit}[1]{\,[\mathrm{#1}]}%
}

% 試験用物理定数 (計算時はこれを使用)
\newcommand{\ValE}{1.60 \times 10^{-19}}     % 素電荷 [C]
\newcommand{\ValK}{1.38 \times 10^{-23}}     % ボルツマン定数 [J/K]
\newcommand{\ValA}{1.20 \times 10^6}         % リチャードソン定数 [A/m2K2]
\newcommand{\ValH}{6.63 \times 10^{-34}}     % プランク定数 [Js]
\newcommand{\ValC}{3.00 \times 10^8}         % 光速 [m/s]
\newcommand{\ValMe}{9.11 \times 10^{-31}}    % 電子質量 [kg]

% 解説用ボックス定義
\newtcolorbox{pointbox}[1]{colback=blue!5!white, colframe=blue!75!black, title=\textbf{#1}, breakable}
\newtcolorbox{termbox}[1]{colback=yellow!5!white, colframe=orange!80!black, title=\textbf{【用語解説】#1}, breakable}
\newtcolorbox{warnbox}[1]{colback=red!5!white, colframe=red!75!black, title=\textbf{【注意】#1}, breakable}
\newtcolorbox{calcbox}[1]{colback=gray!5!white, colframe=black, title=\textbf{計算プロセス詳細: #1}, breakable}
\newtcolorbox{derivationbox}[1]{colback=green!5!white, colframe=green!40!black, title=\textbf{【重要導出】#1}, breakable}

\title{\textbf{電子工学(5E) 試験対策 統合完全版・改}}
\author{詳細解説・計算過程完全記述・用語完全定義版}
\date{}

\begin{document}

\maketitle

\begin{abstract}
本資料は、電子工学の試験対策における**「これ一冊で完結する」**バイブルを目指したものである。
数式の変形過程はもちろん、問題文に登場する「モリブデン線」「鏡像力」といった専門用語についても、初学者がつまずかないよう詳細な定義と物理的イメージを付記した。
教科書やノートを開かずとも、この資料だけで理解と演習が完了するように設計されている。
\end{abstract}

\tableofcontents
\newpage

%==============================================================================
\part{試験範囲ポイント完全解説 (①〜⑲)}
%==============================================================================
\noindent
試験範囲の19ポイントについて、単なる用語説明に留まらず、「なぜそうなるのか」「試験でどう問われるか」まで踏み込んで解説する。

\section{エネルギーバンドと電子放出の基礎 (①〜⑦)}

\subsection*{① 価電子帯、禁制帯、伝導帯とはなにか}
固体中の電子が取りうるエネルギー準位の構造(バンド構造)に関する定義である。

\begin{pointbox}{バンド構造の3要素}
\begin{description}
    \item[価電子帯 (Valence Band)]
    原子核に束縛されている電子(価電子)が充満しているエネルギー帯。
    \begin{itemize}
        \item \textbf{直感的イメージ}: 満員電車。電子がぎゅうぎゅう詰めで身動きが取れない状態。
        \item \textbf{特徴}: ここにある電子は電気伝導(電流)に寄与しない。
        \item \textbf{温度の影響}: 絶対零度($0\unit{K}$)では完全に満たされているが、温度が上がると一部の電子がエネルギーを得て上の階(伝導帯)へ移動し、空席(正孔)ができる。
    \end{itemize}
    
    \item[禁制帯 (Forbidden Band / Band Gap)]
    量子力学的な制約により、電子が存在することのできないエネルギー領域。
    \begin{itemize}
        \item \textbf{直感的イメージ}: 建物の1階(価電子帯)と2階(伝導帯)の間にある「吹き抜け空間」。ここに人は立てない。
        \item \textbf{バンドギャップ $E_g$}: この幅が広いと絶縁体、適度だと半導体、重なっていると金属(導体)になる。
    \end{itemize}
    
    \item[伝導帯 (Conduction Band)]
    原子の束縛を離れ、結晶内を自由に動き回れる電子(自由電子)が存在するエネルギー帯。
    \begin{itemize}
        \item \textbf{直感的イメージ}: ガラガラの2階フロア。電子は自由に走り回れる。
        \item \textbf{電気伝導}: ここに電子が存在して初めて電流が流れる。
    \end{itemize}
\end{description}
\end{pointbox}

\textbf{※金属(導体)の特異性}:
金属では「価電子帯」と「伝導帯」が重なっているか、価電子帯自体が完全に埋まっておらず空き席がある状態である。そのため、わずかなエネルギー(室温の熱など)で電子が自由電子となり、高い導電性を示す。

\subsection*{② 外部エネルギーの入射により電子が放出されるしくみ}
通常、電子は物質内部の「ポテンシャルの井戸(エネルギーの低い安定した場所)」に閉じ込められている。これを脱出させる(電子放出)ためのプロセスの本質は以下の通りである。

\begin{enumerate}
    \item \textbf{定常状態}: 電子は通常、エネルギーの低い状態(価電子帯やフェルミ準位付近)にある。
    \item \textbf{外部励起}: 外部からエネルギー $\Delta E$ を与える。エネルギー源の種類により名称が変わる。
    \begin{itemize}
        \item 熱エネルギー ($kT$) $\to$ \textbf{熱電子放出} (ヒーターで加熱)
        \item 光エネルギー ($h\nu$) $\to$ \textbf{光電子放出} (光を当てる)
        \item 強電界 ($E$) $\to$ \textbf{電界放出} (強い電圧で引っ張る)
        \item 電子衝突 ($1/2 mv^2$) $\to$ \textbf{二次電子放出} (別の電子をぶつける)
    \end{itemize}
    \item \textbf{脱出条件}: 電子の持つ全エネルギーが、表面の障壁高さ(真空準位 $E_{vac}$)を超えたとき、電子は表面を突き抜けて真空中に飛び出す。
\end{enumerate}

\subsection*{③ 金属内電子が金属外に飛び出さない理由}
「なぜ常温の金属を置いておくだけでは電子が飛び出さないのか?」という問いへの物理的解答。

\begin{itemize}
    \item \textbf{電位障壁 (Potential Barrier)}: 金属表面には、電子を閉じ込めるエネルギーの壁が存在する。
    \item \textbf{鏡像力 (Image Force)}: 電子が金属表面からわずかに外に出ようとすると、金属表面に正電荷(ホール)が誘導される。クーロン力により、正電荷が電子(負電荷)を引き戻そうとする力が働く。これが壁の正体である。
    \item \textbf{エネルギー不足}: 室温 ($300\unit{K}$) の熱エネルギー $kT \approx 0.026\unit{eV}$ は、一般的な金属の仕事関数 $\phi \approx 4\sim5\unit{eV}$ に比べてあまりに小さいため、確率的に飛び出す電子はほぼ皆無である。
\end{itemize}

\subsection*{④ 電子放出の共通的基礎(各用語の厳密な定義と関係式)}
記述問題や計算問題の基礎となる定義式。図をイメージしながら理解すること。

\begin{tcolorbox}[colback=white, colframe=black, title=\textbf{重要公式と定義}]
\begin{equation}
    \phi = W - \EF \quad \Longleftrightarrow \quad W = \EF + \phi
\end{equation}
\end{tcolorbox}

\begin{termbox}{用語解説}
    \begin{description}
        \item[真空準位 (Vacuum Level, $E_{vac}$)]
        真空中に静止した「自由な電子」が持つエネルギー。これをエネルギー基準 ($E=0$) とする場合と、ここを脱出ゴールとする場合がある。
        
        \item[全障壁 $W$ (Potential Barrier Height)]
        金属の底(電子が存在しうる最低エネルギー)から、真空準位までの高さ。電子が脱出するために超えなければならない「壁の全高」。
        
        \item[フェルミ準位 $\EF$ (Fermi Level)]
        \textbf{電子のエネルギーの基準点}。
        絶対零度において、電子が詰まっている「水面」の高さのこと。これより下は満員、上は空っぽ。
        
        \item[仕事関数 $\phi$ (Work Function)]
        「フェルミ準位にある電子(表面付近のエリート電子)」を「真空準位(外の世界)」まで引き上げるために最低限必要な追加エネルギー。
        \[ \phi [\unit{eV}] \quad \text{または} \quad e\phi [\unit{J}] \]
        ※材料固有の値であり、表面の状態(汚れや酸化)によって変化する。
    \end{description}
\end{termbox}

\begin{warnbox}{単位の混同に注意}
仕事関数 $\phi$ は通常 $\unit{eV}$ (電子ボルト) で与えられる。しかし、計算式(例えば $h\nu = e\phi + K$)ではジュール $\unit{J}$ に統一する必要がある。
\[ 1 \unit{eV} = 1.60 \times 10^{-19} \unit{J} \]
必ず計算前に換算すること。
\end{warnbox}

\subsection*{⑤ 金属内電子のエネルギー(絶対零度における状態)}
\begin{itemize}
    \item \textbf{パウリの排他律}: 「1つの量子状態(席)には1つの電子しか入れない」という量子力学のルール。
    \item \textbf{積み上げ}: 絶対零度 ($T=0\unit{K}$) では、電子はエネルギーの低い順位から順に、隙間なくびっしりと詰まっていく。
    \item \textbf{フェルミ面}: 詰め込まれた電子の、一番上のエネルギー面がフェルミ準位 $\EF$ と一致する。それより上には電子は1つも存在しない。
\end{itemize}

\subsection*{⑥ フェルミ準位とフェルミ分布関数の意味}
電子が「あるエネルギー準位 $E$」に存在する確率を与える統計力学の基本関数。

\begin{equation}
    F(E) = \frac{1}{1 + \exp\left(\frac{E-\EF}{\kB T}\right)}
\end{equation}

\begin{itemize}
    \item \textbf{意味}: エネルギー $E$ の準位が電子によって占有されている確率 ($0 \le F(E) \le 1$)。
    \item \textbf{$E = \EF$ のとき}: 指数部は $\exp(0)=1$ となるため、温度 $T$ に関わらず $F(\EF) = 1/2$ となる。これがフェルミ準位の定義である。
    \item \textbf{$E \gg \EF$ のとき (ボルツマン近似)}: 
    分母の $1$ が無視でき、$F(E) \approx \exp\left(-\frac{E-\EF}{kT}\right)$ となる。これは古典的なボルツマン分布と一致し、熱電子放出の式の導出根拠となる。
\end{itemize}

\subsection*{⑦ エネルギー準位図のグラフ $F(E), n(E)$}
試験でグラフを描く、あるいは選ぶ問題が出る可能性がある。

\begin{description}
    \item[分布関数 $F(E)$ の形状]
    \begin{itemize}
        \item \textbf{$T=0\unit{K}$}: $\EF$ で垂直に落ちる階段関数(ステップ関数)。$\EF$以下は確率1、以上は確率0。
        \item \textbf{$T>0\unit{K}$}: $\EF$ 付近で角が取れ、なだらかに変化するシグモイド曲線。高温ほど傾きが緩やかになり、高エネルギー側へ裾野が広がる(これが熱電子放出の原因)。
    \end{itemize}
    
    \item[状態密度 $D(E)$ と電子密度 $n(E)$]
    \begin{itemize}
        \item \textbf{状態密度 $D(E)$}: 電子が入れる「席」の数。自由電子モデルでは $D(E) \propto \sqrt{E}$ (放物線)。
        \item \textbf{電子密度 $n(E)$}: 実際にそこにいる電子の数。
        \[ n(E) = D(E) \times F(E) \propto \sqrt{E} \cdot F(E) \]
        グラフ形状は、$\EF$ までは放物線で増え、$\EF$ 付近で急激にゼロになる「山型」となる。
    \end{itemize}
\end{description}

\section{熱電子放出 (⑧〜⑩)}

\subsection*{⑧ 熱電子の飽和電流密度 (ダッシュマン・リチャードソンの式)}
電子工学において最も重要な式の一つ。計算問題の8割はこれに関連する。

\begin{tcolorbox}[colback=white, colframe=blue, title=\textbf{公式: リチャードソン・ダッシュマンの式}]
    \begin{equation}
        J = A T^2 \exp\left( -\frac{e\phi}{\kB T} \right) \quad [\si{A/m^2}]
    \end{equation}
\end{tcolorbox}
\begin{itemize}
    \item $J$: 電流密度(単位面積あたりの電流 $[\si{A/m^2}]$)
    \item $A$: リチャードソン定数(理論値 $\approx 1.20 \times 10^6 [\si{A/m^2 K^2}]$)
    \item $T$: \textbf{絶対温度 [K]} ($= \unit{^\circ C} + 273.15$)
    \item $e\phi$: 仕事関数(エネルギー障壁の高さ $[\si{J}]$)
    \item $k$: ボルツマン定数
\end{itemize}
\textbf{式の物理的意味}: 「$T^2$」は電子の衝突頻度などに由来し、「$\exp$」項はエネルギー障壁を超える確率(ボルツマン因子)を表す。温度 $T$ が少し上がるだけで、指数関数の効果により $J$ は爆発的に増加する。

\subsection*{⑨ リチャードソン線から仕事関数と $A$ を求める方法}
実験データから材料定数 ($\phi, A$) を決定する手法。

式を変形し、線形関係($Y = aX + b$)を見出す。
\[
    \ln\left( \frac{J}{T^2} \right) = \ln A - \frac{e\phi}{\kB} \cdot \frac{1}{T}
\]
\begin{itemize}
    \item 縦軸 ($Y軸$): $\ln(J/T^2)$
    \item 横軸 ($X軸$): $1/T$ (逆温度)
\end{itemize}
このプロット(リチャードソンプロット)は\textbf{右下がりの直線}となる。
\begin{itemize}
    \item \textbf{傾き (Slope)}: $-\frac{e\phi}{k}$ $\to$ 傾きの絶対値から仕事関数 $\phi$ を算出できる。
    \item \textbf{Y切片 (Intercept)}: $\ln A$ $\to$ ここから定数 $A$ を決定できる。
\end{itemize}

\subsection*{⑩ 熱陰極の具備条件}
優れた電子放出材料(カソード)が満たすべき条件。記述問題頻出。

\begin{enumerate}
    \item \textbf{仕事関数 $\phi$ が小さいこと}:
    これが最重要。$\phi$ が小さいほど、低い温度で多量の電子を放出できる(省エネ)。
    \item \textbf{高い放出電流密度 $J$ が得られること}:
    動作温度において十分な電子流を供給できること。
    \item \textbf{融点が高く、蒸気圧が低いこと}:
    動作温度で溶けたり、蒸発して痩せ細ったりしないこと(長寿命化)。
    \item \textbf{機械的強度が強いこと}:
    高温でも変形したり折れたりしないこと。
    \item \textbf{化学的に安定であること}:
    残留ガスと反応して表面が変質(中毒)し、仕事関数が悪化しないこと。
    \item \textbf{陰極能率(放出効率)が高いこと}:
    加熱電力1Wあたりに得られる電流値 [mA/W] が大きいこと。
\end{enumerate}

\begin{pointbox}{代表的な陰極材料}
\begin{itemize}
    \item \textbf{タングステン (W)}: 融点が高い($3655\unit{K}$)が、仕事関数も大きい($4.5\unit{eV}$)。高電圧・高出力管向け。
    \item \textbf{トリウムタングステン (Th-W)}: W中にトリウム(Th)を添加。表面に単原子層を形成し、$\phi$ を $2.6\unit{eV}$ まで下げる。
    \item \textbf{酸化物陰極 (Oxide)}: BaO, SrOなどを塗布。$\phi \approx 1.0\unit{eV}$ と非常に低い。ブラウン管や蛍光灯など一般用途向け。
\end{itemize}
\end{pointbox}

\section{光電子放出 (⑪〜⑬)}

\subsection*{⑪ 光電子放出条件}
光の粒子性(光子)に基づく現象。
\begin{itemize}
    \item 入射光子エネルギー $h\nu$ が、電子を束縛している仕事関数 $e\phi$ に勝つ必要がある。
    \[ h\nu \ge e\phi \]
    \item 光の強さ(明るさ)を上げても、振動数 $\nu$ が足りなければ電子は1個も出ない(古典波動論では説明できない点)。
\end{itemize}

\subsection*{⑫ アインシュタインの式、限界周波数、限界波長}
エネルギー保存則の式である。

\begin{equation}
    K_{\max} = \frac{1}{2}mv_m^2 = h\nu - e\phi
\end{equation}
\begin{itemize}
    \item $K_{\max}$: 放出される電子の最大運動エネルギー
    \item $h\nu$: 入ってくるエネルギー(収入)
    \item $e\phi$: 出るために支払うエネルギー(税金・コスト)
\end{itemize}

\textbf{限界値の定義}:
$K_{\max} = 0$ (ギリギリ放出される状態)とおくことで求められる。
\begin{itemize}
    \item \textbf{限界周波数 $\nu_0$}: $h\nu_0 = e\phi \implies \nu_0 = \frac{e\phi}{h}$
    \item \textbf{限界波長 $\lambda_0$}: $\frac{hc}{\lambda_0} = e\phi \implies \lambda_0 = \frac{hc}{e\phi}$
\end{itemize}
\textbf{重要}: 入射光の波長 $\lambda$ が $\lambda_0$ より\textbf{短くないと}(エネルギーが高くないと)放出されない。長波長側がNGである点に注意。

\subsection*{⑬ 量子効率、光電感度}
効率を表す2つの指標。

\begin{description}
    \item[量子効率 (Quantum Efficiency, $\eta_q$)]
    「個数の比率」。単位なし(または \%)。
    \[ \eta_q = \frac{\text{放出された電子の数 } N_e}{\text{入射した光子の数 } N_p} \]
    通常、金属では $10^{-4}$ 以下と低いが、半導体光電面では $0.1 \sim 0.3$ に達する。
    
    \item[光電感度 (Spectral Responsivity, $S$)]
    「電気的出力の比率」。単位は $[\si{A/W}]$。
    \[ S = \frac{\text{光電流 } I \unit{[A]}}{\text{入射光パワー } P \unit{[W]}} \]
    実用的なセンサー性能評価に使われる。
\end{description}

\section{二次電子放出 (⑭〜⑰)}

\subsection*{⑭ 二次電子放出の原理}
外部から高速の電子(一次電子)を物質に衝突させた際、物質内部の電子がエネルギーを受け取って放出される現象。
\begin{itemize}
    \item 一次電子は物質内部へ侵入しながら、衝突を繰り返してエネルギーを失う。
    \item その過程で励起された内部電子(二次電子)が表面に向かって移動する。
    \item 表面障壁を超えるエネルギーを持ったまま表面に到達できたものだけが放出される。
\end{itemize}

\subsection*{⑮ 放出比の測定方法}
\textbf{二次電子放出比 $\delta$ (Secondary Emission Yield)}:
\[ \delta = \frac{I_s \text{(二次電子流)}}{I_p \text{(一次電子流)}} \]
\begin{itemize}
    \item $\delta > 1$: 入射した電子より多くの電子が出てくる $\to$ \textbf{増幅作用}がある(電子が増える)。
    \item $\delta < 1$: 電子が吸収されて減る。
\end{itemize}

\subsection*{⑯ 放出特性曲線(なぜ山なりになるのか?)}
一次電子の加速電圧 $V_p$(衝突エネルギー)と $\delta$ の関係は極大値を持つ。
\begin{enumerate}
    \item \textbf{上昇域}: $V_p$ が低いときは、エネルギーが増えるほど生成される二次電子の数が増えるため、$\delta$ は上昇する。
    \item \textbf{ピーク $\delta_{max}$}: ある電圧(数百V程度)で最大となる。
    \item \textbf{下降域}: さらに $V_p$ を上げると、一次電子は物質の\textbf{深部}まで侵入してしまう。深い場所で発生した二次電子は、表面まで戻ってくる間に散乱されてエネルギーを失い、脱出できなくなる。そのため、高電圧すぎると逆に効率が落ちる。
\end{enumerate}

\subsection*{⑰ 光電子増倍管 (PMT) の原理}
極微弱な光を検出する真空管デバイス。「光電効果」と「二次電子放出」のハイブリッド。

\begin{enumerate}
    \item \textbf{光電面 (Photocathode)}: 入射光を受け、光電効果により\textbf{光電子}を放出する。(変換: 光子 $\to$ 電子)
    \item \textbf{ダイノード部 (Dynodes)}: 電子増倍電極群。1段あたり $\delta$ 倍に電子を増やす。$n$ 段あれば、トータルのゲインは $G = \delta^n$ となる。
        \begin{itemize}
            \item 例: $\delta=4, n=10$ なら $4^{10} \approx 100\text{万倍}$ の増幅。
        \end{itemize}
    \item \textbf{陽極 (Anode)}: 増倍された電子群(雪崩)を回収し、電流として出力する。
\end{enumerate}

\section{電界放出と電界計算 (⑱〜⑲)}

\subsection*{⑱ ショットキー効果 (Schottky Effect)}
熱電子放出において、電極間に強い電界をかけたときに放出電流が増加する現象。

\begin{termbox}{ショットキー効果の仕組み}
    \begin{enumerate}
        \item \textbf{元々の壁}: 金属表面には「鏡像力」による高い壁がある。
        \item \textbf{電界の追加}: ここに「外部電界(電子を引っ張り出す力)」を加えると、ポテンシャルが右下がりの直線になる。
        \item \textbf{合成}: 「鏡像力」と「外部電界」を足し合わせると、ポテンシャルの山の\textbf{頂点が低くなり ($\Delta \phi$)}、かつ\textbf{壁が薄くなる}。
    \end{enumerate}
\end{termbox}

\begin{itemize}
    \item \textbf{結果}: 実効的な仕事関数が $\phi' = \phi - \Delta \phi$ に減少するため、熱電子が飛び出しやすくなる。
    \item \textbf{数式}: 仕事関数の低下量 $\Delta \phi$ は $\sqrt{E}$ (電界の平方根)に比例する。
\end{itemize}
※これに対し、常温で極めて強い電界をかけ、障壁をトンネル効果で突き抜けるのが「(冷)電界放出 (Field Emission)」である。区別に注意。

\subsection*{⑲ 電界と電位の計算手順(ポアソン・ラプラス)}
記述問題の山場であり、多くの学生が躓くポイント。ここでは、物理法則の根源(ガウスの法則)から出発し、試験で問われる微分方程式の形になるまでを**一切の飛躍なく**解説する。

\subsubsection*{基礎方程式の導出(ガウスの法則からポアソン方程式へ)}
この流れは記述問題で「導出過程を書け」と問われる可能性があるため、論理ステップを暗記すること。

\begin{derivationbox}{ポアソン方程式とラプラス方程式の導出}
\textbf{Step 1: ガウスの法則(積分形)の確認} \\
空間内の任意の閉曲面 $S$ を考え、その内部の体積を $V$、内部にある総電荷を $Q$ とする。
「閉曲面から出る電気力線の総本数(電束)は、内部の電荷量に等しい」という法則である。
\[ \oint_S \bm{D} \cdot \dd \bm{S} = Q = \int_V \rho \, \dd v \]
ここで $\bm{D}$ は電束密度、$\rho$ は体積電荷密度である。

\textbf{Step 2: ガウスの発散定理の適用} \\
数学の定理「面積分 $\to$ 体積分への変換」を用いる。
\[ \oint_S \bm{D} \cdot \dd \bm{S} = \int_V (\nabla \cdot \bm{D}) \, \dd v \]
これと Step 1 の式を比較すると、どちらも体積分 $\int_V \dots \dd v$ の形になる。
\[ \int_V (\nabla \cdot \bm{D}) \, \dd v = \int_V \rho \, \dd v \]

\textbf{Step 3: 微分形(マクスウェル方程式)へ} \\
任意の体積 $V$ で上記が成り立つためには、被積分関数同士が等しくなければならない。
\[ \nabla \cdot \bm{D} = \rho \quad \text{(ガウスの法則の微分形)} \]

\textbf{Step 4: 電界と電位の関係を代入} \\
(1) 構成方程式: $\bm{D} = \varepsilon_0 \bm{E}$ (真空中の場合) \\
(2) 電位の定義: 静電場において電界は電位 $V$ の勾配(マイナス)で表される。 $\bm{E} = -\nabla V$
これらを Step 3 に代入する。
\[ \nabla \cdot (\varepsilon_0 \bm{E}) = \rho \quad \xrightarrow{\bm{E}=-\nabla V} \quad \nabla \cdot (\varepsilon_0 (-\nabla V)) = \rho \]

\textbf{Step 5: ポアソン方程式の完成} \\
$\varepsilon_0$ を定数として外に出し、$\nabla \cdot \nabla = \nabla^2$ (ラプラシアン)を用いると:
\[ -\varepsilon_0 \nabla^2 V = \rho \quad \Longleftrightarrow \quad \nabla^2 V = -\frac{\rho}{\varepsilon_0} \]
これが\textbf{ポアソンの方程式}である。電荷 $\rho$ が存在する場合の電位分布を記述する。

\textbf{Step 6: ラプラス方程式(電荷がない場合)} \\
もし空間に電荷が存在しない($\rho = 0$)ならば、右辺はゼロになる。
\[ \nabla^2 V = 0 \]
これを\textbf{ラプラスの方程式}と呼ぶ。
\end{derivationbox}

\subsubsection*{1次元計算パターン (試験で実際に解く式)}
実際の試験問題では、平板電極などを想定して「1次元($x$方向のみ)」の変化を考えることが多い。この場合、ラプラシアン $\nabla^2$ は単純な2階微分 $\dfrac{\dd^2}{\dd x^2}$ に置き換わる。

\begin{pointbox}{パターン別の微分方程式と解法}
基本式(1次元ポアソン方程式):
\[ \frac{\dd^2 V}{\dd x^2} = -\frac{\rho(x)}{\varepsilon_0} \]

\textbf{【パターン1: 空間電荷なし ($\rho = 0$)】}
方程式: $\dfrac{\dd^2 V}{\dd x^2} = 0$ (ラプラス方程式)
\begin{itemize}
    \item 1回積分: $\dfrac{\dd V}{\dd x} = C_1$ (電界 $E$ は一定)
    \item 2回積分: $V = C_1 x + C_2$ (電位は直線的に変化)
    \item 平行平板コンデンサの電位分布そのものである。
\end{itemize}

\textbf{【パターン2: 一様な空間電荷 ($\rho = \text{const}$)】}
方程式: $\dfrac{\dd^2 V}{\dd x^2} = -\dfrac{\rho}{\varepsilon_0}$
\begin{itemize}
    \item 1回積分: $\dfrac{\dd V}{\dd x} = -\dfrac{\rho}{\varepsilon_0}x + C_1$ (電界は傾きをもつ直線)
    \item 2回積分: $V = -\dfrac{\rho}{2\varepsilon_0}x^2 + C_1 x + C_2$ (電位は放物線を描く)
\end{itemize}

\textbf{【パターン3: 空間電荷制限電流 (Child-Langmuir則の基礎)】}
講義で扱われた「距離に依存する電荷分布 $\rho(x) = -k x^{-1/2}$」のケース(初速度0の電子が加速されるモデル)。
方程式:
\[ \frac{\dd^2 V}{\dd x^2} = \frac{k}{\varepsilon_0} x^{-1/2} \]

\textbf{手順1 (1回積分: 電界)}:
\[ \frac{\dd V}{\dd x} = \int \frac{k}{\varepsilon_0} x^{-1/2} \dd x = \frac{k}{\varepsilon_0} [ 2x^{1/2} ] + C_1 \]
境界条件「カソード($x=0$)で電界0」とすると $C_1 = 0$。
\[ \frac{\dd V}{\dd x} = \frac{2k}{\varepsilon_0} x^{1/2} \]

\textbf{手順2 (2回積分: 電位)}:
\[ V = \int \frac{2k}{\varepsilon_0} x^{1/2} \dd x = \frac{2k}{\varepsilon_0} [ \frac{2}{3}x^{3/2} ] + C_2 \]
境界条件「カソード($x=0$)で電位0」とすると $C_2 = 0$。

\textbf{結論}:
\[ V = \frac{4k}{3\varepsilon_0} x^{3/2} \]
これを逆に解くと $x \propto V^{2/3}$ や、電流 $J$ と電圧 $V$ の関係 ($J \propto V^{3/2}$: 3/2乗則) が導かれる。
\end{pointbox}

\newpage
%==============================================================================
\part{過去問 詳細解法 (2024・2023)}
%==============================================================================
\noindent
計算の途中経過を省略せず、思考のプロセス、式変形、単位変換のタイミングを記述する。「なぜその公式を使うのか?」という着眼点も併記した。

\section{2024年度 試験問題 詳細解説}

\subsection*{問1. エネルギーバンドの名称}
\textbf{問題の意図}: アルカリ金属(Na)の電子配置とバンド構造の対応理解。
\begin{itemize}
    \item 図の下部(内殻): K殻・L殻は電子で満員 $\to$ **[(a) 充満帯]**
    \item 帯の隙間: 電子が入れない $\to$ **[(b) 禁制帯]**
    \item 図の上部(最外殻): Naは3s軌道に電子が1個(定員2個の半分)しか入っていない。空席があるため、電子は自由に動ける。つまり、価電子帯と伝導帯が同一化している $\to$ **[(c) 価電子帯, (d) 伝導帯]** の両性質を持つ(重複選択)。
\end{itemize}

\subsection*{問2. 金属表面のエネルギー準位(穴埋め)}
物理用語の正確な定義を問う問題。
\begin{itemize}
    \item (1) 金属内で最もエネルギーが **[(d) 大きい]** 電子は、絶対零度ではフェルミ準位にある。(小さい、ではない)
    \item (2) $E_F$ は **[(e) フェルミ]** エネルギー。
    \item (3) バンド底部Bの集合体は **[(b) 価電子帯]**。
    \item (4) 真空へ出た電子は束縛がないので **[(j) 自由電子]**。
    \item (5)(6) 足りないエネルギー差 $\phi$ は **[(g) 仕事関数]**。
\end{itemize}

\subsection*{問3. 熱電子放出(タングステン電極の仕事関数 $\phi$)}
\textbf{重要度}: S (計算問題の本丸)
\textbf{問題}: $T=2000 \unit{K}$, 半径 $r=1.25\times 10^{-4} \unit{m}$, 長さ $L=0.1 \unit{m}$, 電流 $I=2.00 \unit{mA}$。$\phi$ [eV] を求めよ。

\begin{calcbox}{計算ステップ}
\textbf{1. 表面積 $S$ の算出}:
円筒の側面積 $S = 2\pi r L$
\begin{align*}
S &= 2 \times 3.14159 \times (1.25 \times 10^{-4}) \times 0.1 \\
  &= (2 \times 1.25) \times 0.1 \times \pi \times 10^{-4} \\
  &= 2.5 \times 0.1 \times \pi \times 10^{-4} = 0.25\pi \times 10^{-4} \approx 7.854 \times 10^{-5} \unit{m^2}
\end{align*}

\textbf{2. 電流密度 $J$ の算出}:
$J = I / S$。$I$ をアンペアに直す ($2\unit{mA} = 2\times 10^{-3}\unit{A}$)。
\begin{align*}
J &= \frac{2.00 \times 10^{-3}}{7.854 \times 10^{-5}} = \frac{2.00}{7.854} \times 10^{2} \approx 0.2546 \times 100 = 25.46 \unit{A/m^2}
\end{align*}

\textbf{3. 式変形 ($\phi$ を求める)}:
\[ J = A T^2 \exp\left(-\frac{e\phi}{kT}\right) \quad \xrightarrow{\text{\(\ln\)をとる}} \quad \ln J = \ln(A T^2) - \frac{e\phi}{kT} \]
\[ \frac{e\phi}{kT} = \ln(A T^2) - \ln J = \ln\left(\frac{A T^2}{J}\right) \]
\[ \phi = \frac{kT}{e} \ln\left(\frac{A T^2}{J}\right) \quad [\unit{eV}] \]
※ $\frac{kT}{e}$ は「熱電圧」相当の値、$\ln$の中身は無次元数となる。

\textbf{4. 数値代入と対数計算}:
\begin{itemize}
    \item 係数項: $\frac{kT}{e} = \frac{1.38 \times 10^{-23} \times 2000}{1.60 \times 10^{-19}} = \frac{2760}{1.60} \times 10^{-4} \approx 0.1725 \unit{eV}$
    \item 対数項の中身 ($X = AT^2/J$):
    $A = 1.20 \times 10^6$ とする。
    \[ X = \frac{(1.20 \times 10^6) \times (2000)^2}{25.46} = \frac{1.20 \times 4.0 \times 10^{12}}{25.46} = \frac{4.8 \times 10^{12}}{25.46} \approx 1.885 \times 10^{11} \]
    \item $\ln X$ の計算 (試験で関数電卓が使えない場合の近似):
    $\ln(1.885 \times 10^{11}) = \ln(1.885) + 11 \ln(10) \approx 0.63 + 11(2.30) = 0.63 + 25.3 = 25.93$
\end{itemize}

\textbf{5. 最終計算}:
\[ \phi = 0.1725 \times 25.93 \approx 4.473 \unit{eV} \]
\end{calcbox}
\textbf{答}: $4.47 \sim 4.48 \unit{eV}$

\subsection*{問4. 光電子の最大速度 $V_m$}
\textbf{問題}: $\phi=1.68 \unit{eV}$, $\lambda=520 \unit{nm}$. 最大速度 $V_m$ を求めよ。

\begin{calcbox}{エネルギー保存則の活用}
\textbf{1. 光子エネルギー $h\nu$ (単位: J)}:
\[ h\nu = \frac{hc}{\lambda} = \frac{6.63 \times 10^{-34} \times 3.00 \times 10^8}{520 \times 10^{-9}} = \frac{19.89 \times 10^{-26}}{5.20 \times 10^{-7}} = 3.825 \times 10^{-19} \unit{J} \]

\textbf{2. 仕事関数 $e\phi$ (単位: J)}:
必ず eV $\to$ J 換算を行う。
\[ e\phi = 1.68 \times (1.60 \times 10^{-19}) = 2.688 \times 10^{-19} \unit{J} \]

\textbf{3. 運動エネルギー $K_{max}$}:
\[ K_{max} = h\nu - e\phi = (3.825 - 2.688) \times 10^{-19} = 1.137 \times 10^{-19} \unit{J} \]

\textbf{4. 速度 $V_m$ への変換}:
$K = \frac{1}{2}mv^2 \implies v = \sqrt{2K/m}$
\[ V_m = \sqrt{ \frac{2 \times 1.137 \times 10^{-19}}{9.11 \times 10^{-31}} } = \sqrt{ \frac{2.274}{9.11} \times 10^{12} } = \sqrt{0.2496} \times 10^6 \]
$\sqrt{0.25} = 0.5$ なので、
\[ V_m \approx 0.4996 \times 10^6 \approx 5.00 \times 10^5 \unit{m/s} \]
\end{calcbox}
\textbf{答}: $5.00 \times 10^5 \unit{m/s}$

\subsection*{問5. 二次電子放出材料}
表(省略)から $\delta_{max}$ が最大のものを選ぶ。
\textbf{答}: **MgO (酸化マグネシウム)**
\textbf{補足}: 一般に金属($\approx 1$)より酸化物や半導体($>1$)の方が二次電子放出比は高い。

\subsection*{問6. 光電子増倍管 (PMT)}
\textbf{(1) 応用}: スーパーカミオカンデ、シンチレーションカウンタ、血液分析装置など。「極微弱光の検出」がキーワード。

\textbf{(2) 出力電流計算}:
$P=1.98\times 10^{-5}\unit{W}, \eta=27.0\unit{mA/W}, \delta=3.51, n=6$段。

\begin{calcbox}{多段増幅計算}
1. \textbf{光電面での初期電流 $I_0$}:
   感度 $\eta = 27 \unit{mA/W} = 0.027 \unit{A/W}$
   \[ I_0 = P \times \eta = (1.98 \times 10^{-5}) \times 0.027 \approx 5.346 \times 10^{-7} \unit{A} \]
2. \textbf{総合利得 $G$}:
   \[ G = \delta^n = 3.51^6 \]
   計算: $3.51^2 \approx 12.32$, $12.32^3 \approx 1869$ (概算)
3. \textbf{出力電流 $I$}:
   \[ I = I_0 \times G = 5.346 \times 10^{-7} \times 1869 \approx 9991 \times 10^{-7} \unit{A} \approx 1.00 \unit{mA} \]
\end{calcbox}
\textbf{答}: $1.00 \unit{mA}$

\subsection*{問7. ショットキー効果}
記述キーワード:「鏡像力ポテンシャル」「外部電界ポテンシャル」「合成障壁」「$\Delta \phi$ の低下」「頂点の金属側への移動」。これらを組み合わせて、「電界により仕事関数が実効的に下がり、熱電子放出が増える現象」と説明する。

\subsection*{問8. 電界計算(記述手順)}
ポアソン方程式 $\frac{d^2V}{dx^2} = -\frac{\rho}{\varepsilon_0}$ を出発点とし、
積分1回目 $\to$ 電界の式(積分定数C1)、積分2回目 $\to$ 電位の式(積分定数C2)、
境界条件($V(0)=0$など)を代入してC1, C2を決定する、という手順を箇条書きにする。

\section{2023年度 試験問題 詳細解説}

\subsection*{問1. 鏡像力と電位障壁}

\begin{termbox}{鏡像力 (Image Force) とは?}
金属表面の前に電子(マイナス)がいると、金属表面にはプラスの電荷が集まります。これをあたかも「鏡の向こう側(壁の奥 $x$ の位置)にプラスの電荷(鏡像電荷)がいる」とみなして計算する手法です。
\end{termbox}

\textbf{(1) クーロン力}: 金属表面から距離 $x$ にある電子($-e$)は、壁の反対側距離 $x$ にある鏡像電荷($+e$)と引き合う。電荷間距離は $2x$。
\[ F = \frac{1}{4\pi\varepsilon_0} \frac{q_1 q_2}{r^2} = \frac{e^2}{4\pi\varepsilon_0 (2x)^2} = \frac{e^2}{16\pi\varepsilon_0 x^2} \]
\textbf{(3) ポテンシャルエネルギー}: 力の積分(無限遠基準)。
\[ W(x) = \int_x^\infty |F| \dd r = \int_{x}^{\infty} \frac{e^2}{16\pi\varepsilon_0 r^2} \dd r \]
積分計算:
\begin{align*}
W &= \frac{e^2}{16\pi\varepsilon_0} \left[ -\frac{1}{r} \right]_{x}^{\infty} \\
  &= \frac{e^2}{16\pi\varepsilon_0} \left( 0 - \left(-\frac{1}{x}\right) \right) = \frac{e^2}{16\pi\varepsilon_0 x} \unit{[J]}
\end{align*}
eV単位で答える場合は $e$ で割る $\to \frac{e}{16\pi\varepsilon_0 x} \unit{[eV]}$

\subsection*{問3. モリブデン線の半径 $r$}
\begin{termbox}{モリブデン (Mo) とは?}
融点が高く ($2623\unit{^\circ C}$)、高温でも強度を保つ金属。タングステンと同様に電子管のヒータやグリッド(網)などの材料として使われる。
試験では「仕事関数 $\phi$ が異なるだけの別の金属」として扱えばOK。
\end{termbox}

2024年の逆問題。
\begin{enumerate}
    \item $T=2000, \phi=4.27\unit{eV}$ から $J$ を計算。
    $\frac{e\phi}{kT} \approx 24.75$
    $J = 1.2\times 10^6 \times 2000^2 \times e^{-24.75} \approx 4.8\times 10^{12} \times 1.78\times 10^{-11} \approx 85.4 \unit{A/m^2}$
    \item $I = J \times S = J \times 2\pi r L$ より $r = \frac{I}{2\pi L J}$。
    $I=22.8\unit{mA}$を代入して $r \approx 4.26 \times 10^{-4} \unit{m}$。
\end{enumerate}

\subsection*{問6. 3次元電位からの電界導出}
$V = (x^2+y^2+z^2)^{-1/2}$。
$E_x = -\frac{\partial V}{\partial x}$。
合成関数の微分: $\frac{\partial}{\partial x}(u^{-1/2}) = -\frac{1}{2}u^{-3/2} \cdot \frac{\partial u}{\partial x}$。
$u = x^2+y^2+z^2$ なので $u_x = 2x$。
\[ E_x = - \left( -\frac{1}{2} (x^2+y^2+z^2)^{-3/2} \cdot 2x \right) = x(x^2+y^2+z^2)^{-3/2} \]
ベクトル表記なら $\bm{E} = \frac{x\bm{i}+y\bm{j}+z\bm{k}}{(x^2+y^2+z^2)^{3/2}} = \frac{\bm{r}}{r^3}$(点電荷の電界と同じ形)。

\newpage
%==============================================================================
\part{重要演習課題 (レポート課題・記述対策)}
%==============================================================================
本番で差がつく計算問題を、検算可能なレベルまで展開する。

\section*{課題1: フィラメント設計 (Lを求める)}
\textbf{条件:} $T = 2500\unit{K}, r = 1.50 \times 10^{-4}\unit{m}, I = 2.00\unit{mA}, \phi = 4.52\unit{eV}$。
必要な長さ $L$ を求めよ。

\begin{calcbox}{逆算のフロー}
\textbf{1. 指数項 $\exp(-e\phi/kT)$ の計算}:
\[ \frac{e\phi}{kT} = \frac{1.60 \times 10^{-19} \times 4.52}{1.38 \times 10^{-23} \times 2500} = \frac{7.232}{3.45} \times 10^4 \times 10^{-4} \approx 20.962 \]
\[ \exp(-20.962) \approx 7.876 \times 10^{-10} \]

\textbf{2. 電流密度 $J$ の計算}:
$A=1.20\times 10^6$ とする。
\[ J = (1.20 \times 10^6) \times (2500)^2 \times (7.876 \times 10^{-10}) \]
\[ J = 7.50 \times 10^{12} \times 7.876 \times 10^{-10} \approx 5907 \unit{A/m^2} \]

\textbf{3. 長さ $L$ の計算}:
$S = I/J$, かつ $S = 2\pi r L$ なので、
\[ L = \frac{I}{2\pi r J} = \frac{2.00 \times 10^{-3}}{2\pi \times (1.50 \times 10^{-4}) \times 5907} \]
分母: $2\pi \times 1.5 \approx 9.42$
分母全体: $9.42 \times 10^{-4} \times 5907 \approx 5.567$
\[ L = \frac{0.002}{5.567} \approx 0.000359 \unit{m} \]
\end{calcbox}
\textbf{答}: $L \approx 3.59 \times 10^{-4} \unit{m}$ ($0.36 \unit{mm}$)

\section*{課題4: 電界ベクトル導出の注意点}
記述でよくある間違いを防ぐためのチェックポイント。

\begin{warnbox}{偏微分の符号ミスに注意}
電界の定義は $\bm{E} = -\nabla V$ である。マイナスを忘れがちなので注意すること。
\begin{itemize}
    \item 正: ポテンシャルが下がる方向に力(電界)が働く。
    \item 計算: $V$ を微分した後、最後に符号を反転させるのを忘れないこと。
\end{itemize}
\end{warnbox}

\[ E_x = - \frac{\partial V}{\partial x} \]
もし $V=kx^2$ なら、$E_x = -2kx$ となる。

\vspace{2em}
\hrule
\begin{center}
    \textbf{Good Luck on Your Exam!} \\
    この資料の計算を一度自分の手でトレースすれば、必ず合格点に到達します。
\end{center}

\end{document}