% !TEX program = lualatex
%==============================================================================
% 電子工学(5E) 後期中間到達度試験 統合対策資料 (完全版)
%==============================================================================
% 制作:試験対策委員会
% 内容:試験範囲ポイント①〜⑲の完全網羅解説 + 2024/2023年度過去問詳細解答 + 課題演習
%==============================================================================

\documentclass[a4paper,11pt]{ltjsarticle}

%------------------------------------------------------------------------------
% パッケージ設定
%------------------------------------------------------------------------------
\usepackage[left=15mm,right=15mm,top=20mm,bottom=20mm]{geometry}
\usepackage{amsmath, amssymb, bm} % 数式
\usepackage{siunitx}              % 単位
\usepackage{graphicx}             % 画像
\usepackage{booktabs}             % 表
\usepackage{tcolorbox}            % 枠線・ボックス
\tcbuselibrary{skins, breakable}
\usepackage{enumitem}             % リスト設定
\usepackage{hyperref}             % リンク
\hypersetup{colorlinks=true, linkcolor=blue, urlcolor=cyan}
\usepackage{cancel}               % 計算過程のキャンセル線

%------------------------------------------------------------------------------
% 定数・コマンド定義
%------------------------------------------------------------------------------
\newcommand{\EF}{E_{\mathrm{F}}} % フェルミ準位
\newcommand{\kB}{k}              % ボルツマン定数
\newcommand{\dd}{\mathrm{d}}     % 微分記号
% Define simple unit macro only if one doesn't already exist to avoid conflicts
% with packages like siunitx which may define their own \unit command.
\providecommand{\unit}[1]{\,[\mathrm{#1}]}

% 試験用物理定数
\newcommand{\ValE}{1.60 \times 10^{-19}}
\newcommand{\ValK}{1.38 \times 10^{-23}}
\newcommand{\ValA}{1.20 \times 10^6}
\newcommand{\ValH}{6.63 \times 10^{-34}}
\newcommand{\ValC}{3.00 \times 10^8}

\title{\textbf{電子工学(5E) 試験対策 統合完全版}}
\author{ポイント①〜⑲網羅解説 \& 過去問詳細解法}
\date{}

\begin{document}

\maketitle

\tableofcontents
\newpage

%==============================================================================
\part{試験範囲ポイント完全解説 (①〜⑲)}
%==============================================================================
\noindent
試験範囲として提示された19個のポイントについて、教科書やノートの内容を補完し、記述問題にも対応できるよう詳細に解説します。

\section{エネルギーバンドと電子放出の基礎 (①〜⑦)}

\subsection*{① 価電子帯、禁制帯、伝導帯とはなにか}
\begin{description}
    \item[価電子帯 (Valence Band)] 原子核に束縛された電子(価電子)が詰まっているエネルギー帯。通常、ここにある電子は電気伝導に寄与しません。
    \item[禁制帯 (Forbidden Band)] 電子が存在することができないエネルギー領域。価電子帯と伝導帯の間のエネルギー差(バンドギャップ)を指します。
    \item[伝導帯 (Conduction Band)] 電子が原子の束縛を離れて自由に動くことができるエネルギー帯。ここに電子が励起されると電流が流れます。
\end{description}
\textbf{※金属の特徴}: 金属(導体)では「価電子帯」と「伝導帯」が重なっているか、価電子帯に空きがあるため、室温でも電子は自由に動くことができます(自由電子)。

\subsection*{② 外部エネルギーの入射により電子が放出されるしくみ}
エネルギーバンド図において、電子は通常、エネルギーの低い「ポテンシャルの井戸」の中にいます。
外部から\textbf{熱・光・強電界・電子衝突}などのエネルギーを与えられると、電子のエネルギー準位が上昇(励起)します。
そのエネルギーが、物質表面の壁の高さである\textbf{真空準位}を超えたとき、電子は原子の束縛を断ち切って外部(真空)へ飛び出します。

\subsection*{③ 金属内電子が金属外に飛び出さない理由}
金属表面には\textbf{電位障壁 (Potential Barrier)} が存在するからです。
電子が表面から外に出ようとすると、金属表面に残された正電荷(プラス)が電子(マイナス)を引き戻そうとするクーロン力が働きます。これを\textbf{鏡像力(イメージ力)}と呼びます。この力が壁となり、常温・無刺激の状態では電子は脱出できません。

\subsection*{④ 電子放出の共通的基礎(各用語の関係)}
以下は電子放出に関わる主要なエネルギー項とその関係です(図や式で把握してください)。
\begin{tcolorbox}[colback=white, colframe=black]
\begin{equation}
    \phi = W - \EF
\end{equation}
\end{tcolorbox}
\begin{itemize}
    \item \textbf{真空準位 (Vacuum level)}: 真空中に"自由"な電子が存在するエネルギーを基準として $E=0$ に取ることが多い。外へ放出するにはこのレベルまでエネルギーを得る必要がある。
    \item \textbf{全障壁 $W$ (Potential barrier)}: 金属側の基準(バンド底や局所ポテンシャル)から真空準位までのエネルギー差。真空側に出るための最大の障壁高さを表す。
    \item \textbf{フェルミ準位 $\EF$ (Fermi level)}: 金属内で電子が占める最高エネルギー($T=0$K での最高占有エネルギー)。通常は金属内部の基準に対する値で表す。
    \item \textbf{仕事関数 $\phi$ (Work function)}: フェルミ準位にある電子を真空準位まで持ち上げるのに必要なエネルギー差。式で表すと
    \[ \phi = W - \EF, \]
    単位に注意($\phi$ が\si{eV} の場合、エネルギーとしては $e\phi$ [J] を用いる)。
    \item \textbf{電子の全エネルギー}: 金属内部の電子の全エネルギーは運動エネルギー $K$ と位置(ポテンシャル)エネルギー $U$ の和で表され、
    \[ E_{\mathrm{total}} = K + U. \]
    真空準位を基準に見たとき、放出条件は
    \[ E_{\mathrm{total}} + \Delta E_{\mathrm{ext}} \ge 0, \]
    すなわち外部から与えられるエネルギー $\Delta E_{\mathrm{ext}}$(熱・光・電界など)により、この不等式が満たされれば電子は脱出できる。
    \item \textbf{光電子放出の条件(単純形)}: 単一光子過程では、入射光子エネルギー $h\nu$ が仕事関数に相当するエネルギーを上回る必要がある。表記の仕方により
    \[ h\nu \ge e\phi\quad(\text{J 単位の }\phi),\qquad\text{あるいは}\qquad h\nu/ e \ge \phi\quad(\phi\text{ が eV の場合})\]
    と表される。
\end{itemize}

\subsection*{⑤ 金属内電子のエネルギー(絶対零度における状態)}
\begin{itemize}
    \item \textbf{絶対温度 $T=\SI{0}{K}$}: 電子はエネルギーの低い順位から順に隙間なく詰まっています。
    \item \textbf{最高エネルギー}: 金属内電子は、最高で\textbf{フェルミ準位 $\EF$} までのエネルギーを持っています。それ以上の準位に電子は存在しません。
\end{itemize}

\subsection*{⑥ フェルミ準位とフェルミ分布関数の意味}
\begin{itemize}
    \item \textbf{フェルミ分布関数 $F(E)$}: あるエネルギー準位 $E$ に電子が存在する確率(0〜1)を表す関数です。
    \begin{equation}
        F(E) = \frac{1}{1 + \exp\left(\frac{E-\EF}{\kB T}\right)}
    \end{equation}
    \item \textbf{フェルミ準位 $\EF$ の定義 ($T>\SI{0}{K}$)}: 電子が存在する確率 $F(E)$ がちょうど \textbf{1/2 (50\%)} になるエネルギー準位のことです。
\end{itemize}

\subsection*{⑦ エネルギー準位図のグラフ $F(E), n(E)$}
\begin{itemize}
    \item \textbf{$F(E)$ (分布関数)}:
        \begin{itemize}
            \item $T=\SI{0}{K}$:$\EF$ までは確率1、それ以上は0(階段状)。
            \item $T>\SI{0}{K}$:$\EF$ 付近でなだらかに変化する曲線(高温ほど裾野が広がる)。
        \end{itemize}
    \item \textbf{$n(E)$ (電子密度)}: 実際に存在する電子の数分布。
        \[ n(E) \propto \sqrt{E} \times F(E) \]
        放物線状の状態密度と、分布関数の積で表されます。$\EF$ 付近に多くの電子が存在します。
\end{itemize}

\section{熱電子放出 (⑧〜⑩)}

\subsection*{⑧ 熱電子の飽和電流密度 (ダッシュマン・リチャードソンの式)}
最も重要な公式です。温度 $T$ と仕事関数 $\phi$ で電流密度 $J$ が決まります。
\begin{tcolorbox}[colback=white, colframe=blue, title=\textbf{公式暗記}]
    \begin{equation}
        J = A T^2 \exp\left( -\frac{e\phi}{\kB T} \right) \quad [\si{A/m^2}]
    \end{equation}
\end{tcolorbox}
\begin{itemize}
    \item $A$: リチャードソン定数 ($1.20 \times 10^6 [\si{A/m^2 K^2}]$)
    \item $\kB$: ボルツマン定数
    \item $T$: \textbf{絶対温度 [K]} ($= \text{摂氏} + 273$)
\end{itemize}

\subsection*{⑨ リチャードソン線から仕事関数と $A$ を求める方法}
リチャードソンの式を変形して対数(log)をとります。
\[
    \ln\left( \frac{J}{T^2} \right) = \ln A - \frac{e\phi}{\kB} \cdot \frac{1}{T}
\]
これを $Y = b - a X$ の形に見立てます。
\begin{itemize}
    \item 縦軸 $Y = \ln(J/T^2)$
    \item 横軸 $X = 1/T$
\end{itemize}
としてグラフ(リチャードソンプロット)を描くと\textbf{右下がりの直線}になります。
\begin{itemize}
    \item \textbf{直線の傾き}: $-\frac{e\phi}{\kB}$ $\to$ ここから \textbf{仕事関数 $\phi$} が求まる。
    \item \textbf{Y切片}: $\ln A$ $\to$ ここから \textbf{定数 $A$} が求まる。
\end{itemize}

\subsection*{⑩ 熱陰極の具備条件}
熱陰極(熱電子放出を利用する陰極)に求められる主な条件を整理します。ここでの指標は、放出電流密度 $J$、放出効率(陰極能率)$S$([A/W])、運用寿命などです。
\begin{enumerate}
    \item \textbf{仕事関数 $\phi$ が小さいこと}: 同じ放出量を得るのに必要な温度(エネルギー)を下げられるため、低温で運用できる。低い $\phi$ は放出電流密度を高める主要因。
    \item \textbf{高い放出電流密度 $J$ を得られること}: 必要な電流を安定して供給できること。単位は [A/m$^2$] で、用途に応じた十分な $J$ を得られる材料が望ましい。
    \item \textbf{陰極能率(放出効率)$S$ が高いこと(A/W)}: ヒータに与える入力パワーあたり得られる放出電流が大きいほど、消費電力が小さくすみ効率が良い。
    \item \textbf{融点・熱安定性が高いこと}: 高温での動作に堪え、形状や性能が劣化しにくい(融点が高い金属は材料的に有利)。
    \item \textbf{寿命が長いこと(化学的・蒸発耐性)}: 高温での蒸発や化学反応、外的なイオン打撃などで性能が低下しにくく、安定して長期間動作すること。
    \item \textbf{真空中での安定性(不活性性・低蒸気圧)}: 真空での表面汚染や反応に強く、作動環境で仕事関数が変わりにくいこと。
\end{enumerate}
	extbf{備考}: 実際には材料ごとにトレードオフがあります。例えばタングステン(W)は融点が高く長寿命である一方、仕事関数が大きく高温が必要です。酸化物陰極(BaO/SrO)は低仕事関数で高効率だが、長時間の高温露出や汚染により劣化する場合があります。
\par\noindent
	extbf{代表材料}: タングステン (W)、酸化物陰極(BaO/SrO:酸化バリウム・酸化ストロンチウム)、酸化モリブデン系など。用途により最適な材料が異なります(高出力用途は W、低消費電力・高効率用途は酸化物陰極など)。

\section{光電子放出 (⑪〜⑬)}

\subsection*{⑪ 光電子放出条件}
入射する光子のエネルギー $h\nu$ が、電子の脱出コスト(仕事関数 $e\phi$)以上である必要があります。
\[ h\nu \ge e\phi \]

\subsection*{⑫ アインシュタインの式、限界周波数、限界波長}
\begin{itemize}
    \item \textbf{アインシュタインの式 (エネルギー保存則)}:
    \[ \frac{1}{2}mv_m^2 = h\nu - e\phi \]
    (飛び出す運動エネルギー) = (入ってきた光エネルギー) - (脱出コスト)
    \item \textbf{限界周波数 $\nu_0$}: 電子放出が始まる最低の振動数。($h\nu_0 = e\phi$)
    \item \textbf{限界波長 $\lambda_0$}: 電子放出が始まる最長の波長。
    \[ \lambda_0 = \frac{c}{\nu_0} = \frac{hc}{e\phi} \]
    これより波長が\textbf{短い}光でないと放出されません(光のエネルギーは波長に反比例するため)。
\end{itemize}

\subsection*{⑬ 量子効率、光電感度}
\begin{description}
    \item[量子効率 $\eta_q$] 「数」の割合。入射した光子1個あたり、何個の電子が放出されたか。
    \item[光電感度 $S$] 「電流」の割合。入射した光パワー1Wあたり、何アンペアの電流が得られたか ($S = I/P$ [A/W])。
\end{description}

\section{二次電子放出 (⑭〜⑰)}

\subsection*{⑭ 二次電子放出の原理}
加速された電子(\textbf{一次電子})が固体表面に衝突し、その運動エネルギーを固体内の電子に与えます。エネルギーを受け取った電子(\textbf{二次電子})が、表面のポテンシャル障壁を超えて外部へ放出される現象です。

\subsection*{⑮ 放出比の測定方法}
二次電子放出比 $\delta$ は、一次電子1個に対して何個の二次電子が出たかの比率です。
\[ \delta = \frac{I_s \text{(二次電子流)}}{I_p \text{(一次電子流)}} \]
$\delta > 1$ のとき、電子が増倍(増幅)されます。

\subsection*{⑯ 放出特性曲線(なぜ山なりになるのか?)}
横軸に一次電子の加速電圧 $V_p$、縦軸に放出比 $\delta$ をとったグラフは山型になります。
\begin{itemize}
    \item \textbf{低電圧領域}: 電圧を上げると衝突エネルギーが増えるため、たたき出される二次電子の数($\delta$)は増加します。
    \item \textbf{ピーク ($V_{pmax}$)}: $\delta$ が最大値 $\delta_{max}$ になります。
    \item \textbf{高電圧領域}: さらに電圧を上げると、一次電子が物質の\textbf{奥深くまで入り込みすぎます}。内部で発生した二次電子が表面まで戻ってくる間にエネルギーを失ってしまうため、逆に放出される数は\textbf{減少}します。
\end{itemize}

\subsection*{⑰ 光電子増倍管 (PMT) の原理}
「光電効果」と「二次電子放出」を組み合わせた高感度センサです。
\begin{enumerate}
    \item \textbf{光電面}: 光を受けて光電子を放出する(光 $\to$ 電子)。
    \item \textbf{ダイノード(増倍部)}: 多段の電極に電子を衝突させ、二次電子放出を繰り返して電子をネズミ算式に増やす(雪崩増幅)。
    \item \textbf{アノード}: 増えた電子を集めて電流として出力する。
\end{enumerate}

\section{電界放出と電界計算 (⑱〜⑲)}

\subsection*{⑱ ショットキー効果}
金属表面に強い電界 $E$ をかけると、電子が放出しやすくなる現象です。
\begin{itemize}
    \item \textbf{原理}: 「鏡像力によるポテンシャルカーブ」と「外部電界による直線ポテンシャル」が合成されます。
    \item \textbf{結果}: 電位障壁の頂点が\textbf{低くなり ($\Delta \phi$)}、かつ位置が\textbf{金属側に移動}します。
    \item 仕事関数が見かけ上減少するため、熱電子放出電流が増加します。
\end{itemize}

\subsection*{⑲ 電界と電位の計算手順(ポアソン・ラプラス)}
空間に電荷密度 $\rho$ が存在する場合の計算手順(記述問題で問われます)。
\begin{enumerate}
    \item \textbf{ポアソンの方程式を立てる}:
    電荷と電位の関係式です。
    \[ \frac{\dd^2 V}{\dd x^2} = -\frac{\rho}{\varepsilon_0} \]
    \item \textbf{1回目の積分}(電界 $E$ を求める):
    $x$ で積分します。積分定数 $C_1$ が現れます。
    \[ \frac{\dd V}{\dd x} = \int \left(-\frac{\rho}{\varepsilon_0}\right) \dd x \quad \Rightarrow \quad -E_x = -\frac{\rho}{\varepsilon_0}x + C_1 \]
    \item \textbf{2回目の積分}(電位 $V$ を求める):
    もう一度 $x$ で積分します。積分定数 $C_2$ が現れます。
    \[ V = \int \left(-\frac{\rho}{\varepsilon_0}x + C_1\right) \dd x \quad \Rightarrow \quad V = -\frac{\rho}{2\varepsilon_0}x^2 + C_1 x + C_2 \]
    \item \textbf{境界条件の適用}:
    問題で与えられた条件(例:$x=0$で$V=0$、 $x=d$で$V=V_a$)を代入して連立方程式を解き、積分定数 $C_1, C_2$ を決定します。
\end{enumerate}

\newpage
%==============================================================================
\part{過去問 詳細解法 (2024・2023)}
%==============================================================================
\noindent
計算問題の解法プロセスを省略せず、すべて記述しました。試験直前の計算練習に活用してください。

\section{2024年度 試験問題}

\subsection*{問1. エネルギーバンドの名称}
ナトリウム金属のバンド図において:
\begin{itemize}
    \item (2) バンドが重なる部分: アルカリ金属では価電子帯が半分しか埋まっておらず、伝導帯と連続しています。よって [(c) 価電子帯, (d) 伝導帯] の両方を選びます。
    \item (3), (4) 内殻準位: 電子で満たされており、電気伝導に寄与しない低いエネルギー準位は [(a) 充満帯] です。
    \item バンド間の隙間: 電子が存在できない領域なので [(b) 禁制帯] です。
\end{itemize}

\subsection*{問2. 金属表面のエネルギー準位(穴埋め)}
\begin{itemize}
    \item 金属内でエネルギーが最も [(d) 大きい] 電子は、底部Bを基準として$E_F$のエネルギーをもつ。これを [(e) フェルミ] エネルギーと呼ぶ。
    \item [(b) 価電子帯] の底部Bにある電子が [(j) 自由電子](真空中に脱出)になるエネルギーは $W$ である。
    \item $\phi = W - E_F$ より [(g) 仕事関数] エネルギーを外部から与えると電子は放出される。この $\phi$ を [(g) 仕事関数] と呼ぶ。
\end{itemize}

\subsection*{問3. タングステン電極の仕事関数 $\phi$}
\textbf{条件}: $T=2000 \unit{K}$, $r=1.25\times 10^{-4} \unit{m}$, $L=0.1 \unit{m}$, $I=2.00 \unit{mA}$. \\
\textbf{解法}:
\begin{enumerate}
    \item \textbf{表面積 $S$}:
    \[ S = 2\pi r L = 2\pi (1.25 \times 10^{-4})(0.1) \approx 7.854 \times 10^{-5} \unit{m^2} \]
    \item \textbf{電流密度 $J$}:
    \[ J = I/S = (2.00 \times 10^{-3}) / (7.854 \times 10^{-5}) \approx 25.46 \unit{A/m^2} \]
    \item \textbf{$\phi$ の計算}: リチャードソンの式 $J = A T^2 \exp(-e\phi/kT)$ より
    \[ \ln(J/AT^2) = -e\phi/kT \quad \Rightarrow \quad \phi = \frac{kT}{e} \ln(AT^2/J) \]
    数値を代入($kT/e \approx 0.1725 \unit{eV}$):
    \[ \phi \approx 0.1725 \times \ln\left( \frac{1.20 \times 10^6 \times 2000^2}{25.46} \right) \approx 4.478 \unit{eV} \]
\end{enumerate}
\textbf{答}: $4.48 \unit{eV}$

\subsection*{問4. 光電子の最大速度 $V_m$}
\textbf{条件}: $\phi=1.68 \unit{eV}$, $\lambda=520 \unit{nm}$. \\
\textbf{解法}:
\begin{enumerate}
    \item \textbf{光エネルギー $h\nu$}:
    \[ h\nu = \frac{hc}{\lambda} \approx 3.825 \times 10^{-19} \unit{J} \]
    \item \textbf{仕事関数 $e\phi$}:
    \[ e\phi = 1.68 \times (1.60 \times 10^{-19}) \approx 2.688 \times 10^{-19} \unit{J} \]
    \item \textbf{運動エネルギー $K$}:
    \[ K = h\nu - e\phi = 1.137 \times 10^{-19} \unit{J} \]
    \item \textbf{速度 $V_m$}:
    \[ V_m = \sqrt{2K/m} \approx \sqrt{2 \times 1.137 \times 10^{-19} / (9.11 \times 10^{-31})} \approx 5.00 \times 10^5 \unit{m/s} \]
\end{enumerate}
\textbf{答}: $5.00 \times 10^5 \unit{m/s}$

\subsection*{問5. 二次電子放出材料}
\textbf{答}: 酸化マグネシウム (MgO) \\
\textbf{理由}: 表の中で二次電子放出比の最大値 $\delta_{max}$ が $4.0$ と最も大きく、効率よく電子を増倍できるため。

\subsection*{問6. 光電子増倍管 (PMT)}
\textbf{(1) 応用例}: スーパーカミオカンデ(ニュートリノ観測)、シンチレーションカウンタ。 \\
\textbf{(2) 出力電流 $I$}:
\begin{enumerate}
    \item 初期電流 $I_0$: $I_0 = P \times \eta = (1.98 \times 10^{-5}) \times (0.027) \approx 5.346 \times 10^{-7} \unit{A}$
    \item 利得 $G$: $G = \delta^n = 3.51^6 \approx 1869$
    \item 出力 $I$: $I = I_0 \times G \approx 1.00 \times 10^{-3} \unit{A}$
\end{enumerate}
\textbf{答}: $1.00 \unit{mA}$

\subsection*{問8. 電界計算(記述)}
\textbf{(1) 方程式}: ポアソンの方程式 $\nabla^2 V = -\rho/\varepsilon_0$ \\
\textbf{(2) 手順}:
\begin{itemize}
    \item ① $x$ で1回積分して電界 $E$ の式を出す。
    \item ② もう一度 $x$ で積分して電位 $V$ の式を出す。
    \item ③ 境界条件で定数を決定し、微分して電界 $E_x$ を求める。
\end{itemize}

\section{2023年度 試験問題}

\subsection*{問1. 鏡像力と電位障壁}
\textbf{(1) 力の大きさ}: $|F| = \frac{e^2}{16\pi\varepsilon_0 x^2}$ (クーロン力、距離 $2x$) \\
\textbf{(3) 電位障壁 $W$}: $W = \int_{x}^{\infty} F dx = \frac{e^2}{16\pi\varepsilon_0 x} \unit{J}$。 eV単位なら $W = \frac{e}{16\pi\varepsilon_0 x} \unit{eV}$。

\subsection*{問3. モリブデン線の半径 $r$}
\textbf{条件}: $T=2000$, $I=22.8\unit{mA}$, $L=0.1$, $\phi=4.27\unit{eV}$. \\
\textbf{解法}:
\begin{enumerate}
    \item \textbf{電流密度 $J$ の計算}: リチャードソンの式より
    \[ J = (1.20 \times 10^6) \times 2000^2 \times \exp\left( -\frac{4.27}{0.1725} \right) \approx 85.25 \unit{A/m^2} \]
    \item \textbf{半径 $r$ の計算}: $I = J \times 2\pi r L$ より
    \[ r = \frac{I}{2\pi L J} = \frac{0.0228}{2\pi (0.1)(85.25)} \approx 4.26 \times 10^{-4} \unit{m} \]
\end{enumerate}
\textbf{答}: $4.26 \times 10^{-4} \unit{m}$

\subsection*{問4. 限界波長 $\lambda_0$}
\textbf{条件}: $\phi=1.72 \unit{eV}$. \\
\textbf{解法}:
\[ \lambda_0 = \frac{hc}{e\phi} \approx \frac{1240 \unit{eV \cdot nm}}{1.72 \unit{eV}} \approx 721 \unit{nm} \]
(正確な定数計算では $7.23 \times 10^{-7} \unit{m}$)\\
\textbf{答}: $723 \unit{nm}$

\subsection*{問5. PMTの出力電流}
\textbf{条件}: $P=6.43 \times 10^{-5}$, $\eta=15.0 \unit{mA/W}$, $\delta=3.4$, $n=5$. \\
\textbf{解法}:
\[ I = (P \times \eta) \times \delta^n = (9.645 \times 10^{-7}) \times (3.4^5) \approx 4.38 \times 10^{-4} \unit{A} \]
\textbf{答}: $0.438 \unit{mA}$

\subsection*{問6. 電界 $E_x$ の導出}
\textbf{条件}: $V = (x^2+y^2+z^2)^{-1/2}$. \\
\textbf{解法}:
電界 $E_x = -\frac{\partial V}{\partial x}$。合成関数の微分を利用。
\[ \frac{\partial V}{\partial x} = -\frac{1}{2}(x^2+y^2+z^2)^{-3/2} \cdot (2x) = -x(x^2+y^2+z^2)^{-3/2} \]
符号を反転させて、
\textbf{答}: $E_x = \frac{x}{(x^2+y^2+z^2)^{3/2}} \unit{V/m}$

\newpage
%==============================================================================
\part{重要演習課題 (レポート課題・記述対策)}
%==============================================================================
提出レポート課題として課された問題の詳細解答です。数値計算のプロセスを正確に追跡します。

\section*{課題1: 熱電子放出とフィラメント設計}
\textbf{条件:}
\begin{itemize}
 \item 温度 $T = 2500 \, [\si{K}]$
 \item 半径 $r = 1.50 \times 10^{-4} \, [\si{m}]$
 \item 全電流 $I = 2.00 \times 10^{-3} \, [\si{A}]$
 \item 仕事関数 $\phi = 4.52 \, [\si{eV}]$
\end{itemize}

\subsection*{【ステップ1】 リチャードソン定数 $A$ の理論値導出}
定数 $A$ は以下の理論式で与えられます。
\begin{align*}
A &= \frac{4\pi m e k^2}{h^3} \\
  &= \frac{4\pi (9.11 \times 10^{-31}) (1.60 \times 10^{-19}) (1.38 \times 10^{-23})^2}{(6.63 \times 10^{-34})^3} \\
  &\approx 1.201 \times 10^6 \, [\si{A \cdot m^{-2} \cdot K^{-2}}]
\end{align*}
※試験では $A = 1.20 \times 10^6$ が与えられることが一般的ですが、導出を問われる可能性もあるため覚えておきましょう。

\subsection*{【ステップ2】 電流密度 $J$ の算出}
リチャードソン・ダッシュマンの式を用います。
\begin{align*}
J &= A T^2 \exp\left( -\frac{e\phi}{kT} \right) \\
\text{指数部の計算:} \quad & \frac{(1.60 \times 10^{-19}) \times 4.52}{(1.38 \times 10^{-23}) \times 2500} \approx \frac{7.232}{3.45} \times 10^{4} \times 10^{-4} \approx 20.962 \\
J &= (1.201 \times 10^6) \times (2500)^2 \times \exp( -20.962 ) \\
  &= (7.506 \times 10^{12}) \times (7.876 \times 10^{-10}) \\
  &\approx 5.912 \times 10^3 \, [\si{A/m^2}]
\end{align*}

\subsection*{【ステップ3】 フィラメント長 $L$ の決定}
全電流 $I$ は電流密度 $J$ と表面積 $S$ の積です。
\[ S = 2\pi r L \quad \text{より} \quad I = J \cdot (2\pi r L) \]
よって、
\begin{align*}
L &= \frac{I}{2\pi r J} \\
  &= \frac{2.00 \times 10^{-3}}{2\pi \times (1.50 \times 10^{-4}) \times (5.912 \times 10^3)} \\
  &= \frac{2.00 \times 10^{-3}}{5.572} \\
  &\approx 3.589 \times 10^{-4}
\end{align*}

\[ \therefore L \approx 3.59 \times 10^{-4} \, [\si{m}] \]

\section*{課題2: 光電子効果と最大速度}
\textbf{条件:}
\begin{itemize}
 \item 仕事関数 $\phi = 4.27 \, [\si{eV}]$
 \item 波長 $\lambda = 45.5 \, [\si{nm}] = 4.55 \times 10^{-8} \, [\si{m}]$
\end{itemize}

\textbf{解答プロセス:}
光電効果の式 $\frac{hc}{\lambda} = e\phi + \frac{1}{2}mv_m^2$ を $v_m$ について解きます。

\begin{enumerate}
    \item \textbf{光エネルギー $h\nu$ (J)}:
    \[ \frac{hc}{\lambda} = \frac{(6.63 \times 10^{-34})(3.00 \times 10^8)}{4.55 \times 10^{-8}} \approx 4.371 \times 10^{-18} \, [\si{J}] \]
    \item \textbf{仕事関数 $e\phi$ (J)}:
    \[ e\phi = (1.60 \times 10^{-19}) \times 4.27 \approx 6.832 \times 10^{-19} = 0.6832 \times 10^{-18} \, [\si{J}] \]
    \item \textbf{運動エネルギー $K$}:
    \[ K = h\nu - e\phi = (4.371 - 0.683) \times 10^{-18} = 3.688 \times 10^{-18} \, [\si{J}] \]
    \item \textbf{速度 $v_m$}:
    \begin{align*}
    v_m &= \sqrt{ \frac{2K}{m} } = \sqrt{ \frac{2 \times 3.688 \times 10^{-18}}{9.11 \times 10^{-31}} } \\
        &= \sqrt{ \frac{7.376}{9.11} \times 10^{13} } = \sqrt{ 0.8096 \times 10^{13} } = \sqrt{ 8.096 \times 10^{12} } \\
        &\approx 2.845 \times 10^6
    \end{align*}
\end{enumerate}

\[ \therefore v_m \approx 2.85 \times 10^6 \, [\si{m/s}] \]

\section*{課題3: 二次電子放出比とPMT入力計算}
\textbf{条件:}
\begin{itemize}
 \item 二次電子放出比 $\delta = 4.0$
 \item 段数 $n = 10$
 \item コレクタ電流(出力) $I_o = 0.125 \times 10^{-3} \, [\si{A}]$
\end{itemize}

\textbf{解答プロセス:}
総合利得(ゲイン)$G$ は $\delta^n$ で表されます。
\[ I_o = I_p \cdot \delta^n \quad \Longleftrightarrow \quad I_p = \frac{I_o}{\delta^n} \]

\begin{align*}
I_p &= \frac{0.125 \times 10^{-3}}{4.0^{10}} \\
    &= \frac{1.25 \times 10^{-4}}{1048576} \quad (\because 2^{20} \approx 10^6) \\
    &\approx \frac{1.25 \times 10^{-4}}{1.05 \times 10^6} \\
    &\approx 1.192 \times 10^{-10}
\end{align*}

\[ \therefore I_p \approx 1.19 \times 10^{-10} \, [\si{A}] \]

\section*{課題4: 電位分布からの電界ベクトル導出}
\textbf{条件:}
\[ V = \frac{1}{\sqrt{x^2 + y^2 + z^2}} \, [\si{V}] \]
電界 $\bm{E}$ を求めよ。

\textbf{解答プロセス:}
電界は電位の勾配(gradient)の逆符号です。$\bm{E} = -\nabla V$
\[ \bm{E} = - \left( \frac{\partial V}{\partial x} \bm{i} + \frac{\partial V}{\partial y} \bm{j} + \frac{\partial V}{\partial z} \bm{k} \right) \]

$r = (x^2 + y^2 + z^2)^{1/2}$ と置くと $V = r^{-1}$ です。
$x$ についての偏微分を行います(合成関数の微分)。
\begin{align*}
\frac{\partial V}{\partial x} &= \frac{\partial}{\partial x} (x^2 + y^2 + z^2)^{-\frac{1}{2}} \\
    &= -\frac{1}{2} (x^2 + y^2 + z^2)^{-\frac{3}{2}} \cdot \frac{\partial}{\partial x}(x^2 + y^2 + z^2) \\
    &= -\frac{1}{2} \frac{1}{(x^2 + y^2 + z^2)^{\frac{3}{2}}} \cdot (2x) \\
    &= -\frac{x}{(x^2 + y^2 + z^2)^{\frac{3}{2}}}
\end{align*}

電界 $E_x$ はこれにマイナスを掛けたものなので、
\[ E_x = - \frac{\partial V}{\partial x} = \frac{x}{(x^2 + y^2 + z^2)^{\frac{3}{2}}} \]

$y, z$ についても対称性より同様の結果となります。

\[ \therefore \bm{E} = \frac{1}{(x^2 + y^2 + z^2)^{\frac{3}{2}}} (x \bm{i} + y \bm{j} + z \bm{k}) \, [\si{V/m}] \]
※ベクトル表記 $(x, y, z)$ でも正解となります。

\end{document}