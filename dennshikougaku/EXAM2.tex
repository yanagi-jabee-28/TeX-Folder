% !TEX program = lualatex
%==============================================================================
% 電子工学(5E) 後期中間到達度試験 完全対策資料
%==============================================================================
% 制作:試験対策委員会
% 対象資料:試験範囲ポイント表(①〜⑲)、2024/2023年度過去問、講義ノート
%==============================================================================

\documentclass[a4paper,11pt]{ltjsarticle}

%------------------------------------------------------------------------------
% パッケージ設定
%------------------------------------------------------------------------------
\usepackage[left=20mm,right=20mm,top=25mm,bottom=25mm]{geometry}
\usepackage{amsmath, amssymb, bm} % 数式
\usepackage{siunitx}              % 単位
\usepackage{graphicx}             % 画像
\usepackage{booktabs}             % 表
\usepackage{tcolorbox}            % 枠線・ボックス
\tcbuselibrary{skins, breakable}
\usepackage{enumitem}             % リスト設定
\usepackage{hyperref}             % リンク
\hypersetup{colorlinks=true, linkcolor=blue, urlcolor=cyan}

%------------------------------------------------------------------------------
% 定数・コマンド定義
%------------------------------------------------------------------------------
\newcommand{\EF}{E_{\mathrm{F}}} % フェルミ準位
\newcommand{\kB}{k}              % ボルツマン定数
% Use \providecommand to avoid conflict if math packages define \dd
\providecommand{\dd}{\mathrm{d}}     % 微分記号

% 試験用物理定数
\newcommand{\ValE}{1.60 \times 10^{-19}}
\newcommand{\ValK}{1.38 \times 10^{-23}}
\newcommand{\ValA}{1.20 \times 10^6}

\title{\textbf{電子工学(5E) 試験範囲ポイント完全解説}}
\author{重要ポイント①〜⑲網羅版}
\date{}

\begin{document}

\maketitle

\begin{tcolorbox}[colback=red!5!white, colframe=red!60!black, title=\textbf{学習の指針}]
配布された「試験範囲ポイント(①〜⑲)」の全てに対し、過去問とノートに基づく解答を作成しました。
\begin{itemize}
    \item \textbf{計算}: 熱電子(リチャードソン)、光電子(アインシュタイン)、PMT増倍率、電界計算。
    \item \textbf{理論}: バンド図の名称、フェルミ準位の定義、各種放出の原理。
\end{itemize}
特に\textbf{赤字}の部分は、記述問題や穴埋め問題で狙われるキーワードです。
\end{tcolorbox}

%==============================================================================
\section{エネルギーバンドと電子放出の基礎 (ポイント①〜⑦)}
%==============================================================================

\subsection{バンド構造の定義 (①)}
\begin{description}
    \item[充満帯 (Filled Band)] 電子が隙間なく詰まっている帯。通常、電流に寄与しない。
    \item[禁制帯 (Forbidden Band)] 電子が存在できないエネルギー領域(バンドギャップ)。
    \item[伝導帯 (Conduction Band)] 電子が自由に動ける帯。金属では充満帯と重なっている。
\end{description}

\subsection{電子放出のメカニズム (②, ③, ④, ⑤)}
\begin{itemize}
    \item \textbf{放出の条件}: 外部エネルギー(熱・光・電界)を受け取り、電子のエネルギーが\textbf{真空準位}(ポテンシャルの壁の高さ $W$)を超えること。
    \item \textbf{飛び出さない理由 (③)}: 表面の原子核(正電荷)による引力(クーロン力・鏡像力)が働き、\textbf{電位障壁}が形成されているため。
    \item \textbf{仕事関数 $\phi$}:
    \begin{equation}
        \phi = W - \EF \quad [\si{eV}]
    \end{equation}
    「フェルミ準位 $\EF$ にある電子を、外(真空)に取り出すのに必要な最小エネルギー」。
    \item \textbf{0Kでのエネルギー (⑤)}: 絶対零度において、金属内電子は最高でフェルミ準位 $\EF$ までのエネルギーを持つ。
\end{itemize}

\subsection{フェルミ準位と分布関数 (⑥, ⑦)}
\begin{itemize}
    \item \textbf{フェルミ分布関数 $f(E)$}: あるエネルギー $E$ に電子が存在する確率。
    \item \textbf{フェルミ準位 $\EF$ の定義}:
    \begin{itemize}
        \item $T=\SI{0}{K}$:電子が存在する上限のエネルギー。
        \item $T>\SI{0}{K}$:電子の存在確率 $f(E)$ が \textbf{1/2 (50\%)} になるエネルギー準位。
    \end{itemize}
    \item \textbf{状態密度 $n(E)$}: エネルギー準位の「座席数」。エネルギーが高くなると増える(放物線状)。
\end{itemize}

%==============================================================================
\section{熱電子放出 (ポイント⑧〜⑩)}
%==============================================================================

\subsection{リチャードソン・ダッシュマンの式 (⑧)}
熱電子の飽和電流密度 $J [\si{A/m^2}]$ を表す最重要公式。
\begin{tcolorbox}[colback=white, colframe=blue]
    \begin{equation}
        J = A T^2 \exp\left( -\frac{e\phi}{\kB T} \right)
    \end{equation}
\end{tcolorbox}
\textbf{計算の注意}: 指数部分 $\frac{e\phi}{\kB T}$ を計算する際、$\phi[\si{eV}]$ は必ず $\times 1.6 \times 10^{-19}$ して \textbf{ジュール$[\si{J}]$} に直すこと。

\subsection{リチャードソン線 (⑨)}
式を変形し、対数をとる:
\[
    \ln\left( \frac{J}{T^2} \right) = \ln A - \frac{e\phi}{\kB} \cdot \frac{1}{T}
\]
\begin{itemize}
    \item \textbf{縦軸}: $\ln(J/T^2)$, \textbf{横軸}: $1/T$
    \item グラフは\textbf{直線}になる。
    \item \textbf{直線の傾き}から「仕事関数 $\phi$」を、\textbf{切片}から「定数 $A$」を求めることができる。
\end{itemize}

\subsection{熱陰極の具備条件 (⑩)}
良い陰極材料の条件:
\begin{enumerate}
    \item \textbf{仕事関数 $\phi$ が小さいこと}(低温で電子が出やすい)。
    \item \textbf{融点が高いこと}(高温に耐える)。
    \item \textbf{寿命が長いこと}(蒸発しにくい)。
\end{enumerate}
※ 例:タングステン(W)、酸化物陰極(BaOなど)。

%==============================================================================
\section{光電子放出 (ポイント⑪〜⑬)}
%==============================================================================

\subsection{アインシュタインの式と放出条件 (⑪, ⑫)}
\begin{equation}
    h\nu = e\phi + \frac{1}{2}mv_m^2
\end{equation}
\begin{itemize}
    \item **放出条件**: 光子エネルギー $h\nu$ が仕事関数 $e\phi$ 以上であること。 ($h\nu \ge e\phi$)
    \item **限界周波数 $\nu_0$**: $h\nu_0 = e\phi$
    \item **限界波長 $\lambda_0$**: $\lambda_0 = \dfrac{hc}{e\phi}$
    \item 波長が $\lambda_0$ より\textbf{短い}光でないと放出されない(短波長=高エネルギー)。
\end{itemize}

\subsection{量子効率と光電感度 (⑬)}
\begin{description}
    \item[量子効率 $\eta_q$] 入射した光子数に対し、放出された電子数の割合。
    \item[光電感度 $S$] 入射した光パワー $[\si{W}]$ に対する電流 $[\si{A}]$ の比 ($S = I/P$)。
\end{description}

%==============================================================================
\section{二次電子放出 (ポイント⑭〜⑰)}
%==============================================================================

\subsection{原理と放出比 (⑭, ⑮, ⑯)}
\begin{itemize}
    \item **原理**: 一次電子が固体に衝突し、そのエネルギーで固体内の電子が弾き飛ばされる現象。
    \item **二次電子放出比 $\delta$**: $\delta = I_s / I_p$ (二次電流 / 一次電流)。
    \item **放出特性曲線 (⑯)**: 一次電子の加速電圧を上げると $\delta$ は増加するが、ある電圧 ($V_{pmax}$) で最大値 ($\delta_{max}$) をとり、それ以降は\textbf{減少}する。(電子が深くまで入りすぎて脱出できなくなるため)。
\end{itemize}

\subsection{光電子増倍管 (PMT) (⑰)}
\begin{itemize}
    \item **構造**: 光電面 $\to$ ダイノード群 $\to$ アノード。
    \item **原理**: 二次電子放出を繰り返して電子を増倍する(雪崩増幅)。
    \item **計算式**: 出力電流 $I = I_{photo} \times \delta^n$ ($n$: ダイノード段数)。
\end{itemize}

%==============================================================================
\section{電界放出と電子の運動 (ポイント⑱, ⑲)}
%==============================================================================

\subsection{ショットキー効果 (⑱)}
強い電界 $E$ をかけることによる効果。
\begin{itemize}
    \item 鏡像力によるポテンシャルと外部電界が合成される。
    \item 結果として、電位障壁の頂点が\textbf{低下}し、かつ\textbf{金属側に移動}する。
    \item 仕事関数が見かけ上減少し、熱電子放出が増加する。
\end{itemize}

\subsection{電界計算の手順 (⑲)}
空間電荷密度 $\rho$ がある場合の計算フロー(記述問題対策)。
\begin{enumerate}
    \item \textbf{ポアソンの方程式}を立てる: $\dfrac{\dd^2 V}{\dd x^2} = -\dfrac{\rho}{\varepsilon_0}$
    \item \textbf{1回目の積分}: $\dfrac{\dd V}{\dd x} = -E_x$ を求める(積分定数 $C_1$ 出現)。
    \item \textbf{2回目の積分}: 電位 $V(x)$ を求める(積分定数 $C_2$ 出現)。
    \item \textbf{境界条件の適用}: $V(0)=0$、 $V(d)=V_a$ などを代入し、定数を決定する。
\end{enumerate}

\end{document}