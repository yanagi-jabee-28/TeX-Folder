% !TEX program = lualatex
%==============================================================================
% 電子工学(5E) 後期中間到達度試験 対策資料
%==============================================================================

\documentclass[a4paper,11pt]{ltjsarticle}

%------------------------------------------------------------------------------
% パッケージ設定
%------------------------------------------------------------------------------
\usepackage[left=20mm,right=20mm,top=25mm,bottom=25mm]{geometry}
\usepackage{amsmath, amssymb, bm} % 数式
\usepackage{siunitx}              % 単位
\usepackage{graphicx}             % 画像
\usepackage{booktabs}             % 表
\usepackage{tcolorbox}            % 枠線
\tcbuselibrary{skins, breakable}
\usepackage{enumitem}             % リスト設定
\usepackage{hyperref}             % リンク
\hypersetup{colorlinks=true, linkcolor=blue, urlcolor=cyan}

%------------------------------------------------------------------------------
% コマンド定義
%------------------------------------------------------------------------------
\newcommand{\EF}{E_{\mathrm{F}}} % フェルミ準位
\newcommand{\dd}{\mathrm{d}}     % 微分記号
\newcommand{\kB}{k}              % ボルツマン定数

% 定数定義(試験用)
\newcommand{\ConstE}{1.60 \times 10^{-19}}
\newcommand{\ConstK}{1.38 \times 10^{-23}}
\newcommand{\ConstH}{6.63 \times 10^{-34}}
\newcommand{\ConstC}{3.00 \times 10^{8}}

\title{\textbf{電子工学(5E) 後期中間試験 完全攻略ガイド}}
\author{試験対策委員会}
\date{対象:2024・2023年度過去問分析に基づく}

\begin{document}

\maketitle

\begin{tcolorbox}[colback=yellow!10!white, colframe=red!50!black, title=\textbf{はじめに:試験の傾向と対策}]
本試験は、過去問(2023年、2024年)の傾向が非常に似通っており、以下の2大テーマから構成されています。
\begin{enumerate}
    \item \textbf{電子放出(第2章)}:熱・光・二次・電界放出のメカニズムと計算。
    \item \textbf{真空中の電子の運動(第3章)}:ポテンシャル、電界計算、ポアソンの方程式。
\end{enumerate}
特に\textbf{「単位変換のミス」}が命取りになります。必ず $[\mathrm{eV}]$ を $[\mathrm{J}]$ に直して計算する習慣をつけてください。
\end{tcolorbox}

%==============================================================================
\section{基礎:エネルギーバンドと用語定義}
%==============================================================================
\textbf{出題実績}:2024年 問1・問2、2023年 問2

穴埋め問題や作図問題で確実に得点すべきセクションです。

\subsection{重要用語の定義}
\begin{description}
    \item[充満帯 (Filled Band)] 電子が隙間なく詰まっているエネルギー帯。
    \item[禁制帯 (Forbidden Band)] 電子が存在できないエネルギーギャップ。
    \item[伝導帯 (Conduction Band)] 電子が自由に動けるエネルギー帯。金属では充満帯と重なっている。
    \item[フェルミ準位 ($\EF$)] 
        \begin{itemize}
            \item $T=\SI{0}{K}$:電子が存在する\textbf{最高の}エネルギー準位。
            \item $T>\SI{0}{K}$:電子の存在確率が\textbf{1/2 (50\%)}になるエネルギー準位。
        \end{itemize}
    \item[仕事関数 ($\phi$)] フェルミ準位にある電子を、表面の外(真空準位)に取り出すのに必要な最小エネルギー。
    \begin{equation}
        \phi = W - \EF \quad (\text{$W$: 金属の底から真空準位までの全障壁高さ})
    \end{equation}
\end{description}

%==============================================================================
\section{熱電子放出(最重要計算)}
%==============================================================================
\textbf{出題実績}:2024年 問3、2023年 問3

\subsection{リチャードソン・ダッシュマンの式}
電流密度 $J$ $[\si{A/m^2}]$ は以下の式で表されます。この式は暗記必須です。
\begin{tcolorbox}[colback=white, colframe=blue]
    \begin{equation}
        J = A T^2 \exp\left( -\frac{e\phi}{\kB T} \right)
    \end{equation}
\end{tcolorbox}
\begin{itemize}
    \item $A$: リチャードソン定数(通常 $1.20 \times 10^6 \, \si{A/m^2 K^2}$)
    \item $T$: 絶対温度 $[\si{K}]$
    \item $\kB$: ボルツマン定数 ($1.38 \times 10^{-23} \, \si{J/K}$)
    \item $e\phi$: 仕事関数(エネルギー単位 $[\si{J}]$ に換算が必要)
\end{itemize}

\subsection{計算の定石とトラップ}
\textbf{パターン1:電流 $I$ を求める(または $I$ から逆算する)} \\
タングステン線などの「円柱形状」の場合、表面積 $S$ の計算を間違えないこと。
\begin{equation}
    I = J \times S = J \times (2 \pi r L) \quad (\text{$r$: 半径, $L$: 長さ})
\end{equation}
※ 断面積 $\pi r^2$ ではありません!電子は\textbf{表面(側面積)}から飛び出します。

\textbf{パターン2:指数の計算} \\
指数の中身 $\frac{e\phi}{\kB T}$ は無次元になります。
\[
    \text{分子} = \phi[\si{eV}] \times (\ConstE) \quad \leftarrow \text{必ずジュールに直す!}
\]

%==============================================================================
\section{光電子放出(光電効果)}
%==============================================================================
\textbf{出題実績}:2024年 問4、2023年 問4

\subsection{アインシュタインの式}
\begin{tcolorbox}[colback=white, colframe=blue]
    \begin{equation}
        h\nu = e\phi + \frac{1}{2} m v_m^2
    \end{equation}
\end{tcolorbox}
\begin{itemize}
    \item $h\nu$: 入射光子のエネルギー(入力)
    \item $e\phi$: 仕事関数(税金のようなコスト)
    \item $\frac{1}{2}mv_m^2$: 電子の最大運動エネルギー(残りカス)
\end{itemize}

\subsection{限界波長 $\lambda_0$}
電子がギリギリ放出される(運動エネルギーが0)条件です。
\begin{equation}
    h \frac{c}{\lambda_0} = e\phi \implies \lambda_0 = \frac{hc}{e\phi}
\end{equation}
\textbf{便利テクニック}:
\[
    \lambda_0 [\si{nm}] \approx \frac{1240}{\phi [\si{eV}]}
\]
※記述式の場合は、基本定数 $h, c, e$ を代入して計算過程を示すのが無難です。

%==============================================================================
\section{二次電子放出・光電子増倍管 (PMT)}
%==============================================================================
\textbf{出題実績}:2024年 問5・6、2023年 問5

\subsection{計算のアルゴリズム}
光電子増倍管の出力電流 $I$ を求める問題は、以下の3ステップで解けます。

\begin{enumerate}
    \item \textbf{光電流 $I_0$ の計算}:
    入射光パワー $P [\si{W}]$ と光電感度 $\eta [\si{A/W}]$ を掛けます。
    \[ I_0 = P \times \eta \]
    \textbf{注意}: $\eta$ が $[\si{mA/W}]$ で与えられた場合、$\times 10^{-3}$ を忘れないこと。

    \item \textbf{増倍率(ゲイン) $G$ の計算}:
    二次電子放出比 $\delta$、ダイノード段数 $n$ のとき、
    \[ G = \delta^n \]

    \item \textbf{出力電流 $I$}:
    \[ I = I_0 \times G = P \eta \delta^n \]
\end{enumerate}

\subsection{記述対策}
\begin{itemize}
    \item \textbf{二次電子放出比特性}: 入射エネルギーが高すぎると、電子が深く潜りすぎて脱出できなくなるため、放出比 $\delta$ はある電圧でピークを持ち、その後減少する。
    \item \textbf{応用例}: 「スーパーカミオカンデ」=ニュートリノ観測。チェレンコフ光をPMTで検出。
\end{itemize}

%==============================================================================
\section{電界放出・真空中の運動}
%==============================================================================
\textbf{出題実績}:2024年 問7・8、2023年 問1・6

\subsection{ショットキー効果}
強い電界 $E$ により、ポテンシャル障壁が変化する現象。
\begin{itemize}
    \item \textbf{変化}: 電位障壁の頂点が\textbf{低くなり}、かつ\textbf{金属側に移動}する。
    \item \textbf{結果}: 仕事関数が見かけ上減少し、熱電子放出電流が増加する。
\end{itemize}

\subsection{鏡像力(イメージ力)}
金属表面から距離 $x$ にある電荷 $-e$ が受ける力 $F$。
\begin{equation}
    F = \frac{1}{4\pi\varepsilon_0} \frac{e^2}{(2x)^2} = \frac{e^2}{16\pi\varepsilon_0 x^2}
\end{equation}
※ 鏡像電荷 $+e$ が、壁の向こう側距離 $x$(合計距離 $2x$)にいると仮定してクーロンの法則を適用する。

\subsection{ポアソン・ラプラスの方程式(解法フロー)}
空間電荷密度 $\rho$ がある場合の電位 $V$ と電界 $E_x$ の求め方。
\begin{enumerate}
    \item \textbf{方程式を立てる}: $\dfrac{\dd^2 V}{\dd x^2} = -\dfrac{\rho}{\varepsilon_0}$
    \item \textbf{1回積分 ($E$)}: $\dfrac{\dd V}{\dd x} = -E_x = -\dfrac{\rho}{\varepsilon_0}x + C_1$
    \item \textbf{もう1回積分 ($V$)}: $V = -\dfrac{\rho}{2\varepsilon_0}x^2 + C_1 x + C_2$
    \item \textbf{境界条件}: 「$x=0$で$V=0$」などを代入して $C_1, C_2$ を決定する。
\end{enumerate}

%==============================================================================
\section{直前チェックリスト(定数・単位)}
%==============================================================================
試験開始直前に確認してください。

\begin{itemize}
    \item $1 \, [\si{eV}] = \ConstE \, [\si{J}]$
    \item $1 \, [\si{nm}] = 10^{-9} \, [\si{m}]$
    \item $1 \, [\si{cm}] = 10^{-2} \, [\si{m}]$
    \item $\mu$ (マイクロ) $= 10^{-6}$, $m$ (ミリ) $= 10^{-3}$, $k$ (キロ) $= 10^{3}$, $M$ (メガ) $= 10^{6}$
    \item 温度は必ず\textbf{ケルビン ($K$)}を使う。 ($\si{\degree C} + 273.15$)
\end{itemize}

\end{document}