% !TEX program = lualatex
%==============================================================================
% プリアンブル (Preamble)
%==============================================================================

% ===== ドキュメントクラス (LuaLaTeX用) =====
% ltjsarticle が自動的に日本語フォント設定を行います
\documentclass[a4paper,11pt]{ltjsarticle}

%------------------------------------------------------------------------------
% パッケージ読み込み
%------------------------------------------------------------------------------

% ===== フォント設定の競合回避 =====
% 【修正】luatexja-preset と luatexja-fontspec を削除しました。
% クラスのデフォルト設定を使用することで、スケーリング警告や機能無視の警告を防ぎます。

% ===== レイアウト関連 =====
\usepackage[left=25mm,right=25mm,top=30mm,bottom=30mm]{geometry}
\usepackage{graphicx}
\usepackage{booktabs}      % 綺麗な表
\usepackage{enumitem}      % リストのカスタマイズ
\setlist[enumerate,1]{label=(\arabic*)}

% ===== 数式・物理単位関連 =====
\usepackage{amsmath}
\usepackage{amssymb}
\usepackage{bm}            % ベクトル太字 (\bm{})
\usepackage{siunitx}       % 単位記述用
\usepackage{cancel}        % 式のキャンセル線

% ===== デザイン・装飾関連 =====
\usepackage{tcolorbox}     % 枠線・強調ボックス
\tcbuselibrary{skins}

% ===== ハイパーリンク =====
\usepackage[
  colorlinks=true,
  linkcolor=blue,
  urlcolor=cyan,
  hidelinks,
]{hyperref}

%------------------------------------------------------------------------------
% カスタムコマンド
%------------------------------------------------------------------------------
\newcommand{\dd}{\mathrm{d}}
\newcommand{\EF}{E_{\mathrm{F}}} % フェルミ準位

% ===== ドキュメント情報 =====
\title{電子工学 中間到達度試験 対策分析まとめ}
\author{}
\date{}

%==============================================================================
% ドキュメント本体 (Body)
%==============================================================================
\begin{document}

\maketitle

提供された資料(試験範囲、2024年過去問、2023年過去問)を詳細に分析しました。
この試験は、\textbf{「電子放出(熱・光・二次・電界)」}と\textbf{「真空中での電子の運動」}の2大テーマで構成されています。

過去問の傾向が極めて明確であり、毎年似た形式で出題されています。以下に、得点源となる重要ポイントと解法の定石を解説します。

%------------------------------------------------------------------------------
\section{エネルギー準位とバンド理論(基礎知識)}
%------------------------------------------------------------------------------
\textbf{【出題箇所】} 2024問1・問2、2023問2 \\
\textbf{【ポイント一覧】} ①〜⑤

まず、言葉の定義と図の関係を暗記してください。

\begin{itemize}
    \item \textbf{孤立原子 vs 金属}: 原子が集合して金属になると、電子の軌道が重なり「バンド(帯)」を形成します。
    \begin{itemize}
        \item \textbf{充満帯 (Filled Band)}: 電子が詰まっているバンド。
        \item \textbf{禁制帯 (Forbidden Band)}: 電子が存在できないエネルギー領域(バンドギャップ)。
        \item \textbf{伝導帯 (Conduction Band)}: 電子が自由に動けるバンド。
    \end{itemize}
    \item \textbf{フェルミ準位 ($\EF$)}: 絶対零度(\SI{0}{K})で電子が詰まっている最大のエネルギー準位。
    \item \textbf{仕事関数 ($\phi$)}: 電子を固体表面から真空中に取り出すのに必要な最小エネルギー。
    \begin{itemize}
        \item 数式定義:$\phi = W - \EF$ (真空準位 $W$ とフェルミ準位の差)
        \item \textbf{重要}: $\phi$ は「表面の障壁」の高さです。
    \end{itemize}
\end{itemize}

%------------------------------------------------------------------------------
\section{熱電子放出(計算問題の最重要)}
%------------------------------------------------------------------------------
\textbf{【出題箇所】} 2024問3、2023問3 \\
\textbf{【ポイント一覧】} ⑧〜⑩

\textbf{ダッシュマン・リチャードソンの式}を使って計算させられます。

\begin{equation}
    J = AT^2 \exp\left(-\frac{e\phi}{kT}\right)
\end{equation}

\begin{itemize}
    \item $J$: 電流密度 $[\mathrm{A/m}^2]$
    \item $A$: リチャードソン定数(問題文で与えられます)
    \item $T$: 温度 $[\mathrm{K}]$
    \item $k$: ボルツマン定数
    \item $\phi$: 仕事関数 $[\mathrm{V}]$ または $[\mathrm{eV}]$(※指数の計算時は単位に注意)
\end{itemize}

\subsection*{【解法の定石】}
\begin{enumerate}
    \item \textbf{電流 $I$ と電流密度 $J$ の関係}:
    \[ I = J \times S \]
    ここで $S$ は表面積です。円柱状の電線(タングステン線など)の場合、半径 $r$、長さ $L$ とすると表面積は $S = 2\pi r L$ です。
    
    \item \textbf{式の変形}:
    \begin{itemize}
        \item 2024年は $\phi$ を求める問題なので、対数をとって変形します。
        \item 2023年は $r$ を求める問題なので、$I = (AT^2 \exp(\dots)) \times 2\pi r L$ から $r$ について解きます。
    \end{itemize}
\end{enumerate}

%------------------------------------------------------------------------------
\section{光電子放出(光電効果)}
%------------------------------------------------------------------------------
\textbf{【出題箇所】} 2024問4、2023問4 \\
\textbf{【ポイント一覧】} ⑪〜⑬

アインシュタインの光電効果の式が全てです。

\begin{equation}
    h\nu = \phi + \frac{1}{2}mv_m^2
\end{equation}

\begin{itemize}
    \item $h\nu$: 入射光のエネルギー($h$: プランク定数、$\nu$: 振動数)
    \item $\phi$: 仕事関数(脱出に必要なエネルギー)
    \item $\frac{1}{2}mv_m^2$: 飛び出した電子の最大運動エネルギー
\end{itemize}

\subsection*{【よく使う変換】}
\begin{itemize}
    \item 振動数と波長の関係: $\nu = \frac{c}{\lambda}$ ($c$: 光速)
    \item \textbf{限界波長 $\lambda_0$}: 電子がギリギリ飛び出す(運動エネルギー0)条件なので、$h\frac{c}{\lambda_0} = \phi$ となります。
    \item \textbf{単位の罠}: 仕事関数 $\phi$ は $[\mathrm{eV}]$ で与えられますが、計算式ではジュール $[\mathrm{J}]$ に直す必要があります。
    \[ 1 \, [\mathrm{eV}] = 1.60 \times 10^{-19} \, [\mathrm{J}] \]
\end{itemize}

%------------------------------------------------------------------------------
\section{二次電子放出と光電子増倍管(PMT)}
%------------------------------------------------------------------------------
\textbf{【出題箇所】} 2024問5・問6、2023問5 \\
\textbf{【ポイント一覧】} ⑭〜⑰

ここも毎年必ず計算が出ます。パターンは決まっています。

\subsection*{【光電子増倍管の出力電流 $I$ の求め方】}
プロセスは「光が入る $\to$ 光電子が出る $\to$ 二次電子で増幅される」です。

\begin{enumerate}
    \item \textbf{初段の電流(光電流)}:
    \[ \text{光電流} = \text{入射光パワー } P \, [\mathrm{W}] \times \text{光電感度 } \eta \, [\mathrm{A/W}] \]
    ※2024年問6、2023年問5(2)では、$\eta$ の単位 $[\mathrm{mA/W}]$ に注意。
    
    \item \textbf{増幅(ダイノード)}:
    1段あたりの二次電子放出比を $\delta$、段数を $n$ とすると、増幅率(ゲイン)は $\delta^n$ です。
    
    \item \textbf{最終出力電流 $I$}:
    \begin{equation}
        I = (P \times \eta) \times \delta^n
    \end{equation}
\end{enumerate}

\subsection*{【記述対策】}
\begin{itemize}
    \item \textbf{スーパーカミオカンデ}: 「ニュートリノ観測装置。チェレンコフ光を光電子増倍管で検出し、電気信号に変える」という概要を押さえる(2023問5(3))。
\end{itemize}

%------------------------------------------------------------------------------
\section{電界放出・ショットキー効果・ポアソン方程式}
%------------------------------------------------------------------------------
\textbf{【出題箇所】} 2024問7・問8、2023問1・問6 \\
\textbf{【ポイント一覧】} ⑱〜⑲

\subsection*{ショットキー効果(2024問7)}
\begin{itemize}
    \item 強い電界をかけると、鏡像力(イメージ力)と外部電界の合成により、金属表面の\textbf{電位障壁が下がる}現象。
    \item これにより熱電子が放出しやすくなります。
    \item 図の説明:点線が元の障壁、実線が合成後の障壁(頂点が低くなっている)を示します。
\end{itemize}

\subsection*{電位と電界の計算(2024問8、2023問6)}
\begin{itemize}
    \item \textbf{関係式}: 電界 $\boldsymbol{E} = -\nabla V$ (電位の傾きのマイナス)。
    \item \textbf{手順}:
    \begin{enumerate}
        \item ポアソン方程式($\nabla^2 V = -\frac{\rho}{\epsilon_0}$)を立てる。
        \item 積分して電界、さらに積分して電位を求める(境界条件で積分定数を決定)。
        \item あるいは、電位の式 $V$ が与えられている場合(2023問6)、単に $x$ で偏微分してマイナスをつければ $E_x$ になります。
        \[ E_x = - \frac{\partial V}{\partial x} \]
    \end{enumerate}
\end{itemize}

\vspace{1em}

\begin{tcolorbox}[
  colback=white,
  colframe=black,
  title=\textbf{直前対策まとめ},
  sharp corners=downhill,
  boxrule=1pt
]
これらの方針で学習すれば、60分以内に十分完答可能です。計算ミス(特に指数の桁)にだけ注意してください。

\begin{enumerate}
    \item \textbf{単位換算を徹底する}:
    \begin{itemize}
        \item エネルギー: $\mathrm{eV} \to \mathrm{J}$ ($\times 1.6 \times 10^{-19}$)
        \item 長さ: $\mathrm{nm} \to \mathrm{m}$ ($\times 10^{-9}$)、$\mathrm{cm} \to \mathrm{m}$ ($\times 10^{-2}$)
        \item 電流: $\mathrm{mA} \to \mathrm{A}$ ($\times 10^{-3}$)
    \end{itemize}
    
    \item \textbf{必須公式3選}:
    \begin{itemize}
        \item 熱電子放出: $J = AT^2 e^{-e\phi/kT}$
        \item 光電効果: $h\frac{c}{\lambda} = \phi + K_{\max}$
        \item PMT増幅: $I_{\text{out}} = P \cdot \eta \cdot \delta^n$
    \end{itemize}
    
    \item \textbf{「鏡像法」の力}:
    \begin{itemize}
        \item 金属表面から距離 $x$ にある電子が受ける力 $F = \frac{e^2}{16\pi\epsilon_0 x^2}$ (2023問1)も式変形できるようにしておいてください。
    \end{itemize}
\end{enumerate}
\end{tcolorbox}

\end{document}