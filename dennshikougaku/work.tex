\documentclass[a4paper,11pt]{ltjsarticle} % LuaLaTeX用のクラス

%------------------------------------------------------------------------------
% パッケージ読み込み
%------------------------------------------------------------------------------
\usepackage[top=25mm,bottom=25mm,left=25mm,right=25mm]{geometry} % 余白設定
\usepackage{amsmath,amssymb} % 数式用
\usepackage{bm}              % ベクトルなどの太字用

%------------------------------------------------------------------------------
% コマンド定義
%------------------------------------------------------------------------------
% 単位記述用のコマンド(別のパッケージが提供している場合は上書きしない)
\providecommand{\unit}[1]{\,[\mathrm{#1}]}

%==============================================================================
% 本文
%==============================================================================
\begin{document}

%--- ヘッダー情報 ---
\begin{flushright}
提出期限:2025年 11月 27日(木) 12:30
\end{flushright}

\begin{center}
{\Large \textbf{電子工学 課題1 レポート}}
\end{center}

\vspace{1em}
\begin{flushright}
\underline{学籍番号:\hspace{3cm} 氏名:\hspace{4cm}}
\end{flushright}
\vspace{1em}

%--- 定数リスト ---
\section*{使用定数}
本レポートでは、講義ノートおよび配布資料に基づき以下の物理定数を用いる。
\begin{align*}
k &= 1.38 \times 10^{-23} \unit{J \cdot K^{-1}} & e &= 1.60 \times 10^{-19} \unit{C} \\
m &= 9.11 \times 10^{-31} \unit{kg} & h &= 6.63 \times 10^{-34} \unit{J \cdot s} \\
c &= 3.00 \times 10^{8} \unit{m/s}
\end{align*}

\hrulefill

%--- 課題1 ---
\section*{課題1}
\textbf{条件:}
\begin{itemize}
 \item 温度 $T = 2500 \unit{K}$
 \item タングステンフィラメント半径 $r = 1.50 \times 10^{-4} \unit{m}$
 \item 全電流 $I = 2.00 \times 10^{-3} \unit{A}$
 \item 仕事関数 $\phi = 4.52 \unit{eV}$
\end{itemize}

\noindent
\textbf{解答:}

講義資料より、リチャードソン・ダッシュマンの式(電流密度 $J$)を用いる。
\begin{equation}
J = A T^2 \exp\left( -\frac{e\phi}{kT} \right)
\end{equation}
ここで定数 $A$(リチャードソン定数)は、理論値として $A \approx 1.20 \times 10^6 \unit{A \cdot m^{-2} \cdot K^{-2}}$ を採用する。

まず、指数関数の引数部分(無次元量)に各数値を代入して計算する。
\begin{align*}
\frac{e\phi}{kT} &= \frac{(1.60 \times 10^{-19}) \times 4.52}{(1.38 \times 10^{-23}) \times 2500} \\
                 &= \frac{7.232 \times 10^{-19}}{3.450 \times 10^{-20}} \\
                 &= 2.0962 \times 10^{1} \\
                 &\approx 20.962
\end{align*}

求めた指数部と与えられた条件を式(1)へ代入し、電流密度 $J$ を求める。
\begin{align*}
J &= (1.20 \times 10^6) \times (2500)^2 \times \exp(-20.962) \\
  &= (1.20 \times 10^6) \times (6.25 \times 10^6) \times (7.876 \times 10^{-10}) \\
  &= (1.20 \cdot 6.25 \cdot 7.876) \times 10^{(6+6-10)} \\
  &\approx 5.907 \times 10^3 \unit{A/m^2}
\end{align*}

電極の側面積は $S = 2\pi r L$ であり、全電流は $I = J S$ で表される。これよりフィラメント長 $L$ の式に変形し、数値を代入する。
\begin{align*}
L &= \frac{I}{J \cdot 2\pi r} \\
  &= \frac{2.00 \times 10^{-3}}{(5.907 \times 10^3) \times 2\pi \times (1.50 \times 10^{-4})} \\
  &= \frac{2.00 \times 10^{-3}}{(5.907 \cdot 2\pi \cdot 1.50) \times 10^{-1}} \\
  &= \frac{2.00 \times 10^{-3}}{55.67 \times 10^{-1}} \\
  &= 0.03592 \times 10^{-2} \\
  &= 3.592 \times 10^{-4}
\end{align*}

有効数字3桁で整理する。
\[
\therefore L = 3.59 \times 10^{-4} \unit{m}
\]

%--- 課題2 ---
\section*{課題2}
\textbf{条件:}
\begin{itemize}
 \item 仕事関数 $\phi = 4.27 \unit{eV}$
 \item 入射光波長 $\lambda = 45.5 \unit{nm} = 4.55 \times 10^{-8} \unit{m}$
\end{itemize}

\noindent
\textbf{解答:}

講義ノート p.20 の記述(式2.12改)に基づき、光電子の最大速度 $v_m$ を求める。
エネルギー保存則 $h\nu = e\phi + \frac{1}{2}mv_m^2$ を変形すると次式となる。
\begin{equation}
v_m = \sqrt{ \frac{2}{m} (h\nu - e\phi) } = \sqrt{ \frac{2}{m} \left( \frac{hc}{\lambda} - e\phi \right) }
\end{equation}

この式に各物理定数および条件値を直接代入する。
\begin{align*}
v_m &= \sqrt{ \frac{2}{9.11 \times 10^{-31}} \left\{ \frac{(6.63 \times 10^{-34}) \cdot (3.00 \times 10^8)}{4.55 \times 10^{-8}} - (1.60 \times 10^{-19}) \cdot 4.27 \right\} } \\[8pt]
    &= \sqrt{ \frac{2}{9.11 \times 10^{-31}} \left\{ \left( \frac{19.89}{4.55} \times 10^{-18} \right) - \left( 6.832 \times 10^{-19} \right) \right\} } \\[8pt]
    &= \sqrt{ \frac{2}{9.11 \times 10^{-31}} \left( 4.371 \times 10^{-18} - 0.6832 \times 10^{-18} \right) } \\[8pt]
    &= \sqrt{ \frac{2}{9.11 \times 10^{-31}} \times (3.6878 \times 10^{-18}) } \\[8pt]
    &= \sqrt{ \frac{7.3756 \times 10^{-18}}{9.11 \times 10^{-31}} } \\[8pt]
    &= \sqrt{ 0.8096 \times 10^{13} } = \sqrt{ 8.096 \times 10^{12} } \\[8pt]
    &\approx 2.845 \times 10^6
\end{align*}

有効数字3桁で整理する。
\[
\therefore v_m = 2.85 \times 10^6 \unit{m/s}
\]

%--- 課題3 ---
\section*{課題3}
\textbf{条件:}
\begin{itemize}
 \item 二次電子放出比 $\delta = 4.0$
 \item ダイノード段数 $n = 10$
 \item コレクタ電流 $I_o = 0.125 \times 10^{-3} \unit{A}$
\end{itemize}

\noindent
\textbf{解答:}

光電子増倍管の総合利得 $G = \delta^n$ を用いて、一次光電流 $I_p$ を求める。
$I_o = G I_p = \delta^n I_p$ より、数値を代入する。
\begin{align*}
I_p &= \frac{I_o}{\delta^n} = \frac{0.125 \times 10^{-3}}{4.0^{10}} \\
  &= \frac{0.125 \times 10^{-3}}{4^{10}} \\
  &= \frac{0.125 \times 10^{-3}}{1\,048\,576} \quad(\text{since }4^{10}=2^{20}=1\,048\,576)
\end{align*}
次に数値を正確に代入して計算する。
\begin{align*}
I_p &= \frac{1.25 \times 10^{-4}}{1\,048\,576} \\
    &= 1.1920928955078125 \times 10^{-10}
\end{align*}

有効数字3桁で整理する。
\[
\therefore I_p = 1.19 \times 10^{-10} \unit{A}
\]

%--- 課題4 ---
\section*{課題4}
\textbf{条件:}
電位分布 $V(x,y,z)$ が次式で与えられる。
\[
V = \frac{1}{\sqrt{x^2 + y^2 + z^2}} = (x^2 + y^2 + z^2)^{-\frac{1}{2}} \unit{V}
\]

\noindent
\textbf{解答:}

電界ベクトル $\bm{E}$ は電位の勾配として定義される。
\[
\bm{E} = -\nabla V = - \left( \frac{\partial V}{\partial x} \bm{i} + \frac{\partial V}{\partial y} \bm{j} + \frac{\partial V}{\partial z} \bm{k} \right)
\]
各成分について偏微分計算を行う。まず $x$ 成分について、
\begin{align*}
E_x &= - \frac{\partial}{\partial x} \left( (x^2 + y^2 + z^2)^{-\frac{1}{2}} \right) \\
    &= - \left( -\frac{1}{2} \right) (x^2 + y^2 + z^2)^{-\frac{3}{2}} \cdot \frac{\partial}{\partial x}(x^2 + y^2 + z^2) \\
    &= \frac{1}{2} (x^2 + y^2 + z^2)^{-\frac{3}{2}} \cdot (2x) \\
    &= \frac{x}{(x^2 + y^2 + z^2)^{\frac{3}{2}}}
\end{align*}
式の対称性より、$y, z$ 成分も同様となる。
\begin{align*}
E_y &= \frac{y}{(x^2 + y^2 + z^2)^{\frac{3}{2}}} \\
E_z &= \frac{z}{(x^2 + y^2 + z^2)^{\frac{3}{2}}}
\end{align*}

\noindent
\textbf{答:}
\begin{equation*}
\bm{E} = \frac{1}{(x^2 + y^2 + z^2)^{\frac{3}{2}}} (x, y, z) \unit{V/m}
\end{equation*}

\end{document}