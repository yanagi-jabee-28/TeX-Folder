\documentclass[a4paper,11pt]{ltjsarticle} % LuaLaTeX用のクラスに変更

%------------------------------------------------------------------------------
% パッケージ読み込み
%------------------------------------------------------------------------------
\usepackage[top=25mm,bottom=25mm,left=25mm,right=25mm]{geometry} % 余白設定
\usepackage{amsmath,amssymb} % 数式用
\usepackage{bm}              % ベクトルなどの太字用

%------------------------------------------------------------------------------
% コマンド定義
%------------------------------------------------------------------------------
% 単位記述用のコマンド([m]などを綺麗に出力するため)
\newcommand{\unit}[1]{\,[\mathrm{#1}]}

%==============================================================================
% 本文
%==============================================================================
\begin{document}

%--- ヘッダー情報 ---
\begin{flushright}
提出期限:2025年 11月 27日(木) 12:30
\end{flushright}

\begin{center}
{\Large \textbf{電子工学 課題1 レポート}}
\end{center}

\vspace{1em}
\begin{flushright}
\underline{学籍番号:\hspace{3cm} 氏名:\hspace{4cm}}
\end{flushright}
\vspace{1em}

%--- 定数リスト ---
\section*{使用定数}
本課題では以下の物理定数を用いる。
\begin{align*}
k &= 1.38 \times 10^{-23} \unit{J \cdot K^{-1}} & e &= 1.60 \times 10^{-19} \unit{C} \\
m &= 9.11 \times 10^{-31} \unit{kg} & h &= 6.63 \times 10^{-34} \unit{J \cdot s} \\
c &= 3.00 \times 10^{8} \unit{m/s}
\end{align*}

\hrulefill

%--- 課題1 ---
\section*{課題1}
\textbf{条件:}
\begin{itemize}
 \item 温度 $T = 2500 \unit{K}$
 \item 半径 $r = 1.50 \times 10^{-4} \unit{m}$
 \item 全電流 $I = 2.00 \times 10^{-3} \unit{A}$
 \item 仕事関数 $\phi = 4.52 \unit{eV}$
\end{itemize}

\noindent
\textbf{解答:}

リチャードソン・ダッシュマンの式より、電流密度 $J$ は次式で与えられる。
\begin{equation}
J = A T^2 \exp\left( -\frac{\phi}{kT} \right)
\end{equation}
ここで定数 $A$(リチャードソン定数)の理論値を計算する。
\begin{align*}
A &= \frac{4\pi m e k^2}{h^3} \\
  &= \frac{4 \times \pi \times (9.11 \times 10^{-31}) \times (1.60 \times 10^{-19}) \times (1.38 \times 10^{-23})^2}{(6.63 \times 10^{-34})^3} \\
  &\approx 1.201 \times 10^6 \unit{A \cdot m^{-2} \cdot K^{-2}}
\end{align*}
仕事関数 $\phi$ をジュール単位に換算する。
\begin{align*}
\phi_{\mathrm{J}} &= 4.52 \times (1.60 \times 10^{-19}) \\
                  &= 7.232 \times 10^{-19} \unit{J}
\end{align*}
これらを式(1)に代入し、電流密度 $J$ を求める。指数部は
\[
\frac{\phi_{\mathrm{J}}}{kT} = \frac{7.232 \times 10^{-19}}{(1.38 \times 10^{-23}) \times 2500} \approx 20.962
\]
であるから、
\begin{align*}
J &= (1.201 \times 10^6) \times (2500)^2 \times \exp(-20.962) \\
  &= 7.506 \times 10^{12} \times 7.876 \times 10^{-10} \\
  &\approx 5.912 \times 10^3 \unit{A/m^2}
\end{align*}
電極の側面積 $S$ は、長さ $L$ を用いて $S = 2\pi r L$ と表される。全電流 $I = J \cdot S$ より、
\begin{align*}
L &= \frac{I}{J \cdot 2\pi r} \\
  &= \frac{2.00 \times 10^{-3}}{(5.912 \times 10^3) \times 2\pi \times (1.50 \times 10^{-4})} \\
  &= \frac{2.00 \times 10^{-3}}{5.572} \\
  &\approx 3.589 \times 10^{-4}
\end{align*}
有効数字3桁で整理する。
\[
\therefore L = 3.59 \times 10^{-4} \unit{m}
\]
(※理論定数 $A$ の計算精度のとり方により $3.60 \times 10^{-4}$ となる場合もあるが、上記の通り導出した)

%--- 課題2 ---
\section*{課題2}
\textbf{条件:}
\begin{itemize}
 \item 仕事関数 $\phi = 4.27 \unit{eV}$
 \item 波長 $\lambda = 45.5 \unit{nm} = 45.5 \times 10^{-9} \unit{m}$
\end{itemize}

\noindent
\textbf{解答:}

アインシュタインの光電効果の式より
\begin{equation}
h\nu = \phi + \frac{1}{2}mv^2 \quad \Longleftrightarrow \quad \frac{hc}{\lambda} = \phi + K_{\max}
\end{equation}
まず、入射光のエネルギー $E = hc/\lambda$ を求める。
\begin{align*}
E &= \frac{(6.63 \times 10^{-34}) \times (3.00 \times 10^8)}{45.5 \times 10^{-9}} \\
  &= \frac{1.989 \times 10^{-25}}{4.55 \times 10^{-8}} \\
  &\approx 4.371 \times 10^{-18} \unit{J}
\end{align*}
次に、仕事関数 $\phi$ をジュール単位に換算する。
\begin{align*}
\phi_{\mathrm{J}} &= 4.27 \times (1.60 \times 10^{-19}) \\
                  &= 6.832 \times 10^{-19} \unit{J}
\end{align*}
光電子の最大運動エネルギー $K_{\max}$ は
\begin{align*}
K_{\max} &= E - \phi_{\mathrm{J}} \\
         &= 4.371 \times 10^{-18} - 0.683 \times 10^{-18} \\
         &= 3.688 \times 10^{-18} \unit{J}
\end{align*}
最大速度 $v$ を求める。
\begin{align*}
v &= \sqrt{\frac{2 K_{\max}}{m}} \\
  &= \sqrt{\frac{2 \times (3.688 \times 10^{-18})}{9.11 \times 10^{-31}}} \\
  &= \sqrt{\frac{7.376 \times 10^{-18}}{9.11 \times 10^{-31}}} \\
  &= \sqrt{8.096 \times 10^{12}} \\
  &\approx 2.845 \times 10^6
\end{align*}
有効数字3桁で整理する。
\[
\therefore v = 2.85 \times 10^6 \unit{m/s}
\]

%--- 課題3 ---
\section*{課題3}
\textbf{条件:}
\begin{itemize}
 \item 二次電子放出比 $\delta = 4.0$
 \item 段数 $n = 10$
 \item コレクタ電流 $I_o = 0.125 \times 10^{-3} \unit{A}$
\end{itemize}

\noindent
\textbf{解答:}

光電子増倍管の総合利得(ゲイン)$G$ は次式となる。
\[
G = \delta^n = 4.0^{10}
\]
コレクタ電流 $I_o$ と光電面からの光電流 $I_p$ の関係は $I_o = G I_p$ であるため、
\begin{align*}
I_p &= \frac{I_o}{G} = \frac{0.125 \times 10^{-3}}{4^{10}} \\
    &= \frac{1.25 \times 10^{-4}}{(2^2)^{10}} = \frac{1.25 \times 10^{-4}}{2^{20}}
\end{align*}
ここで $2^{10} = 1024 \approx 1.024 \times 10^3$ より、
\[
2^{20} = (1.024 \times 10^3)^2 \approx 1.049 \times 10^6
\]
よって、
\begin{align*}
I_p &= \frac{1.25 \times 10^{-4}}{1.049 \times 10^6} \\
    &\approx 1.192 \times 10^{-10}
\end{align*}
有効数字3桁で整理する。
\[
\therefore I_p = 1.19 \times 10^{-10} \unit{A}
\]

%--- 課題4 ---
\section*{課題4}
\textbf{条件:}
電位分布が次式で与えられる。
\[
V = \frac{1}{\sqrt{x^2 + y^2 + z^2}} \unit{V}
\]

\noindent
\textbf{解答:}

電界ベクトル $\bm{E}$ と電位 $V$ の関係は $\bm{E} = -\nabla V$ である。すなわち、
\[
\bm{E} = \left( E_x, E_y, E_z \right) = \left( -\frac{\partial V}{\partial x}, -\frac{\partial V}{\partial y}, -\frac{\partial V}{\partial z} \right)
\]
まず $x$ 成分について計算する。$V = (x^2 + y^2 + z^2)^{-\frac{1}{2}}$ として偏微分を行う。
\begin{align*}
\frac{\partial V}{\partial x} &= \frac{\partial}{\partial x} \left( (x^2 + y^2 + z^2)^{-\frac{1}{2}} \right) \\
&= -\frac{1}{2} (x^2 + y^2 + z^2)^{-\frac{3}{2}} \cdot \frac{\partial}{\partial x}(x^2 + y^2 + z^2) \\
&= -\frac{1}{2} (x^2 + y^2 + z^2)^{-\frac{3}{2}} \cdot (2x) \\
&= -x (x^2 + y^2 + z^2)^{-\frac{3}{2}}
\end{align*}
したがって、電界の $x$ 成分 $E_x$ は
\[
E_x = - \frac{\partial V}{\partial x} = x (x^2 + y^2 + z^2)^{-\frac{3}{2}} = \frac{x}{(x^2 + y^2 + z^2)^{\frac{3}{2}}}
\]
$y, z$ 成分についても式の対称性より同様であるため、以下のように求められる。
\begin{align*}
E_y &= \frac{y}{(x^2 + y^2 + z^2)^{\frac{3}{2}}} \\
E_z &= \frac{z}{(x^2 + y^2 + z^2)^{\frac{3}{2}}}
\end{align*}

\noindent
\textbf{答:}
\begin{equation*}
\left\{
\begin{aligned}
E_x &= \frac{x}{(x^2 + y^2 + z^2)^{\frac{3}{2}}} \unit{V/m} \\
E_y &= \frac{y}{(x^2 + y^2 + z^2)^{\frac{3}{2}}} \unit{V/m} \\
E_z &= \frac{z}{(x^2 + y^2 + z^2)^{\frac{3}{2}}} \unit{V/m}
\end{aligned}
\right.
\end{equation*}

\end{document}