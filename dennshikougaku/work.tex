\documentclass[a4paper,11pt]{ltjsarticle} % LuaLaTeX用のクラス

%------------------------------------------------------------------------------
% パッケージ読み込み
%------------------------------------------------------------------------------
\usepackage[top=25mm,bottom=25mm,left=25mm,right=25mm]{geometry} % 余白設定
\usepackage{amsmath,amssymb} % 数式用
\usepackage{bm}              % ベクトルなどの太字用

%------------------------------------------------------------------------------
% コマンド定義
%------------------------------------------------------------------------------
% 単位記述用のコマンド
\newcommand{\unit}[1]{\,[\mathrm{#1}]}

%==============================================================================
% 本文
%==============================================================================
\begin{document}

%--- ヘッダー情報 ---
\begin{flushright}
提出期限:2025年 11月 27日(木) 12:30
\end{flushright}

\begin{center}
{\Large \textbf{電子工学 課題1 レポート}}
\end{center}

\vspace{1em}
\begin{flushright}
\underline{学籍番号:\hspace{3cm} 氏名:\hspace{4cm}}
\end{flushright}
\vspace{1em}

%--- 定数リスト ---
\section*{使用定数}
本講義ノート(第2章 電子放出)および配布資料に基づき、以下の物理定数を用いる。
\begin{align*}
k &= 1.38 \times 10^{-23} \unit{J \cdot K^{-1}} & e &= 1.60 \times 10^{-19} \unit{C} \\
m &= 9.11 \times 10^{-31} \unit{kg} & h &= 6.63 \times 10^{-34} \unit{J \cdot s} \\
c &= 3.00 \times 10^{8} \unit{m/s}
\end{align*}

\hrulefill

%--- 課題1 ---
\section*{課題1}
\textbf{条件:}
\begin{itemize}
 \item 温度 $T = 2500 \unit{K}$
 \item タングステンフィラメント半径 $r = 1.50 \times 10^{-4} \unit{m}$
 \item 全電流 $I = 2.00 \times 10^{-3} \unit{A}$
 \item 仕事関数 $\phi = 4.52 \unit{V}$ (※エネルギー値 $4.52 \unit{eV}$ に相当)
\end{itemize}

\noindent
\textbf{解答:}

講義資料の式(2.7) リチャードソン・ダッシュマンの式より、電流密度 $J$ は次式で表される。
\begin{equation}
J = A T^2 \exp\left( -\frac{e\phi}{kT} \right) \tag{2.7}
\end{equation}
ここでリチャードソン定数 $A$ の理論値を導出・確認する。
\begin{align*}
A &= \frac{4\pi m e k^2}{h^3} \\
  &= \frac{4\pi \times (9.11 \times 10^{-31}) \times (1.60 \times 10^{-19}) \times (1.38 \times 10^{-23})^2}{(6.63 \times 10^{-34})^3} \\
  &= \frac{12.566 \times 9.11 \times 1.60 \times 1.9044}{291.43} \times \frac{10^{-31} \cdot 10^{-19} \cdot 10^{-46}}{10^{-102}} \\
  &= \frac{348.8}{291.4} \times 10^{(-96) - (-102)} \\
  &\approx 1.197 \times 10^6 \unit{A \cdot m^{-2} \cdot K^{-2}}
\end{align*}
本計算では、講義で示された標準的な理論値 $A \approx 1.20 \times 10^6$ を採用する。

次に、指数項の引数(仕事関数のエネルギー換算項)を計算する。
\begin{align*}
\frac{e\phi}{kT} &= \frac{(1.60 \times 10^{-19}) \times 4.52}{(1.38 \times 10^{-23}) \times 2500} \\
                 &= \frac{7.232 \times 10^{-19}}{3.450 \times 10^{-20}} \\
                 &= \frac{7.232}{3.450} \times 10^{1} \\
                 &\approx 20.9623
\end{align*}
これを式(2.7)に代入し、電流密度 $J$ を求める。
\begin{align*}
J &= (1.20 \times 10^6) \times (2500)^2 \times \exp(-20.9623) \\
  &= (1.20 \times 10^6) \times (6.25 \times 10^6) \times (7.874 \times 10^{-10}) \\
  &= (1.20 \times 6.25 \times 7.874) \times 10^{(6+6-10)} \\
  &= 59.055 \times 10^2 \\
  &\approx 5.906 \times 10^3 \unit{A/m^2}
\end{align*}
電極の側面積 $S = 2\pi r L$ と全電流 $I = J S$ の関係より、必要なフィラメント長 $L$ を逆算する。
\begin{align*}
L &= \frac{I}{J \cdot 2\pi r} \\
  &= \frac{2.00 \times 10^{-3}}{(5.906 \times 10^3) \times 2\pi \times (1.50 \times 10^{-4})} \\
  &= \frac{2.00 \times 10^{-3}}{(5.906 \times 2\pi \times 1.50) \times 10^{-1}} \\
  &= \frac{2.00 \times 10^{-3}}{55.66 \times 10^{-1}} \\
  &= 0.03593 \times 10^{-2} \\
  &= 3.593 \times 10^{-4}
\end{align*}
有効数字3桁で整理する。
\[
\therefore L = 3.59 \times 10^{-4} \unit{m}
\]

%--- 課題2 ---
\section*{課題2}
\textbf{条件:}
\begin{itemize}
 \item 仕事関数 $\phi = 4.27 \unit{V}$ (エネルギー $\phi = 4.27 \unit{eV}$)
 \item 入射光波長 $\lambda = 45.5 \unit{nm} = 4.55 \times 10^{-8} \unit{m}$
\end{itemize}

\noindent
\textbf{解答:}

アインシュタインの光電効果の式(講義ノート§2.3参照)より
\begin{equation}
h\nu = e\phi + \frac{1}{2}mv^2 \quad \Longleftrightarrow \quad \frac{hc}{\lambda} = e\phi + K_{\max}
\end{equation}

まず、入射光子のエネルギー $E = hc/\lambda$ を算出する。
\begin{align*}
E &= \frac{(6.63 \times 10^{-34}) \times (3.00 \times 10^8)}{4.55 \times 10^{-8}} \\
  &= \frac{19.89}{4.55} \times \frac{10^{-26}}{10^{-8}} \\
  &= 4.3714 \times 10^{-18} \unit{J}
\end{align*}

次に、仕事関数のエネルギー項 $W = e\phi$ をジュール単位で算出する。
\begin{align*}
W &= 1.60 \times 10^{-19} \times 4.27 \\
  &= 6.832 \times 10^{-19} \unit{J}
\end{align*}

光電子の最大運動エネルギー $K_{\max}$ を求める。桁数を合わせるため $E = 43.71 \times 10^{-19} \unit{J}$ として計算する。
\begin{align*}
K_{\max} &= E - W \\
         &= (43.714 - 6.832) \times 10^{-19} \\
         &= 36.882 \times 10^{-19} \\
         &= 3.688 \times 10^{-18} \unit{J}
\end{align*}

最大速度 $v$ を導出する。
\begin{align*}
v &= \sqrt{\frac{2 K_{\max}}{m}} \\
  &= \sqrt{\frac{2 \times (3.688 \times 10^{-18})}{9.11 \times 10^{-31}}} \\
  &= \sqrt{\frac{7.376}{9.11} \times 10^{13}} \\
  &= \sqrt{0.8096 \times 10 \times 10^{12}} \\
  &= \sqrt{8.096} \times 10^6 \\
  &\approx 2.845 \times 10^6
\end{align*}
有効数字3桁で整理する。
\[
\therefore v = 2.85 \times 10^6 \unit{m/s}
\]

%--- 課題3 ---
\section*{課題3}
\textbf{条件:}
\begin{itemize}
 \item 二次電子放出比 $\delta = 4.0$
 \item ダイノード段数 $n = 10$
 \item コレクタ電流 $I_o = 0.125 \times 10^{-3} \unit{A}$
\end{itemize}

\noindent
\textbf{解答:}

光電子増倍管の総合利得(ゲイン)$G$ は、各段での増幅率の積となるため次式で与えられる。
\[
G = \delta^n = 4.0^{10}
\]
ここで $4^{10}$ の値を近似計算する($2^{10} = 1024 \approx 1.024 \times 10^3$ を利用)。
\begin{align*}
G &= (2^2)^{10} = 2^{20} = (2^{10})^2 \\
  &\approx (1.024 \times 10^3)^2 \\
  &\approx 1.049 \times 10^6
\end{align*}

光電面からの一次光電流 $I_p$ は、関係式 $I_o = G I_p$ より逆算される。
\begin{align*}
I_p &= \frac{I_o}{G} \\
    &= \frac{0.125 \times 10^{-3}}{1.049 \times 10^6} \\
    &= \frac{0.125}{1.049} \times 10^{-9} \\
    &\approx 0.1191 \times 10^{-9} \\
    &= 1.191 \times 10^{-10}
\end{align*}
有効数字3桁で整理する。
\[
\therefore I_p = 1.19 \times 10^{-10} \unit{A}
\]

%--- 課題4 ---
\section*{課題4}
\textbf{条件:}
電位分布 $V(x,y,z)$ が次式で与えられる(点電荷による電位分布の形状)。
\[
V = \frac{1}{\sqrt{x^2 + y^2 + z^2}} = (x^2 + y^2 + z^2)^{-\frac{1}{2}} \unit{V}
\]

\noindent
\textbf{解答:}

静電場における電界ベクトル $\bm{E}$ は、電位 $V$ の勾配(gradient)として定義される(ノート§3.1参照)。
\[
\bm{E} = -\nabla V = - \left( \frac{\partial V}{\partial x} \bm{i} + \frac{\partial V}{\partial y} \bm{j} + \frac{\partial V}{\partial z} \bm{k} \right)
\]

まず $x$ 成分について偏微分を行う。合成関数の微分法を用いる。
\begin{align*}
\frac{\partial V}{\partial x} &= \frac{\partial}{\partial x} \left\{ (x^2 + y^2 + z^2)^{-\frac{1}{2}} \right\} \\
&= -\frac{1}{2} (x^2 + y^2 + z^2)^{-\frac{3}{2}} \cdot \frac{\partial}{\partial x}(x^2 + y^2 + z^2) \\
&= -\frac{1}{2} (x^2 + y^2 + z^2)^{-\frac{3}{2}} \cdot (2x) \\
&= -x (x^2 + y^2 + z^2)^{-\frac{3}{2}}
\end{align*}

したがって、電界の $x$ 成分 $E_x$ は以下の通りとなる。
\[
E_x = - \frac{\partial V}{\partial x} = - \left( - \frac{x}{(x^2 + y^2 + z^2)^{\frac{3}{2}}} \right) = \frac{x}{(x^2 + y^2 + z^2)^{\frac{3}{2}}}
\]

$y, z$ 成分についても式の対称性より同様である。
\begin{align*}
E_y &= - \frac{\partial V}{\partial y} = \frac{y}{(x^2 + y^2 + z^2)^{\frac{3}{2}}} \\
E_z &= - \frac{\partial V}{\partial z} = \frac{z}{(x^2 + y^2 + z^2)^{\frac{3}{2}}}
\end{align*}

\noindent
\textbf{答:}
\begin{equation*}
\bm{E} = \frac{1}{(x^2 + y^2 + z^2)^{\frac{3}{2}}} (x, y, z) \unit{V/m}
\end{equation*}
または成分表示にて、
\begin{equation*}
\left\{
\begin{aligned}
E_x &= \frac{x}{(x^2 + y^2 + z^2)^{\frac{3}{2}}} \\
E_y &= \frac{y}{(x^2 + y^2 + z^2)^{\frac{3}{2}}} \\
E_z &= \frac{z}{(x^2 + y^2 + z^2)^{\frac{3}{2}}}
\end{aligned}
\right.
\end{equation*}

\end{document}