\documentclass[a4paper,11pt]{ltjsarticle}

%------------------------------------------------------------------------------
% パッケージ読み込み
%------------------------------------------------------------------------------
\usepackage[top=25mm,bottom=25mm,left=25mm,right=25mm]{geometry}
\usepackage{amsmath,amssymb}
\usepackage{bm}

%------------------------------------------------------------------------------
% コマンド定義
%------------------------------------------------------------------------------
% Define unit macro only if not defined (avoid collision with siunitx)
\providecommand{\unit}[1]{\,[\mathrm{#1}]}

%==============================================================================
% 本文
%==============================================================================
\begin{document}

%--- ヘッダー情報 ---
\begin{flushright}
提出期限:2025年 11月 27日(木) 12:30
\end{flushright}

\begin{center}
{\Large \textbf{電子工学 課題1 レポート}}
\end{center}

\vspace{1em}
\begin{flushright}
\underline{学籍番号:\hspace{3cm} 氏名:\hspace{4cm}}
\end{flushright}
\vspace{1em}

%--- 定数リスト ---
\section*{使用定数}
\begin{align*}
k &= 1.38 \times 10^{-23} \unit{J \cdot K^{-1}} & e &= 1.60 \times 10^{-19} \unit{C} \\
m &= 9.11 \times 10^{-31} \unit{kg} & h &= 6.63 \times 10^{-34} \unit{J \cdot s} \\
c &= 3.00 \times 10^{8} \unit{m/s}
\end{align*}

\hrulefill

%--- 課題1 ---
\section*{課題1}
\textbf{条件:}
\begin{itemize}
 \item 温度 $T = 2500 \unit{K}$
 \item 半径 $r = 1.50 \times 10^{-4} \unit{m}$
 \item 全電流 $I = 2.00 \times 10^{-3} \unit{A}$
 \item 仕事関数 $\phi = 4.52 \unit{eV}$
\end{itemize}

\noindent
\textbf{解答:}

リチャードソン定数 $A$(理論値)
\begin{align*}
A &= \frac{4\pi m e k^2}{h^3} \\
  &= \frac{4\pi (9.11 \times 10^{-31}) (1.60 \times 10^{-19}) (1.38 \times 10^{-23})^2}{(6.63 \times 10^{-34})^3} \\
  &\approx 1.201 \times 10^6 \unit{A \cdot m^{-2} \cdot K^{-2}}
\end{align*}

リチャードソン・ダッシュマンの式より、電流密度 $J$
\begin{align*}
J &= A T^2 \exp\left( -\frac{e\phi}{kT} \right) \\
  &= (1.201 \times 10^6) \times (2500)^2 \times \exp\left( -\frac{(1.60 \times 10^{-19}) \times 4.52}{(1.38 \times 10^{-23}) \times 2500} \right) \\
  &= (7.506 \times 10^{12}) \times \exp( -20.9623 \dots ) \\
  &\approx 5.912 \times 10^3 \unit{A/m^2}
\end{align*}

全電流の式 $I = J \cdot S = J \cdot (2\pi r L)$ より、フィラメント長 $L$
\begin{align*}
L &= \frac{I}{2\pi r J} \\
  &= \frac{2.00 \times 10^{-3}}{2\pi \times (1.50 \times 10^{-4}) \times (5.912 \times 10^3)} \\
  &\approx 3.589 \times 10^{-4}
\end{align*}

\[
\therefore L = 3.59 \times 10^{-4} \unit{m}
\]

%--- 課題2 ---
\section*{課題2}
\textbf{条件:}
\begin{itemize}
 \item 仕事関数 $\phi = 4.27 \unit{eV}$
 \item 波長 $\lambda = 45.5 \unit{nm} = 4.55 \times 10^{-8} \unit{m}$
\end{itemize}

\noindent
\textbf{解答:}

光電効果の式より最大速度 $v_m$
\begin{equation}
\frac{hc}{\lambda} = e\phi + \frac{1}{2}mv_m^2 \quad \Longleftrightarrow \quad v_m = \sqrt{ \frac{2}{m} \left( \frac{hc}{\lambda} - e\phi \right) }
\end{equation}

数値を代入
\begin{align*}
v_m &= \sqrt{ \frac{2}{9.11 \times 10^{-31}} \left( \frac{(6.63 \times 10^{-34})(3.00 \times 10^8)}{4.55 \times 10^{-8}} - (1.60 \times 10^{-19})(4.27) \right) } \\
    &= \sqrt{ \frac{2}{9.11 \times 10^{-31}} ( 4.3714 \dots \times 10^{-18} - 6.832 \times 10^{-19} ) } \\
    &= \sqrt{ \frac{7.376 \dots \times 10^{-18}}{9.11 \times 10^{-31}} } \\
    &\approx 2.845 \times 10^6
\end{align*}

\[
\therefore v_m = 2.85 \times 10^6 \unit{m/s}
\]

%--- 課題3 ---
\section*{課題3}
\textbf{条件:}
\begin{itemize}
 \item 二次電子放出比 $\delta = 4.0$
 \item 段数 $n = 10$
 \item コレクタ電流 $I_o = 0.125 \times 10^{-3} \unit{A}$
\end{itemize}

\noindent
\textbf{解答:}

総合利得 $G = \delta^n$ より、一次光電流 $I_p$
\begin{align*}
I_p &= \frac{I_o}{\delta^n} \\
    &= \frac{0.125 \times 10^{-3}}{4.0^{10}} \\
    &= \frac{1.25 \times 10^{-4}}{1.048576 \times 10^6} \\
    &\approx 1.192 \times 10^{-10}
\end{align*}

\[
\therefore I_p = 1.19 \times 10^{-10} \unit{A}
\]

%--- 課題4 ---
\section*{課題4}
\textbf{条件:}
\[
V = \frac{1}{\sqrt{x^2 + y^2 + z^2}} \unit{V}
\]

\noindent
\textbf{解答:}

電界 $\bm{E} = -\nabla V$
\[
\bm{E} = - \left( \frac{\partial V}{\partial x} \bm{i} + \frac{\partial V}{\partial y} \bm{j} + \frac{\partial V}{\partial z} \bm{k} \right)
\]

$x$ 成分の計算
\begin{align*}
E_x &= - \frac{\partial}{\partial x} (x^2 + y^2 + z^2)^{-\frac{1}{2}} \\
    &= - \left( -\frac{1}{2} \right) (x^2 + y^2 + z^2)^{-\frac{3}{2}} \cdot \frac{\partial}{\partial x}(x^2 + y^2 + z^2) \\
    &= \frac{1}{2} (x^2 + y^2 + z^2)^{-\frac{3}{2}} \cdot (2x) \\
    &= \frac{x}{(x^2 + y^2 + z^2)^{\frac{3}{2}}}
\end{align*}

対称性より
\begin{align*}
E_y &= \frac{y}{(x^2 + y^2 + z^2)^{\frac{3}{2}}} \\
E_z &= \frac{z}{(x^2 + y^2 + z^2)^{\frac{3}{2}}}
\end{align*}

\[
\therefore \bm{E} = \frac{1}{(x^2 + y^2 + z^2)^{\frac{3}{2}}} (x, y, z) \unit{V/m}
\]

\end{document}