% !TEX program = lualatex
%==============================================================================
% 電子工学(5E) 後期中間到達度試験 統合対策資料 (完全版・最終改訂)
%==============================================================================
% 制作:試験対策委員会 + AI特別顧問
% 内容:
% 1. 試験範囲ポイント①〜⑲の完全網羅解説 (物理学的定義・数式)
% 2. 2024年度 過去問 模範解答 (全問・計算過程詳細)
% 3. 2023年度 過去問 模範解答 (全問・計算過程詳細)
% 4. レポート課題 (重要演習) の詳細解答
%==============================================================================

\documentclass[a4paper,11pt]{ltjsarticle}

%------------------------------------------------------------------------------
% パッケージ設定
%------------------------------------------------------------------------------
\usepackage{lmodern}              % フォントサイズ警告を解消
\usepackage[left=15mm,right=15mm,top=20mm,bottom=20mm]{geometry}
\usepackage{amsmath, amssymb, bm} % 数式

%--- siunitxとphysicsの競合対策 ---
\usepackage{physics}              % 微分などの記述簡略化
\usepackage{siunitx}              % 単位の統一

% siunitxがphysicsパッケージを検知した際の警告を抑制
\ExplSyntaxOn
\msg_redirect_name:nnn { siunitx } { physics-package } { none }
\ExplSyntaxOff

% \qtyコマンドの競合回避(siunitxの機能を優先)
\AtBeginDocument{\RenewCommandCopy\qty\SI \let\qty\SI}
%--------------------------------

\usepackage{graphicx}             % 画像
\usepackage{booktabs}             % 表
\usepackage{tcolorbox}            % 枠線・ボックス
\tcbuselibrary{skins, breakable, theorems}
\usepackage{enumitem}             % リスト設定
\usepackage{hyperref}             % リンク
\usepackage{cancel}               % 計算過程のキャンセル線

\hypersetup{colorlinks=true, linkcolor=blue, urlcolor=cyan}

%------------------------------------------------------------------------------
% 定数・コマンド定義
%------------------------------------------------------------------------------
\newcommand{\EF}{E_{\mathrm{F}}} % フェルミ準位
\newcommand{\kB}{k}              % ボルツマン定数

% \unitコマンドの定義
\AtBeginDocument{%
    \ifcsname unit\endcsname
        \expandafter\let\csname siunitxUnit\endcsname\unit
    \else
        \providecommand{\siunitxUnit}[1]{\,[\mathrm{#1}]}%
    \fi
    \renewcommand{\unit}[1]{\,[\mathrm{#1}]}%
}

% 試験用物理定数
\newcommand{\ValE}{1.60 \times 10^{-19}}     % 素電荷 [C]
\newcommand{\ValK}{1.38 \times 10^{-23}}     % ボルツマン定数 [J/K]
\newcommand{\ValA}{1.20 \times 10^6}         % リチャードソン定数 [A/m2K2]
\newcommand{\ValH}{6.63 \times 10^{-34}}     % プランク定数 [Js]
\newcommand{\ValC}{3.00 \times 10^8}         % 光速 [m/s]
\newcommand{\ValMe}{9.11 \times 10^{-31}}    % 電子質量 [kg]

% 解説用ボックス定義
\newtcolorbox{pointbox}[1]{colback=blue!5!white, colframe=blue!75!black, title=\textbf{#1}, breakable}
\newtcolorbox{termbox}[1]{colback=yellow!5!white, colframe=orange!80!black, title=\textbf{【用語解説】#1}, breakable}
\newtcolorbox{warnbox}[1]{colback=red!5!white, colframe=red!75!black, title=\textbf{【注意】#1}, breakable}
\newtcolorbox{calcbox}[1]{colback=gray!5!white, colframe=black, title=\textbf{#1}, breakable}
\newtcolorbox{derivationbox}[1]{colback=green!5!white, colframe=green!40!black, title=\textbf{【重要導出】#1}, breakable}
\newtcolorbox{ansbox}[1]{colback=green!5!white, colframe=green!40!black, title=\textbf{解答}, breakable}

\title{\textbf{電子工学(5E) 試験対策 統合完全版}}
\author{理論解説・過去問解答・レポート演習}
\date{}

\begin{document}

\maketitle

\begin{abstract}
本資料は、令和7年度『電子工学』後期中間達成度試験に向けた統合対策資料である。
試験範囲の重要ポイント(①〜⑲)の物理的解説、2024年度および2023年度の過去問の詳細解答、さらに重要演習であるレポート課題の解答プロセスを網羅している。
記述問題・計算問題ともに、論理的な導出過程を省略せず記述した。
\end{abstract}

\tableofcontents
\newpage

%==============================================================================
\part{試験範囲ポイント完全解説 (①〜⑲)}
%==============================================================================
\noindent
試験範囲として指定された19のポイントについて、物理的な意味と試験での問われ方を解説する。

\section{エネルギーバンドと電子放出の基礎 (①〜⑦)}

\subsection*{① 価電子帯、禁制帯、伝導帯とはなにか}
固体中の電子が取りうるエネルギー準位の構造(バンド構造)に関する定義である。

\begin{pointbox}{バンド構造の3要素}
\begin{description}
    \item[価電子帯 (Valence Band)]
    原子核に束縛されている電子(価電子)が充満しているエネルギー帯。
    絶縁体や半導体では、絶対零度においてこの帯域は電子で完全に満たされており、電流は流れない。
    
    \item[禁制帯 (Forbidden Band / Band Gap)]
    量子力学的な制約により、電子が存在することのできないエネルギー領域。
    価電子帯の上端と伝導帯の下端の間のエネルギー差(バンドギャップ $E_g$)を指す。
    
    \item[伝導帯 (Conduction Band)]
    原子の束縛を離れ、結晶内を自由に動き回れる電子(自由電子)が存在するエネルギー帯。
    ここに電子が励起されることで、電気伝導性が生じる。
\end{description}
\end{pointbox}

\textbf{※金属(導体)の特徴}:
金属では「価電子帯」と「伝導帯」が重なっているか、価電子帯自体に空席がある状態であるため、絶対零度付近でも電子は自由に動くことができ、高い導電性を示す。

\subsection*{② 外部エネルギーの入射により電子が放出されるしくみ}
電子放出とは、物質内部のポテンシャル井戸に束縛されている電子が、障壁を乗り越えて外部(真空)へ脱出する現象である。

\begin{enumerate}
    \item \textbf{定常状態}: 電子は通常、フェルミ準位以下の低いエネルギー状態にある。
    \item \textbf{外部励起}: 外部からエネルギー $\Delta E$ を与える。
    \begin{itemize}
        \item 熱エネルギー ($kT$) $\to$ \textbf{熱電子放出}
        \item 光エネルギー ($h\nu$) $\to$ \textbf{光電子放出}
        \item 強電界 ($E$) $\to$ \textbf{電界放出} (障壁の変形・透過)
        \item 電子衝突 ($1/2 mv^2$) $\to$ \textbf{二次電子放出}
    \end{itemize}
    \item \textbf{脱出}: 電子の総エネルギーが真空準位 $E_{vac}$ を超えたとき、電子は表面から放出される。
\end{enumerate}

\subsection*{③ 金属内電子が金属外に飛び出さない理由}
常温の金属から電子が放出されない理由は、エネルギー障壁が存在するためである。

\begin{itemize}
    \item \textbf{電位障壁 (Potential Barrier)}: 金属表面には電子を閉じ込めるエネルギーの壁がある。
    \item \textbf{鏡像力 (Image Force)}: 電子が金属表面から外に出ようとすると、金属表面に誘導された正電荷(鏡像電荷)との間にクーロン引力が働き、引き戻される。これが障壁の物理的実体の一つである。
    \item \textbf{エネルギー不足}: 室温 ($300\unit{K}$) の熱エネルギー $kT \approx 0.026\unit{eV}$ は、一般的な仕事関数 $\phi \approx 4\sim5\unit{eV}$ に比べて極めて小さいため、熱的に脱出できる確率はほぼゼロである。
\end{itemize}

\subsection*{④ 電子放出の共通的基礎(各用語の関係)}
これらの用語は、ポテンシャル井戸モデルにおけるエネルギー保存則で結ばれている。

\begin{tcolorbox}[colback=white, colframe=black, title=\textbf{エネルギー関係式}]
\begin{equation}
    W = \EF + \phi \quad \Longleftrightarrow \quad \phi = W - \EF
\end{equation}
\end{tcolorbox}

\begin{termbox}{用語の物理的定義}
    \begin{description}
        \item[全障壁 $W$]: 井戸の底(電子が存在しうる最低エネルギー準位)から、真空準位(脱出に必要なエネルギー)までの全高。
        \item[フェルミ準位 $\EF$]: 絶対零度において電子が占有している最高のエネルギー準位(井戸の底からの高さ)。
        \item[仕事関数 $\phi$]: フェルミ準位にある電子を、真空準位まで引き上げるために必要な最小エネルギー。
    \end{description}
\end{termbox}

\subsection*{⑤ 金属内電子のエネルギー(0Kにおける状態)}
\begin{itemize}
    \item \textbf{絶対零度での状態}: パウリの排他律により、電子は低いエネルギー準位から順に隙間なく詰まっていく。
    \item \textbf{最大エネルギー}: 0Kにおいて電子が持つ最大のエネルギーは\textbf{フェルミ準位 $\EF$}(フェルミエネルギー)である。
    \item \textbf{放出エネルギー}: したがって、電子を外部へ放出するために最低限必要なエネルギーは、\textbf{仕事関数 $\phi$} となる。
\end{itemize}

\subsection*{⑥ フェルミ準位とフェルミ分布関数の意味}
電子がエネルギー準位 $E$ を占有する確率は、フェルミ・ディラック統計に従う。

\begin{equation}
    F(E) = \frac{1}{1 + \exp\left(\frac{E-\EF}{\kB T}\right)}
\end{equation}

\begin{itemize}
    \item \textbf{フェルミ分布関数 $F(E)$}: エネルギー $E$ の状態が電子に占有されている確率 ($0 \le F(E) \le 1$)。
    \item \textbf{フェルミ準位 $\EF$ の定義}: 電子の占有確率 $F(E)$ がちょうど **$1/2$ ($50\%$)** となるエネルギー準位のこと。
\end{itemize}

\subsection*{⑦ エネルギー準位図のグラフ $F(E), n(E)$}
温度による分布の変化が重要である。
\begin{itemize}
    \item \textbf{$T=0\unit{K}$}: $\EF$ で確率が $1 \to 0$ に急峻に変化する階段関数(ステップ関数)。
    \item \textbf{$T>0\unit{K}$}: $\EF$ 付近でなだらかに変化するシグモイド曲線。$\EF$ より高いエネルギーを持つ電子が確率的に存在するようになり(高エネルギーの裾野)、これが熱電子放出の源となる。
\end{itemize}

\section{熱電子放出 (⑧〜⑩)}

\subsection*{⑧ 熱電子の飽和電流密度 (ダッシュマン・リチャードソンの式)}
熱電子放出における電流密度を与える基本式。

\begin{tcolorbox}[colback=white, colframe=blue, title=\textbf{公式: リチャードソン・ダッシュマンの式}]
    \begin{equation}
        J = A T^2 \exp\left( -\frac{e\phi}{\kB T} \right) \quad [\si{A/m^2}]
    \end{equation}
\end{tcolorbox}
\begin{itemize}
    \item $T$: 絶対温度 [K]
    \item $e\phi$: 仕事関数 [J]
    \item $\exp$項: エネルギー障壁を超える高エネルギー電子の存在確率(ボルツマン因子)に由来する。
\end{itemize}

\subsection*{⑨ リチャードソン線から仕事関数と $A$ を求める方法}
実験データから材料定数を決定する手法。式を対数変換して直線化する。
\[
    \ln\left( \frac{J}{T^2} \right) = \ln A - \frac{e\phi}{\kB} \cdot \frac{1}{T}
\]
縦軸に $\ln(J/T^2)$、横軸に $1/T$ をとってプロット(リチャードソンプロット)する。
\begin{itemize}
    \item \textbf{傾き}: $-\frac{e\phi}{k}$ $\to$ 仕事関数 $\phi$ が求まる。
    \item \textbf{切片}: $\ln A$ $\to$ リチャードソン定数 $A$ が求まる。
\end{itemize}

\subsection*{⑩ 熱陰極の具備条件}
優れたカソード材料が満たすべき条件。
\begin{enumerate}
    \item **仕事関数 $\phi$ が小さいこと**(低温で放出可能)。
    \item **融点が高く、蒸気圧が低いこと**(高温動作に耐え、寿命が長い)。
    \item **放出電流密度が大きいこと**。
    \item **機械的強度が強く、化学的に安定であること**。
\end{enumerate}

\section{光電子放出 (⑪〜⑬)}

\subsection*{⑪ 光電子放出条件}
入射光子のエネルギーが仕事関数を上回ること。
\[ h\nu \ge e\phi \]

\subsection*{⑫ 限界周波数、限界波長}
電子放出が起こるギリギリの条件(運動エネルギー $K=0$)から導かれる。
\begin{itemize}
    \item **アインシュタインの式**: $h\nu = e\phi + \frac{1}{2}mv^2$
    \item **限界周波数 $\nu_0$**: $h\nu_0 = e\phi \implies \nu_0 = e\phi/h$
    \item **限界波長 $\lambda_0$**: $hc/\lambda_0 = e\phi \implies \lambda_0 = hc/e\phi$
    \item 注意点: 入射光の波長は $\lambda_0$ より**短く**なければならない(短波長ほどエネルギーが高い)。
\end{itemize}

\subsection*{⑬ 量子効率、光電感度}
\begin{description}
    \item[量子効率 $\eta_q$]: 入射光子数に対する放出電子数の割合(個数比)。
    \item[光電感度 $S$]: 入射光パワー[W]に対する光電流[A]の割合($S = I/P$)。
\end{description}

\section{二次電子放出 (⑭〜⑰)}

\subsection*{⑭ 二次電子放出の原理}
一次電子が物質に衝突し、そのエネルギーを受け取った内部電子が表面障壁を超えて放出される現象。
放出条件: 励起された電子のエネルギー $E$ が仕事関数 $e\phi$ を超えること ($E \ge e\phi$)。

\subsection*{⑮ 放出比の測定方法}
**二次電子放出比 $\delta$**:
\[ \delta = \frac{I_s \text{(二次電子流)}}{I_p \text{(一次電子流)}} \]
$\delta > 1$ であれば電流増幅作用がある。

\subsection*{⑯ 放出特性曲線}
一次電子の加速電圧 $V_p$ に対して $\delta$ は極大値を持つ。
\begin{itemize}
    \item 低電圧域: エネルギー増加に伴い生成される二次電子数が増えるため $\delta$ は上昇。
    \item 高電圧域: 一次電子が深くまで侵入しすぎるため、深部で発生した二次電子が表面まで脱出できず $\delta$ は減少。
\end{itemize}

\subsection*{⑰ 光電子増倍管 (PMT) の原理}
「光電効果」と「二次電子増倍」を組み合わせた高感度光センサ。
\begin{enumerate}
    \item **光電面**: 光子 $\to$ 光電子に変換。
    \item **ダイノード**: 二次電子放出により電子を増倍($n$段で $\delta^n$ 倍)。
    \item **陽極**: 電流として出力。
\end{enumerate}

\section{電界放出と電界計算 (⑱〜⑲)}

\subsection*{⑱ ショットキー効果}
外部電界により熱電子放出が促進される現象。
\begin{itemize}
    \item **原理**: 金属表面の「鏡像力ポテンシャル」と、外部からの「電界ポテンシャル」が合成されることで、ポテンシャル障壁の頂点が **$\Delta \phi$ だけ低下**し、かつ壁の厚さが薄くなる。
    \item **結果**: 実効的な仕事関数が減少し、放出電流が増大する。
\end{itemize}

\subsection*{⑲ 電界と電位の計算手順(ポアソン・ラプラス)}
空間電荷密度 $\rho$ がある場合の電位分布 $V$ の求め方。
\begin{enumerate}
    \item **基礎方程式**: ポアソン方程式 $\nabla^2 V = -\rho/\varepsilon_0$ を立てる。
    \item **積分**: 1次元なら $x$ で2回積分し、一般解(積分定数含む)を出す。
    \item **境界条件**: 電極の電位などの条件を代入し、定数を決定する。
    \item **電界導出**: $E = -\nabla V$ (1次元なら $E = -dV/dx$)により求める。
\end{enumerate}

\newpage
%==============================================================================
\part{2024年度 試験問題 模範解答}
%==============================================================================

\section*{問1. エネルギーバンドの名称}
図の(a)は原子単体(内殻電子)、(b)は結晶化してバンド構造を持った状態を示している。
\begin{itemize}
    \item (1) 最もエネルギーが低い、電子が詰まった内殻準位 $\rightarrow$ \textbf{[(a) 充満帯]}
    \item (2) バンド間の電子が存在できない領域 $\rightarrow$ \textbf{[(b) 禁制帯]}
    \item (3)・(4) ナトリウム(Na)はアルカリ金属であり、最外殻のs軌道が半分しか埋まっていない。したがって、価電子帯と伝導帯が重なっている(あるいは連続している)状態である。
\end{itemize}
\begin{ansbox}{答え}
(1) \textbf{(a) 充満帯} \quad (2) \textbf{(b) 禁制帯} \quad (3)(4) \textbf{(c) 価電子帯} および \textbf{(d) 伝導帯}
\end{ansbox}

\section*{問2. 金属表面のエネルギー準位(穴埋め)}
\begin{itemize}
    \item (1) 金属内でエネルギーが最も \textbf{[(d) 大きい]} 電子は(絶対零度において)
    \item (2) \textbf{[(e) フェルミ]} 準位にある。
    \item (3) \textbf{[(b) 価電子帯]} の底部Bにある電子が(※金属内の電子が存在する帯域として選択)
    \item (4) 真空中に飛び出すと \textbf{[(j) 自由電子]} になる。
    \item (5) $\phi$ に相当する \textbf{[(g) 仕事関数]} エネルギーを与えることで
    \item (6) このエネルギー $\phi$ を \textbf{[(g) 仕事関数]} と呼ぶ。
\end{itemize}

\section*{問3. [計算] タングステンの仕事関数 $\phi$}
\textbf{問題}: $T=2000 \unit{K}$, 半径 $r=1.25\times 10^{-4} \unit{m}$, 長さ $L=0.1 \unit{m}$, 電流 $I=2.00 \unit{mA}$。

\begin{calcbox}{計算プロセス}
\textbf{1. 表面積 $S$ の計算}
\[ S = 2\pi r L = 2 \times 3.14159 \times (1.25 \times 10^{-4}) \times 0.1 \approx 7.854 \times 10^{-5} \unit{m^2} \]

\textbf{2. 電流密度 $J$ の計算}
\[ J = \frac{I}{S} = \frac{2.00 \times 10^{-3}}{7.854 \times 10^{-5}} \approx 25.46 \unit{A/m^2} \]

\textbf{3. 仕事関数 $\phi$ の導出}
リチャードソン・ダッシュマンの式 $J = AT^2 \exp(-e\phi/kT)$ を変形する。
\[ \ln\left(\frac{J}{AT^2}\right) = -\frac{e\phi}{kT} \quad \Longleftrightarrow \quad \phi = \frac{kT}{e} \ln\left(\frac{AT^2}{J}\right) \]

\textbf{4. 数値代入}
係数項: $\frac{kT}{e} = \frac{1.38 \times 10^{-23} \times 2000}{1.60 \times 10^{-19}} \approx 0.1725 \unit{eV}$ \\
対数の中身: $\frac{AT^2}{J} = \frac{(1.20 \times 10^6) \times (2000)^2}{25.46} \approx 1.885 \times 10^{11}$

対数計算: $\ln(1.885 \times 10^{11}) \approx \ln(1.885) + 25.33 \approx 0.63 + 25.33 = 25.96$

最終計算: $\phi = 0.1725 \times 25.96 \approx 4.478 \unit{eV}$
\end{calcbox}
\begin{ansbox}{答え}
\textbf{4.48 [eV]}
\end{ansbox}

\section*{問4. [計算] 光電子の最大速度 $V_m$}
\textbf{問題}: $\phi=1.68 \unit{eV}$, $\lambda=520 \unit{nm}$。

\begin{calcbox}{計算プロセス}
\textbf{1. 光子エネルギー $h\nu$ (J)}
\[ h\nu = \frac{hc}{\lambda} = \frac{6.63 \times 10^{-34} \times 3.00 \times 10^8}{520 \times 10^{-9}} = 3.825 \times 10^{-19} \unit{J} \]

\textbf{2. 仕事関数 $e\phi$ (J)}
\[ e\phi = 1.68 \times (1.60 \times 10^{-19}) = 2.688 \times 10^{-19} \unit{J} \]

\textbf{3. 最大運動エネルギー $K_{max}$}
\[ K_{max} = h\nu - e\phi = (3.825 - 2.688) \times 10^{-19} = 1.137 \times 10^{-19} \unit{J} \]

\textbf{4. 速度 $V_m$}
$K_{max} = \frac{1}{2}mV_m^2$ より
\[ V_m = \sqrt{\frac{2K_{max}}{m}} = \sqrt{\frac{2 \times 1.137 \times 10^{-19}}{9.11 \times 10^{-31}}} = \sqrt{0.2496 \times 10^{12}} \]
\[ V_m \approx 0.50 \times 10^6 \unit{m/s} \]
\end{calcbox}
\begin{ansbox}{答え}
\textbf{5.00 $\times 10^5$ [m/s]}
\end{ansbox}

\section*{問5. 二次電子放出}
\textbf{(1) 放出条件}:
一次電子のエネルギーが物質内の電子を励起し、その電子が表面の\textbf{電位障壁(仕事関数や電子親和力)を超えて真空中に脱出できるだけのエネルギーを持つこと}。
\[ \frac{1}{2}mv^2 \ge e\phi \]

\textbf{(2) 材料選択}:
\begin{itemize}
    \item \textbf{選択}: \textbf{酸化マグネシウム (MgO)}
    \item \textbf{理由}: 表の中で二次電子放出比の最大値 $\delta_{max}$ が \textbf{4.0} と最も大きく($\delta > 1$)、一次電子1個あたりに放出される二次電子の数が最も多いため。
\end{itemize}

\section*{問6. 光電子増倍管 (PMT)}
\textbf{(1) 測定対象と応用}:
\begin{itemize}
    \item \textbf{対象}: 極微弱な光(フォトン単位の光)。
    \item \textbf{応用例}: スーパーカミオカンデ(ニュートリノ観測)、シンチレーションカウンタなど。
\end{itemize}

\textbf{(2) 出力電流 $I$ の計算}:
\begin{calcbox}{計算プロセス}
\textbf{1. 光電面電流 $I_k$}:
\[ I_k = P \times \eta = (1.98 \times 10^{-5}) \times (27.0 \times 10^{-3}) = 5.346 \times 10^{-7} \unit{A} \]
\textbf{2. 増倍率 $G$}: ダイノード6段なので $\delta^6$
\[ G = 3.51^6 \approx 1869 \]
\textbf{3. 出力電流 $I$}:
\[ I = I_k \times G = 5.346 \times 10^{-7} \times 1869 \approx 9.99 \times 10^{-4} \unit{A} \]
\end{calcbox}
\begin{ansbox}{答え}
\textbf{1.00 $\times 10^{-3}$ [A] (1.00 [mA])}
\end{ansbox}

\section*{問7. ショットキー効果 (記述)}
金属表面に強い外部電界を加えると、外部電界のポテンシャルと鏡像力のポテンシャルが合成(足し合わせ)され、\textbf{電位障壁の頂点が下がり($\Delta \phi$)、かつ壁の厚さが薄くなる}現象。
これにより、実効的な仕事関数が $\phi' = \phi - \Delta \phi$ に低下するため、熱電子放出電流が増加する。

\section*{問8. 電界の求め方 (記述)}
\textbf{(1) 方程式}: ポアソンの方程式 $\nabla^2 V = -\frac{\rho}{\varepsilon_0}$ (または $\frac{d^2V}{dx^2} = -\frac{\rho}{\varepsilon_0}$)

\textbf{(2) 手順}:
方程式に $\rho$ の分布関数を代入 → 電位変化の条件を入れて整理 \\
→ \textbf{① $x$ で1回積分する(電界の式が出る)} \\
→ \textbf{② さらに $x$ で積分する(電位 $V$ の式が出る)} \\
→ 境界条件を代入して定数を求める \\
→ \textbf{③ 求めた電位の式 $V(x)$ を $x$ で微分し、符号を反転させる ($E_x = -\frac{dV}{dx}$)} \\
→ $x$ 方向の電界の強さ $E_x$ を求める。

\newpage
%==============================================================================
\part{2023年度 試験問題 模範解答}
%==============================================================================

\section*{問1. 電子放出の基礎(鏡像力)}
\textbf{(1) 力の大きさ $|F|$}:
金属表面を鏡とし、距離 $x$ の反対側に正電荷 $+e$ があるとみなす(クーロンの法則)。電荷間距離は $2x$ となる。
\[ |F| = \frac{1}{4\pi\varepsilon_0} \frac{e \times e}{(2x)^2} = \frac{e^2}{16\pi\varepsilon_0 x^2} \unit{[N]} \]

\textbf{(2) 力の方向}:
図中の電子 $-e$ から、\textbf{金属表面に向かう左向きの矢印}を描く(引力)。

\textbf{(3) 電位障壁 $W$}: 無限遠を基準として積分する。
\[ W = \int_x^{\infty} |F| dx = \frac{e^2}{16\pi\varepsilon_0} \left[ -\frac{1}{x} \right]_x^{\infty} = \frac{e^2}{16\pi\varepsilon_0 x} \unit{[J]} \]
eV単位にするため $e$ で割る:
\[ W = \frac{e}{16\pi\varepsilon_0 x} \unit{[eV]} \]

\textbf{(4) グラフ}:
縦軸 $W$、横軸 $x$ のグラフを描く。
$x$ が小さいほど $W$ は大きく(無限大へ発散)、$x$ が大きくなると $W$ はゼロに近づく\textbf{反比例の曲線(双曲線)}を描く。

\section*{問2. エネルギー準位図の描画}
ポテンシャル井戸の図に対して以下を記入する。
\begin{itemize}
    \item \textbf{① 価電子帯}: 金属内部(左側の深い部分)の底からフェルミ準位までの、電子が詰まっている領域全体。
    \item \textbf{② フェルミ準位}: 電子が詰まっている最上面のライン(水面)。点線を引き「$E_F$」と書く。
    \item \textbf{③ フェルミエネルギー $E_F$}: 井戸の底からフェルミ準位までの高さを示す矢印。
    \item \textbf{④ 仕事関数 $\phi$}: フェルミ準位から、右側の真空準位(障壁の平らな頂上)までの高さを示す矢印。
\end{itemize}

\section*{問3. [計算] モリブデン線の半径 $r$}
\textbf{問題}: $L=0.1\unit{m}, T=2000\unit{K}, I=22.8\unit{mA}, \phi=4.27\unit{eV}$。

\begin{calcbox}{計算プロセス}
\textbf{1. 電流密度 $J$ の算出}
指数部: $\frac{e\phi}{kT} = \frac{1.60 \times 10^{-19} \times 4.27}{1.38 \times 10^{-23} \times 2000} \approx 24.75$
\[ J = AT^2 \exp(-24.75) = (1.20 \times 10^6) \times (2000)^2 \times (1.78 \times 10^{-11}) \]
\[ J = (4.8 \times 10^{12}) \times (1.78 \times 10^{-11}) \approx 85.44 \unit{A/m^2} \]

\textbf{2. 半径 $r$ の逆算}
全電流 $I = 22.8 \unit{mA} = 0.0228 \unit{A}$
必要な表面積 $S = \frac{I}{J} = \frac{0.0228}{85.44} \approx 2.668 \times 10^{-4} \unit{m^2}$
$S = 2\pi r L$ より
\[ r = \frac{S}{2\pi L} = \frac{2.668 \times 10^{-4}}{2 \times 3.14159 \times 0.1} \approx 4.25 \times 10^{-4} \unit{m} \]
\end{calcbox}
\begin{ansbox}{答え}
\textbf{4.25 $\times 10^{-4}$ [m] (0.425 [mm])}
\end{ansbox}

\section*{問4. [計算] 限界波長 $\lambda_0$}
\textbf{問題}: $\phi = 1.72 \unit{eV}$。

\begin{calcbox}{計算プロセス}
$h\nu_0 = \frac{hc}{\lambda_0} = e\phi$ より
\[ \lambda_0 = \frac{hc}{e\phi} \]
\[ \lambda_0 = \frac{(6.63 \times 10^{-34}) \times (3.00 \times 10^8)}{(1.60 \times 10^{-19}) \times 1.72} \]
\[ \lambda_0 = \frac{19.89 \times 10^{-26}}{2.752 \times 10^{-19}} \approx 7.227 \times 10^{-7} \unit{m} \]
\end{calcbox}
\begin{ansbox}{答え}
\textbf{7.23 $\times 10^{-7}$ [m] (723 [nm])}
\end{ansbox}

\section*{問5. 光電子増倍管とスーパーカミオカンデ}
\textbf{(1) 原理図}:
(図示の手順) 左から「光」が入射 $\to$ 「光電面」から「電子」が放出 $\to$ 「ダイノード」で「二次電子」が増殖(ねずみ算式に矢印を描く) $\to$ 「陽極」で回収。

\textbf{(2) 出力電流 $I$}:
\[ I = (P \eta) \times \delta^n = (6.43 \times 10^{-5} \times 15.0 \times 10^{-3}) \times 3.4^5 \]
\[ I = 9.645 \times 10^{-7} \times 454 \approx 4.38 \times 10^{-4} \unit{A} \]
\begin{ansbox}{答え}
\textbf{4.38 $\times 10^{-4}$ [A]}
\end{ansbox}

\textbf{(3) スーパーカミオカンデの概要}:
岐阜県の地下深くにある、純水を満たした巨大なタンクの内壁に多数の\textbf{光電子増倍管}を配置した装置。ニュートリノが水中の原子核と反応した際に発生する微弱な\textbf{チェレンコフ光}(荷電粒子が水中を光速以上で走る際に発する青白い光)を検出し、ニュートリノの観測を行う。

\section*{問6. [計算] 電位分布からの電界導出}
\textbf{問題}: $V = \frac{1}{\sqrt{x^2 + y^2 + z^2}}$ のとき、$E_x$ を求めよ。

\begin{calcbox}{計算プロセス}
電界は電位の勾配(マイナス)である: $E_x = -\frac{\partial V}{\partial x}$ \\
$u = x^2 + y^2 + z^2$ とおくと、$V = u^{-1/2}$ である。合成関数の微分を行う。
\[ \frac{\partial V}{\partial x} = \frac{d V}{d u} \cdot \frac{\partial u}{\partial x} = -\frac{1}{2}u^{-3/2} \cdot (2x) = -x(x^2 + y^2 + z^2)^{-3/2} \]
したがって、電界 $E_x$ は符号を反転させて、
\[ E_x = - \left( -x(x^2 + y^2 + z^2)^{-3/2} \right) = \frac{x}{(x^2 + y^2 + z^2)^{3/2}} \]
\end{calcbox}
\begin{ansbox}{答え}
\textbf{$\displaystyle E_x = \frac{x}{(x^2 + y^2 + z^2)^{3/2}}$ [V/m]}
\end{ansbox}

\newpage
%==============================================================================
\part{重要演習課題 (レポート解説)}
%==============================================================================
レポート課題の計算プロセスは、試験における計算問題の完全な練習となる。

\section*{課題1: フィラメントの設計 (Lの算出)}
\textbf{条件:} $T = 2500 \unit{K}$, $r = 1.50 \times 10^{-4} \unit{m}$, $I = 2.00 \times 10^{-3} \unit{A}$, $\phi = 4.52 \unit{eV}$。

\begin{calcbox}{詳細計算プロセス}
\textbf{1. リチャードソン定数 $A$ の理論値確認}
\[ A = \frac{4\pi m e k^2}{h^3} \approx 1.20 \times 10^6 \unit{A \cdot m^{-2} \cdot K^{-2}} \]

\textbf{2. 電流密度 $J$ の算出}
\[ J = A T^2 \exp\left( -\frac{e\phi}{kT} \right) \]
指数部: $\frac{e\phi}{kT} = \frac{1.60 \times 10^{-19} \times 4.52}{1.38 \times 10^{-23} \times 2500} \approx 20.962$
\[ J = (1.20 \times 10^6) \times (2500)^2 \times \exp(-20.962) \]
\[ J = 7.50 \times 10^{12} \times 7.876 \times 10^{-10} \approx 5907 \unit{A/m^2} \]

\textbf{3. フィラメント長 $L$ の算出}
$I = J \cdot S = J \cdot (2\pi r L)$ より
\[ L = \frac{I}{2\pi r J} = \frac{2.00 \times 10^{-3}}{2\pi \times (1.50 \times 10^{-4}) \times 5907} \]
\[ L = \frac{0.002}{5.567} \approx 3.59 \times 10^{-4} \unit{m} \]
\end{calcbox}
\begin{ansbox}{解答}
$L = 3.59 \times 10^{-4} \unit{m}$
\end{ansbox}

\section*{課題2: 光電子の最大速度}
\textbf{条件:} $\phi = 4.27 \unit{eV}$, $\lambda = 45.5 \unit{nm}$。

\begin{calcbox}{詳細計算プロセス}
光電効果の式: $\frac{hc}{\lambda} = e\phi + \frac{1}{2}mv_m^2$ より
\[ v_m = \sqrt{ \frac{2}{m} \left( \frac{hc}{\lambda} - e\phi \right) } \]
\textbf{1. 光子エネルギー}: $\frac{hc}{\lambda} = \frac{19.89 \times 10^{-26}}{4.55 \times 10^{-8}} \approx 4.37 \times 10^{-18} \unit{J}$ \\
\textbf{2. 仕事関数}: $e\phi = 1.60 \times 10^{-19} \times 4.27 \approx 6.83 \times 10^{-19} \unit{J}$ \\
\textbf{3. 運動エネルギー差}: $K = 43.7 \times 10^{-19} - 6.83 \times 10^{-19} = 36.87 \times 10^{-19} \unit{J}$ \\
\textbf{4. 速度}: $v_m = \sqrt{\frac{2 \times 36.87 \times 10^{-19}}{9.11 \times 10^{-31}}} \approx \sqrt{8.09 \times 10^{12}} \approx 2.84 \times 10^6$
\end{calcbox}
\begin{ansbox}{解答}
$v_m = 2.85 \times 10^6 \unit{m/s}$
\end{ansbox}

\section*{課題3: PMTの一次光電流}
\textbf{条件:} $\delta = 4.0$, $n = 10$, $I_o = 0.125 \unit{mA}$。

\begin{calcbox}{詳細計算プロセス}
増幅式 $I_o = I_p \times \delta^n$ より、$I_p = I_o / \delta^n$。
\[ I_p = \frac{0.125 \times 10^{-3}}{4.0^{10}} \]
$4^{10} = (2^2)^{10} = 2^{20} \approx 1.05 \times 10^6$
\[ I_p = \frac{1.25 \times 10^{-4}}{1.05 \times 10^6} \approx 1.19 \times 10^{-10} \unit{A} \]
\end{calcbox}
\begin{ansbox}{解答}
$I_p = 1.19 \times 10^{-10} \unit{A}$
\end{ansbox}

\section*{課題4: 3次元電界ベクトル}
\textbf{条件:} $V = (x^2 + y^2 + z^2)^{-1/2}$。

\begin{calcbox}{詳細計算プロセス}
$\bm{E} = -\nabla V = -(\frac{\partial V}{\partial x}\bm{i} + \frac{\partial V}{\partial y}\bm{j} + \frac{\partial V}{\partial z}\bm{k})$。
$x$成分の偏微分(過去問の問6と同様):
\[ \frac{\partial V}{\partial x} = -x(x^2 + y^2 + z^2)^{-3/2} \]
よって $E_x = x(x^2 + y^2 + z^2)^{-3/2}$。$y, z$も同様に対称性を持つ。
\end{calcbox}
\begin{ansbox}{解答}
\[ \bm{E} = \frac{1}{(x^2 + y^2 + z^2)^{3/2}} (x\bm{i} + y\bm{j} + z\bm{k}) \unit{V/m} \]
※これは点電荷が作る電界の式と一致する。
\end{ansbox}

\end{document}