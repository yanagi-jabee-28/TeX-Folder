\documentclass[a4paper,11pt]{ltjsreport}
\usepackage{amsmath,amssymb}
\usepackage{luatexja-fontspec}
\usepackage{lmodern}
% 自動幅調整用: 表が本文幅を超える問題を防ぐ
\usepackage{tabularx}
\usepackage{array} % \newcolumntype のため
% X 列を左寄せにした簡易列型(表記を短くするため)
\newcolumntype{Y}{>{\raggedright\arraybackslash}X}% enumitem を読み込む(description の表示は本文側で明示的に書く)
\usepackage{enumitem}
\usepackage[margin=2.5cm]{geometry}

\title{電子工学 — 後期期末達成度試験:完全対策資料}
\author{}
\date{}

\begin{document}

\maketitle

\begin{abstract}
提示された手書き講義資料(PDF)および試験範囲のテキスト情報に基づく、\textbf{試験直前の完全対策ノート}。
物理的意味と導出プロセスに重点を置き、計算・記述ともに試験で得点しやすい形に整理してある。
\end{abstract}

\section{基本戦略}
\begin{enumerate}
  \item \textbf{計算問題}: 公式を暗記するだけでなく、必ず運動方程式(例: $F=ma$)から導出できるようにする。
  \item \textbf{記述問題}: 因果関係(「なぜそうなるか」)を短く明確に説明できるようにする。
  \item \textbf{式の導出}: タウンゼント理論などは等比級数の和へ落とし込むことが鍵。過程を書けることを重視。
\end{enumerate}

\section{1. 静磁界中の電子の運動(計算・導出)}
\subsection{出題ポイント}
運動方程式から加速度・速度・位置・軌跡(円運動・螺旋運動)を導く。

\subsection{基礎知識}
\begin{itemize}
  \item 磁束密度: $\vec{B}$ [T]
  \item 電子の電荷: $-e$ [C]
  \item 電子の速度: $\vec{v}$ [m/s]
  \item ローレンツ力: $\vec{F} = -e(\vec{v}\times\vec{B})$(大きさ $F = evB\sin\theta$)
\end{itemize}

\subsection{垂直入射時(円運動)}
ローレンツ力を向心力とみなして
\begin{equation}
  m\frac{v^2}{r} = evB
\end{equation}
より軌道半径と角周波数は
\begin{equation}
  r = \frac{mv}{eB},\qquad T = \frac{2\pi m}{eB},\qquad f=\frac{1}{2\pi}\frac{eB}{m}.
\end{equation}
特に周期・周波数は速度に依存しない(サイクロトロンの原理)。

\subsection{斜め入射時(螺旋運動)}
速度の分解: $v_{\perp}=v\sin\theta,\; v_{\parallel}=v\cos\theta$.\
$ v_{\perp}$ 成分は円運動(半径 $r=\dfrac{m(v\sin\theta)}{eB}$)、$v_{\parallel}$ は等速直線運動。

\section{2. 気体分子の電離現象(説明)}
\subsection{出題ポイント}
3 種類の電離を区別して説明できること。

% 自動幅(X 列のみ)に任せつつ、表を少しだけ縮めて余白は自動に任せる
% - 使用幅は \linewidth にし、フォントサイズを小さくして overfull を防ぐ
\noindent\small
\begin{tabularx}{\linewidth}{YYY}
\hline
\textbf{電離の種類} & \textbf{要点} & \textbf{備考}\\
\hline
電界電離 & 衝突で増幅される & タウンゼント作用 \\
熱電離 & 熱運動で電離する & アーク放電 \\
光電離 & 光吸収で電離する & 初期電子供給 \\
\hline
\end{tabularx}
\normalsize\par

% 詳細は見出しを独立行にして記述(labelwidth に依存しない)
\noindent\textbf{電界電離(衝突電離)}\mbox{}\\
電界により加速された自由電子が気体分子と衝突し、分子を電離して二次電子を生成する現象。タウンゼント放電の基礎となる。

\noindent\textbf{熱電離(衝突電離)}\mbox{}\\
高温下で分子の熱運動が激しくなり、衝突エネルギーで電離が起きる。アーク放電において顕著である。

\noindent\textbf{光電離}\mbox{}\\
外来光(紫外線など)を吸収して電離が起きる現象。条件は $h\nu\ge W$ で、放電の種火(初期電子)を供給する。

\section{3. タウンゼントの理論と火花放電の条件(導出)}
\subsection{定義と記号}
$\alpha$ : 単位長あたりに作られる電子・正イオン対数(一次増幅)\\
$\gamma$ : 陰極に衝突した陽イオン1つ当たりの二次電子放出数\\
$\ell$ : 電極間距離,\; $n_0$ : 陰極から出た初期電子数

\subsection{導出(要点)}
陰極から出た $n_0$ 個が陽極に到達するまでに一次増幅を受けると
\begin{equation}
  n_1 = n_0 e^{\alpha\ell}.
\end{equation}
発生した陽イオン数は $n_0(e^{\alpha\ell}-1)$。これが陰極へ戻り二次電子を出す数は
\begin{equation}
  \gamma n_0(e^{\alpha\ell}-1).
\end{equation}
この過程が繰り返されると全通過電子数 $N$ は無限等比級数となり、
\begin{equation}
  N = \frac{n_0 e^{\alpha\ell}}{1-\gamma(e^{\alpha\ell}-1)}.
\end{equation}
分母がゼロになると発散して火花放電が起きるため、タウンゼントの発火条件は
\begin{equation}
  \boxed{\;\gamma(e^{\alpha\ell}-1)=1\;}
\end{equation}

\section{4. パッシェンの法則(説明)}
放電開始電圧 $V_c$ は $p\ell$ の関数であり、ある $p\ell$ で最小値(パッシェン・ミニマム)を持つ。

\subsection{物理的説明(要点)}
\begin{itemize}
  \item $p\ell$ が大きい側: 分子密度が高く電子がすぐ散逸するため、電離に必要なエネルギーを与えにくい → 高い電圧が必要。
  \item $p\ell$ が小さい側: 衝突相手が少なく効率的な電離が起きにくい → 高い電圧が必要。
\end{itemize}

\section{5. グロー放電を利用した機器}
\subsection{動作原理(要点)}
グロー放電は低〜中圧の気体中で、電子・イオンの局所的な衝突電離と再結合により安定な放電輝線(発光)を維持する現象である。電子温度は比較的低く、放電は管壁近傍に集中することが多い。

\subsection{代表的な機器と用途}
\begin{itemize}
  \item \textbf{ネオン管(ネオンサイン)}: 希ガスの輝線を直接利用する低圧封入管。色は封入ガスと蛍光体で決まる。
  \item \textbf{蛍光灯}: グロー放電で水銀蒸気を励起し発生する紫外線を蛍光体で可視光に変換する。
  \item \textbf{プラズマディスプレイ(PDP)}: 微小セル内のグロー放電で紫外線を発生させ、各セルの蛍光体を発光させる(マイクロセル単位でスイッチング)。
  \item \textbf{スパッタリング(薄膜堆積)}: 低圧アルゴン雰囲気でグロー放電を発生させ、アルゴンイオンをターゲットに衝突させて材料原子をはじき出す(物理蒸着)。
\end{itemize}

\subsection{代表的仕様(概数・試験で押さえる点)}
\begin{itemize}
  \item 圧力: 低〜中圧(数 Pa〜数百 Pa の範囲が典型) — 機器により大きく異なる(蛍光灯は非常に低圧、ネオン管はやや高め)。
  \item 電圧/電流: 起電圧は数百 V 程度(ネオン管・蛍光灯の点火は高圧が必要)、運転時は比較的低電流で定電圧領域を持つ。\
  \item ガス組成: ネオン・アルゴン・水銀蒸気などが代表。
\end{itemize}

\subsection{利点・欠点・安全上の注意}
\begin{itemize}
  \item 利点: 低消費電力で安定発光が得られる、狭い領域での均一放電が可能(PDP 等)。
  \item 欠点: 起動(点火)に高電圧が必要、封入ガスや蛍光体・水銀などの環境配慮が必要。\
  \item 安全: 高電圧・紫外線・オゾン生成の危険、破損時のガラス飛散や有害ガスに注意。
\end{itemize}

\subsection{試験で問われやすいポイント}
\begin{itemize}
  \item グロー放電とアーク放電の違い(圧力・電流密度・温度の違い)。
  \item 蛍光灯の発光機構(グロー放電→紫外線→蛍光体→可視光)。
  \item スパッタリングでの荷電粒子の役割(イオン加速→ターゲット衝突による原子放出)。
\end{itemize}

\section{6. アーク放電を利用した機器}
\subsection{動作原理(要点)}
アーク放電は高電流密度・高温度で電流が流れる状態で、電子衝突と熱電子放出が支配的となる。プラズマ温度は非常に高く(数千〜万 K)、電極近傍での熱的・電気的作用が顕著である。

\subsection{代表的な機器と用途}
\begin{itemize}
  \item \textbf{アーク溶接}: 電極と母材間のアークにより金属を局所的に融解して接合する。電流は数十〜数百 A、電圧は数十 V 程度。
  \item \textbf{アーク炉}: 大電流アークで高温を作り、鉄スクラップ等を溶解する(製鉄用途)。
  \item \textbf{高輝度放電ランプ(HID)}: 水銀や金属ハライドを用いたアーク放電で高効率・高輝度の光源を実現(スタジアム灯など)。
  \item \textbf{遮断器のアーク消弧}: 開閉時に発生するアークを短時間で消弧する技術(ガス置換、磁場で引き延ばして細断、アークチャンバで冷却等)。
\end{itemize}

\subsection{代表的仕様(概数・試験で押さえる点)}
\begin{itemize}
  \item 圧力: 通常は大気圧下で発生(真空や低圧のアークもあるが区別すること)。
  \item 電流/電圧: アークは高電流・低電圧領域(溶接: 数十〜数百 A、電圧は数十 V)。
  \item 温度: プラズマ温度は高く、材料の蒸発・イオン化が起き得る。
\end{itemize}

\subsection{利点・欠点・安全上の注意}
\begin{itemize}
  \item 利点: 極めて高温を短時間で得られるため溶接・溶解等の高エネルギー処理に適する。
  \item 欠点: 電極消耗・ノイズ(電磁的)・高温による周辺損傷のリスクがある。
  \item 安全: 強い紫外線・高温・スパッタ(飛散粒子)、高電流取り扱いのための適切な遮断・接地が必須。
\end{itemize}

\subsection{試験で問われやすいポイント}
\begin{itemize}
  \item アーク放電とグロー放電の識別(電流密度・温度・発光の性質)。
  \item アーク溶接における電流・電圧の役割(加熱量はおおむね電力 $P=VI$ に依存)。
  \item 遮断器でのアーク消弧方式(磁気的引き延ばし、ガス置換、真空中での消弧など)とその利点・短所。
\end{itemize}

% 参考: 添付PDFや講義ノートの図・数値を参照して要点をまとめた。

\section{試験直前チェックリスト}
\begin{itemize}
  \item 計算: $r=mv/(eB)$ の式変形がスムーズか(速度・周期を求められるか)。
  \item 導出: $N$ を等比級数の和の形に整理できるか。
  \item 記述: パッシェンの法則で「衝突しすぎるからダメ(右側)」と「衝突相手がいないからダメ(左側)」を明快に書き分けられるか。
  \item 用語: $\alpha$(電子なだれ), $\gamma$(二次電子放出), $\alpha\ell$, $V_c$。
\end{itemize}

\section*{参考}
講義ノート(手書きPDF)に準拠。健闘を祈ります。

\end{document}
