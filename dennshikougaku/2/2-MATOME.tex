\documentclass[a4paper,11pt]{ltjsreport}
\usepackage{amsmath,amssymb}
\usepackage{luatexja-fontspec}
\usepackage{lmodern}
% 自動幅調整用: 表が本文幅を超える問題を防ぐ
\usepackage{tabularx}
\usepackage{array} % \newcolumntype のため
% X 列を左寄せにした簡易列型(表記を短くするため)
\newcolumntype{Y}{>{\raggedright\arraybackslash}X}% enumitem を読み込む(description の表示は本文側で明示的に書く)
\usepackage{enumitem}
\usepackage[margin=2.5cm]{geometry}

\title{電子工学 — 後期期末達成度試験:完全対策資料}
\author{}
\date{}

\begin{document}

\maketitle

\begin{abstract}
提示された手書き講義資料(PDF)および試験範囲のテキスト情報に基づく、\textbf{試験直前の完全対策ノート}。
物理的意味と導出プロセスに重点を置き、計算・記述ともに試験で得点しやすい形に整理してある。
\end{abstract}

\chapter{基本戦略}
\begin{enumerate}
  \item \textbf{計算問題}: 公式を暗記するだけでなく、必ず運動方程式(例: $F=ma$)から導出できるようにする。
  \item \textbf{記述問題}: 因果関係(「なぜそうなるか」)を短く明確に説明できるようにする。
  \item \textbf{式の導出}: タウンゼント理論などは等比級数の和へ落とし込むことが鍵。過程を書けることを重視。
\end{enumerate}

\chapter{真空中の電子の運動(電位分布・静電界)}
\section{電位分布と電界の基本}
\begin{itemize}
  \item 電界と電位: $E = -\dfrac{dV}{dx}$(1次元)
  \item 電束密度: $D=\epsilon_0 E$(真空中)
  \item 空間電荷密度: $\rho$ [C/m$^3$]
\end{itemize}

\section{ポアッソン・ラプラス方程式}
\begin{itemize}
  \item ポアッソン方程式: $\nabla^2 V = -\dfrac{\rho}{\epsilon_0}$
  \item ラプラス方程式($\rho=0$): $\nabla^2 V = 0$
\end{itemize}

\section{平行平板電極の電位分布}
\subsection{空間電荷なし}
$y,z$ 方向の変化を無視すると
\begin{equation}
  \frac{\partial^2 V}{\partial x^2} = 0
\end{equation}
境界条件 $V(0)=0$, $V(D)=V_a$ より
\begin{equation}
  V = \frac{V_a}{D}x, \qquad E_x = -\frac{V_a}{D}.
\end{equation}

\subsection{空間電荷あり($\rho\neq 0$ の場合)}
ポアッソン方程式
\begin{equation}
  \frac{d^2 V}{dx^2} = -\frac{\rho}{\epsilon_0}
\end{equation}
を解く必要がある。講義資料の例として $\rho=-kx^{-1/2}$ を仮定すると、
\begin{equation}
  \frac{d^2 V}{dx^2} = \frac{k}{\epsilon_0}x^{-1/2}
\end{equation}
となり、積分して
\begin{equation}
  \frac{dV}{dx} = \frac{2k}{\epsilon_0}x^{1/2} + C_1,
  \qquad V = \frac{4k}{3\epsilon_0}x^{3/2} + C_1 x + C_2.
\end{equation}
境界条件で整理した結果は資料の式に従うが、重要なのは\textbf{電位分布が直線ではなく下に凸になる}点である(空間電荷が電位を押し下げる)。

\section{静電界中の電子の運動}
\subsection{運動方程式}
\begin{equation}
  m\frac{d\vec{v}}{dt} = -e\vec{E}
\end{equation}
平行平板($E_x=-V_a/D$)で初速度0なら
\begin{equation}
  \frac{d^2 x}{dt^2} = \frac{e}{m}\frac{V_a}{D},\quad
  v_x = \frac{e}{m}\frac{V_a}{D}t,\quad
  x = \frac{1}{2}\frac{e}{m}\frac{V_a}{D}t^2.
\end{equation}
到達時間(走行時間)は
\begin{equation}
  \tau = \sqrt{\frac{2m}{eV_a}}\,D.
\end{equation}

\subsection{電子ボルトと速度}
エネルギー保存より
\begin{equation}
  \frac{1}{2}mv^2 = eV,\qquad v=\sqrt{\frac{2eV}{m}}.
\end{equation}

\section{静電偏向(CRT の基本式)}
加速電圧 $V_0$、偏向板長さ $l$、間隔 $d$、偏向電圧 $V$、スクリーンまでの距離 $L$ のとき、
\begin{equation}
  y_1 = \frac{eVlL}{2dV_0}.
\end{equation}
偏向感度は $V_0$ に反比例する。

\chapter{静磁界中の電子の運動(計算・導出)}
\section{出題ポイント}
運動方程式から加速度・速度・位置・軌跡(円運動・螺旋運動)を導く。

\section{基礎知識}
\begin{itemize}
  \item 磁束密度: $\vec{B}$ [T]
  \item 電子の電荷: $-e$ [C]
  \item 電子の速度: $\vec{v}$ [m/s]
  \item ローレンツ力: $\vec{F} = -e(\vec{v}\times\vec{B})$(大きさ $F = evB\sin\theta$)
\end{itemize}

\subsection{垂直入射時(円運動)}
ローレンツ力を向心力とみなして
\begin{equation}
  m\frac{v^2}{r} = evB
\end{equation}
より軌道半径と角周波数は
\begin{equation}
  r = \frac{mv}{eB},\qquad T = \frac{2\pi m}{eB},\qquad f=\frac{1}{2\pi}\frac{eB}{m}.
\end{equation}
特に周期・周波数は速度に依存しない(サイクロトロンの原理)。

\subsection{斜め入射時(螺旋運動)}
速度の分解: $v_{\perp}=v\sin\theta,\; v_{\parallel}=v\cos\theta$.\
$ v_{\perp}$ 成分は円運動(半径 $r=\dfrac{m(v\sin\theta)}{eB}$)、$v_{\parallel}$ は等速直線運動。

\section{一様磁界中の周期と角周波数}
\begin{equation}
  T = \frac{2\pi m}{eB},\qquad \omega_c = \frac{eB}{m}.
\end{equation}

\section{静電磁界中($\vec{E}\perp\vec{B}$)の運動}
直交電磁界では運動方程式が連立となり、軌跡はサイクロイドになる。代表解は
\begin{equation}
  x = \frac{A}{\omega_c^2}(1-\cos\omega_c t),\quad
  y = \frac{A}{\omega_c^2}(\omega_c t-\sin\omega_c t),
\end{equation}
ただし $A=\dfrac{eV_a}{mD}$、$\omega_c=\dfrac{eB}{m}$。

\chapter{気体中の放電と電離(説明)}
\section{出題ポイント}
3 種類の電離を区別して説明できること.

\section{電離・励起・発光の基本}
\begin{itemize}
  \item 電離: 外部から $eV_i$ 以上のエネルギーを与えると電子が飛び出し、原子は $+$ イオンとなる。
  \item 基底状態: 軌道電子のエネルギーが最低の状態。
  \item 電離電圧: 電離エネルギー $eV_i$ に対応する電圧 $V_i$。
  \item 励起: $eV_i$ 未満のエネルギーで軌道が高い準位へ遷移。
  \item 発光: 励起状態から基底状態へ戻るとき、$\Delta E = h\nu$ を放射。
\end{itemize}

\section{準安定電圧と電離電圧(代表値)}
\noindent\small
\begin{tabularx}{\linewidth}{YYYY}
\hline
\textbf{気体} & \textbf{元素記号} & \textbf{準安定電圧 [V]} & \textbf{電離電圧 [V]}\\
\hline
ヘリウム & He & 19.81 & 24.580 \\
ネオン & Ne & 16.62 & 21.559 \\
アルゴン & Ar & 11.53 & 15.755 \\
キセノン & Xe & 8.28 & 12.127 \\
水銀 & Hg & 4.67 & 10.434 \\
\hline
\end{tabularx}
\normalsize\par

% 自動幅(X 列のみ)に任せつつ、表を少しだけ縮めて余白は自動に任せる
% - 使用幅は \linewidth にし、フォントサイズを小さくして overfull を防ぐ
\noindent\small
\begin{tabularx}{\linewidth}{YYY}
\hline
\textbf{電離の種類} & \textbf{要点} & \textbf{備考}\\
\hline
電界電離 & 衝突で増幅される & タウンゼント作用 \\
熱電離 & 熱運動で電離する & アーク放電 \\
光電離 & 光吸収で電離する & 初期電子供給 \\
\hline
\end{tabularx}
\normalsize\par

% 詳細は見出しを独立行にして記述(labelwidth に依存しない)
\noindent\textbf{電界電離(衝突電離)}\mbox{}\\
電界により加速された自由電子が気体分子と衝突し、分子を電離して二次電子を生成する現象。タウンゼント放電の基礎となる。

\noindent\textbf{熱電離(衝突電離)}\mbox{}\\
高温下で分子の熱運動が激しくなり、衝突エネルギーで電離が起きる。アーク放電において顕著である。

\noindent\textbf{光電離}\mbox{}\\
外来光(紫外線など)を吸収して電離が起きる現象。条件は $h\nu\ge W$ で、放電の種火(初期電子)を供給する。

\section{電離の要因と消滅}
\begin{itemize}
  \item 要因: 自然放射線・宇宙線・核種壊変由来の放射線など。
  \item 電離プロセス: 気体分子 $\to$ 電子 $+$ 陽イオン。
  \item 消滅: 拡散・再結合により電子と陽イオンが消滅。
\end{itemize}

\section{放電の電圧-電流特性(低圧気体)}
電圧上昇で暗電流 $\to$ タウンゼント放電 $\to$ グロー放電 $\to$ アーク放電の順に遷移する。
\begin{itemize}
  \item 暗電流: $10^{-15}$--$10^{-10}$ A 程度。外因(光・放射線)が遮断されると停止。
  \item タウンゼント放電: $10^{-10}$--$10^{-5}$ A 程度。電圧 $V_s$ で急増。
  \item グロー放電: $10^{-4}$--$10^{-1}$ A 程度。電圧はほぼ一定。
  \item アーク放電: $10^0$--$10^2$ A 以上。電圧が数十 V まで低下。
\end{itemize}

\section{重要グラフ(描画対策)}
以下のグラフは概形を描けるようにしておくこと。
\begin{itemize}
  \item \textbf{放電の $V$--$I$ 特性}: 横軸 $V$、縦軸 $\log I$。暗電流(一定)→タウンゼント(立ち上がり)→グロー(定電圧)→アーク(低電圧・大電流)。
  \item \textbf{パッシェン曲線}: 横軸 $p\ell$、縦軸 $V_c$。下に凸の U 字型で最小値を持つ。
  \item \textbf{電位分布}: 空間電荷なし(直線)と空間電荷あり(下に凸)の比較。
\end{itemize}

\section{暗電流と電子なだれ}
光電子放出や自然放射線で生じた電子が微弱電流を形成する。電界が強くなると衝突電離が進み、電子数が
$2,4,8,\ldots$ と指数的に増加する(電子なだれ)。

\subsection*{補足:電子とイオンの運動}
\begin{itemize}
  \item 質量比: $m_p / m_e \approx 1836$。
  \item 陽イオンは電子に比べて質量が大きく加速されにくいため、初期の衝突電離($\beta$ 作用)は小さく、$\alpha$ 作用が支配的となる($\alpha \gg \beta$)。
\end{itemize}

\section{グロー放電の空間構造(名称)}
アストン暗部、陰極グロー、陰極暗部、負グロー、ファラデー暗部、陽光柱。発光色は気体に依存。

\chapter{タウンゼントの理論と火花放電の条件(導出)}
\section{定義と記号}
$\alpha$ : 単位長あたりに作られる電子・正イオン対数(一次増幅)\\
$\gamma$ : 陰極に衝突した陽イオン1つ当たりの二次電子放出数\\
$\ell$ : 電極間距離,\; $n_0$ : 陰極から出た初期電子数

\subsection{導出(要点)}
陰極から出た $n_0$ 個が陽極に到達するまでに一次増幅を受けると
\begin{equation}
  n_1 = n_0 e^{\alpha\ell}.
\end{equation}
発生した陽イオン数は $n_0(e^{\alpha\ell}-1)$。これが陰極へ戻り二次電子を出す数は
\begin{equation}
  \gamma n_0(e^{\alpha\ell}-1).
\end{equation}
この過程が無限に繰り返されると、全通過電子数 $N$ は初項 $n_1$、公比 $\gamma(e^{\alpha\ell}-1)$ の無限等比級数となる。
\begin{equation}
  N = n_1 + n_1[\gamma(e^{\alpha\ell}-1)] + n_1[\gamma(e^{\alpha\ell}-1)]^2 + \cdots
    = \frac{n_1}{1-\gamma(e^{\alpha\ell}-1)}.
\end{equation}
分母がゼロになると発散して火花放電が起きるため、タウンゼントの発火条件は
\begin{equation}
  \boxed{\;\gamma(e^{\alpha\ell}-1)=1\;}
\end{equation}

% --- タウンゼント復習(答案に直結する要点集/問題文は掲載しない) ---
\subsection*{タウンゼント復習 — 答案に使える要点とテンプレ}
以下は試験でそのまま書ける\textbf{定義・物理的意味・答案テンプレ}のみを示す(問題文は載せていません)。

\subsubsection*{即答ワンライナー(そのまま書ける)}
\begin{itemize}
  \item $\alpha$(定義): \underline{単位長さ}あたりに生成される電子対(\underline{\alpha\ 対})の平均個数。\underline{電界}で加速された\underline{電子}が\underline{気体分子・原子}と衝突して\underline{電離}する確率の指標。
  \item $\gamma$(定義): 陰極表面で放出される\underline{二次電子}の平均個数(\underline{$\gamma$\ 個})。陽イオンの陰極衝突による二次過程の強さを表す。
\end{itemize}

\subsubsection*{答案テンプレ(短く・確実に点を取る)}
\begin{enumerate}
  \item 定義(1文): 「$\alpha$ は...」「$\gamma$ は...」。
  \item 物理的意味(1文): 「$\alpha$ が大きいと $n_1=n_0e^{\alpha\ell}$ のように指数的増幅が起きる。$\gamma$ は陰極で種電子を回復し放電を持続させる。」
  \item 一語挿入(必須): 指定語句を\underline{1〜2語}入れる(例: \underline{単位長さ}, \underline{二次電子}, \underline{陰極 (-極)})。
\end{enumerate}

\subsubsection*{短い模範表現(暗記用)}
\begin{itemize}
  \item $\alpha$ — 「単位長で増える電子対の数(\underline{単位長さ}・\underline{電界}・\underline{電離})。」
  \item $\gamma$ — 「陰極衝突で出る二次電子の個数(\underline{二次電子}・\underline{陰極 (-極)})。」
\end{itemize}

\subsubsection*{チェックリスト(答案を書く前に)}
\begin{itemize}
  \item 定義を一文で書いたか(必須)。
  \item 物理的理由(衝突・電界・陰極二次放出)を1文で示したか。
  \item 文字式($n_1=n_0e^{\alpha\ell}$、あるいは $\gamma(e^{\alpha\ell}-1)=1$ の意味)を一つ入れたか。
\end{itemize}

\subsubsection*{よくある減点ポイント(短く)}
長い見出しを本文と同じ行に書いて語句が押し出される/$\alpha$(一次増幅)と$\gamma$(二次放出)を混同する/「何を数えているか」を書かない。

\vspace{4mm}
% (この節を覚えれば、出題形式に依らず $\alpha,\gamma$ を確実に説明できます。)


\section{パッシェンの法則(説明)}
放電開始電圧 $V_c$ は $p\ell$ の関数であり、ある $p\ell$ で最小値(パッシェン・ミニマム)を持つ。

\subsection{物理的理由(記述対策)}
$V_c$ が最小値(パッシェン・ミニマム)を持つ理由は、電子のエネルギー獲得と衝突回数のトレードオフによる。
\begin{description}
  \item[右側($p\ell$ 大 / 高気圧)] 分子密度が高く平均自由行程が短い。電子は加速される前に衝突し、電離に必要なエネルギーを得にくい $\to$ $V_c$ 上昇。
  \item[左側($p\ell$ 小 / 低気圧)] 電子は十分加速されるが、分子密度が低すぎて衝突回数が不足する $\to$ $V_c$ 上昇。
\end{description}

\chapter{グロー放電を利用した機器}
\section{動作原理(要点)}
グロー放電は低〜中圧の気体中で、電子・イオンの局所的な衝突電離と再結合により安定な放電輝線(発光)を維持する現象である。電子温度は比較的低く、放電は管壁近傍に集中することが多い。

\subsection{代表的な機器と用途(名称中心)}
\begin{itemize}
  \item \textbf{ネオン管(ネオンサイン)}: 希ガスの発光を利用する低圧封入管。
  \item \textbf{蛍光灯}: グロー放電による紫外線を蛍光体で可視光に変換。
  \item \textbf{プラズマディスプレイ(PDP)}: グロー放電を表示に利用する方式(詳細は講義範囲に合わせる)。
  \item \textbf{スパッタリング(薄膜堆積)}: 低圧放電を用いる薄膜形成法(詳細は講義範囲に合わせる)。
\end{itemize}

\subsection{代表的仕様(概数・試験で押さえる点)}
\begin{itemize}
  \item 圧力: 低〜中圧(数 Pa〜数百 Pa の範囲が典型) — 機器により大きく異なる(蛍光灯は非常に低圧、ネオン管はやや高め)。
  \item 電圧/電流: 起電圧は数百 V 程度(ネオン管・蛍光灯の点火は高圧が必要)、運転時は比較的低電流で定 電圧領域を持つ。\
  \item ガス組成: ネオン・アルゴン・水銀蒸気などが代表。
\end{itemize}

\subsection{利点・欠点・安全上の注意}
\begin{itemize}
  \item 利点: 低消費電力で安定発光が得られる、狭い領域での均一放電が可能(PDP 等)。
  \item 欠点: 起動(点火)に高電圧が必要、封入ガスや蛍光体・水銀などの環境配慮が必要。\
  \item 安全: 高電圧・紫外線・オゾン生成の危険、破損時のガラス飛散や有害ガスに注意。
\end{itemize}

\subsection{試験で問われやすいポイント}
\begin{itemize}
  \item グロー放電とアーク放電の違い(圧力・電流密度・温度の違い)。
  \item 蛍光灯の発光機構(グロー放電→紫外線→蛍光体→可視光)。
  \item スパッタリングでの荷電粒子の役割(イオン加速→ターゲット衝突による原子放出)。
\end{itemize}

\chapter{アーク放電を利用した機器}
\section{動作原理(要点)}
アーク放電は高電流密度・高温度で電流が流れる状態で、電子衝突と熱電子放出が支配的となる。プラズマ温度は非常に高く(数千〜万 K)、電極近傍での熱的・電気的作用が顕著である。

\subsection{代表的な機器と用途(名称中心)}
\begin{itemize}
  \item \textbf{アーク溶接}: アークで金属を局所的に融解して接合する。
  \item \textbf{アーク炉}: 大電流アークで高温を作り金属を溶解する。
  \item \textbf{高輝度放電ランプ(HID)}: アーク放電を用いた高輝度光源。
  \item \textbf{遮断器のアーク消弧}: 開閉時のアークを消弧する技術(詳細は講義範囲に合わせる)。
\end{itemize}

\subsection{代表的仕様(概数・試験で押さえる点)}
\begin{itemize}
  \item 圧力: 通常は大気圧下で発生(真空や低圧のアークもあるが区別すること)。
  \item 電流/電圧: アークは高電流・低電圧領域(溶接: 数十〜数百 A、電圧は数十 V)。
  \item 温度: プラズマ温度は高く、材料の蒸発・イオン化が起き得る。
\end{itemize}

\subsection{利点・欠点・安全上の注意}
\begin{itemize}
  \item 利点: 極めて高温を短時間で得られるため溶接・溶解等の高エネルギー処理に適する。
  \item 欠点: 電極消耗・ノイズ(電磁的)・高温による周辺損傷のリスクがある。
  \item 安全: 強い紫外線・高温・スパッタ(飛散粒子)、高電流取り扱いのための適切な遮断・接地が必須。
\end{itemize}

\subsection{試験で問われやすいポイント}
\begin{itemize}
  \item アーク放電とグロー放電の識別(電流密度・温度・発光の性質)。
  \item アーク溶接における電流・電圧の役割(加熱量はおおむね電力 $P=VI$ に依存)。
  \item 遮断器でのアーク消弧方式(磁気的引き延ばし、ガス置換、真空中での消弧など)とその利点・短所。
\end{itemize}

% 参考: 添付PDFや講義ノートの図・数値を参照して要点をまとめた。

\chapter{試験直前チェックリスト}
\begin{itemize}
  \item 計算: $r=mv/(eB)$ の式変形がスムーズか(速度・周期を求められるか)。
  \item 導出: $N$ を等比級数の和の形に整理できるか。
  \item 記述: パッシェンの法則で「衝突しすぎるからダメ(右側)」と「衝突相手がいないからダメ(左側)」を明快に書き分けられるか。
  \item 用語: $\alpha$(電子なだれ), $\gamma$(二次電子放出), $\alpha\ell$, $V_c$。
\end{itemize}

\chapter*{参考}
講義ノート(手書きPDF)に準拠。健闘を祈ります。

\end{document}
