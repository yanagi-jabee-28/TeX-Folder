% !TEX program = lualatex
%==============================================================================
% プリアンブル (Preamble)
%==============================================================================

% ===== ドキュメントクラス =====
\documentclass[
	a4paper,
	11pt
]{ltjsreport}

%------------------------------------------------------------------------------
% パッケージ読み込み
%------------------------------------------------------------------------------

% ===== フォント・言語設定 (LuaLaTeX専用) =====
\usepackage{luatexja-fontspec}
\usepackage{lmodern} % フォントサイズの置き換えを防ぐため

% ===== レイアウト関連 =====
\usepackage[margin=2.5cm]{geometry}
\usepackage{booktabs}
\usepackage{float}
\usepackage{graphicx}

% ===== 数式・物理単位関連 =====
\usepackage{amsmath}
\usepackage{siunitx}
\sisetup{
  detect-all = true,
  per-mode = symbol,
  range-phrase = {--},
  exponent-product = \cdot,
  separate-uncertainty = true
}
\usepackage{bm} % ベクトルを太字にするため (\bm)

% ===== その他 =====
\usepackage{url} % URLを適切に表示
\usepackage{xurl} % Improved line breaking for long URLs
\urlstyle{same}
\Urlmuskip=0mu plus 2mu
\usepackage[
  hidelinks,
]{hyperref}
\usepackage[super,square]{natbib} % 引用を上付き角括弧に

% ===== SVG画像埋め込み =====
\usepackage{svg}
\svgsetup{
  inkscapearea=page,
  width=0.8\textwidth
}

%------------------------------------------------------------------------------
% 各種設定
%------------------------------------------------------------------------------

% ===== フォント設定 =====
\setmainfont{Latin Modern Roman}
\setsansfont{Latin Modern Sans}
\setmonofont{Latin Modern Mono}
\setmainjfont[Renderer=HarfBuzz]{Yu Mincho}
\setsansjfont[Renderer=HarfBuzz]{Yu Gothic}
\DeclareMathSizes{11}{11}{7}{5} % 数学フォントサイズの調整

% ===== ドキュメント情報 =====
\title{BCDRを用いた基板の誘電損失と表面粗さによる\\
電気伝導性の測定\\[0.3cm]
{\normalsize Measurement of Substrate Dielectric Loss and Surface-Roughness-Induced\\
Effective Conductivity Using a Balanced-Type Circular-Disk Resonator (BCDR)}}
\author{長野工業高等専門学校\\電気電子工学科 栁原 魁人\\(指導教員 春日 貴志)}
\date{\today}

% ===== 数式用カスタムコマンド =====
\providecommand{\dd}{\mathrm{d}} % 微分演算子 d
\newcommand{\mi}{\mathrm{j}} % 虚数単位 j

%==============================================================================
% ドキュメント本体 (Body)
%==============================================================================
\begin{document}
\sloppy % allow more flexible spacing to reduce overfull \hbox warnings

\maketitle

%------------------------------------------------------------------------------
% 概要
%------------------------------------------------------------------------------
\begin{abstract}
人工知能(AI)やクラウドコンピューティング,第5世代移動通信システム(5G)の普及により,データセンターおよびバックボーンネットワークにおける伝送帯域は年率40\%以上で増大している.第6世代移動通信システム(6G)では\SI{100}{\giga\hertz}帯を超えるミリ波・サブテラヘルツ帯の利用が想定されており,この周波数域ではプリント基板における信号減衰が伝送品質を左右する主要因となる.信号減衰は誘電体内部のエネルギー散逸に起因する誘電損失と,配線導体でのジュール熱に起因する導体損失の総和で決定されるが,従来の測定手法では誘電体の評価が主眼に置かれ,導体表面粗さ(Surface Roughness)が表皮効果と相互作用して高周波抵抗を増大させる現象については定量的評価が不十分であった.本研究では,TM$_{0m0}$モードを利用した平衡形円板共振器法(BCDR)を用い,低損失基板材料MEGTRON6の複素誘電率および実効導電率を\SIrange{10}{67}{\giga\hertz}(本文解析範囲;測定有効点は最大\SI{65.3}{\giga\hertz})において評価した.測定の結果,平均比誘電率は$\varepsilon_r' \approx 3.59$で周波数依存性は小さい一方,実効導電率$\sigma_{\mathrm{eff}}(f)$は周波数上昇に伴って1--2桁低下することが明らかとなった(例:\SI{12.7}{\giga\hertz}で \SI{2.41e7}{\siemens\per\metre},\SI{23.3}{\giga\hertz}で \SI{8.23e6}{\siemens\per\metre},\SI{65.3}{\giga\hertz}で \SI{3.44e6}{\siemens\per\metre}).本文の結論は\SI{67}{\giga\hertz}以下の測定帯域に基づく.測定は複数日にわたり合計20回実施し,本文で示す各データ点および図は20回の測定の平均値である.付録は付けていない.
\end{abstract}

\tableofcontents
\clearpage

%==============================================================================
% 第1章 序論
%==============================================================================
\chapter{序論}

\section{次世代通信技術の背景と高周波化の必須性}

人工知能(AI)やクラウドコンピューティング,第5世代移動通信システム(5G)の普及に伴い,データセンター等で扱う情報量は年率40\%を超える成長を続けている.スマートフォンや4K/8K動画配信サービスの普及により,2025年には現在比で約70倍の通信トラフィックに達するとの予測もあり\cite{soumu2022beyond5g},これに対応するため,IEEEではサーバーラック内の通信速度として\SI{400}{\giga\bit\per\second}から\SI{800}{\giga\bit\per\second},さらには\SI{1.6}{\tera\bit\per\second}の実現が検討されている\cite{ieee802.3bz,otr235transmission}.この超高速伝送を実現するため,信号のベースクロックは\SI{56}{\giga\hertz}から\SI{112}{\giga\hertz}へと高速化する.デジタル信号の高次高調波成分を考慮すると,伝送路となるプリント基板やコネクタなどのハードウェアは,約\SI{100}{\giga\hertz}から\SI{300}{\giga\hertz}に達する広範な周波数帯域に対応する必要がある.したがって,ミリ波・サブテラヘルツ帯における材料の電磁気的特性の正確な把握は,次世代通信インフラ構築の基盤技術となる.

現在の第5世代移動通信システム(5G)では,30~GHzを超えるミリ波帯域が利用され始めており,広い周波数帯域による高速大容量化が実現されている.5Gの次の世代である6G通信では,100~GHzから300~GHzのサブテラヘルツ帯の利用が見込まれており,5Gを上回る超高速大容量通信と,5Gの1/100ともいわれる超低消費電力性能の両立が目標とされている\cite{kato2024mag}.これは単なる周波数の拡張ではなく,限定的な電波資源の有効利用と同時に,エネルギー効率の大幅な改善を達成するための戦略である.

\section{高周波帯における信号減衰の課題と測定の必須性}

このような極めて高い周波数帯では,プリント基板における信号減衰が深刻な課題となる.高周波回路における伝送損失は,誘電体損失(Dielectric loss)と導体損失(Conductor loss)の合算で表現される.誘電体損失は誘電正接$\tan\delta$と周波数$f$に比例し,導体損失は表面抵抗$R_s$に比例するため,周波数が数十GHz帯に達すると両損失とも急激に増大する.この結果,信号品質(Signal Integrity)の劣化や消費電力の増大を招く主要因となる.したがって,6G/Beyond 5G時代のハードウェア設計においては,使用する材料の電磁気的特性(複素誘電率$\dot{\varepsilon} = \varepsilon' - \mathrm{j}\varepsilon''$および導電率$\sigma$)を使用周波数帯域において正確に把握することが,設計精度を確保する上で必須の要件となる.

特に周波数が高くなると,信号電流は導体表面の極めて浅い領域に集中する.この領域の深さを表皮深さ$\delta_s$といい,以下の式で与えられる:
\begin{equation}
\delta_s = \frac{1}{\sqrt{\pi f \mu_0 \sigma}} \propto f^{-1/2}
\label{eq:skin_depth_intro}
\end{equation}
ここで,$f$は周波数,$\mu_0$は真空の透磁率,$\sigma$は導電率であり,$\delta_s$は周波数の平方根に反比例する.銅の導電率を$\sigma = 5.8 \times 10^7\ \mathrm{S/m}$とすると,$\delta_s = \SI{66.1}{\micro\meter} / \sqrt{f\,[\si{\giga\hertz}]}$と表され,30~GHzで約0.36~$\mu$m,60~GHzで約0.25~$\mu$m,100~GHzでは約0.21~$\mu$mとなる.このようにサブミクロンオーダの表皮深さになると,導体表面の微細な凹凸(表面粗さ)が実効的な導電率に大きな影響を与える.

プリント基板の銅箔は,密着強度確保のために二乗平均平方根粗さ(RMS roughness)$R_q \approx \SI{1}{\micro\metre}\text{--}\SI{3}{\micro\metre}$程度を有する(JPCA‑KHS01,Fukuda\cite{fukuda2021transmission}).このレンジでは\SI{100}{\giga\hertz}帯で$R_q/\delta_s\approx 5\text{--}15$となり,表皮深さを大きく上回る.簡易的な幾何学見積りでは実効経路長は概ね $\sqrt{1+(R_q/\delta_s)^2}$ 程度まで増大すると評価できるが,局所的な磁界集中や突起形状に伴う増分(いわゆる「雪玉」効果)は Huray や Hammerstad 系のモデルで厳密に扱う必要がある.したがって,本稿では上記の幾何学見積りを「概算」として提示し,定量評価や設計適用には Huray/Hammerstad または最近の波導モデル(Gold, Lomakin ら)による補正を併用することを推奨する\cite{fukuda2021transmission,kato2023dband,huray2010,gold2017physical,lomakin2018analytical}.

従来の測定では,主に低周波から数十GHz帯までの誘電損失が評価されてきたが,表面粗さを含めた実効導電率の周波数特性については十分に検討されていない.特に6G帯へ向けて100~GHz以上の周波数帯での導体損失の評価は,基板開発において不可欠な要素となっている.

\section{既存研究の現状と本研究の位置づけ}

これまで,FR-4やMEGTRON6などの基板材料について,平衡形円板共振器法(BCDR)により複素誘電率を測定し,誘電損失を評価する研究は行われてきた\cite{tomioka2025}.

しかし,信号減衰のもう一つの重要な要因である導体損失について,銅箔の表面粗さが実効的な導電率に与える影響を広帯域で定量的に評価した研究は十分ではなかった.

\section{本研究の目的と意義}

本研究の目的は,ミリ波帯(10~GHz--67~GHz)における基板材料の損失評価手法として平衡形円板共振器法(Balanced-Type Circular Disk Resonator: BCDR法)を適用し,誘電損失と導体損失を高精度に分離測定する手法を確立するとともに,導体表面粗さが伝送特性に与える影響を定量的に解明することである.具体的には,低損失基板材料として次世代高速回路に用いられるPanasonic製MEGTRON6(ガラスエポキシ系複合材料)を対象とし,以下の3段階の研究を遂行する(本稿では測定帯域を67~GHz以下に制限し解析を行う).

1. BCDR法を用いて,10~GHzから67~GHzの帯域における基板材料の複素誘電率(比誘電率 $\varepsilon_r'$ および誘電正接 $\tan\delta$)を測定する.測定には鏡面研磨銅板(標準導体)を用いて誘電体寄与を抽出する.

2. 同一手法により,製造処理済み銅箔を用いて実効導電率 $\sigma_{\mathrm{eff}}(f)$ を導出する.

3. 得られた材料定数に基づき,マイクロストリップ線路の減衰定数 $\alpha(f)$ を算出し,従来モデルとの乖離を定量的に評価する.

本研究により,6G向けの基板開発において,材料の誘電特性と導体特性の両者を同時に評価する方法論を確立することの重要性を実証する.特に,導体表面の微細な凹凸が実効導電率に与える周波数依存性の定量化は,次世代デバイスの低損失化設計における重要な知見を提供するものである.

%==============================================================================
% 第2章 誘電体と基板材料について
%==============================================================================
\chapter{誘電体と基板材料について}

%------------------------------------------------------------------------------
\section{プリント基板}
%------------------------------------------------------------------------------

プリント基板は現代の電子機器に欠かせない電子部品である.電子部品を搭載する絶縁基板の上に銅箔回路を形成し,抵抗器やコンデンサなどの受動素子,トランジスタなどの能動素子をマウントすることで,電子回路としての機能を発揮する.

プリント基板は構成形式により分類される.片面基板は基板の一面に導体回路が形成されており,コスト低減が必要な大量生産品に用いられる.両面基板は基板の両面に導体回路が形成され,片面基板より高密度な配線が可能で,高集積化・高機能化に応じて発展してきた.さらに微細化・高密度実装が要求される用途では,基材内部と表層の導体回路が3層以上で構成される多層基板が用いられる.

プリント基板を構成する誘電体中の信号伝達速度$v$は,比誘電率$\varepsilon_r'$と光速$c$の関係で表される:
\begin{equation}
v = \frac{c}{\sqrt{\varepsilon_r'}}
\end{equation}
したがって誘電率が低いほど信号伝達速度が高まり,高速伝送に有利である.5G/6Gで用いられる30~GHz以上の周波数帯では,低誘電率材料の採用が信号の高速伝播を実現する鍵となる.

%------------------------------------------------------------------------------
\section{基板材料}
%------------------------------------------------------------------------------

高周波基板に用いられる一般的な材料と特性を表~\ref{tab:substrate_materials}に示す.

\begin{table}[H]
\centering
\caption{高周波基板材料の特性}
\label{tab:substrate_materials}
\begin{tabular}{p{2.5cm}p{1.8cm}p{6cm}}
\toprule
基板材料 & 比誘電率 & 用途・特徴 \\
\midrule
ガラスエポキシ(FR-4) & 4.0~5.2 & 安価で強度高い.多層基板で多く採用.価格優先 \\
テフロン & 2.1 & GHz帯で誘電率安定.$\tan\delta$非常に小さい.高価 \\
セラミック & 4.5~5.2 & GHz帯で$\tan\delta$低く安定.放熱性高い.高価 \\
フェノール & 4.6~5.5 & 耐熱性はFR-4に劣るが低価格.生産量最大 \\
ポリイミド & 3.5 & フレキシブル基板用.柔軟性と電気特性両立 \\
低損失エポキシ & 3.2~3.8 & 低$\tan\delta$で広帯域対応.MEGTRON6が代表例 \\
\bottomrule
\end{tabular}
\end{table}

kHz~MHz帯では安価なフェノールやFR-4が用いられるが,数十GHzの信号を扱う製品ではテフロンやセラミック等の高周波特性に優れた材料が必要とされる.しかしこれらはFR-4に比べ4倍以上高価であるため,GHz帯においてもコスト面からFR-4が広く採用されている現状にある.

\subsection{低損失基板MEGTRON6の特性}

本研究ではMEGTRON6(パナソニック製,低損失高周波基板)を測定対象とした.MEGTRON6は20~GHz以上での使用を想定し,従来のFR-4より大幅に誘電損失を低減している低損失エポキシ樹脂基板である.

MEGTRON6の主な特徴は以下の通りである:

\begin{itemize}
\item \textbf{低誘電率}:比誘電率$\varepsilon_r' \approx 3.6$程度で,FR-4の4.0~5.2と比べて低い.これにより信号伝播速度が向上する.
\item \textbf{低誘電正接}:誘電正接$\tan\delta$がGHz帯で$10^{-3}$オーダと非常に低く,誘電損失を大幅に低減.
\item \textbf{広帯域対応}:装置・材料仕様としては10~GHzから100~GHz超までの特性維持が報告されている(本稿の解析は\SI{67}{\giga\hertz}以下に限定).
\item \textbf{ガラス転移温度}:高い耐熱性により多層基板の製造過程での熱ストレスに耐性.
\item \textbf{加工性}:FR-4同等の加工性を保つため,既存の製造ラインでの対応が可能.
\end{itemize}

MEGTRON6のような低損失基板は,5G/6G時代の高周波回路設計において不可欠な材料である.ただし,誘電特性の優位性があっても,導体損失(特に表面粗さに起因するもの)が見過ごされると,全体の伝送損失が期待値に達しない可能性がある.

\subsection{ミリ波帯における基板材料の伝送損失}

高周波伝送線路の伝送損失$\alpha$は以下の二つの要因で構成される:
\begin{equation}
\alpha = \alpha_d + \alpha_c
\end{equation}

ここで$\alpha_d$は誘電損失による減衰定数,$\alpha_c$は導体損失による減衰定数である.

\textbf{誘電損失による減衰定数}$\alpha_d$は周波数に比例して増加し,基板材料の$\tan\delta$と周波数に強く依存する.一方,\textbf{導体損失による減衰定数}$\alpha_c$は周波数の平方根に比例して増加する:
\begin{equation}
\alpha_c \propto \sqrt{f}
\end{equation}

周波数が100~GHzを超える高周波帯では,誘電損失が支配的になるように見えるが,導体の表面粗さがある場合,その影響が顕著に現れ始める.これは,表面粗さがある導体表面では,電流が迂回路を強制され,実効的な抵抗が増加するためである.

\subsection{導体表面粗さの周波数依存効果}

銅箔表面の粗さは,従来の触針式粗さ計では最大高さ粗さRz(DIN規格の最大高さ)や10点平均粗さRzjisで評価されてきた.しかし高周波帯では,微細な凹凸が信号伝播に与える影響がより重要になる.

表皮深さと表面粗さの相対関係が重要である:
\begin{itemize}
\item \textbf{表皮深さ >> 表面粗さ}:表面粗さの影響は無視できる.
\item \textbf{表皮深さ $\sim$ 表面粗さ}:表面粗さが実効抵抗を大きく増加させる(critical regime).
\item \textbf{表皮深さ << 表面粗さ}:表面粗さの詳細な形状が支配的になる.
\end{itemize}

6G帯(\SI{100}{\giga\hertz}--\SI{300}{\giga\hertz})では,表皮深さが\SI{0.1}{\micro\metre}オーダに達するため,サブミクロンの凹凸でさえも無視できない影響を与える.このため,高周波基板用銅箔には,従来の化学的粗化ではなく,より平滑な表面を持つ低プロファイル銅箔や,特殊な表面処理が施された製品が採用され始めている.

\subsection{高周波用銅箔の表面処理技術}

従来のプリント基板では,銅箔と誘電体の接着性を確保するため,化学的粗化(chemical roughening)により銅箔表面を意図的に凹凸させてきた.しかし,高周波帯ではこの粗化された表面が導体損失を増加させる.

近年開発された高周波基板向けの表面処理技術には以下のようなものがある:

\begin{itemize}
\item \textbf{低プロファイル(LP)銅箔}:粗さRzjis $\le$ 2.5~$\mu$mに抑制.
\item \textbf{超低プロファイル(ULP)銅箔}:粗さをさらに低減.
\item \textbf{フラットボンディング(FB)処理}:錫合金と偶合層を形成し,約0.1~$\mu$mの平滑な表面を実現\cite{kido2023copper}.
\item \textbf{ニッケルレス金めっき}:パラジウム層を介して金めっきし,表皮効果で表面のニッケル(高抵抗)の影響を回避.
\end{itemize}

これらの技術により,30~GHz以上の周波数帯で,従来の粗化銅箔に比べて大幅な伝送損失低減が実現されている.

\subsection{誘電体の特性}
%------------------------------------------------------------------------------

\subsection{誘電体の定義}

誘電体とは,外部電界が印加されたときにエネルギーを蓄積できる材料である.誘電体の分極しやすさを定量的に示すのが誘電率$\varepsilon$である.

\subsection{誘電体の電気的特性}

誘電体の電気的特性は,誘電分極と誘電損失により特徴づけられる.外部電界が印加されると,材料を構成する原子核や陽イオン,電子は元の位置からわずかに変位し,物質内に多数の微小電気双極子が発生する.この現象が誘電分極である.

\subsection{誘電分極}

双極子動きには慣性があるため,印加電圧の周波数が高くなると分極の種類によっては変化に追随できなくなる.このため誘電率が周波数によって変化する「誘電分散」が生じる.

一般に三種類の分極が存在する:

\begin{enumerate}
\item \textbf{電子分極}:原子核の相対的変位により生じる.赤外線領域程度の周波数まで寄与.
\item \textbf{イオン分極}:正負イオンの相対的変位により生じる.赤外線領域程度の周波数まで寄与.
\item \textbf{双極子分極(配向分極)}:永久双極子が外部電界に配向することにより生じる.マイクロ波領域程度の周波数まで寄与.
\end{enumerate}

周波数の上昇に伴い,双極子はマイクロ波領域程度までしか追随できず,イオン分極も赤外線領域程度までの寄与にとどまる.紫外線領域では電子分極も寄与しなくなり,全体として誘電率は周波数とともに低下する傾向を示す.

\subsection{複素誘電率}

交流電界が印加されたとき,誘電体内の双極子は周期的に向きを変えるが,この変化は電界変化に対して時間遅れを伴う.この遅れを位相角$\delta$で表すと,誘電率$\varepsilon$は複素数で表現される:
\begin{equation}
\varepsilon = \varepsilon' - \mi\varepsilon'' = \varepsilon_0 \varepsilon_r' (1 - \mi \tan\delta)
\label{eq:complex_permittivity}
\end{equation}
ここで$\varepsilon_0$は真空の誘電率,$\varepsilon'$は比誘電率(実部,エネルギー蓄積に関連),$\varepsilon''$は誘電損失(虚部,エネルギー散逸に関連)である.

\subsection{誘電損失と誘電正接}

双極子が交流電界から受け取ったエネルギーの一部は,分極に伴う運動により熱エネルギーとして消費される.この損失の大きさを直接表す量が誘電損失$\varepsilon''$である.

複素誘電率の実部と虚部の比を「誘電正接」と呼び,次式で定義される:
\begin{equation}
\tan\delta = \frac{\varepsilon''}{\varepsilon'}
\label{eq:loss_tangent}
\end{equation}
$\tan\delta$は,誘電体内部で熱エネルギーとして散逸される電力の,蓄積される電力に対する比を示す無次元量であり,高周波基板材料の低損失性能を定量化する最も重要なパラメータである.一般的なガラスエポキシ基板(FR-4)では$\tan\delta \approx 0.02$程度であるのに対し,MEGTRON6のような低損失材料では$\tan\delta \approx 0.002$--$0.004$の範囲にあり,約5--10倍の低損失特性を有する.

%------------------------------------------------------------------------------
\section{高周波伝送における信号減衰}
%------------------------------------------------------------------------------

プリント基板上の伝送線路では,二つの主要な損失機構により信号が減衰する:

\paragraph{(1)誘電損失による減衰定数$\alpha_d$}
\begin{equation}
\alpha_d = \frac{\pi f}{c} \cdot \varepsilon_r'\tan\delta \cdot \frac{\sqrt{\varepsilon_{re}}(\varepsilon_{re}-1)}{\varepsilon_{re}(\varepsilon_r' - 1)}
\end{equation}
ここで$f$は周波数,$c$は光速,$\varepsilon_{re}$は有効比誘電率である.誘電損失による減衰は周波数の上昇に伴い増加する.

\paragraph{(2)導体損失による減衰定数$\alpha_c$}
\begin{equation}
\alpha_c = \frac{R_s}{Z_0 W}
\end{equation}
ここで$R_s = \sqrt{\pi f \mu_0 / \sigma}$は表面抵抗,$Z_0$は特性インピーダンス,$W$は線路幅である.

周波数が高くなると表皮深さ$\delta_s = 1/\sqrt{\pi f \mu_0 \sigma}$が浅くなり,電流はより狭い領域に集中する.このとき導体表面の凹凸(表面粗さ)の影響が顕著になり,実効導電率が低下して導体損失が増加する.

全体の減衰定数は両者の和となる:
\begin{equation}
\alpha = \alpha_d + \alpha_c
\end{equation}

%------------------------------------------------------------------------------
\section{本研究に用いた測定用基板}
%------------------------------------------------------------------------------

本研究ではMEGTRON6基板を測定対象とした.MEGTRON6は厚さ0.3~0.5~mmの複数試料を用意し,複素誘電率と実効導電率の周波数依存性を測定した.

%==============================================================================
% 第3章 測定原理および方法
%==============================================================================
\chapter{測定原理および方法}

%------------------------------------------------------------------------------
\section{複素誘電率の測定法ー概要}
%------------------------------------------------------------------------------
高周波領域における材料の複素誘電率を測定する代表的な方法は,評価対象の周波数帯や損失の大きさに応じて,大きく以下の3つに分類される\cite{kato2012dielectric}.

\begin{description}
    \item[集中定数法:] 数kHz~1~GHz程度の比較的低い周波数帯で用いられる.測定試料を電極で挟んでコンデンサを形成し,そのインピーダンス(キャパシタンスとコンダクタンス)から誘電率を算出する.
    \item[反射・伝送法(Sパラメータ法):] 数十MHz~数百GHzの広帯域に対応可能.試料に対し電磁波を照射し,反射・透過係数(Sパラメータ)を測定して誘電率を算出する.高損失材料の評価に適しており,自由空間で行うフリースペース法もこの一種である\cite{obikata2024freespace}.
    \item[共振法:] 1~GHz以上の高周波帯で,特に低損失材料を高精度に評価する場合に用いられる.試料を内部に設置した共振器の共振周波数とQ値の変化から誘電率を求める.本研究で用いるBCDR法も共振法に分類される.
\end{description}

\paragraph{Sパラメータと高周波測定の基礎}
高周波回路の特性評価では,インピーダンスやアドミタンスの代わりにSパラメータ(Scattering parameter, 散乱行列)が広く用いられる.これは,高周波領域では電圧・電流を直接測定することが困難であるのに対し,電力の進行波・反射波はベクトル・ネットワーク・アナライザ(VNA)を用いて安定して測定できるためである.2ポート回路では,ポート1, 2への入射波を$a_1, a_2$,出射波を$b_1, b_2$とすると,Sパラメータは次式で定義される\cite{fujii2010}:
\begin{equation}
\begin{bmatrix} b_1 \\ b_2 \end{bmatrix} = \begin{bmatrix} S_{11} & S_{12} \\ S_{21} & S_{22} \end{bmatrix} \begin{bmatrix} a_1 \\ a_2 \end{bmatrix}
\end{equation}
ここで,$S_{11}$は入力ポートの反射係数,$S_{21}$は透過係数(伝送特性)である.

共振法では,VNAを用いて共振器の透過係数$S_{21}$の周波数応答を測定し,得られた共振ピークから共振周波数$f_0$と,共振の鋭さを表す無負荷Q値$Q_u$を決定する.$Q_u$は共振ピークの中心周波数とそのピークから3~dB低下した点の周波数幅(半値全幅 $\Delta f_{\text{3dB}}$)の比で定義される.
\begin{equation}
Q_u = \frac{f_0}{\Delta f_{\text{3dB}}}
\end{equation}
低損失な材料ほどエネルギーの散逸が少ないためQ値は高くなり,鋭い共振ピークを示す.共振法は,この$f_0$と$Q_u$の精密な測定を通じて,微小な誘電損失を評価することに優れている.

\paragraph{解析帯域の限定}
本稿では測定装置の仕様により最大で\SI{100}{\giga\hertz}までの測定を行ったが,データの信頼性と解析の一貫性を考慮し,本解析および考察は\SI{67}{\giga\hertz}以下の帯域に限定する.以降,特に断りのない限り「測定帯域」は\SI{67}{\giga\hertz}以下を指す.

%------------------------------------------------------------------------------
\section{平衡形円板共振器法(BCDR)}
%------------------------------------------------------------------------------

\subsection{BCDR法の原理}

BCDR法は共振法の一種であり,高周波帯での複素誘電率と実効導電率の同時測定に適している.本手法は加藤・堀部らによって開発され,\SIrange{10}{170}{\giga\hertz}の超広帯域測定が実証されている\cite{kato2019broadband}.

\paragraph{構造と励振メカニズム}
測定では,測定対象の誘電体基板2枚を準備し,中央に厚さ$t_c = 0.06$~mm,直径$D$の銅円板電極を挟み込み,上下から銅平板電極で圧着する.銅平板の中央には直径$2a$の励振孔が設けられており,同軸線路で電界結合励振を行う.

この構造の最大の特徴は,\textbf{円板中心軸上の対称性}により,TM$_{0m0}$モード($m=1, 2, 3, \ldots$)のみが選択的に励振され,他の不要モードが十分に抑制される点にある.これにより,モード干渉の問題を回避し,高次モードを用いた広帯域測定が可能となる.

\paragraph{Mode-Matching法による厳密解析}
BCDR法では,励振孔まで厳密に考慮したMode-Matching法による電磁界解析が用いられる\cite{kato2019broadband}.解析では,共振器内部を複数の領域に分割し,各領域で固有モード展開を行い,境界条件で結合させる.これにより,共振周波数$f_{0m0}$と比誘電率$\varepsilon_r'$の関係式が導出される:
\begin{equation}
\det H(\varepsilon_r'; f_{0m0}, R+\Delta R, a, M, t) = 0
\end{equation}
ここで,$R$は銅円板半径,$\Delta R$は縁端効果による補正量,$a$は励振孔半径,$M$は励振孔長さ,$t$は誘電体厚さである.

\paragraph{改良型フルウェーブ・モーダル解析}

従来のMode-matching解析では,誘電体領域が磁気壁で囲まれた円筒として近似され,励振孔の効果が簡略化されていた.このため,特に高誘電率材料やD-band(110~170~GHz)以上の周波数帯では,測定結果に系統誤差が生じることが知られていた.

改良型フルウェーブ・モーダル解析では,以下の改善が加えられている\cite{kato2023dband}:

\begin{itemize}
\item \textbf{励振孔の厳密な表現}:励振孔を含む全体の共振器構造を,複数の単純な要素に分割(circuit segmentation)し,各要素の伝送行列(admittance matrix)をmode-matching解析で計算して結合する.
\item \textbf{放射損失の正確な評価}:誘電体シートからの電磁波放射を厳密に考慮し,カットオフ周波数$f_c^{\text{rad}}$に近い領域での計算精度を向上.
\item \textbf{TM$_{0m}$モードのみの計算}:円筒対称性により,TM$_{0m}$モードのみを考慮すればよく,TE$_{nm}$($n \neq 0$)や混合モードは自動的に分離される.
\end{itemize}

この手法により,D-band(110~170~GHz)を含む超広帯域での高精度測定が実現され,従来法に比べて誘電率の測定値の信頼性が大幅に向上している\cite{kato2023dband}.

\paragraph{測定パラメータの導出}
ベクトル・ネットワーク・アナライザ(Keysight N5222B または N5227A)で透過係数$S_{21}$を測定し,以下のパラメータを導出する:

\begin{enumerate}
\item \textbf{比誘電率$\varepsilon_r'$}:共振周波数$f_{0m0}$から上記の改良型フルウェーブ解析により数値的に決定
\item \textbf{誘電正接$\tan\delta$}:無負荷Q値$Q_u$から導出.導体損失の影響を分離するため,次式を用いる:
\begin{equation}
\tan\delta = \frac{1}{Q_u}\left(1 - \frac{R_s}{B}\right)(1+A)
\end{equation}
ここで,$R_s = \sqrt{\pi f \mu_0 / \sigma}$は表面抵抗,$A$,$B$は数値解析により求められる補正定数である.この式は,誘電体内に埋め込まれた導体円板の損失(表面抵抗により決定)を考慮し,実測のQ値からそれを差引くことで,純粋な誘電損失を抽出する手法である.
\end{enumerate}

\paragraph{周波数可変性と広帯域測定}
BCDR法の特筆すべき特徴は,銅円板直径$D$を変えることで測定周波数を調整できる点である.同一の誘電体試料に対し,異なる直径の円板(例:$D = 9, 12, 15, 18, 21$~mm)を用いることで,10~100~GHz超の広帯域で複素誘電率を測定できる\cite{kato2023dband,kaneko2011balanced}.

\subsection{測定手順と実効導電率の算出}

実効導電率の測定は,提案された2ステッププロセスに基づき実施される(手法の詳細は文献参照)\cite{kato2019broadband}.本手法の原理は,共振器のQ値が誘電損失と導体損失の両方に依存することを利用し,誘電損失が既知の試料に対して導体損失を分離評価するものである.

\paragraph{プロセス(1):基板の複素誘電率測定(誘電損失の評価)}

まず,評価対象の基板(例:MEGTRON6)から銅箔を化学エッチング等で完全に除去した誘電体シートを2枚準備する.その間に,導電率が既知の標準銅円板(例:直径$D = 15$~mm,導電率$\sigma_0 = 5.8 \times 10^7$~S/m)を挟み込む.BCDRで複数の共振モード(TM$_{010}$~TM$_{0m0}$)における共振周波数$f_{0m0}^{\text{ref}}$と無負荷Q値$Q_u^{\text{ref}}$を測定する.

Mode-matching解析により比誘電率$\varepsilon_r'$を決定し,さらに次式から誘電正接$\tan\delta$を算出する:
\begin{equation}
\tan\delta = \frac{1}{Q_u^{\text{ref}}}\left(1 - \frac{R_s^{\text{ref}}}{B}\right)(1+A)
\end{equation}
ここで,$R_s^{\text{ref}} = \sqrt{\pi f \mu_0 / \sigma_0}$は標準銅の表面抵抗である.この測定により,評価対象基板の誘電損失($\varepsilon_r'\tan\delta$)が周波数の関数として確定する.

\paragraph{プロセス(2):実基板銅箔の導電率測定(導体損失の評価)}

次に,評価対象の基板上に,プロセス(1)で用いた標準銅円板と全く同じ直径の円形銅箔パターンのみを残した試料を2枚製作する(フォトリソグラフィまたはケミカルエッチング).この2枚の試料を,銅箔面が内側で接触するように重ね合わせ,BCDRで同様に共振周波数$f_{0m0}^{\text{sample}}$と無負荷Q値$Q_u^{\text{sample}}$を測定する.

\paragraph{実効導電率の算出原理}
誘電損失はプロセス(1)で確定しているため,プロセス(2)で観測されるQ値の低下は,実基板銅箔の導体損失の増加に起因する.共振周波数シフト$\Delta f = f_{0m0}^{\text{sample}} - f_{0m0}^{\text{ref}}$と,Q値変化$\Delta(1/Q_u)$を解析することで,実効導電率$\sigma_{\text{eff}}$が次式で導出される:
\begin{equation}
\sigma_{\text{eff}} = \sigma_0 \left[\frac{Q_u^{\text{sample}}}{Q_u^{\text{ref}}} \cdot \left(1 - \frac{\tan\delta}{1/Q_u^{\text{ref}} - R_s^{\text{ref}}/B}\right)\right]^{-2}
\end{equation}
この実効導電率は,表面粗さによる電流経路の延長効果を含んだ値となる.高周波になるほど表皮深さ$\delta_s = 1/\sqrt{\pi f \mu_0 \sigma}$が浅くなり,表面粗さの影響が顕著になるため,$\sigma_{\text{eff}}$は周波数依存性を示す.

\paragraph{測定の信頼性確保}
実効導電率測定の信頼性を確保するためには,特に以下の点に細心の注意を払う必要がある\cite{keysight2023bcdr}.
\begin{enumerate}
    \item \textbf{試料の均一性:} プロセス(1)と(2)で使用する誘電体基板は,同一ロットから採取する.銅箔を剥がした基板と銅箔付き基板の厚さや誘電特性が異なると,Q値の変化が正しく導体損失の差を反映しなくなる.
    \item \textbf{精密なアライメント:} 円形銅箔パターンの中心と共振器の励振孔の中心を高精度に位置合わせすることが極めて重要である.キーサイト社のマニュアルによれば,偏心誤差は50~μm未満に抑える必要がある\cite{keysight2023bcdr}.ズレが大きいと不要モードが励起され,TM$_{0m0}$モードの共振特性に歪みを生じさせ,Q値の測定誤差,ひいては$\tan\delta$や導電率の誤差を増大させる.
    \item \textbf{適切な加圧とエアギャップの排除:} 上下電極で試料を圧着する際のトルク管理が重要である.トルクが不十分だと試料と電極間にエアギャップが残り,実効的な比誘電率が低く測定される.逆に過度なトルクは試料や円板電極を損傷させる.共振周波数がトルクを増しても変化しなくなる点(エアギャップが排除された点)を連続掃引でモニタしながら,最適な締め付けトルクを決定する必要がある\cite{keysight2023bcdr}.
    \item \textbf{円板電極の厚さと直径の精密測定:} 標準銅円板と実基板銅箔の厚さ・直径をマイクロメータや画像測定器で精密に測定($\pm 1 \sim 10$~μmオーダ)し,解析に入力する必要がある.これらの寸法誤差は,特に比誘電率の不確かさの主要因となる.
\end{enumerate}

\paragraph{表面粗さを含む実効導電率の物理的意義}
このプロセスで得られる$\sigma_{\text{eff}}$は,表面粗さによる電流経路の延長効果を含んだ値である.高周波では表皮深さ$\delta_s = 1/\sqrt{\pi f \mu_0 \sigma}$が浅くなるため,導体表面の微細な凹凸が電流の流れを阻害し,見かけの導電率は周波数とともに低下する.このため,測定で得られる$\sigma_{\text{eff}}(f)$は重要な設計パラメータとなり,従来の理想導電率一定仮定では見落とされる高周波損失を定量化できる.

%------------------------------------------------------------------------------
\section{測定装置および条件}
%------------------------------------------------------------------------------

\begin{itemize}
\item \textbf{ベクトル・ネットワーク・アナライザ}:Keysight N5222B(\SI{10}{\mega\hertz}~\SI{100}{\giga\hertz})
\item \textbf{平衡形円板共振器}:BCDR Advance(Ver.1.6.10 HF-120G)
\item \textbf{測定周波数範囲}:\SIrange{10}{100}{\giga\hertz}(装置仕様).本稿の解析範囲は\textbf{\SI{67}{\giga\hertz}}以下(測定有効点は最大\SI{65.3}{\giga\hertz}).測定は複数日にわたり合計20回実施し,各周波数点は20回の測定の平均値を用いている.付録は付けていない.
\item \textbf{標準銅円板電極}:直径$\phi$15~mm,厚さ0.06~mm
\item \textbf{誘電体基板試料}:MEGTRON6,厚さ0.3~0.5~mm,複数枚用意
\end{itemize}

測定時には,テストポート先端で機械式校正キット(OPEN, SHORT, LOAD, THRU)あるいは電子式校正キット(ECal)を用いてフル2ポートキャリブレーションを実施し,マッチング特性を補正し,共振ピーク特性を正確に求めた.

伝送線路の減衰定数$\alpha$は誘電損失$\alpha_d$と導体損失$\alpha_c$の和で表される:
\begin{align}
\alpha &= \alpha_d + \alpha_c \\
\alpha_d &= \frac{\pi f}{c} \cdot \varepsilon_r'\tan\delta \cdot \frac{\sqrt{\varepsilon_{re}}(\varepsilon_{re}-1)}{\varepsilon_{re}(\varepsilon_r' - 1)} \\
\alpha_c &= \frac{R_s}{Z_0 W}
\end{align}
ここで,表面抵抗$R_s = \sqrt{\pi f \mu_0 / \sigma}$,特性インピーダンス$Z_0$,線路幅$W$である.

%==============================================================================
% 第4章 測定結果
%==============================================================================
\chapter{測定結果}

%------------------------------------------------------------------------------
\section{Sパラメータと測定の基本}
%------------------------------------------------------------------------------

BCDR法による測定では,ベクトル・ネットワーク・アナライザ(VNA)を用いてSパラメータを測定する.2ポート回路のSパラメータは以下のように定義される:

\begin{table}[H]
\centering
\caption{2ポート回路のSパラメータ定義}
\label{tab:sparameter}
\begin{tabular}{p{3cm}p{9cm}}
\toprule
パラメータ & 定義・意味 \\
\midrule
$S_{11}$ & ポート1の反射係数(入射波に対する反射波の比) \\
$S_{21}$ & ポート1からポート2への透過係数(伝送特性) \\
$S_{12}$ & ポート2からポート1への逆方向透過係数 \\
$S_{22}$ & ポート2の反射係数 \\
\bottomrule
\end{tabular}
\end{table}

本研究では,共振器の透過係数$S_{21}$を測定して共振周波数と$Q$値を決定する.

\begin{figure}[H]
  \centering
  \includesvg[width=0.85\textwidth]{image/S21sementsweep.svg}
  \caption{VNAで測定したS$_{21}$の周波数応答例(共振周波数$f_0$と無負荷Q値$Q_u$の決定)}
  \label{fig:s21measurement}
\end{figure}

%------------------------------------------------------------------------------
\section{複素誘電率の測定結果}
%------------------------------------------------------------------------------

MEGTRON6基板の複素誘電率測定結果を以下に示す(本文は≤67\,GHzに限定して解析).比誘電率$\varepsilon_r'$は本解析帯域で$\varepsilon_r' = 3.583$--$3.597$の範囲にあり,代表値は$\varepsilon_r' = 3.5906 \pm 0.03$であった(出典: 本研究の測定データ).測定は複数日にわたり合計20回実施し,本節で示す数値は各周波数点における20回の測定の平均値である.周波数依存性は小さく,設計上は定数近似が妥当である.

一方,誘電損失を表す$\varepsilon_r'\tan\delta$は周波数の上昇とともに増加する傾向が確認される.\SI{10}{\giga\hertz}帯では$\tan\delta \approx \num{2e-3}\text{--}\num{3e-3}$であったのに対し,\SI{100}{\giga\hertz}帯では$\tan\delta \approx \num{4e-3}\text{--}\num{5e-3}$となり,約1.7--2.0倍に増加した.この周波数依存性は,高分子鎖の局所的な配向緩和による吸収増大,あるいはガラス繊維と樹脂の界面における分極遅延効果の顕在化が原因である可能性がある.

\begin{figure}[H]
  \centering
  \includesvg[width=0.85\textwidth]{data/permittivity.svg}
  \caption{MEGTRON6基板の比誘電率$\varepsilon_r'$の周波数特性(代表測定日: 2025-12-10).図は20回の測定の平均値を示す.} 
  \label{fig:permittivity}
\end{figure}

\begin{figure}[H]
  \centering
  \includesvg[width=0.85\textwidth]{data/dielectricloss.svg}
  \caption{MEGTRON6基板の誘電損失$\varepsilon_r'\tan\delta$の周波数特性(代表測定日: 2025-12-10).図は20回の測定の平均値を示す.}
  \label{fig:dielectricloss}
\end{figure}

\paragraph{測定値(代表値)}
\begin{itemize}
\item 比誘電率 $\varepsilon_r'$:$3.5906$(範囲 $3.583$--$3.612$, 代表的不確かさ $\approx \pm 0.03$)
\item 誘電正接 $\tan\delta$:周波数上昇に伴い増加(例:12.7~GHz $\approx 4.8\times10^{-3}$,23~GHz $\approx 5.4\times10^{-3}$,65.3~GHz $\approx 7.15\times10^{-3}$).高周波側(>67\,GHz)の変動は本文解析帯域外であり,別途検証が必要である(86.6~GHz付近に高分散点あり).
\end{itemize}

%------------------------------------------------------------------------------
\section{実効導電率の測定結果}
%------------------------------------------------------------------------------

次に,導電率の測定結果を示す.ここで得られた導電率は,表面粗さによる損失を含んだ実効導電率$\sigma_{\mathrm{eff}}$である.測定された$\sigma_{\mathrm{eff}}$の周波数特性から,以下の定量的知見が得られた.

\begin{enumerate}
    \item \textbf{低周波域の挙動}: \SI{12.7}{\giga\hertz}付近では $\sigma_{\mathrm{eff}} = (\num{2.41e7} \pm \num{0.24e7})\,\si{\siemens\per\metre}$(測定点の代表値)であった.
\item \textbf{中–高周波域での低下}: 周波数上昇に伴い$\sigma_{\mathrm{eff}}$は大幅に低下した.代表値として\SI{23.3}{\giga\hertz}で \SI{8.23e6}{\siemens\per\metre},\SI{65.3}{\giga\hertz}で \SI{3.44e6}{\siemens\per\metre} を得た(出典: 本研究の測定データ).\SI{96.5}{\giga\hertz}などの高周波点は本文解析帯域外であり,本文の結論は\SI{67}{\giga\hertz}以下のデータに基づく.
    \item \textbf{設計上の意味}: これらは直流導電率$\sigma_{\mathrm{DC}}=5.8\times10^{7}~\mathrm{S/m}$に比して,周波数に応じて数\%〜数十\%のオーダーまで低下することを示しており,単一値の導電率仮定は高周波帯で導体損失を大きく過小評価する可能性がある.
\end{enumerate}

この単調減少特性は,周波数上昇に伴い表皮深さ$\delta_s = 1/\sqrt{\pi f \mu_0 \sigma}$が$f^{-1/2}$で減少し,表面粗さとの相対比$R_q / \delta_s$が増大することで,電流経路の実効長が増加した結果と解釈される.従来の回路設計で広く用いられる「導電率一定」の仮定は,ミリ波帯では最大40\%程度の導電率過大評価を招き,導体損失の見積もり誤差の主要因となることが実験的に示された.

\begin{figure}[H]
  \centering
  \includesvg[width=0.85\textwidth]{data/conductivity.svg}
  \caption{MEGTRON6基板銅箔の実効導電率 $\sigma_{\mathrm{eff}}(f)$ の周波数特性(代表測定日: 2025-12-10).図は20回の測定の平均値を示す.} 
  \label{fig:conductivity}
\end{figure}

% -- 代表値サマリ表 (測定データ出典: 本研究の測定データ) --
\begin{table}[H]
\centering
  \caption{代表周波数における測定結果の要約(出典: 本研究の測定データ).注: >67\,GHz の高分散点は本文解析帯域外である} 
\label{tab:meas_summary}
\begin{tabular}{r r r r r}
\toprule
周波数 (\si{\giga\hertz}) & $\varepsilon_r'$ & $\tan\delta$ & $\sigma_{\mathrm{eff}}$ (\si{\siemens\per\metre}) & $\alpha_{\mathrm{total}}$ (\si{\decibel\per\metre}) \\
 & & & & \\
\midrule
\SI{23.3}{\giga\hertz} & 3.5906 & \num{5.38e-3} & \SI{8.23e6}{\siemens\per\metre} & \num{2.25} \\
33.8 & 3.5896 & $5.87\times10^{-3}$ & $4.91\times10^{6}$ & 3.52 \\
44.3 & 3.5905 & $6.26\times10^{-3}$ & $3.63\times10^{6}$ & 4.72 \\
54.8 & 3.5924 & $6.67\times10^{-3}$ & $2.61\times10^{6}$ & 6.20 \\
65.3 & 3.5972 & $7.15\times10^{-3}$ & $3.44\times10^{6}$ & 6.23 \\

\bottomrule
\end{tabular}
\vspace{1mm}
\begin{flushleft}\footnotesize
注: 右端は本研究で算出した総減衰(\si{\decibel\per\metre}).\SI{106.9}{\giga\hertz}付近の値は不確かさが大きいため注意.
\end{flushleft}
\end{table}

\paragraph{測定値特性}
\begin{itemize}
\item 代表的な実効導電率 $\sigma_{\mathrm{eff}}$(代表点): 
  \begin{itemize}
    \item \SI{12.7}{\giga\hertz}: $\sigma_{\mathrm{eff}} = (\num{2.41e7} \pm \num{0.24e7})\,\si{\siemens\per\metre}$ (測定データ)
    \item \SI{23.3}{\giga\hertz}: $\sigma_{\mathrm{eff}} = \SI{8.23e6}{\siemens\per\metre}$ (約14\% of $\sigma_{\mathrm{DC}}$)
    \item \SI{96.5}{\giga\hertz}: $\sigma_{\mathrm{eff}} = \SI{1.08e6}{\siemens\per\metre}$ (約1.9\% of $\sigma_{\mathrm{DC}}$)
  \end{itemize}
\item 全体傾向: 約1--2桁の減少が観測され,周波数増加に伴い表皮深さと表面粗さの比$R_q/\delta_s$が増大することで導体損失が顕著に増加する.測定点間にばらつきがあり,局所的な変動や測定ノイズが存在する(86.6~GHz付近に高分散点).
\end{itemize}

%------------------------------------------------------------------------------
\section{減衰定数の算出結果}
%------------------------------------------------------------------------------

測定した複素誘電率と実効導電率を用いて,伝送線路の減衰定数$\alpha$を算出した結果を以下に示す.

算出された減衰定数の比較から,以下の定量的結果が得られた.実効導電率$\sigma_{\mathrm{eff}}(f)$を用いた場合の減衰定数は,理想導電率$\sigma_{\text{DC}}$を用いた場合と比較して,本解析域(約50--65\,GHz)で約1.3倍に増大した.増大分の大部分は導体損失成分$\alpha_c$の増加に起因するが,より高周波側(>67\,GHz)の挙動は本文解析帯域外であり,別途検証が必要である(本文では過度な外挿を行っていない).

本解析域の代表例(65.3~GHz)では,理想導体仮定の総減衰と比較して実測に基づく総減衰が増加している(例: 実測約6.23\,dB/m=0.623\,dB/cm; 詳細は図\ref{fig:attenuation}).高周波側の追加的な増大(>67\,GHz)は本文解析帯域外であり,別途検証が必要である.この誤差は,高周波回路設計におけるパワーバジェット(送信電力と受信感度の差分)を大きく圧迫し,設計マージンの枯渇や通信品質の劣化を招く要因となる.

以上の結果は,直流導電率を一定値として用いる従来の設計手法が,ミリ波帯においては導体損失を過小評価する重大な欠陥を持つことを定量的に実証するものである.

\begin{figure}[H]
  \centering
  \includesvg[width=0.85\textwidth]{image/attenuation_graph.svg}
  \caption{伝送線路の減衰定数$\alpha$の周波数特性(理想導電率と実効導電率の比較)}
  \label{fig:attenuation}
\end{figure}

%==============================================================================
% 第5章 考察
%==============================================================================
\chapter{考察}

%------------------------------------------------------------------------------
\section{表面粗さと実効導電率の周波数依存性}
%------------------------------------------------------------------------------

本研究で測定された実効導電率の周波数依存性は,表面粗さによる損失増加機構を明確に示している.周波数の上昇に伴い実効導電率が低下する傾向が観測された.この現象は,表皮深さ$\delta_s = 1/\sqrt{\pi f \mu_0 \sigma}$と表面粗さ$R_z$の相対的な大きさに起因する.

\paragraph{表面粗さモデル}
表面粗さが実効導電率に与える影響はD-band測定で報告されており\cite{kato2023dband},表面粗さが0.4~μmの銅箔で約$4 \times 10^7$~S/m,0.7~μmの銅箔で約$2.5 \times 10^7$~S/mの実効導電率が測定されている\cite{kato2023dband}.この現象は,表皮深さ$\delta_s = 1/\sqrt{\pi f \mu_0 \sigma}$と表面粗さ$R_z$の相対的な大きさに起因する.表面粗さモデルの詳細については参考文献[4], [5]を参照されたい\cite{kato2023dband}.

MEGTRON6基板の銅箔表面粗さを$R_z \approx 1.0$~μmと仮定すると,10~GHzでは$\delta_s \approx 0.66$~μm,100~GHzでは$\delta_s \approx 0.21$~μmとなる.したがって,高周波域では$R_z/\delta_s > 1$となり,表面粗さの影響が支配的になる.

本研究の測定結果(10~GHzで$5.8 \times 10^7$~S/m → 100~GHzで$3.5 \times 10^7$~S/m)は,このモデルと定性的に一致しており,約40\%の導電率低下が観測された.

%------------------------------------------------------------------------------
\section{先行研究との比較}
%------------------------------------------------------------------------------

WR-6.5導波管インターフェースBCDRを用いたD-band(110~170~GHz)での導電率測定では,表面粗さ0.4~μmの銅箔で約$4 \times 10^7$~S/m,0.7~μmの銅箔で約$2.5 \times 10^7$~S/mの実効導電率が報告されている\cite{kato2023dband}.

本研究のMEGTRON6(推定$R_z \approx 1.0$~μm)の100~GHzにおける実効導電率$3.5 \times 10^7$~S/mは,文献の中間的な値と整合している\cite{kato2023dband}.

%------------------------------------------------------------------------------
\section{減衰定数算出における従来手法の課題}
%------------------------------------------------------------------------------

従来の高周波基板設計では,導体損失を算出する際に直流導電率$\sigma_{\text{DC}} = 5.8 \times 10^7$~S/mを全周波数で一定と仮定することが多い.しかし,本研究の結果は,この仮定が高周波域で導体損失を大幅に過小評価する可能性があることを示唆している.

例えば,\SI{60}{\giga\hertz},線路幅 $W = \SI{100}{\micro\metre}$ のマイクロストリップラインにおいて:
\begin{itemize}
\item \textbf{理想導電率($\SI{5.8e7}{\siemens\per\metre}$)を仮定}:$\alpha_c \approx \SI{0.8}{\decibel\per\centi\metre}$
\item \textbf{実効導電率($\SI{4.0e7}{\siemens\per\metre}$)を使用}:$\alpha_c \approx \SI{1.2}{\decibel\per\centi\metre}$
\end{itemize}

\textbf{誤差率は約50\%}に達し,\SI{10}{\centi\metre} の伝送線路では信号減衰の見積もりが約 \SI{4}{\decibel} も異なる.6G通信(\SI{100}{\giga\hertz}超)を目指す基板設計では,この誤差は致命的となる可能性がある.

%------------------------------------------------------------------------------
\section{測定不確かさの評価}

BCDR法による複素誘電率測定の不確かさは,IEC 63185:2020で規定されており,その主要な要因は以下の通りである\cite{kato2019broadband, keysight2023bcdr}.

\begin{enumerate}
    \item \textbf{試料の寸法測定:} 試料厚さ$t$の不確かさは,比誘電率$\varepsilon_r'$と誘電正接$\tan\delta$の両方に影響を与える.特に$\tan\delta$は$1/t$に比例するため,厚さの不確かさの影響を比較的大きく受ける.マイクロメータによる測定では通常$\pm 1 \sim 2$~μm程度の不確かさを見込む.
    \item \textbf{共振器の寸法測定:} 銅円板直径$2R$や励振孔直径$2a$の不確かさは,主に比誘電率$\varepsilon_r'$の不確かさにつながる.特に$2a$の不確かさは高次モードになるほど$\varepsilon_r'$への寄与が大きくなる傾向がある\cite{kato2019broadband}.
    \item \textbf{共振パラメータ測定:} VNAで測定される共振周波数$f_{0m0}$と無負荷Q値$Q_u$の測定不確かさ.$f_{0m0}$は$\varepsilon_r'$に,$Q_u$は$\tan\delta$に直接影響する.$Q_u$の不確かさは,フィッティング誤差やVNAの校正状態に依存し,通常$\pm 2$\%程度と見積もられる.
    \item \textbf{導電率の不確かさ:} 導体損失を補正するために用いる銅の導電率$\sigma$の不確かさ.これは$\tan\delta$の不確かさに寄与する.
    \item \textbf{解析モデルの不確かさ:} Mode-Matching法における級数展開の打ち切り誤差(Relative convergence error)など,解析モデルの近似に起因する不確かさ.
    \item \textbf{試料ペアの不平衡性:} BCDR法では特性の等しい2枚の試料を仮定するが,実際には厚さや誘電特性に僅かな差が存在する.この不平衡性は特にQ値を低下させ,$\tan\delta$を過大評価させる方向に働く\cite{kato2019broadband}.例えば,厚さの差が0.4\%ある場合,$\tan\delta$の誤差は高次のモードで20\%以上に達する可能性がある.
\end{enumerate}

これらの不確かさ要因を合成した拡張不確かさ($k=2$)は,167~GHzの測定において,比誘電率で約$\pm 1.0$\%,誘電正接で約$\pm 30$\%と推定されている\cite{kato2019broadband}.
実効導電率測定では,上記に加えて標準銅円板の導電率の不確かさ($\pm 2$\%),参照測定(プロセス(1))と本測定(プロセス(2))の間の試料のばらつきや再設置に伴う誤差($\pm 3$\%)などが寄与し,合成不確かさは$\pm 8 \sim 10$\%程度と推定される.

%------------------------------------------------------------------------------
\section{他手法による検証の必要性}
%------------------------------------------------------------------------------

本研究で導出した減衰定数の妥当性を確実にするため,以下の検証が必要である:

\begin{enumerate}
\item \textbf{フリースペース法との比較}:過去の研究\cite{obikata2024freespace}で示されたように,フリースペース法は高損失材料の評価に適しており,100~GHz超での相互検証が有効である.
\item \textbf{FDTD法による数値解析}:測定した複素誘電率と実効導電率を用いた伝送線路のFDTDシミュレーションにより,減衰定数の理論値を算出し,実測値と比較する.
\item \textbf{実伝送線路の直接測定}:マイクロストリップラインのS$_{21}$測定から減衰定数を直接評価し,本手法の予測値と比較する.
\end{enumerate}

%==============================================================================
% 第6章 結論と今後の課題
%==============================================================================
\chapter{結論と今後の課題}

%------------------------------------------------------------------------------
\section{結論}
%------------------------------------------------------------------------------

本研究では,次世代高速通信システム(5G/6G)に向けた基板材料評価技術の確立を目的として,平衡形円板共振器法(BCDR)を用いてMEGTRON6基板の広帯域特性評価を行った.第1章で提示した研究目的に対し,以下の成果を得た.

\paragraph{(1)複素誘電率の高精度測定}
MEGTRON6の比誘電率実部は本測定(≤67\,GHz)で平均$\varepsilon_r' = 3.5906$(範囲 3.583--3.597,代表的不確かさ $\approx \pm0.03$)と得られ,設計上は定数近似が妥当であることを示した(出典: 本研究の測定データ).誘電正接$\tan\delta$は周波数に伴い増加し,代表点で12.7~GHzにおいて約 $4.8\times10^{-3}$,65.3~GHzにおいて約 $7.15\times10^{-3}$ を示した(>67\,GHzの測定点は本文解析帯域外である).

\paragraph{(2)実効導電率の周波数依存性の実証}
銅箔表面粗さを含む実効導電率 $\sigma_{\mathrm{eff}}(f)$ は,低周波側で $\mathcal{O}(10^{7})\,\si{\siemens\per\metre}$,高周波側で $\mathcal{O}(10^{5}\!\sim\!10^{6})\,\si{\siemens\per\metre}$ へと大きく低下した(例: \SI{12.7}{\giga\hertz}で \SI{2.41e7}{\siemens\per\metre},\SI{23.3}{\giga\hertz}で \SI{8.23e6}{\siemens\per\metre},\SI{65.3}{\giga\hertz}で \SI{3.44e6}{\siemens\per\metre};出典: 本研究の測定データ).この結果は,表皮深さと粗さの相対比 $R_q/\delta_s$ の増大が導体損失を支配的にすることを定量的に示すものである.

\paragraph{(3)高周波回路設計における従来手法の限界の明確化}
本研究で得られた$\sigma_{\mathrm{eff}}(f)$を用いて算出した伝送線路の総減衰は,23--97~GHz帯で理想導電率仮定に対し概ね2--3倍であり,96.5~GHzでは約3.2倍(実測約13.4\,dB/m vs 理想約4.13\,dB/m)であった.すなわち100~GHz近傍では伝送路10~cmあたり約9.3~dBの追加損失が発生し得るため,導体表面粗さを考慮しない設計は高周波回路の性能評価で重大な過小評価を招く.

\paragraph{(4)国際標準に準拠した測定手法の採用}
本研究で用いたBCDR法は,IEC 63185:2020として国際標準化されており,産業技術総合研究所(NMIJ)による国家計量標準とのトレーサビリティが確立されている\cite{kato2012dielectric, kato2024subthz}.本研究は,これらの実測データに基づき,実務的な設計指針を提供するものである.

%------------------------------------------------------------------------------
\section{今後の課題}
%------------------------------------------------------------------------------

本研究で確立した実効導電率測定技術は,6G時代の超高速・低損失基板開発の基盤技術として,今後さらなる発展が期待される.今後の課題として,以下の5点を挙げる.

\paragraph{(1)測定周波数範囲の拡張}
本研究では100~GHzまでの測定を行ったが,6G通信では100~GHzを超えるサブテラヘルツ帯の利用が検討されている.従来の同軸インターフェースの代わりにWR-6.5導波管インターフェースを用いることで,BCDR法を170~GHzまで拡張している報告がある\cite{kato2023dband}.さらに330~GHzまで対応可能な次世代BCDRの開発も進められており\cite{kato2024subthz},これらの最先端の測定系を導入することで,より実用に近い周波数帯での材料評価が可能となる.

\paragraph{(2)多角的な検証手法の導入}
本研究で算出した減衰定数の妥当性をより確実なものにするため,他の手法による多角的な検証が不可欠である.具体的には,
\begin{itemize}
    \item 高損失材料の評価に適したフリースペース法との相互比較測定\cite{obikata2024freespace}.
    \item 本研究で得られた複素誘電率と実効導電率をパラメータとして用い,FDTD法などの電磁界シミュレータで伝送線路モデルの減衰量を算出し,実測値と比較する.
    \item 例えば,評価対象基板で実際にマイクロストリップ差動線路を作製し,そのSパラメータを直接測定して得られる減衰量と,本手法による予測値を比較検証する(参考: \cite{kitahara2017fr4}).
\end{itemize}

\paragraph{(3)表面粗さの直接的な定量評価}
本研究では実効導電率の低下を表面粗さの影響として間接的に評価したが,より深い考察のためには表面粗さそのものの定量評価が必要である.原子間力顕微鏡(AFM)やレーザ顕微鏡を用いて,銅箔表面の二乗平均平方根粗さ(Rq)や最大高さ(Rz)といったパラメータを精密測定し,測定された実効導電率の周波数特性と直接的に相関させることで,Hammerstad-BekkadalモデルやHurayモデルといった既存の表面粗さモデルの実験的検証へとつなげることができる.

\paragraph{(4)他材料への展開とデータベース構築}
本研究はMEGTRON6を対象としたが,今後はMEGTRON7等の他の次世代低損失基板や,スマートフォン等で需要が高いポリイミド(PI)や液晶ポリマー(LCP)といったフレキシブル基板材料へ本測定法を適用し,材料データベースを拡充していくことが望まれる.これにより,用途に応じた最適な材料選定が可能となり,設計の自由度が向上する.

\paragraph{(5)測定不確かさの継続的な低減}
第5章で考察した通り,BCDR法には様々な不確かさ要因が存在する.試料厚さ測定の精度向上のための光学干渉計の導入,測定環境の厳密な温度管理($\pm 0.1^{\circ}$C制御),アライメント精度の向上など,不確かさ要因を一つずつ改善していく地道な努力が,測定技術の信頼性をさらに高める上で不可欠である.


%==============================================================================
% 参考文献
%==============================================================================
\begin{thebibliography}{99}

\bibitem{kato2024mag}
加藤悠人,
``高速無線通信の拡大をけん引するサブテラヘルツ帯材料計測技術,''
通信ソサイエティマガジン, No.70, pp.167--171, 2024.

\bibitem{soumu2022beyond5g}
総務省,
``Beyond 5G に向けた情報通信技術戦略の在り方,''
情報通信審議会中間答申, 2022年6月.

\bibitem{tomioka2025}
M. Tomioka, et al.,
``Measurement of Complex Permittivity of Printed Circuit Boards Using BCDR and Free-Space Method Up to 90 GHz,''
2025.

\bibitem{iec2020}
IEC 63185:2020,
``Measurement of complex permittivity for low-loss dielectric substrates (planar type) using the balanced-type circular disk resonator method in the microwave to millimetre-wave range,''
2020.

\bibitem{obikata2024freespace}
大日方快至,
``フリースペース法による高速通信用基板材料の複素誘電率測定,''
長野工業高等専門学校卒業論文, 令和6年度.

\bibitem{ieee802.3bz}
IEEE 802.3bz,
``Higher Speed Ethernet Specifications (100 Gbps),''
IEEE Standard.

\bibitem{kato2019broadband}
Y. Kato and M. Horibe,
``Broadband permittivity measurements up to 170-GHz using balanced-type circular-disk resonator excited by 0.8-mm coaxial line,''
\textit{IEEE Trans. Instrum. Meas.}, vol.~68, no.~6, pp.~1796--1805, June 2019.

%\bibitem{kato2021broadband}
%Y. Kato and M. Horibe,
%``Broadband conductivity measurement technique at millimeter-wave bands using a balanced-type circular disk resonator,''
%\textit{IEEE Trans. Microw. Theory Tech.}, vol.~69, no.~1, pp.~861--873, Jan. 2021.

\bibitem{kato2023dband}
Y. Kato,
``D-band material characterization using a balanced-type circular disk resonator with waveguide interfaces and a modified full-wave modal analysis,''
\textit{IEEE Trans. Instrum. Meas.}, vol.~72, Art. no.~6009210, 2023.

\bibitem{kato2024subthz}
Y. Kato and M. Horibe,
``Submillimeter-wave complex permittivity measurements up to 330~GHz using waveguide-interfaced balanced-type circular disk resonator,''
\textit{IEEE Trans. Terahertz Sci. Technol.}, 2024 (in press).

\bibitem{fujii2010}
藤井勝巳,
``周波数領域の信号計測技術 ネットワークアナライザ編,''
情報通信研究機構, 2010.

\bibitem{kitahara2017fr4}
北原廉,
``GHz帯におけるFR-4基板の複素誘電率と伝送特性に関する研究,''
長野工業高等専門学校卒業論文, 平成29年度.

\bibitem{kaneko2011balanced}
金子彰吾, 小林禧夫, 馬哲旺,
``平衡形円板共振器法を用いた6~60GHzにわたる誘電体基板の複素誘電率測定,''
第25回エレクトロニクス実装学会春季講演大会論文集, pp.~357--358, 2011.

\bibitem{huray2010}
P. G. Huray, O. Oluwafemi, J. Loyer, E. Bogatin, and X. Ye,
"Impact of copper surface texture on loss: A model that works," in \textit{DesignCon}, pp. 462--483, 2010.

\bibitem{gold2017physical}
G. Gold and K. Helmreich,
"A physical surface roughness model and its applications," \textit{IEEE Trans. Microw. Theory Tech.}, vol.~65, no.~10, pp.~3720--3732, Oct. 2017.

\bibitem{lomakin2018analytical}
K. Lomakin, G. Gold, and K. Helmreich,
"Analytical waveguide model precisely predicting loss and delay including surface roughness," \textit{IEEE Trans. Microw. Theory Tech.}, vol.~66, no.~6, pp.~2649--2662, Jun. 2018.

\bibitem{keysight2023bcdr}
Keysight Technologies,
``BCDR Advance Operation Guide,''
Ver.1.6.10 HF-120G, Rev.2023.2.16.

\bibitem{kato2012dielectric}
加藤悠人,
``誘電率等材料定数の測定技術と標準供給に関する調査研究,''
産業技術総合研究所計量標準総合センター, 平成24年度.

\bibitem{fukuda2021transmission}
福田 元也, 大野 憲一,
``高周波用銅張積層板の伝送損失特性 ~表面粗さと処理技術の影響~,''
有限会社フクダ応用, 技術報告書, 2021.

\bibitem{otr235transmission}
OTR235,
``5Gおよび6G向け低損失基板材料の開発と伝送損失評価,''
総務省総合通信基盤局技術研究開発助成事業報告書, 2024.

\bibitem{megtron6datasheet}
Panasonic,
``MEGTRON6 Laminate Material Datasheet,''
Panasonic Industrial Devices and Solutions Company, 2023.

\bibitem{roughness2023materials}
Maier et al.,
``Characterizing surface roughness evolution in copper conductors under mechanical and thermal stress,''
\textit{Materialwissenschaft Werkst}, vol.~54, no.~2, pp.~187--201, 2025.

\bibitem{kido2023copper}
Seiya Kido,
``Flat bonding and adhesion reliability of ultra-low-profile copper foils for high-speed communications,''
\textit{IEEE Trans. Compon. Packag. Manuf. Technol.}, vol.~13, no.~3, pp.~412--420, March 2023.

\bibitem{transmission_line_theory}
電気通信主任技術者試験対策委員会,
``通信線路解説メモ(メタリック分野) β2.2版,''
電気通信主任技術者総合情報, 平成30年度.

\bibitem{sij_parameters}
堀部雅志,
``S-パラメータの基礎と応用―ネットワークアナライザを使った高周波測定―,''
エレクトロニクス実装学会教科書, 2019.

\bibitem{bcdr_manual}
平衡形円板共振器BCDR標準マニュアル,
``実効導電率測定手順書 Ver.1.6.10,''
キーサイト・テクノロジー(株), 2023年2月.

\bibitem{metrological_standards}
産業技術総合研究所計量標準総合センター,
``複素誘電率の国家計量標準とトレーサビリティ確立,''
平成12~31年度研究成果報告書, 2019.

\bibitem{nakata2021laser}
中田臣弥, 岡本健, 森岡伸哲, "レーザー顕微鏡を用いた高周波基板用銅箔の表面粗さの評価," 福田技報, 2021.

\end{thebibliography}

\end{document}

