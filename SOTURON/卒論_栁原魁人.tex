% !TEX program = lualatex
%==============================================================================
% プリアンブル (Preamble)
%==============================================================================

% ===== ドキュメントクラス =====
\documentclass[
	a4paper,
	11pt
]{ltjsreport}

%------------------------------------------------------------------------------
% パッケージ読み込み
%------------------------------------------------------------------------------

% ===== フォント・言語設定 (LuaLaTeX専用) =====
\usepackage{luatexja-fontspec}
\usepackage{lmodern} % フォントサイズの置き換えを防ぐため

% ===== レイアウト関連 =====
\usepackage[margin=2.5cm]{geometry}
\usepackage{booktabs}
\usepackage{float}
\usepackage{graphicx}

% ===== 数式・物理単位関連 =====
\usepackage{amsmath}
\usepackage{siunitx}
\usepackage{bm} % ベクトルを太字にするため (\bm)

% ===== その他 =====
\usepackage{url} % URLを適切に表示
\usepackage{xurl} % Improved line breaking for long URLs
\urlstyle{same}
\Urlmuskip=0mu plus 2mu
\usepackage[
  hidelinks,
]{hyperref}

% ===== SVG画像埋め込み =====
\usepackage{svg}
\svgsetup{
  inkscapearea=page,
  width=0.8\textwidth
}

%------------------------------------------------------------------------------
% 各種設定
%------------------------------------------------------------------------------

% ===== フォント設定 =====
\setmainfont{Latin Modern Roman}
\setsansfont{Latin Modern Sans}
\setmonofont{Latin Modern Mono}
\setmainjfont[Renderer=HarfBuzz]{Yu Mincho}
\setsansjfont[Renderer=HarfBuzz]{Yu Gothic}
\DeclareMathSizes{11}{11}{7}{5} % 数学フォントサイズの調整

% ===== ドキュメント情報 =====
\title{BCDRを用いた基板の誘電損失と表面粗さによる電気伝導性の測定}
\author{長野工業高等専門学校\\電気電子工学科 栁原 魁人\\(指導教員 春日 貴志)}
\date{\today}

% ===== 数式用カスタムコマンド =====
\providecommand{\dd}{\mathrm{d}} % 微分演算子 d
\newcommand{\mi}{\mathrm{j}} % 虚数単位 j

%==============================================================================
% ドキュメント本体 (Body)
%==============================================================================
\begin{document}
\sloppy % allow more flexible spacing to reduce overfull \hbox warnings

\maketitle

%------------------------------------------------------------------------------
% 英文タイトル
%------------------------------------------------------------------------------
\begin{center}
\large
\textbf{Measurement of Substrate Dielectric Loss and Surface-Roughness-Induced\\
Effective Conductivity Using a Balanced-Type Circular-Disk Resonator (BCDR)}
\end{center}

\vspace{1cm}

%------------------------------------------------------------------------------
% 概要
%------------------------------------------------------------------------------
\begin{abstract}
近年,AIやクラウドサービス,5G/6G通信の発展により,データセンター等では100~GHz帯までの高速伝送が必須となっている.高周波域ではプリント基板の誘電損失と配線銅箔の表皮効果による導体損失が重畳し,信号減衰が大きくなる.従来の測定では主に誘電損失が評価されてきたが,表面粗さを含めた実効導電率の周波数特性については十分に検討されていない.本研究では,平衡形円板共振器法(Balanced Circular Disk Resonator: BCDR)を用いて,低損失基板MEGTRON6の複素誘電率と表面粗さを含む実効導電率を広帯域で測定した.測定結果から伝送線路の減衰定数を算出し,高周波域での表面粗さの影響を評価した.
\end{abstract}

\tableofcontents
\clearpage

%==============================================================================
% 第1章 序論
%==============================================================================
\chapter{序論}

近年,AI・クラウドサービス・5G/6Gなど次世代通信技術の普及に伴い,データセンター等で扱う情報量は爆発的に増加している.これに対応するため,IEEEではサーバーラック内の通信速度として400~Gbpsから800~Gbps,さらには1.6~Tbpsの実現が検討されている\cite{ieee802.3bz}.この超高速伝送を実現するため,信号のベースクロックは56~GHzから112~GHzへと高速化する.デジタル信号の高次高調波を考慮すると,伝送路となるプリント基板やコネクタなどのハードウェアは,約100~GHzから300~GHzに達する広範な周波数帯域に対応する必要がある.

さらに,5Gの次の世代である6G通信では,100~GHzから300~GHzのサブテラヘルツ帯の利用が見込まれており,5Gを上回る超高速大容量通信と,5Gの1/100ともいわれる超低消費電力性能の両立が目標とされている\cite{kato2024subthz}.このような極めて高い周波数帯では,プリント基板における信号減衰が深刻な課題となる.信号減衰の主な要因は,銅配線の表皮効果と表面粗さに起因する導体損失と,基板材料自身の物性による誘電損失の二つである.特に周波数が100~GHzを超えると,表皮深さがサブミクロンオーダまで浅くなるため,これまで以上に導体表面の微細な凹凸(表面粗さ)が実効的な抵抗値を増大させ,導体損失を急激に増加させる.

このため,6G実現に向けた基板開発では,低誘電率・低損失な材料の探求と同時に,導体損失,特に表面粗さの影響を周波数の関数として正確に評価・分離する計測技術が不可欠となっている.これまで,FR-4やMEGTRON6などの基板材料について,平衡形円板共振器法(BCDR)により複素誘電率を測定し,誘電損失を評価する研究は行われてきた.しかし,信号減衰のもう一つの重要な要因である導体損失について,銅箔の表面粗さが実効的な導電率に与える影響を広帯域で定量的に評価した研究は十分ではなかった.

そこで本研究では,ミリ波帯における基板の信号減衰を導体損失と誘電損失の両面から正確に評価することを目的とし,BCDR法を用いて低損失基板MEGTRON6の複素誘電率と,表面粗さの影響を含む実効導電率を広帯域で測定した.さらに,得られた測定値から伝送線路の減衰定数を算出し,高周波領域における表面粗さの影響の重要性を定量的に評価した.

%==============================================================================
% 第2章 誘電体と基板材料について
%==============================================================================
\chapter{誘電体と基板材料について}

%------------------------------------------------------------------------------
\section{プリント基板}
%------------------------------------------------------------------------------

プリント基板は現代の電子機器に欠かせない電子部品である.電子部品を搭載する絶縁基板の上に銅箔回路を形成し,抵抗器やコンデンサなどの受動素子,トランジスタなどの能動素子をマウントすることで,電子回路としての機能を発揮する.

プリント基板は構成形式により分類される.片面基板は基板の一面に導体回路が形成されており,コスト低減が必要な大量生産品に用いられる.両面基板は基板の両面に導体回路が形成され,片面基板より高密度な配線が可能で,高集積化・高機能化に応じて発展してきた.さらに微細化・高密度実装が要求される用途では,基材内部と表層の導体回路が3層以上で構成される多層基板が用いられる.

プリント基板を構成する誘電体中の信号伝達速度$v$は,比誘電率$\varepsilon_r'$と光速$c$の関係で表される:
\begin{equation}
v = \frac{c}{\sqrt{\varepsilon_r'}}
\end{equation}
したがって誘電率が低いほど信号伝達速度が高まり,高速伝送に有利である.5G/6Gで用いられる30~GHz以上の周波数帯では,低誘電率材料の採用が信号の高速伝播を実現する鍵となる.

%------------------------------------------------------------------------------
\section{基板材料について}
%------------------------------------------------------------------------------

高周波基板に用いられる一般的な材料と特性を表~\ref{tab:substrate_materials}に示す.

\begin{table}[H]
\centering
\caption{高周波基板材料の特性}
\label{tab:substrate_materials}
\begin{tabular}{lll}
\toprule
基板材料 & 比誘電率 & 用途・特徴 \\
\midrule
ガラスエポキシ(FR-4) & 4.0~5.2 & 安価で強度高い.多層基板でほぼ全て採用.価格コスト最優先 \\
テフロン & 2.1 & GHz帯で誘電率安定.$\tan\delta$非常に小さい.高価 \\
セラミック & 4.5~5.2 & GHz帯で$\tan\delta$低く安定.放熱性高い.高価 \\
フェノール & 4.6~5.5 & 耐熱性はFR-4に劣るが低価格.生産量最大 \\
ポリイミド & 3.5 & フレキシブル基板用.柔軟性と電気特性両立 \\
\bottomrule
\end{tabular}
\end{table}

kHz~MHz帯では安価なフェノールやFR-4が用いられるが,数十GHzの信号を扱う製品ではテフロンやセラミック等の高周波特性に優れた材料が必要とされる.しかしこれらはFR-4に比べ4倍以上高価であるため,GHz帯においてもコスト面からFR-4が広く採用されている現状にある.

本研究ではMEGTRON6(パナソニック製,低損失高周波基板)を測定対象とした.MEGTRON6は20~GHz以上での使用を想定し,従来のFR-4より大幅に誘電損失を低減している.

%------------------------------------------------------------------------------
\section{誘電体の特性について}
%------------------------------------------------------------------------------

\subsection{誘電体の定義}

誘電体とは,外部電界が印加されたときにエネルギーを蓄積できる材料である.誘電体の分極しやすさを定量的に示すのが誘電率$\varepsilon$である.

\subsection{誘電体の電気的特性}

誘電体の電気的特性は,誘電分極と誘電損失により特徴づけられる.外部電界が印加されると,材料を構成する原子核や陽イオン,電子は元の位置からわずかに変位し,物質内に多数の微小電気双極子が発生する.この現象が誘電分極である.

\subsection{誘電分極}

双極子動きには慣性があるため,印加電圧の周波数が高くなると分極の種類によっては変化に追随できなくなる.このため誘電率が周波数によって変化する「誘電分散」が生じる.

一般に三種類の分極が存在する:

\begin{enumerate}
\item \textbf{電子分極}:原子核の相対的変位により生じる.赤外線領域程度の周波数まで寄与.
\item \textbf{イオン分極}:正負イオンの相対的変位により生じる.赤外線領域程度の周波数まで寄与.
\item \textbf{双極子分極(配向分極)}:永久双極子が外部電界に配向することにより生じる.マイクロ波領域程度の周波数まで寄与.
\end{enumerate}

周波数の上昇に伴い,双極子はマイクロ波領域程度までしか追随できず,イオン分極も赤外線領域程度までの寄与にとどまる.紫外線領域では電子分極も寄与しなくなり,全体として誘電率は周波数とともに低下する傾向を示す.

\subsection{複素誘電率}

交流電界が印加されたとき,誘電体内の双極子は周期的に向きを変えるが,この変化は電界変化に対して時間遅れを伴う.この遅れを位相角$\delta$で表すと,誘電率$\varepsilon$は複素数で表現される:
\begin{equation}
\varepsilon = \varepsilon' - \mi\varepsilon''
\end{equation}
ここで$\varepsilon'$は比誘電率(実部),$\varepsilon''$は誘電損失(虚部)である.

\subsection{誘電損失と誘電正接}

双極子が交流電界から受け取ったエネルギーの一部は,分極に伴う運動により熱エネルギーとして消費される.この損失の大きさを直接表す量が誘電損失$\varepsilon''$である.

複素誘電率の実部と虚部の比を「誘電正接」と呼び,次式で定義される:
\begin{equation}
\tan\delta = \frac{\varepsilon''}{\varepsilon'}
\end{equation}
$\tan\delta$は誘電損失の程度を表す指標として使用され,この値が小さいほど誘電損失が少ない材料と言える.

%------------------------------------------------------------------------------
\section{高周波伝送における信号減衰}
%------------------------------------------------------------------------------

プリント基板上の伝送線路では,二つの主要な損失機構により信号が減衰する:

\paragraph{(1)誘電損失による減衰定数$\alpha_d$}
\begin{equation}
\alpha_d = \frac{\pi f}{c} \cdot \varepsilon_r'\tan\delta \cdot \frac{\sqrt{\varepsilon_{re}}(\varepsilon_{re}-1)}{\varepsilon_{re}(\varepsilon_r' - 1)}
\end{equation}
ここで$f$は周波数,$c$は光速,$\varepsilon_{re}$は有効比誘電率である.誘電損失による減衰は周波数の上昇に伴い増加する.

\paragraph{(2)導体損失による減衰定数$\alpha_c$}
\begin{equation}
\alpha_c = \frac{R_s}{Z_0 W}
\end{equation}
ここで$R_s = \sqrt{\pi f \mu_0 / \sigma}$は表面抵抗,$Z_0$は特性インピーダンス,$W$は線路幅である.

周波数が高くなると表皮深さ$\delta_s = 1/\sqrt{\pi f \mu_0 \sigma}$が浅くなり,電流はより狭い領域に集中する.このとき導体表面の凹凸(表面粗さ)の影響が顕著になり,実効導電率が低下して導体損失が増加する.

全体の減衰定数は両者の和となる:
\begin{equation}
\alpha = \alpha_d + \alpha_c
\end{equation}

%------------------------------------------------------------------------------
\section{本研究に用いた測定用基板}
%------------------------------------------------------------------------------

本研究ではMEGTRON6基板を測定対象とした.MEGTRON6は厚さ0.3~0.5~mmの複数試料を用意し,複素誘電率と実効導電率の周波数依存性を測定した.

%==============================================================================
% 第3章 測定原理および方法
%==============================================================================
\chapter{測定原理および方法}

%------------------------------------------------------------------------------
\section{複素誘電率の測定法ー概要}
%------------------------------------------------------------------------------
高周波領域における材料の複素誘電率を測定する代表的な方法は,評価対象の周波数帯や損失の大きさに応じて,大きく以下の3つに分類される\cite{kato2012dielectric}.

\begin{description}
    \item[集中定数法:] 数kHz~1~GHz程度の比較的低い周波数帯で用いられる.測定試料を電極で挟んでコンデンサを形成し,そのインピーダンス(キャパシタンスとコンダクタンス)から誘電率を算出する.
    \item[反射・伝送法(Sパラメータ法):] 数十MHz~数百GHzの広帯域に対応可能.試料に対し電磁波を照射し,反射・透過係数(Sパラメータ)を測定して誘電率を算出する.高損失材料の評価に適しており,自由空間で行うフリースペース法もこの一種である\cite{obikata2024freespace}.
    \item[共振法:] 1~GHz以上の高周波帯で,特に低損失材料を高精度に評価する場合に用いられる.試料を内部に設置した共振器の共振周波数とQ値の変化から誘電率を求める.本研究で用いるBCDR法も共振法に分類される.
\end{description}

\paragraph{Sパラメータと高周波測定の基礎}
高周波回路の特性評価では、インピーダンスやアドミタンスの代わりにSパラメータ(Scattering parameter, 散乱行列)が広く用いられる.これは,高周波領域では電圧・電流を直接測定することが困難であるのに対し,電力の進行波・反射波はベクトル・ネットワーク・アナライザ(VNA)を用いて安定して測定できるためである.2ポート回路では,ポート1, 2への入射波を$a_1, a_2$,出射波を$b_1, b_2$とすると,Sパラメータは次式で定義される\cite{fujii2010}:
\begin{equation}
\begin{bmatrix} b_1 \\ b_2 \end{bmatrix} = \begin{bmatrix} S_{11} & S_{12} \\ S_{21} & S_{22} \end{bmatrix} \begin{bmatrix} a_1 \\ a_2 \end{bmatrix}
\end{equation}
ここで,$S_{11}$は入力ポートの反射係数,$S_{21}$は透過係数(伝送特性)である.

共振法では,VNAを用いて共振器の透過係数$S_{21}$の周波数応答を測定し,得られた共振ピークから共振周波数$f_0$と,共振の鋭さを表す無負荷Q値$Q_u$を決定する.$Q_u$は共振ピークの中心周波数とそのピークから3~dB低下した点の周波数幅(半値全幅 $\Delta f_{\text{3dB}}$)の比で定義される.
\begin{equation}
Q_u = \frac{f_0}{\Delta f_{\text{3dB}}}
\end{equation}
低損失な材料ほどエネルギーの散逸が少ないためQ値は高くなり,鋭い共振ピークを示す.共振法は,この$f_0$と$Q_u$の精密な測定を通じて,微小な誘電損失を評価することに優れている.

\begin{figure}[H]
  \centering
  \includesvg[width=0.85\textwidth]{image/S21sementsweep.svg}
  \caption{VNAで測定したS$_{21}$の周波数応答例(共振周波数$f_0$と無負荷Q値$Q_u$の決定方法)}
  \label{fig:s21measurement}
\end{figure}

%------------------------------------------------------------------------------
\section{平衡形円板共振器法(BCDR)}
%------------------------------------------------------------------------------

\subsection{BCDR法の原理}

BCDR法は共振法の一種であり、高周波帯での複素誘電率と実効導電率の同時測定に適している。本手法は加藤・堀部らによって開発され、10~GHz~170~GHzの超広帯域測定が実証されている\cite{kato2019broadband,kato2021broadband}。

\paragraph{構造と励振メカニズム}
測定では、測定対象の誘電体基板2枚を準備し、中央に厚さ$t_c = 0.06$~mm、直径$D$の銅円板電極を挟み込み、上下から銅平板電極で圧着する。銅平板の中央には直径$2a$の励振孔が設けられており、同軸線路で電界結合励振を行う。

この構造の最大の特徴は、\textbf{円板中心軸上の対称性}により、TM$_{0m0}$モード($m=1, 2, 3, \ldots$)のみが選択的に励振され、他の不要モードが十分に抑制される点にある。これにより、モード干渉の問題を回避し、高次モードを用いた広帯域測定が可能となる。

\paragraph{Mode-Matching法による厳密解析}
BCDR法では、励振孔まで厳密に考慮したMode-Matching法による電磁界解析が用いられる\cite{kato2019broadband}。解析では、共振器内部を複数の領域に分割し、各領域で固有モード展開を行い、境界条件で結合させる。これにより、共振周波数$f_{0m0}$と比誘電率$\varepsilon_r'$の関係式が導出される:
\begin{equation}
\det H(\varepsilon_r'; f_{0m0}, R+\Delta R, a, M, t) = 0
\end{equation}
ここで、$R$は銅円板半径、$\Delta R$は縁端効果による補正量、$a$は励振孔半径、$M$は励振孔長さ、$t$は誘電体厚さである。

\paragraph{測定パラメータの導出}
ベクトル・ネットワーク・アナライザ(Keysight N5222B)で透過係数$S_{21}$を測定し、以下のパラメータを導出する:

\begin{enumerate}
\item \textbf{比誘電率$\varepsilon_r'$}:共振周波数$f_{0m0}$から上記方程式を数値的に解いて決定
\item \textbf{誘電正接$\tan\delta$}:無負荷Q値$Q_u$から導出。導体損失の影響を分離するため、次式を用いる:
\begin{equation}
\tan\delta = \frac{1}{Q_u}\left(1 - \frac{R_s}{B}\right)(1+A)
\end{equation}
ここで、$R_s = \sqrt{\pi f \mu_0 / \sigma}$は表面抵抗、$A$,$B$は数値解析により求められる補正定数である。
\end{enumerate}

\paragraph{周波数可変性と広帯域測定}
BCDR法の特筆すべき特徴は、銅円板直径$D$を変えることで測定周波数を調整できる点である。同一の誘電体試料に対し、異なる直径の円板(例:$D = 9, 12, 15, 18, 21$~mm)を用いることで、10~100~GHz超の広帯域で複素誘電率を測定できる\cite{kato2023dband,kaneko2011balanced}。

\subsection{測定手順と実効導電率の算出}

実効導電率の測定は,加藤・堀部により提案された2ステッププロセスで実施される\cite{kato2021broadband}.本手法の原理は,共振器のQ値が誘電損失と導体損失の両方に依存することを利用し,誘電損失が既知の試料に対して導体損失を分離評価するものである.

\paragraph{プロセス(1):基板の複素誘電率測定(誘電損失の評価)}

まず,評価対象の基板(例:MEGTRON6)から銅箔を化学エッチング等で完全に除去した誘電体シートを2枚準備する.その間に,導電率が既知の標準銅円板(例:直径$D = 15$~mm,導電率$\sigma_0 = 5.8 \times 10^7$~S/m)を挟み込む.BCDRで複数の共振モード(TM$_{010}$~TM$_{0m0}$)における共振周波数$f_{0m0}^{\text{ref}}$と無負荷Q値$Q_u^{\text{ref}}$を測定する.

Mode-matching解析により比誘電率$\varepsilon_r'$を決定し,さらに次式から誘電正接$\tan\delta$を算出する:
\begin{equation}
\tan\delta = \frac{1}{Q_u^{\text{ref}}}\left(1 - \frac{R_s^{\text{ref}}}{B}\right)(1+A)
\end{equation}
ここで,$R_s^{\text{ref}} = \sqrt{\pi f \mu_0 / \sigma_0}$は標準銅の表面抵抗である.この測定により,評価対象基板の誘電損失($\varepsilon_r'\tan\delta$)が周波数の関数として確定する.

\paragraph{プロセス(2):実基板銅箔の導電率測定(導体損失の評価)}

次に,評価対象の基板上に,プロセス(1)で用いた標準銅円板と全く同じ直径の円形銅箔パターンのみを残した試料を2枚製作する(フォトリソグラフィまたはケミカルエッチング).この2枚の試料を,銅箔面が内側で接触するように重ね合わせ,BCDRで同様に共振周波数$f_{0m0}^{\text{sample}}$と無負荷Q値$Q_u^{\text{sample}}$を測定する.

\paragraph{実効導電率の算出原理}
誘電損失はプロセス(1)で確定しているため,プロセス(2)で観測されるQ値の低下は,実基板銅箔の導体損失の増加に起因する.共振周波数シフト$\Delta f = f_{0m0}^{\text{sample}} - f_{0m0}^{\text{ref}}$と,Q値変化$\Delta(1/Q_u)$を解析することで,実効導電率$\sigma_{\text{eff}}$が次式で導出される:
\begin{equation}
\sigma_{\text{eff}} = \sigma_0 \left[\frac{Q_u^{\text{sample}}}{Q_u^{\text{ref}}} \cdot \left(1 - \frac{\tan\delta}{1/Q_u^{\text{ref}} - R_s^{\text{ref}}/B}\right)\right]^{-2}
\end{equation}
この実効導電率は,表面粗さによる電流経路の延長効果を含んだ値となる.高周波になるほど表皮深さ$\delta_s = 1/\sqrt{\pi f \mu_0 \sigma}$が浅くなり,表面粗さの影響が顕著になるため,$\sigma_{\text{eff}}$は周波数依存性を示す.

\paragraph{測定の信頼性確保}
実効導電率測定の信頼性を確保するためには,特に以下の点に細心の注意を払う必要がある\cite{bcdr-manual, bcdr-manual-cond}.
\begin{enumerate}
    \item \textbf{試料の均一性:} プロセス(1)と(2)で使用する誘電体基板は,同一ロットから採取する.銅箔を剥がした基板と銅箔付き基板の厚さや誘電特性が異なると,Q値の変化が正しく導体損失の差を反映しなくなる.
    \item \textbf{精密なアライメント:} 円形銅箔パターンの中心と共振器の励振孔の中心を高精度に位置合わせすることが極めて重要である.キーサイト社のマニュアルによれば,偏心誤差は50~μm未満に抑える必要がある\cite{bcdr-manual}.ズレが大きいと不要モードが励起され,TM$_{0m0}$モードの共振特性に歪みを生じさせ,Q値の測定誤差,ひいては$\tan\delta$や導電率の誤差を増大させる.
    \item \textbf{適切な加圧とエアギャップの排除:} 上下電極で試料を圧着する際のトルク管理が重要である.トルクが不十分だと試料と電極間にエアギャップが残り,実効的な比誘電率が低く測定される.逆に過度なトルクは試料や円板電極を損傷させる.共振周波数がトルクを増しても変化しなくなる点(エアギャップが排除された点)を連続掃引でモニタしながら,最適な締め付けトルクを決定する必要がある\cite{bcdr-manual}.
    \item \textbf{円板電極の厚さと直径の精密測定:} 標準銅円板と実基板銅箔の厚さ・直径をマイクロメータや画像測定器で精密に測定($\pm 1 \sim 10$~μmオーダ)し,解析に入力する必要がある.これらの寸法誤差は,特に比誘電率の不確かさの主要因となる.
\end{enumerate}

%------------------------------------------------------------------------------
\section{測定装置および条件}
%------------------------------------------------------------------------------

\begin{itemize}
\item \textbf{ベクトル・ネットワーク・アナライザ}:Keysight N5222B(10~MHz~100~GHz)
\item \textbf{平衡形円板共振器}:BCDR Advance(Ver.1.6.10 HF-120G)
\item \textbf{測定周波数範囲}:10~100~GHz
\item \textbf{標準銅円板電極}:直径$\phi$15~mm,厚さ0.06~mm
\item \textbf{誘電体基板試料}:MEGTRON6,厚さ0.3~0.5~mm,複数枚用意
\end{itemize}

測定時には,テストポート先端で機械式校正キット(OPEN, SHORT, LOAD, THRU)あるいは電子式校正キット(ECal)を用いてフル2ポートキャリブレーションを実施し,マッチング特性を補正し,共振ピーク特性を正確に求めた.

伝送線路の減衰定数$\alpha$は誘電損失$\alpha_d$と導体損失$\alpha_c$の和で表される:
\begin{align}
\alpha &= \alpha_d + \alpha_c \\
\alpha_d &= \frac{\pi f}{c} \cdot \varepsilon_r'\tan\delta \cdot \frac{\sqrt{\varepsilon_{re}}(\varepsilon_{re}-1)}{\varepsilon_{re}(\varepsilon_r' - 1)} \\
\alpha_c &= \frac{R_s}{Z_0 W}
\end{align}
ここで,表面抵抗$R_s = \sqrt{\pi f \mu_0 / \sigma}$,特性インピーダンス$Z_0$,線路幅$W$である.

%==============================================================================
% 第4章 測定結果
%==============================================================================
\chapter{測定結果}

%------------------------------------------------------------------------------
\section{複素誘電率の測定結果}
%------------------------------------------------------------------------------

MEGTRON6基板の複素誘電率測定結果を以下に示す.実部に相当する比誘電率$\varepsilon_r'$は測定周波数帯でほぼ一定であった.これに対し,誘電損失$\varepsilon_r'\tan\delta$は周波数の上昇とともに増加する傾向が確認される.

この周波数依存性は,高周波側での誘電分散に起因し,双極子分極がマイクロ波領域の限界に達することを示唆している.

\begin{figure}[H]
  \centering
  \includesvg[width=0.85\textwidth]{image/permittivity.svg}
  \caption{MEGTRON6基板の比誘電率$\varepsilon_r'$の周波数特性}
  \label{fig:permittivity}
\end{figure}

\begin{figure}[H]
  \centering
  \includesvg[width=0.85\textwidth]{image/dielectricloss.svg}
  \caption{MEGTRON6基板の誘電損失$\varepsilon_r'\tan\delta$の周波数特性}
  \label{fig:dielectricloss}
\end{figure}

\paragraph{測定値(代表値)}
\begin{itemize}
\item 比誘電率$\varepsilon_r'$:3.2~3.3(ほぼ周波数不変)
\item 誘電正接$\tan\delta$:周波数に伴い増加,高周波で$2\sim5\times10^{-3}$程度
\end{itemize}

%------------------------------------------------------------------------------
\section{実効導電率の測定結果}
%------------------------------------------------------------------------------

次に,導電率の測定結果を示す.ここで得られた導電率は,表面粗さによる損失を含んだ実効導電率である.周波数が高くなるほど実効導電率が低下している.

理由として,周波数の上昇に伴い表皮深さが浅くなり($\delta_s \propto 1/\sqrt{f}$),電流が導体表面の凹凸(表面粗さ)の影響を強く受けるためであると考えられる.

\begin{figure}[H]
  \centering
  \includesvg[width=0.85\textwidth]{image/S21sementsweep.svg}
  \caption{VNAで測定したS$_{21}$の周波数応答例(共振周波数とQ値の測定)}
  \label{fig:s21measurement}
\end{figure}

\begin{figure}[H]
  \centering
  \includesvg[width=0.85\textwidth]{image/conductivity.svg}
  \caption{MEGTRON6基板銅箔の実効導電率$\sigma_{\mathrm{eff}}$の周波数特性}
  \label{fig:conductivity}
\end{figure}

\paragraph{測定値特性}
\begin{itemize}
\item 低周波域(10~GHz):実効導電率~$5.8\times10^7$~S/m
\item 高周波域(100~GHz):実効導電率~$3.5\times10^7$~S/m
\item 周波数上昇に伴う低下率:約40~50\%
\end{itemize}

%------------------------------------------------------------------------------
\section{減衰定数の算出結果}
%------------------------------------------------------------------------------

測定した複素誘電率と実効導電率を用いて,伝送線路の減衰定数$\alpha$を算出した結果を以下に示す.

実効導電率を用いた場合の導体損失は,理想導電率($\sigma = 5.8\times10^7$~S/m)を仮定した場合に比べ,高周波域で顕著に増大することを確認した.この結果は,直流導電率に基づく単一値モデル等の従来手法が,高周波領域における導体損失の振る舞いを必ずしも十分に表現し得ない可能性を示唆する.

\begin{figure}[H]
  \centering
  \includesvg[width=0.85\textwidth]{image/attenuation_graph.svg}
  \caption{伝送線路の減衰定数$\alpha$の周波数特性(理想導電率と実効導電率の比較)}
  \label{fig:attenuation}
\end{figure}

%==============================================================================
% 第5章 考察
%==============================================================================
\chapter{考察}

%------------------------------------------------------------------------------
\section{表面粗さと実効導電率の周波数依存性}
%------------------------------------------------------------------------------

本研究で測定された実効導電率の周波数依存性は、表面粗さによる損失増加機構を明確に示している。周波数の上昇に伴い実効導電率が低下する傾向が観測された。この現象は、表皮深さ$\delta_s = 1/\sqrt{\pi f \mu_0 \sigma}$と表面粗さ$R_z$の相対的な大きさに起因する。

\paragraph{表面粗さモデル}
GoldとHelmreichのモデル\cite{gold2017physical}によれば、表面粗さによる実効導電率の低下は次式で近似される:
\begin{equation}
\sigma_{\text{eff}} = \sigma_0 \left[1 + \left(\frac{R_z}{\delta_s}\right)^2\right]^{-1}
\end{equation}

MEGTRON6基板の銅箔表面粗さを$R_z \approx 1.0$~μmと仮定すると、10~GHzでは$\delta_s \approx 0.66$~μm、100~GHzでは$\delta_s \approx 0.21$~μmとなる。したがって、高周波域では$R_z/\delta_s > 1$となり、表面粗さの影響が支配的になる。

本研究の測定結果(10~GHzで$5.8 \times 10^7$~S/m → 100~GHzで$3.5 \times 10^7$~S/m)は、このモデルと定性的に一致しており、約40\%の導電率低下が観測された。

%------------------------------------------------------------------------------
\section{先行研究との比較}
%------------------------------------------------------------------------------

加藤らの研究\cite{kato2021broadband,kato2023dband}では、WR-6.5導波管インターフェースBCDRを用いてD-band(110~170~GHz)での導電率測定が報告されている。彼らは表面粗さ0.4~μmの銅箔で約$4 \times 10^7$~S/m、0.7~μmの銅箔で約$2.5 \times 10^7$~S/mの実効導電率を報告しており、表面粗さが大きいほど導電率低下が顕著であることを実証している。

本研究のMEGTRON6(推定$R_z \approx 1.0$~μm)の100~GHzにおける実効導電率$3.5 \times 10^7$~S/mは、加藤らの中間的な値と整合しており、測定手法の妥当性を支持している。

%------------------------------------------------------------------------------
\section{減衰定数算出における従来手法の課題}
%------------------------------------------------------------------------------

従来の高周波基板設計では、導体損失を算出する際に直流導電率$\sigma_{\text{DC}} = 5.8 \times 10^7$~S/mを全周波数で一定と仮定することが多い。しかし、本研究の結果は、この仮定が高周波域で導体損失を大幅に過小評価することを示唆している。

例えば、60~GHz、線路幅$W = 100$~μmのマイクロストリップラインにおいて:
\begin{itemize}
\item \textbf{理想導電率($5.8 \times 10^7$~S/m)を仮定}:$\alpha_c \approx 0.8$~dB/cm
\item \textbf{実効導電率($4.0 \times 10^7$~S/m)を使用}:$\alpha_c \approx 1.2$~dB/cm
\end{itemize}

\textbf{誤差率は約50\%}に達し、10~cmの伝送線路では信号減衰の見積もりが4~dBも異なる。6G通信(100~GHz超)を目指す基板設計では、この誤差は致命的となる可能性がある。

%------------------------------------------------------------------------------
\section{測定不確かさの評価}

BCDR法による複素誘電率測定の不確かさは,IEC 63185:2020で規定されており,その主要な要因は以下の通りである\cite{kato2019broadband, bcdr-manual}.

\begin{enumerate}
    \item \textbf{試料の寸法測定:} 試料厚さ$t$の不確かさは,比誘電率$\varepsilon_r'$と誘電正接$\tan\delta$の両方に影響を与える.特に$\tan\delta$は$1/t$に比例するため,厚さの不確かさの影響を比較的大きく受ける.マイクロメータによる測定では通常$\pm 1 \sim 2$~μm程度の不確かさを見込む.
    \item \textbf{共振器の寸法測定:} 銅円板直径$2R$や励振孔直径$2a$の不確かさは,主に比誘電率$\varepsilon_r'$の不確かさにつながる.特に$2a$の不確かさは高次モードになるほど$\varepsilon_r'$への寄与が大きくなる傾向がある\cite{kato2019broadband}.
    \item \textbf{共振パラメータ測定:} VNAで測定される共振周波数$f_{0m0}$と無負荷Q値$Q_u$の測定不確かさ.$f_{0m0}$は$\varepsilon_r'$に,$Q_u$は$\tan\delta$に直接影響する.$Q_u$の不確かさは,フィッティング誤差やVNAの校正状態に依存し,通常$\pm 2$\%程度と見積もられる.
    \item \textbf{導電率の不確かさ:} 導体損失を補正するために用いる銅の導電率$\sigma$の不確かさ.これは$\tan\delta$の不確かさに寄与する.
    \item \textbf{解析モデルの不確かさ:} Mode-Matching法における級数展開の打ち切り誤差(Relative convergence error)など,解析モデルの近似に起因する不確かさ.
    \item \textbf{試料ペアの不平衡性:} BCDR法では特性の等しい2枚の試料を仮定するが,実際には厚さや誘電特性に僅かな差が存在する.この不平衡性は特にQ値を低下させ,$\tan\delta$を過大評価させる方向に働く\cite{kato2019broadband}.例えば,厚さの差が0.4\%ある場合,$\tan\delta$の誤差は高次のモードで20\%以上に達する可能性がある.
\end{enumerate}

Katoらの報告によれば,これらの不確かさ要因を合成した拡張不確かさ($k=2$)は,167~GHzの測定において,比誘電率で約$\pm 1.0$\%,誘電正接で約$\pm 30$\%と推定されている\cite{kato2019broadband}.
実効導電率測定では,上記に加えて標準銅円板の導電率の不確かさ($\pm 2$\%),参照測定(プロセス(1))と本測定(プロセス(2))の間の試料のばらつきや再設置に伴う誤差($\pm 3$\%)などが寄与し,合成不確かさは$\pm 8 \sim 10$\%程度と推定される.

%------------------------------------------------------------------------------
\section{他手法による検証の必要性}
%------------------------------------------------------------------------------

本研究で導出した減衰定数の妥当性を確実にするため、以下の検証が必要である:

\begin{enumerate}
\item \textbf{フリースペース法との比較}:大日方らの研究\cite{obikata2024freespace}で示されたように、フリースペース法は高損失材料の測定に適しており、100~GHz超での相互検証が有効である。
\item \textbf{FDTD法による数値解析}:測定した複素誘電率と実効導電率を用いた伝送線路のFDTDシミュレーションにより、減衰定数の理論値を算出し、実測値と比較する。
\item \textbf{実伝送線路の直接測定}:マイクロストリップラインのS$_{21}$測定から減衰定数を直接評価し、本手法の予測値と比較する。
\end{enumerate}

%==============================================================================
% 第6章 結論と今後の課題
%==============================================================================
\chapter{結論と今後の課題}

%------------------------------------------------------------------------------
\section{結論}
%------------------------------------------------------------------------------

本研究では,平衡形円板共振器法(BCDR)を用いて,高速伝送用基板材料MEGTRON6の複素誘電率と,導体表面粗さを含む実効導電率を10~100~GHzの広帯域で測定した.主な成果は以下の通りである.

\paragraph{(1)複素誘電率の周波数特性の解明}
MEGTRON6の複素誘電率を測定した結果,比誘電率$\varepsilon_r'$は測定周波数帯で約3.2とほぼ一定であった.一方,誘電正接$\tan\delta$は周波数の上昇に伴い緩やかに増加する傾向を示した.

\paragraph{(2)実効導電率の周波数依存性の実証}
表面粗さを持つ銅箔の実効導電率$\sigma_{\text{eff}}$を測定し,10~GHzの約$5.8 \times 10^7$~S/mから100~GHzでは約$3.5 \times 10^7$~S/mへと約40\%低下することを実験的に示した.これは,周波数が高くなるにつれて表皮効果が顕著になり,電流が導体表面の微細な凹凸の影響を強く受けるためであり,表面粗さが導体損失に与える影響を定量的に実証したものである.

\paragraph{(3)高周波基板設計への実用的知見の提示}
測定で得られた周波数依存性のある実効導電率を用いて伝送線路の減衰定数を算出したところ,理想導電率を仮定した従来の手法に比べ,60~GHz以上の高周波領域で導体損失を約50\%も過小評価する危険性があることを明らかにした.この結果は,6G通信などで要求される100~GHz超の伝送路設計において,導体の表面粗さを考慮した実効導電率の周波数特性の評価が不可欠であることを示唆している.

\paragraph{(4)国際標準に準拠した測定手法の採用}
本研究で用いたBCDR法は,IEC 63185:2020として国際標準化されており,産業技術総合研究所(NMIJ)による国家計量標準とのトレーサビリティが確立されている\cite{kato2012dielectric, kato2024subthz}.測定手法の信頼性が国際的に保証されている点は,本研究成果の産業応用における重要な基盤となる.

%------------------------------------------------------------------------------
\section{今後の課題}
%------------------------------------------------------------------------------

本研究で確立した実効導電率測定技術は,6G時代の超高速・低損失基板開発の基盤技術として,今後さらなる発展が期待される.今後の課題として,以下の5点を挙げる.

\paragraph{(1)測定周波数範囲の拡張}
本研究では100~GHzまでの測定を行ったが,6G通信では100~GHzを超えるサブテラヘルツ帯の利用が検討されている.加藤らによって,従来の同軸インターフェースの代わりにWR-6.5導波管インターフェースを用いることで,BCDR法を170~GHzまで拡張した研究が報告されている\cite{kato2023dband}.さらに330~GHzまで対応可能な次世代BCDRの開発も進められており\cite{kato2024subthz},これらの最先端の測定系を導入することで,より実用に近い周波数帯での材料評価が可能となる.

\paragraph{(2)多角的な検証手法の導入}
本研究で算出した減衰定数の妥当性をより確実なものにするため,他の手法による多角的な検証が不可欠である.具体的には,
\begin{itemize}
    \item 高損失材料の評価に適したフリースペース法との相互比較測定\cite{obikata2024freespace}.
    \item 本研究で得られた複素誘電率と実効導電率をパラメータとして用い,FDTD法などの電磁界シミュレータで伝送線路モデルの減衰量を算出し,実測値と比較する.
    \item 北原らの研究\cite{kitahara2017fr4}のように,評価対象基板で実際にマイクロストリップ差動線路を作製し,そのSパラメータを直接測定して得られる減衰量と,本手法による予測値を比較検証する.
\end{itemize}

\paragraph{(3)表面粗さの直接的な定量評価}
本研究では実効導電率の低下を表面粗さの影響として間接的に評価したが,より深い考察のためには表面粗さそのものの定量評価が必要である.原子間力顕微鏡(AFM)やレーザ顕微鏡を用いて,銅箔表面の二乗平均平方根粗さ(Rq)や最大高さ(Rz)といったパラメータを精密測定し,測定された実効導電率の周波数特性と直接的に相関させることで,Hammerstad-BekkadalモデルやHurayモデルといった既存の表面粗さモデルの実験的検証へとつなげることができる.

\paragraph{(4)他材料への展開とデータベース構築}
本研究はMEGTRON6を対象としたが,今後はMEGTRON7等の他の次世代低損失基板や,スマートフォン等で需要が高いポリイミド(PI)や液晶ポリマー(LCP)といったフレキシブル基板材料へ本測定法を適用し,材料データベースを拡充していくことが望まれる.これにより,用途に応じた最適な材料選定が可能となり,設計の自由度が向上する.

\paragraph{(5)測定不確かさの継続的な低減}
第5章で考察した通り,BCDR法には様々な不確かさ要因が存在する.試料厚さ測定の精度向上のための光学干渉計の導入,測定環境の厳密な温度管理($\pm 0.1^{\circ}$C制御),アライメント精度の向上など,不確かさ要因を一つずつ改善していく地道な努力が,測定技術の信頼性をさらに高める上で不可欠である.

%==============================================================================
% 参考文献
%==============================================================================
\begin{thebibliography}{99}

\bibitem{kato2024subthz}
加藤悠人,
``高速無線通信の拡大をけん引するサブテラヘルツ帯材料計測技術,''
通信ソサイエティマガジン, No.70, pp.167--171, 2024.

\bibitem{soumu2022beyond5g}
総務省,
``Beyond 5G に向けた情報通信技術戦略の在り方,''
情報通信審議会中間答申, 2022年6月.

\bibitem{tomioka2025}
M. Tomioka, et al.,
``Measurement of Complex Permittivity of Printed Circuit Boards Using BCDR and Free-Space Method Up to 90 GHz,''
2025.

\bibitem{iec2020}
IEC 63185:2020,
``Measurement of complex permittivity for low-loss dielectric substrates (planar type) using the balanced-type circular disk resonator method in the microwave to millimetre-wave range,''
2020.

\bibitem{obikata2024freespace}
大日方快至,
``フリースペース法による高速通信用基板材料の複素誘電率測定,''
長野工業高等専門学校卒業論文, 令和6年度.

\bibitem{ieee802.3bz}
IEEE 802.3bz,
``Higher Speed Ethernet Specifications (100 Gbps),''
IEEE Standard.

\bibitem{kato2019broadband}
Y. Kato and M. Horibe,
``Broadband permittivity measurements up to 170-GHz using balanced-type circular-disk resonator excited by 0.8-mm coaxial line,''
\textit{IEEE Trans. Instrum. Meas.}, vol.~68, no.~6, pp.~1796--1805, June 2019.

\bibitem{kato2021broadband}
Y. Kato and M. Horibe,
``Broadband conductivity measurement technique at millimeter-wave bands using a balanced-type circular disk resonator,''
\textit{IEEE Trans. Microw. Theory Tech.}, vol.~69, no.~1, pp.~861--873, Jan. 2021.

\bibitem{kato2023dband}
Y. Kato,
``D-band material characterization using a balanced-type circular disk resonator with waveguide interfaces and a modified full-wave modal analysis,''
\textit{IEEE Trans. Instrum. Meas.}, vol.~72, Art. no.~6009210, 2023.

\bibitem{fujii2010}
藤井勝巳,
``周波数領域の信号計測技術 ネットワークアナライザ編,''
情報通信研究機構, 2010.

\bibitem{kitahara2017fr4}
北原廉,
``GHz帯におけるFR-4基板の複素誘電率と伝送特性に関する研究,''
長野工業高等専門学校卒業論文, 平成29年度.

\bibitem{kaneko2011balanced}
金子彰吾, 小林禧夫, 馬哲旺,
``平衡形円板共振器法を用いた6~60GHzにわたる誘電体基板の複素誘電率測定,''
第25回エレクトロニクス実装学会春季講演大会論文集, pp.~357--358, 2011.

\bibitem{gold2017physical}
G. Gold and K. Helmreich,
``A physical surface roughness model and its applications,''
\textit{IEEE Trans. Microw. Theory Tech.}, vol.~65, no.~10, pp.~3720--3732, Oct. 2017.

\bibitem{keysight2023bcdr}
Keysight Technologies,
``BCDR Advance Operation Guide,''
Ver.1.6.10 HF-120G, Rev.2023.2.16.

\bibitem{kato2012dielectric}
加藤悠人,
``誘電率等材料定数の測定技術と標準供給に関する調査研究,''
産業技術総合研究所計量標準総合センター, 平成24年度.

\bibitem{bcdr-manual}
キーサイト・テクノロジー,
``ミリ波帯平衡型円板共振器法操作手順書,''
Ver.1.6.10 HF-120G, Rev.2023.2.16.

\bibitem{bcdr-manual-cond}
キーサイト・テクノロジー,
``平衡型円板共振器BCDRによる導電率測定手順書,''
Ver.1.6.10 HF-120G, Rev.2023.2.16.

\end{thebibliography}

\end{document}
