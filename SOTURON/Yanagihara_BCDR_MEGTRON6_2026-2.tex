% !TEX program = lualatex
%==============================================================================
% プリアンブル (Preamble)
%==============================================================================

% ===== ドキュメントクラス =====
\documentclass[
  a4paper,
  11pt
]{ltjsreport}

%------------------------------------------------------------------------------
% パッケージ読み込み
%------------------------------------------------------------------------------

% ===== フォント・言語設定 (LuaLaTeX専用) =====
\usepackage{luatexja-fontspec}
\usepackage{lmodern} % フォントサイズの置き換えを防ぐため

% ===== レイアウト関連 =====
\usepackage[margin=2.5cm]{geometry}
\usepackage{booktabs}
\usepackage{float}
\usepackage[section]{placeins}
\usepackage{graphicx}
\setcounter{topnumber}{2}
\setcounter{bottomnumber}{1}
\setcounter{totalnumber}{3}
\renewcommand{\topfraction}{0.9}
\renewcommand{\bottomfraction}{0.8}
\renewcommand{\textfraction}{0.1}
\renewcommand{\floatpagefraction}{0.9}
\setlength{\textfloatsep}{10pt plus 2pt minus 2pt}
\setlength{\floatsep}{8pt plus 2pt minus 2pt}

% ===== 数式・物理単位関連 =====
\usepackage{amsmath}
\usepackage{siunitx}
\sisetup{
  detect-all = true,
  per-mode = symbol,
  range-phrase = {--},
  exponent-product = \cdot,
  separate-uncertainty = true
}
\usepackage{bm} % ベクトルを太字にするため (\bm)

% ===== その他 =====
\usepackage{url} % URLを適切に表示
\usepackage{xurl} % Improved line breaking for long URLs
\urlstyle{same}
\Urlmuskip=0mu plus 2mu
\usepackage[super,square]{natbib} % 引用を上付き角括弧に

% ===== SVG画像埋め込み =====
\usepackage{svg}
\svgsetup{
  inkscapearea=page,
  width=0.8\textwidth
}

% ===== ハイパーリンク設定(最後に読み込む) =====
\usepackage[hidelinks]{hyperref}

%------------------------------------------------------------------------------
% 各種設定
%------------------------------------------------------------------------------

% ===== フォント設定 =====
\setmainfont{Latin Modern Roman}
\setsansfont{Latin Modern Sans}
\setmonofont{Latin Modern Mono}
\setmainjfont[Renderer=HarfBuzz]{Yu Mincho}
\setsansjfont[Renderer=HarfBuzz]{Yu Gothic}
\DeclareMathSizes{11}{11}{7}{5} % 数学フォントサイズの調整

% ===== ドキュメント情報 =====
\title{BCDRを用いた基板の誘電損失と表面粗さによる\\
電気伝導性の測定\\[0.3cm]
{\normalsize Measurement of Substrate Dielectric Loss and Surface-Roughness-Induced\\
Effective Conductivity Using a Balanced-Type Circular-Disk Resonator (BCDR)}}
\author{長野工業高等専門学校\\電気電子工学科 栁原 魁人\\(指導教員 春日 貴志)}
\date{\today}

% ===== 数式用カスタムコマンド =====
\providecommand{\dd}{\mathrm{d}} % 微分演算子 d
\newcommand{\mi}{\dot{\jmath}} % 虚数単位 j

%==============================================================================
% ドキュメント本体 (Body)
%==============================================================================
\begin{document}
\sloppy

\maketitle

\tableofcontents
\clearpage

%==============================================================================
% 第1章 序論
%==============================================================================
\chapter{序論}

\section{背景}
近年,人工知能(AI)やクラウドサービスの急速な普及により,データ通信量が爆発的に増大している.これに対応するため,5Gや次世代の6G通信では,\SI{100}{\giga\hertz}帯という非常に高い周波数が使われようとしている.
しかし,電気信号は周波数が高くなればなるほど,基板を通る際に信号が減衰する.この減衰を正確に見積もることが,高性能な通信機器を作る上で不可欠となっている\cite{soumu2022beyond5g}.

\section{本研究の課題と目的}
基板での信号減衰には,大きく分けて2つの原因がある.
\begin{enumerate}
    \item \textbf{誘電損失}:基板の材料そのものが電気を熱に変えてしまう損失.
    \item \textbf{導体損失}:配線(銅箔)を流れる電気が抵抗によって熱になる損失.
\end{enumerate}

これまで本研究室では,誘電損失の評価が主に行われ,その評価手法は確立されてきた\cite{tomioka2025}.しかし,周波数が高くなると,電流は導体の表面の浅い部分に集中する.この現象は表皮効果と呼ばれる.高周波ほど電流が表面の浅い部分により強く集中し,配線表面のわずかな粗さ(凸凹)が損失を増大させることが知られている\cite{fukuda2021transmission}.銅箔は基板への接着性を高めるために粗化処理されている.

本研究の目的は,この「表面粗さを含めた導体損失」を実測し,実際の信号減衰にどれくらい影響するかを明らかにすることである.

%==============================================================================
% 第2章 測定原理および方法
%==============================================================================
\chapter{測定原理および方法}

\section{BCDR法の概要}
本研究では「平衡形円板共振器法(BCDR法)」という測定方法を用いた.この方法は,2枚の基板で銅の円板を挟み込み,特定の周波数で共振させることで,材料の性質を高精度に測ることができる手法である\cite{kato2019broadband}.

この手法の最大のメリットは,以下の手順を踏むことで,「材料の損失(誘電損失)」と「表面の損失(導体損失)」を切り分けて評価できる点にある.

\begin{figure}[htbp]
  \centering
  % 元のファイルにある図を使用(構成図などが適しているが、ここではS21の例を表示)
  \includesvg[width=0.85\textwidth]{image/セグメントスイープ3.svg}
  \caption{BCDR法による共振波形の測定例(S$_{21}$)}
  \label{fig:segment_sweep}
\end{figure}

\section{測定手順}

測定は以下の2ステップで行った.

\subsection*{ステップ1:基準となる誘電率の測定}
まず,基板(MEGTRON6)の銅箔をすべて溶かして取り除き,代わりに表面が平滑な標準銅円板(直径\SI{15}{\milli\metre})を挟んで測定を行う.
標準銅板の性質は既知であるため,この測定で基板そのものの「比誘電率($\varepsilon_r'$)」と「誘電正接($\tan\delta$)」を正確に決めることができる.

\subsection*{ステップ2:実効導電率の測定}
次に,基板にもともと付いている銅箔を円形(直径\SI{15}{\milli\metre})に残したサンプルを作成し,同様に測定を行う.
ステップ1で基板の性質はわかっているので,ステップ1とステップ2の結果の「差」を見れば,元々の銅箔が持っている表面粗さの影響(実効導電率 $\sigma_{\mathrm{eff}}$)を逆算することができる.

%==============================================================================
% 第3章 測定結果
%==============================================================================
\chapter{測定結果}

周波数\SI{10}{\giga\hertz}から\SI{67}{\giga\hertz}の範囲で測定を行った結果を示す.

\section{複素誘電率(基板の性質)}
基板材料(MEGTRON6)の比誘電率$\varepsilon_r'$は,周波数が変わっても約3.59でほぼ一定であった(図\ref{fig:permittivity}).一方で,損失の割合を示す誘電損失 $\varepsilon_r'\tan\delta$ は,周波数が高くなるにつれて増加する傾向が見られた(図\ref{fig:dielectricloss}).

\begin{figure}[htbp]
  \centering
  \includesvg[width=0.85\textwidth]{data/permittivity.svg}
  \caption{MEGTRON6の比誘電率 $\varepsilon_r'$(mean, n=20)}
  \label{fig:permittivity}
\end{figure}

\begin{figure}[htbp]
  \centering
  \includesvg[width=0.70\textwidth]{data/dielectricloss.svg}
  \caption{MEGTRON6の誘電損失 $\varepsilon_r'\tan\delta$(mean, n=20)}
  \label{fig:dielectricloss}
\end{figure}

\section{実効導電率(銅箔の性質)}
本研究で最も重要な結果が,導電率の変化である(図\ref{fig:conductivity}).
本来,銅の導電率は一定(約 \SI{5.8e7}{\siemens\per\metre})とされることが多い.しかし測定の結果,表面粗さを含めた「実効導電率 $\sigma_{\mathrm{eff}}$」は,周波数が高くなるにつれて明らかに低下した.

例えば,低い周波数では高い導電率を保っていたが,\SI{65}{\giga\hertz}付近では値が大きく下がっている.これは,高周波になるほど電気が表面の凸凹の影響を強く受け,流れにくくなっていることを示している.

\begin{figure}[htbp]
  \centering
  \includesvg[width=0.85\textwidth]{data/conductivity.svg}
  \caption{MEGTRON6銅箔の実効導電率 $\sigma_{\mathrm{eff}}(f)$(mean, n=20)}
  \label{fig:conductivity}
\end{figure}

\section{信号減衰量の算出}
測定されたデータをもとに,実際の回路でどれくらい信号が弱まるか(減衰定数 $\alpha$)を計算した(図\ref{fig:attenuation}).
「銅の導電率は一定」と仮定して計算した場合に比べ,今回の実測値(導電率低下を考慮)を使って計算した場合,信号の減衰量は高周波帯で明らかに大きくなることが確認された.

\begin{figure}[htbp]
  \centering
  \includesvg[width=0.70\textwidth]{image/attenuation_graph.svg}
  \caption{伝送線路の減衰定数 $\alpha$(理想 vs 実効)}
  \label{fig:attenuation}
\end{figure}

%==============================================================================
% 第4章 考察
%==============================================================================
\chapter{考察}

\section{表面粗さの影響について}
測定結果から,高周波回路の設計において「銅の導電率は一定である」という従来の仮定を使うと,損失を実際よりも小さく見積もってしまう危険性があることがわかった.

周波数が上がると「表皮深さ(電気が流れる深さ)」が浅くなり,数ミクロンの表面の凸凹さえも電気の通り道としては大きな障害物となる.これが図\ref{fig:conductivity}で見られた実効導電率の低下の主要因であると考えられる.

%==============================================================================
% 第5章 結論
%==============================================================================
\chapter{結論}

本研究では,BCDR法を用いてMEGTRON6基板の特性評価を行った.得られた結論は以下の通りである.

\begin{enumerate}
    \item 基板の誘電率は周波数によらずほぼ一定だが,誘電損失は周波数とともに増加する.
    \item 銅箔の実効導電率は,周波数が高くなるにつれて表面粗さの影響で低下する.
    \item この導電率の低下を考慮しないと,実際の信号減衰量を正しく予測できないことが示唆された.
\end{enumerate}

今後は,シミュレーション(FDTD法)など別の方法でも検証を行い,今回の測定結果の妥当性をさらに高めていく必要がある.

%==============================================================================
% 謝辞
%==============================================================================
\chapter*{謝辞}
\addcontentsline{toc}{chapter}{謝辞}

本研究および本論文の作成にあたり,終始多大なるご指導ご鞭撻を賜りました春日貴志教授に深く感謝申し上げます.日頃の研究生活において,多くのご指導をいただきました,同研究室の専攻科生の内藤さくらさん,小池凛さんにも心から感謝いたします.

審査を担当いただいた姜天水先生に深く感謝申し上げます.

%==============================================================================
% 参考文献
%==============================================================================
\bibliographystyle{unsrt}
\bibliography{references}

%==============================================================================
% 付録
%==============================================================================
\appendix
\chapter{測定データ(mean ± std, n=20)}
\label{appendix:rawdata}

% 元の表データを保持
\begin{table}[htbp]
\centering
\begingroup\footnotesize\setlength{\tabcolsep}{6pt}\renewcommand{\arraystretch}{1.05}
\caption{比誘電率 $\varepsilon_r'$(mean ± std, n=20)}
\label{tab:app_permittivity}
\begin{tabular}{r S S}
\toprule
\shortstack[c]{Frequency\\(\si{\giga\hertz})} & \shortstack[c]{Mean\\\ensuremath{\varepsilon_r'}} & \shortstack[c]{Std\\\ensuremath{\varepsilon_r'}} \\
\midrule
12.6915 & 3.6009 & 0.0063 \\
23.3084 & 3.5838 & 0.0274 \\
33.8721 & 3.5826 & 0.0274 \\
44.4253 & 3.5835 & 0.0272 \\
54.9713 & 3.5852 & 0.0272 \\
65.4813 & 3.5898 & 0.0286 \\
75.9782 & 3.5927 & 0.0310 \\
86.5899 & 3.5838 & 0.0248 \\
96.6554 & 3.6115 & 0.0238 \\
107.0115 & 3.5921 & 0.0161 \\
\bottomrule
\end{tabular}\endgroup\end{table}

\begin{table}[htbp]
\centering
\begingroup\footnotesize\setlength{\tabcolsep}{6pt}\renewcommand{\arraystretch}{1.05}
\caption{実効導電率 $\sigma_{\mathrm{eff}}$(mean ± std, n=20)}
\label{tab:app_conductivity}
\begin{tabular}{r S S}
\toprule
\shortstack[c]{Frequency\\(\si{\giga\hertz})} & \shortstack[c]{Mean\\\ensuremath{\sigma_{\mathrm{eff}}}\\(\si{\siemens\per\metre})} & \shortstack[c]{Std\\(\si{\siemens\per\metre})} \\
\midrule
12.6638 & 2.4146e7 & 2.4072e6 \\
23.2561 & 8.2279e6 & 1.6116e6 \\
33.7960 & 4.9135e6 & 1.1691e6 \\
44.3247 & 3.6321e6 & 9.6109e5 \\
54.8473 & 2.6096e6 & 1.0277e6 \\
65.3264 & 3.4419e6 & 2.7636e6 \\
75.7767 & 1.4033e6 & 1.6201e6 \\
86.4699 & 3.4072e5 & 7.1732e5 \\
96.5272 & 1.0772e6 & 1.8638e6 \\
106.9216 & 3.6553e5 & 3.0212e4 \\
\bottomrule
\end{tabular}
\endgroup
\end{table}

\end{document}