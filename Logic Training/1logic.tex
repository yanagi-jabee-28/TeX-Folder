% ===== ドキュメントクラスと基本パッケージ =====
\documentclass[
  a4paper,
  11pt,
]{ltjsarticle}
\usepackage{newtxtext, newtxmath}
\usepackage{amsmath,amssymb}
\usepackage{graphicx}
\usepackage{siunitx}
\usepackage{float}
\usepackage{anyfontsize}
\usepackage[margin=2.5cm]{geometry}
\usepackage{booktabs}
\usepackage{hyperref}
\usepackage{parskip} % 段落の空行を見やすく

% ドキュメント情報
\title{論理学 演習問題 — 解答・解説}
\author{氏名}
\date{\today}

\begin{document}
\maketitle
\tableofcontents
\clearpage

% ===================================================================
\section{問題1:真理表による証明}

$I$ を真(True)、$O$ を偽(False)として真理表を作成します.

\subsection*{(1) $p\land (q \lor r)\equiv (p\land q) \lor (p\land r)$ (分配法則)}

下表は,各組合せに対する各項の真理値を示します.左辺と右辺の列が一致するため,同値であることが分かります.

\begin{table}[H]
  \centering
  \caption{$p\land (q \lor r)$ と $(p\land q) \lor (p\land r)$ の真理表}
  \label{tab:dist1}
  \begin{tabular}{cccccccc}
    \toprule
    $p$ & $q$ & $r$ & $q\lor r$ & $p\land(q\lor r)$ & $p\land q$ & $p\land r$ & $(p\land q)\lor(p\land r)$ \\
    \midrule
    $I$ & $I$ & $I$ & $I$ & $I$ & $I$ & $I$ & $I$ \\
    $I$ & $I$ & $O$ & $I$ & $I$ & $I$ & $O$ & $I$ \\
    $I$ & $O$ & $I$ & $I$ & $I$ & $O$ & $I$ & $I$ \\
    $I$ & $O$ & $O$ & $O$ & $O$ & $O$ & $O$ & $O$ \\
    $O$ & $I$ & $I$ & $I$ & $O$ & $O$ & $O$ & $O$ \\
    $O$ & $I$ & $O$ & $I$ & $O$ & $O$ & $O$ & $O$ \\
    $O$ & $O$ & $I$ & $I$ & $O$ & $O$ & $O$ & $O$ \\
    $O$ & $O$ & $O$ & $O$ & $O$ & $O$ & $O$ & $O$ \\
    \bottomrule
  \end{tabular}
\end{table}

\subsection*{(2) $p\lor (q \land r)\equiv (p\lor q) \land (p\lor r)$ (分配法則)}

同様に,次の真理表で左辺と右辺が一致することが確認できます.

\begin{table}[H]
  \centering
  \caption{$p\lor (q \land r)$ と $(p\lor q)\land(p\lor r)$ の真理表}
  \label{tab:dist2}
  \begin{tabular}{cccccccc}
    \toprule
    $p$ & $q$ & $r$ & $q\land r$ & $p\lor(q\land r)$ & $p\lor q$ & $p\lor r$ & $(p\lor q)\land(p\lor r)$ \\
    \midrule
    $I$ & $I$ & $I$ & $I$ & $I$ & $I$ & $I$ & $I$ \\
    $I$ & $I$ & $O$ & $O$ & $I$ & $I$ & $I$ & $I$ \\
    $I$ & $O$ & $I$ & $O$ & $I$ & $I$ & $I$ & $I$ \\
    $I$ & $O$ & $O$ & $O$ & $I$ & $I$ & $I$ & $I$ \\
    $O$ & $I$ & $I$ & $I$ & $I$ & $I$ & $I$ & $I$ \\
    $O$ & $I$ & $O$ & $O$ & $O$ & $I$ & $O$ & $O$ \\
    $O$ & $O$ & $I$ & $O$ & $O$ & $O$ & $I$ & $O$ \\
    $O$ & $O$ & $O$ & $O$ & $O$ & $O$ & $O$ & $O$ \\
    \bottomrule
  \end{tabular}
\end{table}

% ===================================================================
\section{問題3:同値変形による証明}

使用する主な法則:
\begin{itemize}
  \item $A\to B \equiv \overline{A} \lor B$
  \item ド・モルガンの法則: $\overline{A \lor B} \equiv \bar{A} \land \bar{B}$, \quad $\overline{A \land B} \equiv \bar{A} \lor \bar{B}$
  \item 分配律: $A \lor (B \land C) \equiv (A \lor B) \land (A \lor C)$, \quad $A \land (B \lor C) \equiv (A \land B) \lor (A \land C)$
  \item その他の恒等式: $A \lor \bar{A} \equiv I$, $A \land \bar{A} \equiv O$, $A \lor O \equiv A$, $A \land I \equiv A$, $A \lor I \equiv I$, $A \land O \equiv O$
\end{itemize}

\subsection*{(1) $(p\lor q)\land (p\lor \bar{q})\equiv p$}

\begin{align*}
(p\lor q)\land (p\lor \bar{q})
&\equiv p \lor (q \land \bar{q}) \quad (\text{分配法則})\\
&\equiv p \lor O \quad (\text{矛盾律 } q \land \bar{q} \equiv O)\\
&\equiv p \quad (\text{同一律 } p \lor O \equiv p)
\end{align*}

よって示された.

\subsection*{(2) $(p\to q)\to (p\land q)\equiv p$}

\begin{align*}
(p\to q)\to (p\land q)
&\equiv (\overline{p\to q}) \lor (p\land q) \quad (A \to B \equiv \bar{A} \lor B)\\
&\equiv (\overline{\bar{p} \lor q}) \lor (p\land q) \quad (p \to q \equiv \bar{p} \lor q)\\
&\equiv (p \land \bar{q}) \lor (p\land q) \quad (\text{ド・モルガンの法則})\\
&\equiv p \land (\bar{q} \lor q) \quad (\text{分配法則})\\
&\equiv p \land I \quad (\text{排中律 } \bar{q} \lor q \equiv I)\\
&\equiv p \quad (\text{同一律 } p \land I \equiv p)
\end{align*}

よって示された.

\subsection*{(3) $(p\land q)\to (p\to q)\equiv I$}

\begin{align*}
(p\land q)\to (p\to q)
&\equiv (\overline{p\land q}) \lor (p\to q) \quad (A \to B \equiv \bar{A} \lor B)\\
&\equiv (\bar{p} \lor \bar{q}) \lor (\bar{p} \lor q) \quad (\text{ド・モルガン},\; p \to q \equiv \bar{p} \lor q)\\
&\equiv \bar{p} \lor \bar{p} \lor \bar{q} \lor q \quad (\text{結合・交換})\\
&\equiv \bar{p} \lor (\bar{q} \lor q) \quad (\bar{p} \lor \bar{p} \equiv \bar{p})\\
&\equiv \bar{p} \lor I \quad (\text{排中律 } \bar{q} \lor q \equiv I)\\
&\equiv I \quad (\text{同一律 } \bar{p} \lor I \equiv I)
\end{align*}

よって示された.

\subsection*{(4) $(p\lor q)\to (p \land q) \equiv (p\lor \bar{q})\land(\bar{p}\lor q)$}

\begin{align*}
(p\lor q) &\to (p \land q)\\ 
&\equiv \overline{(p\lor q)} \lor (p \land q) \quad (\text{含意の定義 } A \to B \equiv \bar{A} \lor B)\\
&\equiv (\bar{p} \land \bar{q}) \lor (p \land q) \quad (\text{ド・モルガンの法則})\\
&\equiv \big((\bar{p} \land \bar{q}) \lor p\big) \land \big((\bar{p} \land \bar{q}) \lor q\big) \quad (\text{分配法則})\\
&\equiv (\bar{p} \lor p) \land (\bar{q} \lor p) \land (\bar{p} \lor q) \land (\bar{q} \lor q) \quad (\text{分配法則})\\
&\equiv I \land (\bar{q} \lor p) \land (\bar{p} \lor q) \land I \quad (\text{排中律 } \bar{p} \lor p \equiv I, \bar{q} \lor q \equiv I)\\
&\equiv (\bar{q} \lor p) \land (\bar{p} \lor q) \quad (\text{同一律 } I \land A \equiv A)\\
&\equiv (p \lor \bar{q}) \land (\bar{p} \lor q) \quad (\text{交換律})
\end{align*}

よって示された.

\clearpage
\appendix
\section{補遺:記法と恒等式}
本ドキュメントで使用した略記と恒等式の一覧をまとめます.授業や試験での参照に便利です.

\begin{itemize}
  \item $I$:真 (True)
  \item $O$:偽 (False)
  \item $\bar{A}$:命題 $A$ の否定
  \item $A\to B$:$A$ ならば $B$(含意)
\end{itemize}

\end{document}
