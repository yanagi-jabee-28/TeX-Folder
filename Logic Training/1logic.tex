% !TEX program = lualatex
%==============================================================================
% プリアンブル (Preamble)
%==============================================================================

% ===== ドキュメントクラス =====
\documentclass[
  a4paper,
  11pt,
]{ltjsarticle}

% ===== フォント設定 (LuaLaTeX専用) =====
\usepackage{luatexja-fontspec} % モダンなフォントシステムを有効化

% --- 安全なフォント設定 ---
% システムに依存しにくい TeX Gyre 系フォントを明示的に指定します。
% Windows 環境で "Arial" や "Courier New" が見つからない場合に備え、
% TeX Live に同梱されている TeX Gyre 系を使うのが安定します。

% 欧文(本文): 安全なデフォルトとして Latin Modern を使用(LuaLaTeX 向け)
% 依然としてシステムに TeX Gyre 系がある場合は置き換えても構いません。
\setmainfont{Latin Modern Roman}[Ligatures=TeX,Scale=MatchLowercase]

% 和文(本文): 利用可能なら Yu Mincho を指定します。
% 注意: Windows 環境でこのフォント名が見つからないとコンパイルエラーになります。
% 利用環境で日本語フォントが未インストールの場合は以下の行をコメントアウトするか
% ご自身のシステムにある日本語フォント名に置き換えてください(例: "IPAMincho", "Noto Serif CJK JP" 等)。
%\setmainjfont[Renderer=HarfBuzz]{Yu Mincho}

% 等幅フォント(ソース等): Latin Modern Mono を使用
\setmonofont{Latin Modern Mono}[Scale=MatchLowercase]

% サンセリフ体: Latin Modern Sans を使用
\setsansfont{Latin Modern Sans}[Scale=MatchLowercase]

% ===== 数式・物理単位関連 =====
\usepackage{amsmath}
\usepackage{newtxmath} % 数式用フォント。これはluatexja-fontspecと競合しないため継続使用
\usepackage{siunitx}
\usepackage{cancel}            % 数式の打ち消し線

% ===== 図表・レイアウト関連 =====
\usepackage[margin=2.5cm]{geometry} % 余白の設定
\usepackage{graphicx}          % 画像の挿入
\usepackage{booktabs}          % 見栄えの良い表
\usepackage{float}             % 図表の位置調整 (`[H]`オプション)
\usepackage{wrapfig}           % 図の回り込み
% \usepackage{parskip}         % (任意) 段落間のスペースを空ける場合、この行のコメントを外す

% ===== 回路図・グラフ描画関連 (TikZ/PGF) =====
\usepackage{tikz}
\usepackage{circuitikz}
\usepackage{pgfplots}          % 高機能なグラフ描画
\usepackage{pgfplotstable}     % グラフ描画のための表データ操作
\pgfplotsset{compat=1.18}      % pgfplotsのバージョン互換性設定
\usepgfplotslibrary{statistics} % 統計計算(回帰分析など)ライブラリ
\usetikzlibrary{positioning}   % ノードの相対配置

% ===== プログラミング・アルゴリズム関連 =====
\usepackage{listings}          % ソースコードの表示
\usepackage{algorithm}         % アルゴリズム記述の環境
\usepackage{algpseudocode}     % algorithm環境内で使う疑似コード

% ===== ハイパーリンク設定 (原則として最後に読み込む) =====
\usepackage[hidelinks,colorlinks=true,linkcolor=blue,citecolor=green!60!black]{hyperref}

% ===== (削除したパッケージ) =====
% anyfontsize: ltjsarticleでは不要なため削除
% amssymb: newtxmathに含まれるため不要

% ドキュメント情報
\title{論理学 演習問題 — 解答・解説}
\author{氏名}
\date{\today}

\begin{document}
\maketitle
\tableofcontents
\clearpage

% ===================================================================
\section{問題1:真理表による証明}

$I$ を真(True)、$O$ を偽(False)として真理表を作成します.

\subsection*{(1) $p\land (q \lor r)\equiv (p\land q) \lor (p\land r)$ (分配法則)}

下表は,各組合せに対する各項の真理値を示します.左辺と右辺の列が一致するため,同値であることが分かります.

\begin{table}[H]
  \centering
  \caption{$p\land (q \lor r)$ と $(p\land q) \lor (p\land r)$ の真理表}
  \label{tab:dist1}
  \begin{tabular}{cccccccc}
    \toprule
    $p$ & $q$ & $r$ & $q\lor r$ & $p\land(q\lor r)$ & $p\land q$ & $p\land r$ & $(p\land q)\lor(p\land r)$ \\
    \midrule
    $I$ & $I$ & $I$ & $I$ & $I$ & $I$ & $I$ & $I$ \\
    $I$ & $I$ & $O$ & $I$ & $I$ & $I$ & $O$ & $I$ \\
    $I$ & $O$ & $I$ & $I$ & $I$ & $O$ & $I$ & $I$ \\
    $I$ & $O$ & $O$ & $O$ & $O$ & $O$ & $O$ & $O$ \\
    $O$ & $I$ & $I$ & $I$ & $O$ & $O$ & $O$ & $O$ \\
    $O$ & $I$ & $O$ & $I$ & $O$ & $O$ & $O$ & $O$ \\
    $O$ & $O$ & $I$ & $I$ & $O$ & $O$ & $O$ & $O$ \\
    $O$ & $O$ & $O$ & $O$ & $O$ & $O$ & $O$ & $O$ \\
    \bottomrule
  \end{tabular}
\end{table}

\subsection*{(2) $p\lor (q \land r)\equiv (p\lor q) \land (p\lor r)$ (分配法則)}

同様に,次の真理表で左辺と右辺が一致することが確認できます.

\begin{table}[H]
  \centering
  \caption{$p\lor (q \land r)$ と $(p\lor q)\land(p\lor r)$ の真理表}
  \label{tab:dist2}
  \begin{tabular}{cccccccc}
    \toprule
    $p$ & $q$ & $r$ & $q\land r$ & $p\lor(q\land r)$ & $p\lor q$ & $p\lor r$ & $(p\lor q)\land(p\lor r)$ \\
    \midrule
    $I$ & $I$ & $I$ & $I$ & $I$ & $I$ & $I$ & $I$ \\
    $I$ & $I$ & $O$ & $O$ & $I$ & $I$ & $I$ & $I$ \\
    $I$ & $O$ & $I$ & $O$ & $I$ & $I$ & $I$ & $I$ \\
    $I$ & $O$ & $O$ & $O$ & $I$ & $I$ & $I$ & $I$ \\
    $O$ & $I$ & $I$ & $I$ & $I$ & $I$ & $I$ & $I$ \\
    $O$ & $I$ & $O$ & $O$ & $O$ & $I$ & $O$ & $O$ \\
    $O$ & $O$ & $I$ & $O$ & $O$ & $O$ & $I$ & $O$ \\
    $O$ & $O$ & $O$ & $O$ & $O$ & $O$ & $O$ & $O$ \\
    \bottomrule
  \end{tabular}
\end{table}

% ===================================================================
\section{問題2:カードと発言のパズル}

命題変数 $p$, $q$, $r$ をそれぞれ以下のように定義します:
\begin{itemize}
  \item $p$: でてきたのはダムである
  \item $\bar{p}$: でてきたのはディーである
  \item $q$: 赤いカードをもっている
  \item $\bar{q}$: 黒いカードをもっている
  \item $I$: 恒真命題(常に真)
  \item $O$: 恒偽命題(常に偽)
\end{itemize}

ルール:
\begin{itemize}
  \item 赤のカード ($q$) をもっている人は,正しいことを言う。
  \item 黒のカード ($\bar{q}$) をもっている人は,間違ったことを言う。
\end{itemize}

発言 $S$ の内容:
「僕は黒のカードをもったダムか,または赤のカードをもったディーだ。」
これを論理式で表すと,以下のようになります:
\[
S \equiv (p \land \bar{q}) \lor (\bar{p} \land q)
\]

パズルの条件:
このパズルのルールは,「発言 $S$ が真である」ことと「発言者が赤のカードを持っている ($q$)」ことが同値($ \equiv $)であることを意味します。
\begin{itemize}
  \item $q \equiv I$ (赤カード)ならば,$S \equiv I$ (真実を言う)
  \item $q \equiv O$ (黒カード)ならば,$S \equiv O$ (嘘を言う)
\end{itemize}

この2つの条件を同時に満たすのが,$S \equiv q$ という関係です。

\subsection*{指定された論理展開による証明}

このパズルが成立するためには,発言者が赤カードを持っていると仮定した場合(ケース1)と,黒カードを持っていると仮定した場合(ケース2)の両方で,矛盾なく同じ結論($p$ が $I$ か $O$ か)が導かれなければなりません。

\subsubsection*{ケース1: $q \equiv I$ と仮定する (赤カードを持っている場合)}
\begin{enumerate}
  \item 仮定:  
    $q \equiv I$ (赤カード)  
    このとき,$\bar{q} \equiv \bar{I} \equiv O$ (黒カードではない)となります。
  \item ルールの適用:  
    赤カードを持っているので,発言 $S$ は真実でなければなりません。  
    よって,$S \equiv I$ となります。
  \item 発言 $S$ の内容を評価:  
    発言 $S \equiv (p \land \bar{q}) \lor (\bar{p} \land q)$ に,仮定 $q \equiv I$ と $\bar{q} \equiv O$ を代入します。  
    \[
    S \equiv (p \land O) \lor (\bar{p} \land I)
    \]
  \item 論理式の簡略化:  
    性質 $P \land O = O$ および $P \land I = P$ を用います。  
    \[
    S \equiv O \lor \bar{p}
    \]
    性質 $P \lor O = P$ を用います。  
    \[
    S \equiv \bar{p}
    \]
  \item 結論の導出:  
    ステップ2(ルール)より $S \equiv I$ であり,ステップ4(発言内容)より $S \equiv \bar{p}$ です。  
    したがって,この2つは等しくなければなりません。  
    \[
    I \equiv \bar{p}
    \]
    これは,$p \equiv O$ を意味します。  
    この仮定(赤カード)は,発言者がディーである場合 ($p \equiv O$) に矛盾なく成立します。
\end{enumerate}

\subsubsection*{ケース2: $q \equiv O$ と仮定する (黒カードを持っている場合)}
\begin{enumerate}
  \item 仮定:  
    $q \equiv O$ (赤カードではない=黒カード)  
    このとき,$\bar{q} \equiv \bar{O} \equiv I$ (黒カード)となります。
  \item ルールの適用:  
    黒カードを持っているので,発言 $S$ は嘘(間違い)でなければなりません。  
    よって,$S \equiv O$ となります。
  \item 発言 $S$ の内容を評価:  
    発言 $S \equiv (p \land \bar{q}) \lor (\bar{p} \land q)$ に,仮定 $q \equiv O$ と $\bar{q} \equiv I$ を代入します。  
    \[
    S \equiv (p \land I) \lor (\bar{p} \land O)
    \]
  \item 論理式の簡略化:  
    性質 $P \land I = P$ および $P \land O = O$ を用います。  
    \[
    S \equiv p \lor O
    \]
    性質 $P \lor O = P$ を用います。  
    \[
    S \equiv p
    \]
  \item 結論の導出:  
    ステップ2(ルール)より $S \equiv O$ であり,ステップ4(発言内容)より $S \equiv p$ です。  
    したがって,この2つは等しくなければなりません。  
    \[
    O \equiv p
    \]
    これは,$p \equiv O$ を意味します。  
    この仮定(黒カード)も,発言者がディーである場合 ($p \equiv O$) に矛盾なく成立します。
\end{enumerate}

\subsection*{結論}
ケース1(赤カード)の場合も,$p \equiv O$(ダムではない)が導かれました。  
ケース2(黒カード)の場合も,$p \equiv O$(ダムではない)が導かれました。  
どちらの場合も一貫して $p \equiv O$ となり,矛盾は生じません。  
$p$ は「でてきたのはダムである」という命題だったので,$p \equiv O$ は「でてきたのがダムであることは偽である」を意味します。  
したがって,でてきたのは\textbf{ディー} ($\bar{p} \equiv I$) です。

% ===================================================================
\section{問題3:同値変形による証明}

使用する主な法則:
\begin{itemize}
  \item $A\to B \equiv \overline{A} \lor B$
  \item ド・モルガンの法則: $\overline{A \lor B} \equiv \bar{A} \land \bar{B}$, \quad $\overline{A \land B} \equiv \bar{A} \lor \bar{B}$
  \item 分配律: $A \lor (B \land C) \equiv (A \lor B) \land (A \lor C)$, \quad $A \land (B \lor C) \equiv (A \land B) \lor (A \land C)$
  \item その他の恒等式: $A \lor \bar{A} \equiv I$, $A \land \bar{A} \equiv O$, $A \lor O \equiv A$, $A \land I \equiv A$, $A \lor I \equiv I$, $A \land O \equiv O$
\end{itemize}

\subsection*{(1) $(p\lor q)\land (p\lor \bar{q})\equiv p$}

\begin{align*}
(p\lor q)\land (p\lor \bar{q})
&\equiv p \lor (q \land \bar{q}) \quad (\text{分配法則 } (A\lor B)\land(A\lor C)\equiv A\lor(B\land C))\\
&\equiv p \lor O \quad (\text{矛盾律 } q \land \bar{q} \equiv O)\\
&\equiv p \quad (\text{同一律 } p \lor O \equiv p)
\end{align*}

よって示された.

\subsection*{(2) $(p\to q)\to (p\land q)\equiv p$}

\begin{align*}
(p\to q)\to (p\land q)
&\equiv (\overline{ p\to q }) \lor (p\land q) \quad (A \to B \equiv \bar{A} \lor B)\\
&\equiv (\bar{p} \lor \bar{q}) \lor (p\land q) \quad (p \to q \equiv \bar{p} \lor q)\\
&\equiv (p \land \bar{q}) \lor (p\land q) \quad (\text{ド・モルガンの法則})\\
&\equiv p \land (\bar{q} \lor q) \quad (\text{分配法則 } (X\land Y)\lor(X\land Z)\equiv X\land(Y\lor Z))\\
&\equiv p \land I \quad (\text{排中律 } \bar{q} \lor q \equiv I)\\
&\equiv p \quad (\text{同一律 } p \land I \equiv p)
\end{align*}

よって示された.

\subsection*{(3) $(p\land q)\to (p\to q)\equiv I$}

\begin{align*}
(p\land q)\to (p\to q)
&\equiv (\overline{p\land q}) \lor (p\to q) \quad (A \to B \equiv \bar{A} \lor B)\\
&\equiv (\bar{p} \lor \bar{q}) \lor (\bar{p} \lor q) \quad (\text{ド・モルガン},\; p \to q \equiv \bar{p} \lor q)\\
&\equiv \bar{p} \lor \bar{p} \lor \bar{q} \lor q \quad (\text{結合・交換})\\
&\equiv \bar{p} \lor (\bar{q} \lor q) \quad (\text{冪等律 } \bar{p} \lor \bar{p} \equiv \bar{p})\\
&\equiv \bar{p} \lor I \quad (\text{排中律 } \bar{q} \lor q \equiv I)\\
&\equiv I \quad (\text{同一律 } \bar{p} \lor I \equiv I)
\end{align*}

よって示された.

\subsection*{(4) $(p\lor q)\to (p \land q) \equiv (p\lor \bar{q})\land(\bar{p}\lor q)$}

\begin{align*}
(p\lor q) &\to (p \land q)\\ 
&\equiv (\overline{p\lor q}) \lor (p \land q) \quad (\text{含意の定義 } A \to B \equiv \bar{A} \lor B)\\
&\equiv (\bar{p} \land \bar{q}) \lor (p \land q) \quad (\text{ド・モルガンの法則})\\
&\equiv \big((\bar{p} \land \bar{q}) \lor p\big) \land \big((\bar{p} \land \bar{q}) \lor q\big) \quad (\text{分配法則})\\
&\equiv (\bar{p} \lor p) \land (\bar{q} \lor p) \land (\bar{p} \lor q) \land (\bar{q} \lor q) \quad (\text{分配法則})\\
&\equiv I \land (\bar{q} \lor p) \land (\bar{p} \lor q) \land I \quad (\text{排中律 } \bar{p} \lor p \equiv I, \bar{q} \lor q \equiv I)\\
&\equiv (\bar{q} \lor p) \land (\bar{p} \lor q) \quad (\text{同一律 } I \land A \equiv A)\\
&\equiv (p \lor \bar{q}) \land (\bar{p} \lor q) \quad (\text{交換律})
\end{align*}

よって示された.

\clearpage
\appendix

本ドキュメントでは、命題の否定を表すために、単一の変数には $\bar{a}$ のような形式を用い、複合式には $\overline{aB}$ のような形式を用いています.これは、視覚的な明確さを保つためです.

\section{補遺:記法と恒等式}
本ドキュメントで使用した略記と恒等式の一覧をまとめます.授業や試験での参照に便利です.

\begin{itemize}
  \item $I$:真 (True)
  \item $O$:偽 (False)
  \item $\overline{A}$:命題 $A$ の否定
  \item $A\to B$:$A$ ならば $B$(含意)
\end{itemize}

\end{document}
