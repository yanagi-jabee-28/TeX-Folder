% !TEX program = lualatex
%==============================================================================
% プリアンブル (Preamble)
%==============================================================================

% ===== ドキュメントクラス =====
\documentclass[
	a4paper,
	11pt
]{ltjsreport}

%------------------------------------------------------------------------------
% パッケージ読み込み
%------------------------------------------------------------------------------

% ===== フォント・言語設定 (LuaLaTeX専用) =====
\usepackage{luatexja-fontspec}
\usepackage{lmodern} % フォントサイズの置き換えを防ぐため

% ===== レイアウト関連 =====
\usepackage[margin=2.5cm]{geometry}
\usepackage{booktabs}
\usepackage{float}
\usepackage{graphicx}

% ===== 数式・物理単位関連 =====
\usepackage{amsmath}
\usepackage{amssymb} % \mathbb等の数学記号
\usepackage{siunitx}
\usepackage{bm} % ベクトルを太字にするため (\bm)

% ===== その他 =====
\usepackage{url} % URLを適切に表示
\usepackage{xurl} % Improved line breaking for long URLs
\urlstyle{same}
\Urlmuskip=0mu plus 2mu
\usepackage[
  hidelinks,
]{hyperref}
\usepackage[super,square]{natbib} % 引用を上付き角括弧に

%------------------------------------------------------------------------------
% 各種設定
%------------------------------------------------------------------------------

% ===== フォント設定 =====
\setmainfont{Latin Modern Roman}
\setsansfont{Latin Modern Sans}
\setmonofont{Latin Modern Mono}
\setmainjfont[Renderer=HarfBuzz]{Yu Mincho}
\setsansjfont[Renderer=HarfBuzz]{Yu Gothic}
\DeclareMathSizes{11}{11}{7}{5} % 数学フォントサイズの調整

% ===== ドキュメント情報 =====
\title{論理トレーニング \\ レポート課題}
\author{科  番  氏名}
\date{\today}

% ===== 数式用カスタムコマンド =====
\providecommand{\dd}{\mathrm{d}} % 微分演算子 d
\newcommand{\mi}{\mathrm{j}} % 虚数単位 j

%==============================================================================
% ドキュメント本体 (Body)
%==============================================================================
\begin{document}
\sloppy % allow more flexible spacing to reduce overfull \hbox warnings

\maketitle

\chapter*{問題}

\section*{1. 命題論理の同値変形}

$p, q, r, s$ を命題とするとき、次の (1), (2) を同値変形により示せ。

\begin{enumerate}
\item[(1)] 
\begin{equation}
(p \land q) \lor (r \land s) \equiv (p \lor r) \land (p \lor s) \land (q \lor r) \land (q \lor s)
\end{equation}

\item[(2)] 
\begin{equation}
(p \lor q) \land (r \lor s) \equiv (p \land r) \lor (p \land s) \lor (q \land r) \lor (q \land s)
\end{equation}
\end{enumerate}

\section*{2. 述語論理と量化子}

命題関数 $p(x), q(x) \ (x \in X)$ に対して、次が成り立つ。

\begin{enumerate}
\item[(1)] 
\begin{equation}
\forall x \ p(x) \land \forall x \ q(x) \equiv \forall x \ (p(x) \land q(x))
\end{equation}

\item[(2)] 
\begin{equation}
\forall x \ p(x) \lor \forall x \ q(x) \Rightarrow \forall x \ (p(x) \lor q(x))
\end{equation}

\item[(3)] 
\begin{equation}
\exists x \ (p(x) \lor q(x)) \equiv \exists x \ p(x) \lor \exists x \ q(x)
\end{equation}

\item[(4)] 
\begin{equation}
\exists x \ (p(x) \land q(x)) \Rightarrow \exists x \ p(x) \land \exists x \ q(x)
\end{equation}
\end{enumerate}

$X = \{a_1, a_2\}$ とするとき、同値変形により (2) を示せ。

\vspace{0.3cm}
\noindent
(Hint: $\forall x \ p(x) \lor \forall x \ q(x) \rightarrow \forall x \ (p(x) \lor q(x)) \equiv I$ を示す。)

\section*{3. 命題関数の真理値判定}

次の命題関数 $p(\epsilon, N)$ に対して、(1), (2) はそれぞれどんな命題か。また、その真理値を答えよ。
\begin{align}
\epsilon &\in \{x \in \mathbb{R} \mid x > 0\} \\
N &\in \mathbb{N} \\
p(\epsilon, N) &: N\epsilon > 1
\end{align}

\begin{enumerate}
\item[(1)] 
\begin{equation}
\forall \epsilon \ \exists N \ p(\epsilon, N)
\end{equation}

\item[(2)] 
\begin{equation}
\overline{\forall \epsilon \ \exists N \ p(\epsilon, N)}
\end{equation}
\end{enumerate}

\section*{4. ε-N論法による極限証明}

$\displaystyle \lim_{n \to \infty} \frac{1}{n} = 0$ であることを、ε-N論法を用いて証明せよ。

\section*{5. ε-δ論法による極限証明}

$\displaystyle \lim_{x \to 1} x^2 = 1$ であることを、ε-δ論法を用いて証明せよ。

\end{document}
