% !TEX program = lualatex
\documentclass[a4paper,11pt]{ltjsreport}

%==============================================================================
% Packages & Settings
%==============================================================================
\usepackage{amsmath, amssymb, amsthm}
\usepackage[margin=2.5cm]{geometry}
\usepackage{bm}

% Redefine proof symbol
\renewcommand{\qedsymbol}{$\blacksquare$}

% Title Info
\title{論理トレーニング レポート課題}
\author{学籍番号 氏名}
\date{\today}

\begin{document}

\maketitle

\chapter*{解答}

\section*{1. 命題論理の同値変形}

$p, q, r, s$ を命題とする。

\subsection*{(1) $(p \land q) \lor (r \land s) \equiv (p \lor r) \land (p \lor s) \land (q \lor r) \land (q \lor s)$}

\begin{proof}
右辺に対し、分配法則 $A \lor (B \land C) \equiv (A \lor B) \land (A \lor C)$ を適用して整理する。
\begin{align*}
\text{RHS} 
&= \bigl[ (p \lor r) \land (p \lor s) \bigr] \land \bigl[ (q \lor r) \land (q \lor s) \bigr] \\
&= \bigl[ p \lor (r \land s) \bigr] \land \bigl[ q \lor (r \land s) \bigr] && (\because \text{Distributive Law}) \\
&= (r \land s) \lor (p \land q) && (\because \text{Distributive Law}) \\
&= (p \land q) \lor (r \land s) \\
&= \text{LHS}
\end{align*}
\end{proof}

\subsection*{(2) $(p \lor q) \land (r \lor s) \equiv (p \land r) \lor (p \land s) \lor (q \land r) \lor (q \land s)$}

\begin{proof}
左辺に対し、分配法則を順次適用して展開する。
\begin{align*}
\text{LHS} 
&= (p \lor q) \land (r \lor s) \\
&= \bigl( (p \lor q) \land r \bigr) \lor \bigl( (p \lor q) \land s \bigr) && (\because \text{Distributive Law}) \\
&= (p \land r) \lor (q \land r) \lor (p \land s) \lor (q \land s) && (\because \text{Distributive Law}) \\
&= (p \land r) \lor (p \land s) \lor (q \land r) \lor (q \land s) && (\because \text{Commutativity}) \\
&= \text{RHS}
\end{align*}
\end{proof}

\section*{2. 述語論理と量化子}

$X = \{a_1, a_2\}$ とする。
\[ \forall x \in X, p(x) \lor \forall x \in X, q(x) \implies \forall x \in X, (p(x) \lor q(x)) \]

\begin{proof}
定義域が有限集合であるため、$\forall x$ を要素ごとの論理積に書き換える。
$p(a_i) = p_i, \ q(a_i) = q_i$ とおく。

\begin{align*}
\text{Conclusion (RHS)} 
&= (p_1 \lor q_1) \land (p_2 \lor q_2) \\
&= (p_1 \land p_2) \lor (p_1 \land q_2) \lor (q_1 \land p_2) \lor (q_1 \land q_2) && (\text{展開})
\end{align*}

一方、仮定(LHS)は $(p_1 \land p_2) \lor (q_1 \land q_2)$ である。
論理和の導入則 $A \implies A \lor B$ より、
\[ (p_1 \land p_2) \lor (q_1 \land q_2) \implies (p_1 \land p_2) \lor (q_1 \land q_2) \lor \underbrace{(p_1 \land q_2) \lor (q_1 \land p_2)}_{\text{Additional terms}} \]
したがって、LHS $\implies$ RHS が成立する。
\end{proof}

\section*{3. 命題関数の真理値判定}

定義:$p(\epsilon, N) : N\epsilon > 1 \quad (\epsilon > 0, N \in \mathbb{N})$

\subsection*{(1) $\forall \epsilon \ \exists N \ p(\epsilon, N)$}

\textbf{真理値:真}

アルキメデスの公理(Archimedean Property)より、任意の $\epsilon > 0$ に対し、
\[ \exists N \in \mathbb{N} \quad \text{s.t.} \quad N > \frac{1}{\epsilon} \]
このとき $N\epsilon > 1$ が成り立つ。

\subsection*{(2) $\overline{\forall \epsilon \ \exists N \ p(\epsilon, N)}$}

\textbf{真理値:偽}

命題 (1) の否定を考える。
\[ \neg (\forall \epsilon \exists N, N\epsilon > 1) \equiv \exists \epsilon > 0, \forall N \in \mathbb{N}, N\epsilon \le 1 \]
これは「ある定数 $1/\epsilon$ が全ての自然数 $N$ の上界である」ことを意味するが、$\mathbb{N}$ は上に有界ではないため矛盾。
よって元の命題は偽である。

\section*{4. $\epsilon-N$ 論法による極限証明}

命題:$\displaystyle \lim_{n \to \infty} \frac{1}{n} = 0$

\begin{proof}
$\forall \epsilon > 0$ に対し、アルキメデスの公理より
\[ \exists N \in \mathbb{N} \quad \text{s.t.} \quad N > \frac{1}{\epsilon} \]
が存在する。任意の $n > N$ について、
\begin{align*}
n > N > \frac{1}{\epsilon} \implies \frac{1}{n} < \epsilon
\end{align*}
よって、
\[ \left| \frac{1}{n} - 0 \right| = \frac{1}{n} < \epsilon \]
以上より、$\displaystyle \lim_{n \to \infty} \frac{1}{n} = 0$。
\end{proof}

\section*{5. $\epsilon-\delta$ 論法による極限証明}

命題:$\displaystyle \lim_{x \to 1} x^2 = 1$

\begin{proof}
$\forall \epsilon > 0$ に対し、$\delta = \min\left\{1, \frac{\epsilon}{3}\right\}$ とおく。

$0 < |x - 1| < \delta$ なる $x$ について評価を行う。
\begin{enumerate}
    \item $\delta \le 1$ より、
    \[ |x - 1| < 1 \implies 0 < x < 2 \implies |x + 1| < 3 \]
    \item $\delta \le \frac{\epsilon}{3}$ より、
    \[ |x - 1| < \frac{\epsilon}{3} \]
\end{enumerate}
したがって、
\begin{align*}
|x^2 - 1| &= |x - 1||x + 1| \\
&< \frac{\epsilon}{3} \cdot 3 \\
&= \epsilon
\end{align*}
$\therefore \displaystyle \lim_{x \to 1} x^2 = 1$
\end{proof}

\end{document}