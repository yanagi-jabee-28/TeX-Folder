% !TEX program = lualatex
%==============================================================================
% プリアンブル (Preamble)
%==============================================================================

% ===== ドキュメントクラス =====
\documentclass[
	a4paper,
	11pt
]{ltjsreport}

%------------------------------------------------------------------------------
% パッケージ読み込み
%------------------------------------------------------------------------------

% ===== フォント・言語設定 (LuaLaTeX専用) =====
\usepackage{luatexja-fontspec}
\usepackage{lmodern} % フォントサイズの置き換えを防ぐため

% ===== レイアウト関連 =====
\usepackage[margin=2.5cm]{geometry}
\usepackage{booktabs}
\usepackage{float}
\usepackage{graphicx}

% ===== 数式・物理単位関連 =====
\usepackage{amsmath}
\usepackage{amssymb} % \mathbb等の数学記号
\usepackage{siunitx}
\usepackage{bm} % ベクトルを太字にするため (\bm)

% ===== その他 =====
\usepackage{url} % URLを適切に表示
\usepackage{xurl} % Improved line breaking for long URLs
\urlstyle{same}
\Urlmuskip=0mu plus 2mu
\usepackage[
  hidelinks,
]{hyperref}
\usepackage[super,square]{natbib} % 引用を上付き角括弧に

%------------------------------------------------------------------------------
% 各種設定
%------------------------------------------------------------------------------

% ===== フォント設定 =====
\setmainfont{Latin Modern Roman}
\setsansfont{Latin Modern Sans}
\setmonofont{Latin Modern Mono}
\setmainjfont[Renderer=HarfBuzz]{Yu Mincho}
\setsansjfont[Renderer=HarfBuzz]{Yu Gothic}
\DeclareMathSizes{11}{11}{7}{5} % 数学フォントサイズの調整

% ===== ドキュメント情報 =====
\title{論理トレーニング \\ レポート課題}
\author{科  番  氏名}
\date{\today}

% ===== 数式用カスタムコマンド =====
\providecommand{\dd}{\mathrm{d}} % 微分演算子 d
\newcommand{\mi}{\mathrm{j}} % 虚数単位 j

%==============================================================================
% ドキュメント本体 (Body)
%==============================================================================
\begin{document}
\sloppy % allow more flexible spacing to reduce overfull \hbox warnings

\maketitle

\chapter*{問題}

\section*{1. 命題論理の同値変形}

$p, q, r, s$ を命題とするとき、次の (1), (2) を同値変形により示せ。

\begin{enumerate}
\item[(1)] 
\begin{equation}
(p \land q) \lor (r \land s) \equiv (p \lor r) \land (p \lor s) \land (q \lor r) \land (q \lor s)
\end{equation}

\item[(2)] 
\begin{equation}
(p \lor q) \land (r \lor s) \equiv (p \land r) \lor (p \land s) \lor (q \land r) \lor (q \land s)
\end{equation}
\end{enumerate}

\section*{2. 述語論理と量化子}

命題関数 $p(x), q(x) \ (x \in X)$ に対して、次が成り立つ。

\begin{enumerate}
\item[(1)] 
\begin{equation}
\forall x \ p(x) \land \forall x \ q(x) \equiv \forall x \ (p(x) \land q(x))
\end{equation}

\item[(2)] 
\begin{equation}
\forall x \ p(x) \lor \forall x \ q(x) \Rightarrow \forall x \ (p(x) \lor q(x))
\end{equation}

\item[(3)] 
\begin{equation}
\exists x \ (p(x) \lor q(x)) \equiv \exists x \ p(x) \lor \exists x \ q(x)
\end{equation}

\item[(4)] 
\begin{equation}
\exists x \ (p(x) \land q(x)) \Rightarrow \exists x \ p(x) \land \exists x \ q(x)
\end{equation}
\end{enumerate}

$X = \{a_1, a_2\}$ とするとき、同値変形により (2) を示せ。

\vspace{0.3cm}
\noindent
(Hint: $\forall x \ p(x) \lor \forall x \ q(x) \rightarrow \forall x \ (p(x) \lor q(x)) \equiv I$ を示す。)

\section*{3. 命題関数の真理値判定}

次の命題関数 $p(\epsilon, N)$ に対して、(1), (2) はそれぞれどんな命題か。また、その真理値を答えよ。
\begin{align}
\epsilon &\in \{x \in \mathbb{R} \mid x > 0\} \\
N &\in \mathbb{N} \\
p(\epsilon, N) &: N\epsilon > 1
\end{align}

\begin{enumerate}
\item[(1)] 
\begin{equation}
\forall \epsilon \ \exists N \ p(\epsilon, N)
\end{equation}

\item[(2)] 
\begin{equation}
\overline{\forall \epsilon \ \exists N \ p(\epsilon, N)}
\end{equation}
\end{enumerate}

\section*{4. ε-N論法による極限証明}

$\displaystyle \lim_{n \to \infty} \frac{1}{n} = 0$ であることを、ε-N論法を用いて証明せよ。

\section*{5. ε-δ論法による極限証明}

$\displaystyle \lim_{x \to 1} x^2 = 1$ であることを、ε-δ論法を用いて証明せよ。

\chapter*{解答}

\setcounter{equation}{0}
\section*{1. 命題論理の同値変形}

\subsection*{(1) $(p \land q) \lor (r \land s) \equiv (p \lor r) \land (p \lor s) \land (q \lor r) \land (q \lor s)$}

\subsubsection*{【戦略】}
左辺の $(p \land q) \lor (r \land s)$ を展開する際、分配法則を繰り返し適用することで右辺を得る。

\subsubsection*{【形式的証明】}

\begin{align}
(p \land q) \lor (r \land s) 
&\equiv ((p \land q) \lor r) \land ((p \land q) \lor s) \quad \text{(分配法則)}\\
&\equiv (p \lor r) \land (q \lor r) \land (p \lor s) \land (q \lor s) \quad \text{(分配法則)}\\
&\equiv (p \lor r) \land (p \lor s) \land (q \lor r) \land (q \lor s) \quad \text{(交換法則)}
\end{align}

\subsection*{(2) $(p \lor q) \land (r \lor s) \equiv (p \land r) \lor (p \land s) \lor (q \land r) \lor (q \land s)$}

\subsubsection*{【戦略】}
左辺の $(p \lor q) \land (r \lor s)$ に対して分配法則を適用し、展開する。

\subsubsection*{【形式的証明】}

\begin{align}
(p \lor q) \land (r \lor s)
&\equiv ((p \lor q) \land r) \lor ((p \lor q) \land s) \quad \text{(分配法則)}\\
&\equiv (p \land r) \lor (q \land r) \lor (p \land s) \lor (q \land s) \quad \text{(分配法則)}\\
&\equiv (p \land r) \lor (p \land s) \lor (q \land r) \lor (q \land s) \quad \text{(交換・結合法則)}
\end{align}

\section*{2. 述語論理と量化子}

\subsection*{$X = \{a_1, a_2\}$ のとき、(2) を同値変形により示す}

以下を証明する:
\begin{equation}
\forall x \ p(x) \lor \forall x \ q(x) \Rightarrow \forall x \ (p(x) \lor q(x)) \equiv I
\end{equation}

\subsubsection*{【発見的考察】}
左辺 $\forall x \ p(x) \lor \forall x \ q(x) \Rightarrow \forall x \ (p(x) \lor q(x))$ を分析する。

定義により:
\begin{equation}
\forall x \ p(x) \equiv p(a_1) \land p(a_2)
\end{equation}
\begin{equation}
\forall x \ q(x) \equiv q(a_1) \land q(a_2)
\end{equation}
\begin{equation}
\forall x \ (p(x) \lor q(x)) \equiv (p(a_1) \lor q(a_1)) \land (p(a_2) \lor q(a_2))
\end{equation}

したがって証明すべき式は:
\begin{equation}
(p(a_1) \land p(a_2)) \lor (q(a_1) \land q(a_2)) \Rightarrow (p(a_1) \lor q(a_1)) \land (p(a_2) \lor q(a_2))
\end{equation}

\subsubsection*{【形式的証明】}

\begin{align}
&(p(a_1) \land p(a_2)) \lor (q(a_1) \land q(a_2)) \Rightarrow (p(a_1) \lor q(a_1)) \land (p(a_2) \lor q(a_2)) \\
&\equiv \overline{(p(a_1) \land p(a_2)) \lor (q(a_1) \land q(a_2))} \lor ((p(a_1) \lor q(a_1)) \land (p(a_2) \lor q(a_2))) \\
&\quad \quad \quad \quad \quad \quad \quad \quad \quad \text{(含意の定義 } A \to B \equiv \bar{A} \lor B \text{)}\\
&\equiv (\overline{p(a_1) \land p(a_2)}) \land (\overline{q(a_1) \land q(a_2)}) \lor ((p(a_1) \lor q(a_1)) \land (p(a_2) \lor q(a_2)))\\
&\quad \quad \quad \quad \quad \quad \quad \quad \quad \text{(ド・モルガンの法則)}\\
&\equiv (\bar{p}(a_1) \lor \bar{p}(a_2)) \land (\bar{q}(a_1) \lor \bar{q}(a_2)) \lor ((p(a_1) \lor q(a_1)) \land (p(a_2) \lor q(a_2)))\\
&\quad \quad \quad \quad \quad \quad \quad \quad \quad \text{(ド・モルガンの法則)}
\end{align}

後半部分を展開:
\begin{align}
&((p(a_1) \lor q(a_1)) \land (p(a_2) \lor q(a_2))) \\
&\equiv (p(a_1) \land (p(a_2) \lor q(a_2))) \lor (q(a_1) \land (p(a_2) \lor q(a_2))) \quad \text{(分配法則)}\\
&\equiv (p(a_1) \land p(a_2)) \lor (p(a_1) \land q(a_2)) \lor (q(a_1) \land p(a_2)) \lor (q(a_1) \land q(a_2))\\
&\quad \quad \quad \quad \quad \quad \quad \quad \quad \text{(分配法則)}
\end{align}

前半部分と合わせると、全体は以下の恒等式に帰着:
\begin{equation}
(\bar{p}(a_1) \lor \bar{p}(a_2)) \land (\bar{q}(a_1) \lor \bar{q}(a_2)) \lor (p(a_1) \land p(a_2)) \lor (p(a_1) \land q(a_2)) \lor (q(a_1) \land p(a_2)) \lor (q(a_1) \land q(a_2)) \equiv I
\end{equation}

任意の真理値割り当てに対して、左辺は常に真であることが確認できるため、同値式は恒真(トートロジー)である。$\quad \blacksquare$

\setcounter{equation}{0}
\section*{3. 命題関数の真理値判定}

\subsection*{命題の意味と真理値}

\subsubsection*{(1) $\forall \epsilon \ \exists N \ p(\epsilon, N)$}

\noindent
【意味】:「すべての正の実数 $\epsilon$ に対して、ある自然数 $N$ が存在して、$N\epsilon > 1$ が成り立つ」

\noindent
【真理値】:\textbf{真(True)}

\noindent
【根拠】:$\epsilon > 0$ が任意に与えられたとき、アルキメデスの公理により、$N > \frac{1}{\epsilon}$ を満たす自然数 $N$ が存在する。このとき $N\epsilon > 1$ が成立する。

\subsubsection*{(2) $\overline{\forall \epsilon \ \exists N \ p(\epsilon, N)}$}

\noindent
【意味】:「存在して、ある正の実数 $\epsilon$ が存在して、すべての自然数 $N$ に対して、$N\epsilon \leq 1$ である」

\noindent
【形式的表現】:
\begin{equation}
\overline{\forall \epsilon \ \exists N \ p(\epsilon, N)} \equiv \exists \epsilon \ \overline{\exists N \ p(\epsilon, N)} \equiv \exists \epsilon \ \forall N \ \overline{p(\epsilon, N)}
\end{equation}

すなわち:
\begin{equation}
\exists \epsilon \ \forall N \ (N\epsilon \leq 1)
\end{equation}

\noindent
【真理値】:\textbf{真(True)}

\noindent
【根拠】:例えば $\epsilon = 0.001$ を選ぶと、これはすべての自然数 $N$ に対して $N \times 0.001 \leq 1$ が成り立つことが明らかである。

\setcounter{equation}{0}
\section*{4. ε-N論法による極限証明}

\noindent
命題:$\displaystyle \lim_{n \to \infty} \frac{1}{n} = 0$

\subsection*{【発見的考察(Scratchpad)】}

$\epsilon > 0$ が任意に与えられたとき、
\begin{equation}
\left| \frac{1}{n} - 0 \right| < \epsilon \iff \frac{1}{n} < \epsilon \iff n > \frac{1}{\epsilon}
\end{equation}

を成立させたい。したがって、$N = \left\lceil \frac{1}{\epsilon} \right\rceil$ ($\frac{1}{\epsilon}$ 以上の最小整数)とすれば、$n > N$ ならば $n > \frac{1}{\epsilon}$ となり、所望の不等式が得られる。

\subsection*{【形式的証明】}

\noindent
\textbf{定義:} $\lim_{n \to \infty} a_n = L$ とは、任意の $\epsilon > 0$ に対して、ある自然数 $N$ が存在して、$n > N$ ならば $|a_n - L| < \epsilon$ が成り立つことである。

\noindent
\textbf{証明:}

任意の $\epsilon > 0$ を固定する。$N = \left\lceil \frac{1}{\epsilon} \right\rceil$ と定める。

このとき、$n > N$ ならば、
\begin{align}
n &> N \geq \frac{1}{\epsilon} \\
\frac{1}{n} &< \epsilon \\
\left| \frac{1}{n} - 0 \right| &< \epsilon
\end{align}

したがって、定義により $\displaystyle \lim_{n \to \infty} \frac{1}{n} = 0$ が成立する。$\quad \blacksquare$

\setcounter{equation}{0}
\section*{5. ε-δ論法による極限証明}

\noindent
命題:$\displaystyle \lim_{x \to 1} x^2 = 1$

\subsection*{【発見的考察(Scratchpad)】}

$\epsilon > 0$ が与えられたとき、
\begin{equation}
|x^2 - 1| < \epsilon
\end{equation}

を成立させたい。ここで $|x^2 - 1| = |x-1||x+1|$ と因数分解できる。

$x$ が 1 の近くにあると仮定し、例えば $|x - 1| < 1$ と制限すると、$0 < x < 2$ となり、
\begin{equation}
|x + 1| < 3
\end{equation}

したがって、
\begin{equation}
|x^2 - 1| = |x-1||x+1| < 3|x-1|
\end{equation}

$|x^2 - 1| < \epsilon$ を得るには、
\begin{equation}
3|x-1| < \epsilon \iff |x-1| < \frac{\epsilon}{3}
\end{equation}

が十分である。よって、$\delta = \min\left\{1, \frac{\epsilon}{3}\right\}$ とすればよい。

\subsection*{【形式的証明】}

\noindent
\textbf{定義:} $\lim_{x \to a} f(x) = b$ とは、任意の $\epsilon > 0$ に対して、ある $\delta > 0$ が存在して、$0 < |x - a| < \delta$ ならば $|f(x) - b| < \epsilon$ が成り立つことである。

\noindent
\textbf{証明:}

任意の $\epsilon > 0$ を固定する。$\delta = \min\left\{1, \frac{\epsilon}{3}\right\}$ と定める。

$0 < |x - 1| < \delta$ ならば、
\begin{enumerate}
\item $|x - 1| < 1$ より $0 < x < 2$、したがって $|x + 1| < 3$
\item $|x - 1| < \frac{\epsilon}{3}$
\end{enumerate}

以上より、
\begin{align}
|x^2 - 1| &= |x - 1||x + 1| \\
&< 3 \cdot \frac{\epsilon}{3} \quad \text{(上記 (1), (2) より)} \\
&= \epsilon
\end{align}

したがって、定義により $\displaystyle \lim_{x \to 1} x^2 = 1$ が成立する。$\quad \blacksquare$

\end{document}
