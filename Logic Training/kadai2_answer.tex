% !TEX program = lualatex
\documentclass[a4paper,11pt]{ltjsreport}

%==============================================================================
% Packages & Settings
%==============================================================================
\usepackage{amsmath, amssymb, amsthm}
\usepackage[margin=2.5cm]{geometry}
\usepackage{bm}

% Redefine proof symbol
\renewcommand{\qedsymbol}{$\blacksquare$}

% Title Info
\title{論理トレーニング レポート課題}
\author{学籍番号 氏名}
\date{\today}

\begin{document}

\maketitle

\chapter*{問題}

\section*{1. 命題論理の同値変形}

$p, q, r, s$ を命題とするとき、次の (1), (2) を同値変形により示せ。

\begin{enumerate}
\item[(1)] 
\begin{equation*}
(p \land q) \\lor (r \land s) \equiv (p \\lor r) \land (p \\lor s) \land (q \\lor r) \land (q \\lor s)
\end{equation*}

\item[(2)] 
\begin{equation*}
(p \\lor q) \land (r \\lor s) \equiv (p \land r) \\lor (p \land s) \\lor (q \land r) \\lor (q \land s)
\end{equation*}
\end{enumerate}

\section*{2. 述語論理と量化子}

命題関数 $p(x), q(x) \ (x \in X)$ に対して、次が成り立つ。

\begin{enumerate}
\item[(1)] 
\begin{equation*}
\forall x \ p(x) \land \forall x \ q(x) \equiv \forall x \ (p(x) \land q(x))
\end{equation*}

\item[(2)] 
\begin{equation*}
\forall x \ p(x) \\lor \forall x \ q(x) \Rightarrow \forall x \ (p(x) \\lor q(x))
\end{equation*}

\item[(3)] 
\begin{equation*}
\exists x \ (p(x) \\lor q(x)) \equiv \exists x \ p(x) \\lor \exists x \ q(x)
\end{equation*}

\item[(4)] 
\begin{equation*}
\exists x \ (p(x) \land q(x)) \Rightarrow \exists x \ p(x) \land \exists x \ q(x)
\end{equation*}
\end{enumerate}

$X = \{a_1, a_2\}$ とするとき、同値変形により (2) を示せ。

\vspace{0.3cm}
\noindent
(Hint: $\forall x \ p(x) \lor \forall x \ q(x) \rightarrow \forall x \ (p(x) \lor q(x)) \equiv I$ を示す。)

\section*{3. 命題関数の真理値判定}

次の命題関数 $p(\epsilon, N)$ に対して、(1), (2) はそれぞれどんな命題か。また、その真理値を答えよ。
\begin{align*}
\epsilon &\in \{x \in \mathbb{R} \mid x > 0\} \\ 
N &\in \mathbb{N} \\ 
p(\epsilon, N) &: N\epsilon > 1
\end{align*}

\begin{enumerate}
\item[(1)] 
\begin{equation*}
\forall \epsilon \ \exists N \ p(\epsilon, N)
\end{equation*}

\item[(2)] 
\begin{equation*}
\overline{\forall \epsilon \ \exists N \ p(\epsilon, N)}
\end{equation*}
\end{enumerate}

\section*{4. ε-N論法による極限証明}

$\displaystyle \lim_{n \to \infty} \frac{1}{n} = 0$ であることを、ε-N論法を用いて証明せよ。

\section*{5. ε-δ論法による極限証明}

$\displaystyle \lim_{x \to 1} x^2 = 1$ であることを、ε-δ論法を用いて証明せよ。


\chapter*{解答}

\section*{1. 命題論理の同値変形}

$p, q, r, s$ を命題とする。

\subsection*{(1) $(p \land q) \lor (r \land s) \equiv (p \lor r) \land (p \lor s) \land (q \lor r) \land (q \lor s)$}

\begin{proof}
右辺に対し、分配法則 $A \lor (B \land C) \equiv (A \lor B) \land (A \lor C)$ を適用して整理する。
\begin{align*}
\text{右辺} 
&= \bigl[ (p \lor r) \land (p \lor s) \bigr] \land \bigl[ (q \lor r) \land (q \lor s) \bigr] \\
&= \bigl[ p \lor (r \land s) \bigr] \land \bigl[ q \lor (r \land s) \bigr] && (\because \text{Distributive Law}) \\
&= (r \land s) \lor (p \land q) && (\because \text{Distributive Law}) \\
&= (p \land q) \lor (r \land s) \\
&= \text{左辺}
\end{align*}
\end{proof}

\subsection*{(2) $(p \lor q) \land (r \lor s) \equiv (p \land r) \lor (p \land s) \lor (q \land r) \lor (q \land s)$}

\begin{proof}
左辺に対し、分配法則を順次適用して展開する。
\begin{align*}
\text{左辺} 
&= (p \lor q) \land (r \lor s) \\
&= \bigl( (p \lor q) \land r \bigr) \lor \bigl( (p \lor q) \land s \bigr) && (\because \text{Distributive Law}) \\
&= (p \land r) \lor (q \land r) \lor (p \land s) \lor (q \land s) && (\because \text{Distributive Law}) \\
&= (p \land r) \lor (p \land s) \lor (q \land r) \lor (q \land s) && (\because \text{Commutativity}) \\
&= \text{右辺}
\end{align*}
\end{proof}

\section*{2. 述語論理と量化子}

$X = \{a_1, a_2\}$ とする。

\begin{proof}
$X = \{a_1, a_2\}$ に対し、
\begin{align*}
\forall x \, p(x) &\equiv p(a_1) \land p(a_2), \quad \exists x \, p(x) \equiv p(a_1) \lor p(a_2)
\end{align*}
$p(a_i) = p_i, \ q(a_i) = q_i$ とおく。

\noindent
\textbf{左辺:}
\begin{align*}
\forall x \, p(x) \lor \forall x \, q(x) &\equiv (p_1 \land p_2) \lor (q_1 \land q_2)
\end{align*}

\noindent
\textbf{右辺:}
\begin{align*}
\forall x \, (p(x) \lor q(x)) &\equiv (p_1 \lor q_1) \land (p_2 \lor q_2) \\
&= (p_1 \land (p_2 \lor q_2)) \lor (q_1 \land (p_2 \lor q_2)) && (\text{分配}) \\
&= (p_1 \land p_2) \lor (p_1 \land q_2) \lor (q_1 \land p_2) \lor (q_1 \land q_2) && (\text{分配})
\end{align*}

\noindent
左辺は右辺に含まれるため、
\[
(p_1 \land p_2) \lor (q_1 \land q_2) \implies (p_1 \land p_2) \lor (p_1 \land q_2) \lor (q_1 \land p_2) \lor (q_1 \land q_2)
\]

$\therefore \ \text{左辺} \implies \text{右辺}$。
\end{proof}

\section*{3. 命題関数の真理値判定}

定義:$p(\epsilon, N) : N\epsilon > 1 \quad (\epsilon > 0, N \in \mathbb{N})$

\subsection*{(1) $\forall \epsilon \ \exists N \ p(\epsilon, N)$}

\textbf{意味:}「任意の正数 $\epsilon$ に対し、$N\epsilon > 1$ を満たす自然数 $N$ が存在する」

\textbf{真理値:真}

\begin{align*}
\forall \epsilon > 0, \ \exists N \in \mathbb{N} \text{ s.t. } N > \frac{1}{\epsilon}
\end{align*}

このとき,
\begin{align*}
N > \frac{1}{\epsilon} \implies N\epsilon > 1
\end{align*}
(Archimedean Property)

\subsection*{(2) $\overline{\forall \epsilon \ \exists N \ p(\epsilon, N)}$}

\textbf{真理値:偽}

\begin{align*}
\overline{\forall \epsilon \ \exists N \ p(\epsilon, N)} 
&\equiv \exists \epsilon \ \overline{\exists N \ p(\epsilon, N)} && (\text{De Morgan}) \\
&\equiv \exists \epsilon \ \forall N \ \overline{p(\epsilon, N)} && (\text{De Morgan}) \\
&\equiv \exists \epsilon > 0, \forall N \in \mathbb{N}, \overline{p(\epsilon, N)} \\
&\equiv \exists \epsilon > 0, \forall N \in \mathbb{N}, \overline{N\epsilon > 1} && (p(\epsilon, N) \text{ の定義より}) \\
&\equiv \exists \epsilon > 0, \forall N \in \mathbb{N}, N\epsilon \le 1 \\
&\equiv \exists \epsilon > 0, \forall N \in \mathbb{N}, N \le \frac{1}{\epsilon}
\end{align*}

\textbf{意味:}「ある正数 $\epsilon$ が存在して、全ての自然数 $N$ に対して $N\epsilon \le 1$ が成り立つ」

しかし、任意の実数 $M = \frac{1}{\epsilon} > 0$ に対して,$N > M$ を満たす自然数 $N$ が存在する(アルキメデスの公理)。
したがって、いかなる $\epsilon > 0$ に対しても、$N\epsilon > 1$ を満たす自然数が存在し,命題と矛盾する。
\begin{align*}
\therefore \text{(2) の命題は偽}
\end{align*}

\section*{4. $\epsilon-N$ 論法による極限証明}

命題:$\displaystyle \lim_{n \to \infty} \frac{1}{n} = 0$

\begin{proof}
$\forall \epsilon > 0$ に対し、アルキメデスの公理より
\[ \exists N \in \mathbb{N} \quad \text{s.t.} \quad N > \frac{1}{\epsilon} \]
が存在する。任意の $n > N$ について、
\begin{align*}
n > N > \frac{1}{\epsilon} \implies \frac{1}{n} < \epsilon
\end{align*}
よって、
\[ \left| \frac{1}{n} - 0 \right| = \frac{1}{n} < \epsilon \]
以上より、$\displaystyle \lim_{n \to \infty} \frac{1}{n} = 0$。
\end{proof}

\section*{5. $\epsilon-\delta$ 論法による極限証明}

\subsection*{【ノート】$\delta$ の導出過程}

$|x - 1| < \delta$ ならば $|x^2 - 1| < \epsilon$ となる $\delta$ を逆算する。

\begin{align*}
|x^2 - 1| &= |(x - 1)(x + 1)| = |x - 1| \cdot |x + 1|
\end{align*}

$\delta \le 1$ ととれば、
\begin{align*}
|x - 1| < 1 &\implies -1 < x - 1 < 1 \\
&\implies 0 < x < 2 \\
&\implies |x + 1| < 3
\end{align*}

したがって $|x^2 - 1| = |x - 1| \cdot |x + 1| < 3|x - 1|$。これが $\epsilon$ より小さくするには、$|x - 1| < \frac{\epsilon}{3}$ が必要。両条件を満たすため,
\[
\delta = \min\left\{1, \frac{\epsilon}{3}\right\}
\]

\subsection*{【証明】}

命題:$\displaystyle \lim_{x \to 1} x^2 = 1$

\begin{proof}
$\forall \epsilon > 0$ に対し、$\delta = \min\left\{1, \frac{\epsilon}{3}\right\}$ とおく。

$0 < |x - 1| < \delta$ なる $x$ について評価を行う。
\begin{enumerate}
    \item $\delta \le 1$ より、
    \[ |x - 1| < 1 \implies 0 < x < 2 \implies |x + 1| < 3 \]
    \item $\delta \le \frac{\epsilon}{3}$ より、
    \[ |x - 1| < \frac{\epsilon}{3} \]
\end{enumerate}
したがって、
\begin{align*}
|x^2 - 1| &= |x - 1||x + 1| \\
&< \frac{\epsilon}{3} \cdot 3 \\
&= \epsilon
\end{align*}
$\therefore \displaystyle \lim_{x \to 1} x^2 = 1$
\end{proof}

\end{document}