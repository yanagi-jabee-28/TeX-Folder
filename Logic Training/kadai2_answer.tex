% !TEX program = lualatex
\documentclass[a4paper,11pt]{ltjsreport}

% 最小限のパッケージのみ読み込み
\usepackage{amsmath, amssymb}
\usepackage[margin=2.5cm]{geometry}
\usepackage{bm}

% 便利なコマンド
\newcommand{\qedbox}{\hfill $\blacksquare$}

\title{論理トレーニング レポート課題}
\author{氏名}
\date{\today}

\begin{document}

\maketitle

\chapter*{解答}

\section*{1. 命題論理の同値変形}

\subsection*{(1) $(p \land q) \lor (r \land s) \equiv (p \lor r) \land (p \lor s) \land (q \lor r) \land (q \lor s)$}

\textbf{証明:}
右辺を展開して、左辺と一致することを示す。
右辺に対し、分配法則 $(A \lor B) \land (A \lor C) \equiv A \lor (B \land C)$ を適用して整理する。
\begin{align*}
(\text{右辺}) 
&= \bigl( (p \lor r) \land (p \lor s) \bigr) \land \bigl( (q \lor r) \land (q \lor s) \bigr) \\
&= \bigl( p \lor (r \land s) \bigr) \land \bigl( q \lor (r \land s) \bigr) \quad \text{(分配法則の逆向き)} \\
&= (r \land s) \lor (p \land q) \quad \text{(分配法則)} \\
&= (p \land q) \lor (r \land s) \\
&= (\text{左辺})
\end{align*}
よって示された。\qedbox

\subsection*{(2) $(p \lor q) \land (r \lor s) \equiv (p \land r) \lor (p \land s) \lor (q \land r) \lor (q \land s)$}

\textbf{証明:}
左辺に対し、分配法則を順次適用して展開する。
\begin{align*}
(\text{左辺}) 
&= (p \lor q) \land (r \lor s) \\
&= \bigl( (p \lor q) \land r \bigr) \lor \bigl( (p \lor q) \land s \bigr) \quad \text{(分配法則)} \\
&= (p \land r) \lor (q \land r) \lor (p \land s) \lor (q \land s) \quad \text{(分配法則)} \\
&= (p \land r) \lor (p \land s) \lor (q \land r) \lor (q \land s) \quad \text{(交換法則・結合法則)} \\
&= (\text{右辺})
\end{align*}
よって示された。\qedbox

\section*{2. 述語論理と量化子}

$X = \{a_1, a_2\}$ のとき、以下を示す。
\[ \forall x \, p(x) \lor \forall x \, q(x) \implies \forall x \, (p(x) \lor q(x)) \]

\textbf{証明:}
$p(a_i)$ を $p_i$、$q(a_i)$ を $q_i$ と略記する。
このとき、それぞれの命題は以下のように書き下せる。
\begin{align*}
\text{左辺} (\text{仮定}) &: (p_1 \land p_2) \lor (q_1 \land q_2) \\
\text{右辺} (\text{結論}) &: (p_1 \lor q_1) \land (p_2 \lor q_2)
\end{align*}

ここで、結論(右辺)を分配法則で展開してみる。
\begin{align*}
(\text{右辺}) 
&= (p_1 \lor q_1) \land (p_2 \lor q_2) \\
&= (p_1 \land p_2) \lor (p_1 \land q_2) \lor (q_1 \land p_2) \lor (q_1 \land q_2) \quad \text{(分配法則)}
\end{align*}
展開した式の中に、左辺の「$(p_1 \land p_2)$」と「$(q_1 \land q_2)$」が含まれていることがわかる。
「$A$ ならば $A \lor B$」は恒真(同一律・包含律)であるため、
\[ (\text{左辺}) \implies (\text{右辺}) \]
は成り立つ。\qedbox

\section*{3. 命題関数の真理値判定}

$p(\epsilon, N) : N\epsilon > 1$ とする。

\subsection*{(1) $\forall \epsilon \, \exists N \, p(\epsilon, N)$}

\textbf{答え:真}

\textbf{理由:}
どんなに小さな正の数 $\epsilon$ に対しても、それより逆数が大きい自然数 $N$(つまり $N > 1/\epsilon$)を選ぶことができる(アルキメデスの公理)からである。

\subsection*{(2) $\overline{\forall \epsilon \, \exists N \, p(\epsilon, N)}$}

この命題は否定をとると、「ある $\epsilon$ があって、すべての $N$ で $N\epsilon \le 1$」という意味になる。

\textbf{答え:偽}

\textbf{理由:}
もしこれが正しいとすると、ある数 $\epsilon$ に対して $N \le 1/\epsilon$ が全ての自然数 $N$ で成り立つことになってしまう。しかし、自然数 $N$ はいくらでも大きくできるため、上限があることはありえない。よって矛盾するので偽である。

\section*{4. $\epsilon-N$ 論法による極限証明}

$\displaystyle \lim_{n \to \infty} \frac{1}{n} = 0$ を示す。

\textbf{証明:}
任意の $\epsilon > 0$ をとる。
これに対して、$\frac{1}{\epsilon} < N$ となる自然数 $N$ をひとつ選ぶ(アルキメデスの公理により、このような $N$ は必ず存在する)。

このとき、$n > N$ となる全ての $n$ について、
\[ n > N > \frac{1}{\epsilon} \]
が成り立つ(実数の順序の推移律)。逆数をとると不等号が逆転して、
\[ \frac{1}{n} < \epsilon \]
となる(正の数の逆数をとると不等号が反転する性質)。したがって、
\[ \left| \frac{1}{n} - 0 \right| = \frac{1}{n} < \epsilon \]
が言えるので、極限値は $0$ である。\qedbox

\section*{5. $\epsilon-\delta$ 論法による極限証明}

$\displaystyle \lim_{x \to 1} x^2 = 1$ を示す。

\textbf{証明:}
任意の $\epsilon > 0$ をとる。
ここで、$\delta = \min\left\{1, \frac{\epsilon}{3}\right\}$ と定める。

$|x - 1| < \delta$ のとき、$|x^2 - 1|$ がどうなるか計算する。
まず、$|x^2 - 1| = |x - 1| \, |x + 1|$ である(因数分解と絶対値の積の性質)。

さらに $|x-1|<1$ を仮定すれば $0<x<2$ となり、よって $|x+1|<3$ が得られる(実数の順序の性質)。
したがって
\[
|x^2-1|=|x-1||x+1|<3|x-1|
\]
となる。ここで $|x^2-1|<\epsilon$ を保証したいので、
\[
3|x-1|<\epsilon \quad \Longleftrightarrow \quad |x-1|<\frac{\epsilon}{3}
\]
を満たせば十分である(不等式の同値変形)。

1. $\delta \le 1$ なので、$|x - 1| < 1$ である。
   これより $0 < x < 2$ となるので、$|x + 1| < 3$ と評価できる。
2. また、$\delta \le \frac{\epsilon}{3}$ なので、$|x - 1| < \frac{\epsilon}{3}$ である。

以上より、
\[ |x^2 - 1| = |x - 1| \, |x + 1| < \frac{\epsilon}{3} \cdot 3 = \epsilon \quad \text{(積の不等式)} \]
となり、$|x^2 - 1| < \epsilon$ が示された。
よって、極限値は $1$ である。\qedbox

\end{document}