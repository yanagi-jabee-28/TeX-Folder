% !TEX program = lualatex
\documentclass[a4paper,11pt]{ltjsarticle}
\usepackage{amsmath,amssymb}
\usepackage{newtxmath}
\usepackage[margin=25mm]{geometry}
\usepackage[hidelinks,colorlinks=true,linkcolor=blue]{hyperref}

\title{演習問題 解答 — 述語論理と量化子 (2)}
\author{}
\date{}

\begin{document}
\maketitle

\section*{問題 2}
次の命題を示せ:
\[
\forall x\,p(x)\ \lor\ \forall x\,q(x)\ \Longrightarrow\ \forall x\,(p(x)\lor q(x)).
\]

\subsection*{【思考プロセス(下書き)】}
$X=\{a_1,a_2\}$ という有限集合なので,量化子 $\forall$ を論理積 $\land$ に書き換えて命題論理に落とし込む.
題意は LHS $\Rightarrow$ RHS の証明である.
LHS(仮定)は $p$ が全員真 または $q$ が全員真.
RHS(結論)は 各個体について $p$ か $q$ のどちらかが真.
表記簡略化のため $p_i:=p(a_i),\ q_i:=q(a_i)\ (i=1,2)$ とおく.

方針:RHS を分配律で展開し,その中に LHS の項が含まれていることを示す(命題論理の恒真式 $A\implies A\lor B$ を用いる)。

\subsection*{【解答(清書)】}
$p_i=p(a_i),\ q_i=q(a_i)\ (i=1,2)$ とおく.定義より
\begin{align*}
\mathrm{LHS} &= (p_1\land p_2)\lor (q_1\land q_2),\\
\mathrm{RHS} &= (p_1\lor q_1)\land(p_2\lor q_2).
\end{align*}
RHS を分配律で展開すると
\begin{align*}
\mathrm{RHS}
&\equiv ((p_1\lor q_1)\land p_2)\lor((p_1\lor q_1)\land q_2)\\
&\equiv (p_1\land p_2)\lor(q_1\land p_2)\lor(p_1\land q_2)\lor(q_1\land q_2)\\
&\equiv (p_1\land p_2)\lor(q_1\land q_2)\lor(p_1\land q_2)\lor(q_1\land p_2).
\end{align*}
ここで $(p_1\land p_2)\lor(q_1\land q_2)$ は仮定(LHS)に他ならない.命題論理において $A\implies A\lor B$ は恒真であるから,
\[
\mathrm{LHS}\implies\mathrm{RHS}
\]
が成り立つ.すなわち
\[
\forall x\,p(x)\ \lor\ \forall x\,q(x)\ \Longrightarrow\ \forall x\,(p(x)\lor q(x))
\]
が示された.\hfill$\blacksquare$

\end{document}
