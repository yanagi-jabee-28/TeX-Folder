\documentclass[11pt,a4paper]{ltjsarticle}

% --- 基本パッケージ ---
\usepackage{luatexja}
\usepackage{luatexja-fontspec}
\usepackage{amsmath,amssymb}
\usepackage{geometry}
\usepackage{graphicx}
\usepackage{booktabs}
\usepackage{fancyhdr}
\usepackage{lmodern} % スケーラブルなフォント
\usepackage{bm} % 太字の数式用

% --- レイアウト・デザイン ---
\usepackage[most]{tcolorbox}
\usepackage{tikz}
\usepackage{pgfplots}
\pgfplotsset{compat=1.18}

% --- フォント設定 ---
\setmainfont{Times New Roman}
\setsansfont{Arial}
\setmonofont{Consolas}
\setmainjfont{Yu Mincho}
\setsansjfont{Yu Gothic}

% --- ページ設定 ---
\geometry{left=2cm,right=2cm,top=2.5cm,bottom=2.5cm}
\pagestyle{fancy}
\fancyhf{}
\fancyhead[L]{\small 制御工学演習:安定判別法マスター}
\fancyhead[R]{\small \thepage}
\fancyfoot[C]{\small 長野高専 電気電子工学科}
\renewcommand{\headrulewidth}{0.4pt}
\renewcommand{\footrulewidth}{0.4pt}

% --- tcolorbox の設定 ---
\tcbset{
    colback=blue!5!white,
    colframe=blue!75!black,
    fonttitle=\bfseries,
    arc=2mm,
    boxrule=1pt
}

\newtcolorbox{theorem}[1]{
    title={#1},
    colback=green!5!white,
    colframe=green!50!black
}

\newtcolorbox{example}[1]{
    title={#1},
    colback=orange!5!white,
    colframe=orange!80!black
}


\begin{document}

\begin{center}
    {\Huge \bfseries 制御工学における安定判別法} \\
    \vspace{2mm}
    {\Large ラウス・フルビッツ法とナイキスト法の徹底解説}
\end{center}

\vspace{5mm}

\begin{tcolorbox}{はじめに:なぜ安定性解析が重要なのか?}
    制御システムを設計する上で、最も基本的な要件は「安定性」です。システムが不安定だと、入力に対して出力が発散してしまい、制御不能な状態に陥ります。例えば、ロボットアームが暴走したり、温度制御システムが異常加熱したりする原因となります。
    この資料では、システムの安定性を数学的に判別するための2つの強力な手法、「ラウス・フルビッツの安定判別法」と「ナイキストの安定判別法」について、例題を通して体系的に学びます。
\end{tcolorbox}

\section{問題1:ゲインKの安定範囲の導出}

\begin{example}{問題設定}
    フィードバック制御系の開ループ伝達関数が $G(s) = \frac{K}{s(s^2+3s+12)}$、フィードバックが単位フィードバック $H(s)=1$ で与えられる。
    \vspace{2mm}
    このシステムが安定となるためのゲイン $K$ の範囲を求めよ。
\end{example}

\subsection{アプローチ1:ラウス・フルビッツの安定判別法}

\begin{tcolorbox}{ラウス法とは?}
    ラウス・フルビッツの安定判別法は、システムの\textbf{特性方程式}の係数だけを使って、安定性を判断する代数的な手法です。わざわざ方程式の根を解かなくても、根が複素平面の右半面(不安定領域)に存在するかどうかを判定できます。
    \textbf{キーポイント:} ラウス配列と呼ばれる表を作成し、その第1列の符号を調べる。
\end{tcolorbox}

\subsubsection{Step 1: 特性方程式を求める}
安定性は、閉ループ伝達関数の分母、すなわち特性方程式 $1+G(s)H(s)=0$ の根で決まります。
分母を払って整理すると、以下の特性方程式が得られます。
\begin{align*}
    &1 + \frac{K}{s(s^2+3s+12)} \cdot 1 = 0 \\
    &s(s^2+3s+12) + K = 0 \\
    &\implies \bm{s^3 + 3s^2 + 12s + K = 0}
\end{align*}

\subsubsection{Step 2: ラウス配列を構築する}
特性方程式 $s^3+3s^2+12s+K=0$ の係数 $(a_3=1, a_2=3, a_1=12, a_0=K)$ を使って、ラウス配列を作成します。

\begin{center}
\begin{tabular}{c|cc}
\toprule
$s^3$ & $a_3=1$ & $a_1=12$ \\
$s^2$ & $a_2=3$ & $a_0=K$ \\
$s^1$ & $b_1$ & $b_2$ \\
$s^0$ & $c_1$ & $c_2$ \\
\bottomrule
\end{tabular}
\end{center}

$b_1, c_1$ を計算します。
\begin{align*}
b_1 &= \frac{a_2 \cdot a_1 - a_3 \cdot a_0}{a_2} = \frac{3 \cdot 12 - 1 \cdot K}{3} = \frac{36-K}{3} \\
c_1 &= \frac{b_1 \cdot a_0 - a_2 \cdot b_2}{b_1} = \frac{\frac{36-K}{3} \cdot K - 3 \cdot 0}{\frac{36-K}{3}} = K
\end{align*}

完成したラウス配列は以下のようになります。

\begin{center}
\begin{tabular}{c|cc}
\toprule
$s^3$ & 1 & 12 \\
$s^2$ & 3 & $K$ \\
$s^1$ & $\frac{36-K}{3}$ & 0 \\
$s^0$ & $K$ & 0 \\
\bottomrule
\end{tabular}
\end{center}

\subsubsection{Step 3: 安定条件を適用する}
\begin{theorem}{ラウスの安定条件}
    システムが安定であるための必要十分条件は、\textbf{ラウス配列の第1列の要素がすべて正である}こと。
\end{theorem}

第1列の要素 $1, 3, \frac{36-K}{3}, K$ がすべて正である必要があります。
\begin{enumerate}
    \item $1 > 0$ (OK)
    \item $3 > 0$ (OK)
    \item $\frac{36-K}{3} > 0 \implies 36-K > 0 \implies \bm{K < 36}$
    \item $K > 0 \implies \bm{K > 0}$
\end{enumerate}
これらの条件をすべて満たす $K$ の範囲は $0 < K < 36$ となります。

\begin{tcolorbox}[title=ラウス法による結論, colback=yellow!10!white, colframe=yellow!75!black]
    ラウス・フルビッツの安定判別法により、システムが安定となるゲイン $K$ の範囲は $\mathbf{0 < K < 36}$ である。
\end{tcolorbox}

\clearpage
\subsection{アプローチ2:ナイキストの安定判別法}

\begin{tcolorbox}{ナイキスト法とは?}
    ナイキストの安定判別法は、\textbf{開ループ伝達関数の周波数応答} $G(j\omega)H(j\omega)$ のベクトル軌跡(ナイキスト線図)を描き、それが複素平面上の点 $(-1, 0)$ をどのように囲むかによって安定性を判断する図形的な手法です。
    \textbf{キーポイント:} $Z = N+P$ という関係式に基づき、閉ループ系の安定性を判定する。
\end{tcolorbox}

\subsubsection{Step 1: 不安定な開ループ極の数 P を求める}
開ループ伝達関数 $G(s)H(s) = \frac{K}{s(s^2+3s+12)}$ の極(分母=0の根)を調べます。
$s(s^2+3s+12)=0$ の極は $s=0$ と $s^2+3s+12=0$ の解です。$s^2+3s+12=0$ の解は $s = -\frac{3}{2} \pm j\frac{\sqrt{39}}{2}$ です。
これらの極はすべて複素平面の左半面または虚軸上にあります。したがって、右半面にある不安定な開ループ極の数は $\bm{P=0}$ です。

\subsubsection{Step 2: 安定条件を決定する}
\begin{theorem}{ナイキストの安定条件}
    閉ループ系が安定である条件は、閉ループ系の不安定な極の数 $Z$ がゼロであること、すなわち $\bm{Z=0}$。
    ナイキストの定理 $Z=N+P$ より、安定条件は $\bm{N = -P}$ となる。
\end{theorem}
今回は $P=0$ なので、安定条件は $\bm{N=0}$ となります。これは、\textbf{ナイキスト線図が点(-1, 0)を囲まない}ことを意味します。

\subsubsection{Step 3: ナイキスト線図が実軸と交わる点を計算する}
$s=j\omega$ を代入して周波数伝達関数を求め、実部と虚部に分けます。
\begin{align*}
G(j\omega) &= \frac{K}{j\omega(-\omega^2+3j\omega+12)} = \frac{K}{(12-\omega^2)j\omega - 3\omega^2}\\
\text{有理化すると、}& \\
G(j\omega) &= \frac{K}{-3\omega^2 + j\omega(12-\omega^2)} \cdot \frac{-3\omega^2 - j\omega(12-\omega^2)}{-3\omega^2 - j\omega(12-\omega^2)} = \frac{K(-3\omega^2 - j\omega(12-\omega^2))}{9\omega^4 + \omega^2(12-\omega^2)^2} \\
\text{Re}[G(j\omega)] &= \frac{-3K\omega^2}{9\omega^4 + \omega^2(12-\omega^2)^2} \\
\text{Im}[G(j\omega)] &= \frac{-K\omega(12-\omega^2)}{9\omega^4 + \omega^2(12-\omega^2)^2}
\end{align*}
軌跡が実軸と交わるのは、虚部 $\text{Im}[G(j\omega)]=0$ のときです。
$12-\omega^2 = 0 \implies \bm{\omega^2 = 12} \implies \omega = \sqrt{12} = 2\sqrt{3}$

このときの実部の値は、
\begin{align*}
\text{Re}|_{\omega^2=12} &= \frac{-3K(12)}{9(12)^2 + 12(12-12)^2} \\
&= \frac{-36K}{9 \cdot 144 + 0} = \frac{-36K}{1296} = \bm{-\frac{K}{36}}
\end{align*}

\subsubsection{Step 4: 安定条件を適用する}
安定条件 $N=0$ を満たすには、実軸との交点 $-\frac{K}{36}$ が、臨界点 $-1$ よりも右側にある必要があります。
$$-\frac{K}{36} > -1$$
両辺に $-36$ を掛けると、不等号の向きが反転します。
$$\bm{K < 36}$$
ゲイン $K$ は正の値なので、$K>0$ も考慮します。

\begin{tcolorbox}[title=ナイキスト法による結論, colback=yellow!10!white, colframe=yellow!75!black]
    ナイキストの安定判別法により、システムが安定となるゲイン $K$ の範囲は $\mathbf{0 < K < 36}$ である。
\end{tcolorbox}

\section{結論の比較と検証}
ラウス法、ナイキスト法という全く異なる2つのアプローチから、同じ $\mathbf{0 < K < 36}$ という結論が得られました。これにより、解析の正しさが強く裏付けられます。
特に、ラウス法で安定限界を示した $K=36$ は、ナイキスト線図がちょうど点 $(-1,0)$ を通過するゲインと一致しており、両手法の関連性の深さを示しています。

\clearpage
\section{問題2:システムの安定性判別}
\begin{example}{問題設定}
    開ループ伝達関数が $G(s) = \frac{6}{s(s+2)(s+5)}$, $H(s)=1$ で与えられるシステムの安定性を判別せよ。
\end{example}

\subsection{ラウス・フルビッツ法による解析}
\subsubsection{Step 1: 特性方程式}
\begin{align*}
&1 + \frac{6}{s(s+2)(s+5)} = 0 \\
&s(s^2+7s+10) + 6 = 0 \\
&\implies \bm{s^3 + 7s^2 + 10s + 6 = 0}
\end{align*}

\subsubsection{Step 2: ラウス配列}
係数 $(1, 7, 10, 6)$ を用いて配列を作成します。
\begin{align*}
b_1 &= \frac{7 \cdot 10 - 1 \cdot 6}{7} = \frac{64}{7} \\
c_1 &= 6
\end{align*}
\begin{center}
\begin{tabular}{c|cc}
\toprule
$s^3$ & 1 & 10 \\
$s^2$ & 7 & 6 \\
$s^1$ & $\frac{64}{7}$ & 0 \\
$s^0$ & 6 & 0 \\
\bottomrule
\end{tabular}
\end{center}

\subsubsection{Step 3: 判定}
第1列の要素 $(1, 7, \frac{64}{7}, 6)$ は\textbf{すべて正}です。
\begin{tcolorbox}[title=ラウス法による結論, colback=green!10!white, colframe=green!75!black]
    ラウス配列の第1列の符号がすべて正であるため、このシステムは\textbf{安定}である。
\end{tcolorbox}

\subsection{ナイキスト法による解析}
\subsubsection{Step 1: Pの算定}
開ループ極は $s=0, -2, -5$ であり、すべて左半面または虚軸上にあるため $\bm{P=0}$。
安定条件は $\bm{N=0}$ となります。

\subsubsection{Step 2: 実軸との交点}
$s=j\omega$ を代入して周波数伝達関数を求めます。分母を展開し、実部と虚部に整理します。
\begin{align*}
G(j\omega) &= \frac{6}{j\omega(j\omega+2)(j\omega+5)} \\
&= \frac{6}{j\omega(-\omega^2 + 7j\omega + 10)} \\
&= \frac{6}{-j\omega^3 - 7\omega^2 + 10j\omega} \\
&= \frac{6}{-7\omega^2 + j\omega(10 - \omega^2)}
\end{align*}
この式から、虚部がゼロになるのは $10-\omega^2=0 \implies \omega = \sqrt{10}$ のときです。

このときの実部の値は,
\begin{align*}
\text{Re}[G(j\omega)]|_{\omega^2=10} &= \frac{-42\omega^2}{49\omega^4 + \omega^2(10-\omega^2)^2} |_{\omega^2=10} \\
&= \frac{-42(10)}{49(10)^2 + 0} = \frac{-420}{4900} = \bm{-\frac{3}{35}}
\end{align*}

\subsubsection{Step 3: ナイキスト線図の主要な点の評価と安定性判別}
ナイキスト線図の概形を把握するために、いくつかの重要な周波数における$G(j\omega)$の座標を求めます。

\paragraph{1. $\omega \to 0^+$ (始点)}
$\omega$がゼロに近づくとき、伝達関数は積分要素$1/s$の影響で支配されます。
\begin{align*}
    \lim_{\omega \to 0^+} G(j\omega) = \lim_{\omega \to 0^+} \frac{6}{j\omega(2)(5)} = \lim_{\omega \to 0^+} \frac{0.6}{j\omega} = \lim_{\omega \to 0^+} -j\frac{0.6}{\omega}
\end{align*}
これは、ベクトル軌跡が虚軸の負の無限遠から出発することを示します。座標は $\left(0, -\infty\right)$ に相当します。

\paragraph{2. $\omega = \omega_{\pi} = \sqrt{10}$ (実軸との交点)}
虚部がゼロになる周波数 $\omega_{\pi} = \sqrt{10}$ での座標は、既に計算した通りです。
$ G(j\sqrt{10}) = -\frac{3}{35} \approx -0.086 $
この点の座標は $\left(-\frac{3}{35}, 0\right)$ です。

\paragraph{3. $\omega \to \infty$ (終点)}
$\omega$が非常に大きくなるとき、分母の次数が分子より3次高いため、値はゼロに収束します。
\begin{align*}
    \lim_{\omega \to \infty} G(j\omega) = \lim_{\omega \to \infty} \frac{6}{(j\omega)^3} = \lim_{\omega \to \infty} \frac{6}{-j\omega^3} = 0
\end{align*}
これは、ベクトル軌跡が原点 $\left(0,0\right)$ に収束することを示します。

\paragraph{安定性の判定}
実軸との交点 $(-\frac{3}{35}, 0)$ は、臨界点 $(-1, 0)$ よりも右側にあります。
$P=0$ であり、ナイキスト線図は $(-1,0)$ を囲まないため、$N=0$です。
$Z = N + P = 0 + 0 = 0$ となり、閉ループ系は安定です。

\begin{tcolorbox}[title=ナイキスト法による結論, colback=green!10!white, colframe=green!75!black]
    ナイキスト線図が臨界点$(-1,0)$を囲まないため、このシステムは\textbf{安定}である。
\end{tcolorbox}

\begin{figure}[htbp]
\centering
\caption{$G(s) = \frac{6}{s(s+2)(s+5)}$ のナイキスト線図}
\label{fig:nyquist-additional}
\begin{tikzpicture}
\begin{axis}[
    width=12cm,
    height=10cm,
    xlabel={実部 Re},
    ylabel={虚部 Im},
    grid=both,
    xmin=-1.2, xmax=0.2,
    ymin=-5, ymax=5,
    axis equal image,
    legend pos=outer north east,
]

\addplot[domain=0.1:10, samples=200, thick, blue, variable=t]
plot ({-42*t^2 / (49*t^4 + t^2*(10-t^2)^2)}, {-6*t*(10-t^2) / (49*t^4 + t^2*(10-t^2)^2)});
\addlegendentry{ナイキスト軌跡 ($\omega: 0 \to 10$)}

\addplot[mark=*, mark size=3pt, red] coordinates {(-1, 0)};
\addlegendentry{臨界点 (-1,0)}
\node[anchor=north, red] at (axis cs:-1, -0.05) {$(-1,0)$};

\addplot[mark=*, mark size=2pt, green!60!black] coordinates {(-0.0857, 0)};
\addlegendentry{実軸交点 (-3/35, 0)}
\node[anchor=south, green!60!black] at (axis cs:-0.0857, 0.05) {$\left(-\frac{3}{35}, 0\right)$};

\end{axis}
\end{tikzpicture}
\end{figure}

\end{document}