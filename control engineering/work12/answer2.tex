\documentclass[11pt,a4paper]{ltjsarticle}
\usepackage{luatexja}
\usepackage{luatexja-fontspec}
\usepackage{amsmath,amssymb}
\usepackage{geometry}
\geometry{left=2.5cm,right=2.5cm,top=3cm,bottom=3cm}
\usepackage{booktabs}
\usepackage{fancyhdr}
\usepackage{fontspec}
% 欧文フォント設定
\setmainfont{Times New Roman}
% 日本語フォント設定
\setmainjfont{Yu Mincho}
\setsansjfont{Yu Gothic}

\pagestyle{fancy}
\fancyhead{}
\fancyhead[R]{\footnotesize
  制御工学における安定判別法の演習 \\
  長野高専 電気電子工学科 5年 34番 栁原魁人 \\
  2025年7月22日
}
\setlength{\headheight}{34.832pt}

\begin{document}
\title{制御工学における安定判別法の演習}
\author{長野高専 電気電子工学科 5年 34番 栁原魁人}
\date{2025年7月22日}
\maketitle
\thispagestyle{fancy}

\section{伝達関数の極に基づく安定判別}

\subsection{問題1(1) の解答と解説}

\textbf{問題:} 以下の伝達関数の安定性を,極を求めて判定せよ.

$$\frac{1}{s^2-2s+3}$$

\textbf{解答プロセス:}

\begin{enumerate}
\item 特性方程式の特定:\\
伝達関数の分母多項式をゼロとおくことで,特性方程式を得る.
$$s^2-2s+3=0$$

\item 極の計算:\\
この2次方程式を解の公式 $s=\frac{-b\pm\sqrt{b^2-4ac}}{2a}$ を用いて解く.
$$s = \frac{-(-2) \pm \sqrt{(-2)^2 - 4 \cdot 1 \cdot 3}}{2 \cdot 1}$$
$$s = \frac{2 \pm \sqrt{4 - 12}}{2} = \frac{2 \pm \sqrt{-8}}{2}$$
$$s = \frac{2 \pm j2\sqrt{2}}{2} = 1 \pm j\sqrt{2}$$

\item 極の配置と安定性の判定:\\
得られた極は $s_1=1+j\sqrt{2}$ と $s_2=1-j\sqrt{2}$ である.\\
これらの極の実数部は共に $\mathrm{Re}(s)=+1$ であり,正の値である.\\
これは,極が複素$s$平面の右半平面(RHP)に存在することを示す.
\end{enumerate}

\textbf{結論:}\\
右半平面に極が存在するため,このシステムは不安定である.

\subsection{問題1(2) の解答と解説}

\textbf{問題:} 以下の伝達関数の安定性を,極を求めて判定せよ.

$$\frac{1}{s^3+4s^2+7s+6}$$

\textbf{解答プロセス:}

\begin{enumerate}
\item 特性方程式の特定:\\
特性方程式は以下のようになる.
$$P(s)=s^3+4s^2+7s+6=0$$

\item 極の計算(高次方程式の因数分解):\\
3次以上の方程式では,まず有理根定理などを用いて整数の根を探すのが一般的である.定数項6の約数($\pm1,\pm2,\pm3,\pm6$)を候補として代入する.
\begin{itemize}
\item $s=-1$ を試す: $P(-1)=(-1)^3+4(-1)^2+7(-1)+6=-1+4-7+6=2\neq0$
\item $s=-2$ を試す: $P(-2)=(-2)^3+4(-2)^2+7(-2)+6=-8+16-14+6=0$
\end{itemize}
$s=-2$ が根であることがわかったため,$(s+2)$ が $P(s)$ の因数であることがわかる.次に,多項式の割り算(組立除法など)を行い,残りの因子を求める.

$$(s^3+4s^2+7s+6)\div(s+2)=s^2+2s+3$$

これにより,特性方程式は次のように因数分解できる.
$$(s+2)(s^2+2s+3)=0$$

\item すべての極の導出:\\
各因数がゼロとなる条件から,すべての極を求める.
\begin{itemize}
\item 第1の因子から: $s+2=0\Rightarrow s_1=-2$
\item 第2の因子から: $s^2+2s+3=0$ を解の公式で解く.
$$s = \frac{-2 \pm \sqrt{2^2 - 4 \cdot 1 \cdot 3}}{2 \cdot 1} = \frac{-2 \pm \sqrt{4 - 12}}{2} = \frac{-2 \pm \sqrt{-8}}{2}$$
$$s = \frac{-2 \pm j2\sqrt{2}}{2} = -1 \pm j\sqrt{2}$$
\end{itemize}

したがって,3つの極は $s_1=-2$, $s_2=-1+j\sqrt{2}$, $s_3=-1-j\sqrt{2}$ となる.

\item 極の配置と安定性の判定:\\
各極の実数部を確認する.
\begin{itemize}
\item $\mathrm{Re}(s_1)=-2$ (負)
\item $\mathrm{Re}(s_2)=-1$ (負)
\item $\mathrm{Re}(s_3)=-1$ (負)
\end{itemize}
すべての極の実数部が負であり,$s$平面の左半平面(LHP)に配置されている.
\end{enumerate}

\textbf{結論:}\\
すべての極が左半平面に存在するため,このシステムは安定である.

\section{ラウスの安定判別法}

\subsection{問題3(1) の解答と解説}

\textbf{問題:} 特性方程式が $s^4+3s^3+4s^2+3s+2=0$ で与えられる伝達関数の安定性を,ラウスの安定判別法から判定せよ.

\textbf{解答プロセス:}

\begin{enumerate}
\item ラウス配列の構成:\\
特性方程式の係数は $a_4=1, a_3=3, a_2=4, a_1=3, a_0=2$ である.

\begin{table}[ht]
\centering
\caption{ラウス配列の構成過程}
\label{tbl:routh1}
\begin{tabular}{cccc}
\toprule
$s^4$ & 1 & 4 & 2 \\
$s^3$ & 3 & 3 & 0 \\
$s^2$ & $b_1$ & $b_2$ & \\
$s^1$ & $c_1$ & & \\
$s^0$ & $d_1$ & & \\
\bottomrule
\end{tabular}
\end{table}

各要素を計算する.
\begin{itemize}
\item $s^2$ の行:
$$b_1 = \frac{3 \cdot 4 - 1 \cdot 3}{3} = \frac{9}{3} = 3$$
$$b_2 = \frac{3 \cdot 2 - 1 \cdot 0}{3} = \frac{6}{3} = 2$$
\item $s^1$ の行:
$$c_1 = \frac{3 \cdot 3 - 3 \cdot 2}{3} = \frac{3}{3} = 1$$
\item $s^0$ の行:
$$d_1 = \frac{1 \cdot 2 - 3 \cdot 0}{1} = 2$$
\end{itemize}

\item 完成したラウス配列と安定性の判定:\\
完成した配列は表\ref{tbl:routh1_complete}の通りである.

\begin{table}[ht]
\centering
\caption{完成したラウス配列}
\label{tbl:routh1_complete}
\begin{tabular}{cccc}
\toprule
$s^4$ & 1 & 4 & 2 \\
$s^3$ & \textbf{3} & 3 & 0 \\
$s^2$ & \textbf{3} & 2 & \\
$s^1$ & \textbf{1} & & \\
$s^0$ & \textbf{2} & & \\
\bottomrule
\end{tabular}
\end{table}

第1列の要素は $[1, 3, 3, 1, 2]$ である.すべての要素が正であり,符号の変化はない.
\end{enumerate}

\textbf{結論:}\\
ラウス配列の第1列に符号変化がないため,このシステムは安定である.

\subsection{問題3(2) の解答と解説}

\textbf{問題:} 特性方程式が $2s^4+4s^3+s^2+2s+3=0$ で与えられる伝達関数の安定性を判定し,不安定の場合は不安定極の数も求めよ.

\textbf{解答プロセス:}

\begin{enumerate}
\item ラウス配列の構成と特殊ケースの発生:\\
係数は $a_4=2, a_3=4, a_2=1, a_1=2, a_0=3$ である.

\begin{table}[ht]
\centering
\caption{ラウス配列(特殊ケース発生)}
\label{tbl:routh2}
\begin{tabular}{cccc}
\toprule
$s^4$ & 2 & 1 & 3 \\
$s^3$ & 4 & 2 & 0 \\
$s^2$ & $b_1$ & $b_2$ & \\
\bottomrule
\end{tabular}
\end{table}

$s^2$ の行の第1要素 $b_1$ を計算する.
$$b_1 = \frac{4 \cdot 1 - 2 \cdot 2}{4} = \frac{0}{4} = 0$$
第1列に0が現れたため,これは特殊ケース1に該当する.

\item $\varepsilon$法の適用:\\
$b_1=0$ を微小な正の数 $\varepsilon$ で置き換えて計算を続行する.

$$b_2 = \frac{4 \cdot 3 - 2 \cdot 0}{4} = 3$$

修正された配列は表\ref{tbl:routh2_epsilon}のようになる.

\begin{table}[ht]
\centering
\caption{微小量法を適用したラウス配列}
\label{tbl:routh2_epsilon}
\begin{tabular}{cccc}
\toprule
$s^4$ & 2 & 1 & 3 \\
$s^3$ & 4 & 2 & 0 \\
$s^2$ & $\varepsilon$ & 3 & \\
$s^1$ & $c_1$ & & \\
$s^0$ & $d_1$ & & \\
\bottomrule
\end{tabular}
\end{table}

残りの要素を計算する.
\begin{itemize}
\item $s^1$ の行:
$$c_1 = \frac{\varepsilon \cdot 2 - 4 \cdot 3}{\varepsilon} = \frac{2\varepsilon - 12}{\varepsilon}$$
\item $s^0$ の行:
$$d_1 = \frac{c_1 \cdot 3 - \varepsilon \cdot 0}{c_1} = 3$$
\end{itemize}

\item 第1列の符号評価 ($\varepsilon \to 0^+$):\\
第1列の要素は $[2, 4, \varepsilon, \frac{2\varepsilon-12}{\varepsilon}, 3]$ である.$\varepsilon$ を微小な正の数として,各要素の符号を評価する.
\begin{itemize}
\item $s^4$ の行: 2 (正)
\item $s^3$ の行: 4 (正)
\item $s^2$ の行: $\varepsilon$ (正)
\item $s^1$ の行: $\frac{2\varepsilon-12}{\varepsilon} \approx \frac{-12}{\varepsilon}$ (負)
\item $s^0$ の行: 3 (正)
\end{itemize}
符号の系列は $[+, +, +, -, +]$ となる.

\item 符号変化の回数と安定性の判定:\\
第1列の符号は,$s^2$ の行から $s^1$ の行へ移る際に正から負へ変化し,$s^1$ の行から $s^0$ の行へ移る際に負から正へ変化している.符号変化は合計で2回である.
\end{enumerate}

\textbf{結論:}\\
第1列の符号変化が2回あるため,このシステムは不安定であり,右半平面に2個の不安定極を持つ.

\subsection{問題3(3) の解答と解説}

\textbf{問題:} 特性方程式が $2s^4+s^3+4s^2+s+2=0$ で与えられる伝達関数の安定性を判定し,不安定の場合は不安定極の数も求めよ.

\textbf{解答プロセス:}

\begin{enumerate}
\item ラウス配列の構成と特殊ケースの発生:\\
係数は $a_4=2, a_3=1, a_2=4, a_1=1, a_0=2$ である.

\begin{table}[ht]
\centering
\caption{ラウス配列(ケース2発生)}
\label{tbl:routh3}
\begin{tabular}{cccc}
\toprule
$s^4$ & 2 & 4 & 2 \\
$s^3$ & 1 & 1 & 0 \\
$s^2$ & $b_1$ & $b_2$ & \\
$s^1$ & $c_1$ & & \\
\bottomrule
\end{tabular}
\end{table}

$s^2$ の行の要素を計算する.
$$b_1 = \frac{1 \cdot 4 - 2 \cdot 1}{1} = 2$$
$$b_2 = \frac{1 \cdot 2 - 2 \cdot 0}{1} = 2$$

次に,$s^1$ の行の要素 $c_1$ を計算する.
$$c_1 = \frac{2 \cdot 1 - 1 \cdot 2}{2} = \frac{0}{2} = 0$$

$s^1$ の行の他の要素も0となり,この行のすべての要素が0になる.これは特殊ケース2に該当する.

\item 補助多項式の利用:\\
ゼロの行 ($s^1$ の行) の1つ上の行 ($s^2$ の行) の係数 $[2, 2]$ から補助多項式 $P(s)$ を作る.$s^2$ の行は $s^2$ の項から始まるので,
$$P(s) = 2s^2 + 2s^0 = 2s^2 + 2$$

この補助多項式を $s$ で微分する.
$$\frac{dP(s)}{ds} = 4s$$

この微分の係数 $[4, 0]$ を,ゼロになった $s^1$ の行に代入する.

\item 修正されたラウス配列と安定性の判定:\\
修正された配列で計算を続行する.

\begin{table}[ht]
\centering
\caption{補助多項式を用いて修正したラウス配列}
\label{tbl:routh3_corrected}
\begin{tabular}{cccc}
\toprule
$s^4$ & 2 & 4 & 2 \\
$s^3$ & \textbf{1} & 1 & 0 \\
$s^2$ & \textbf{2} & 2 & \\
$s^1$ & \textbf{4} & 0 & \\
$s^0$ & $d_1$ & & \\
\bottomrule
\end{tabular}
\end{table}

$s^0$ の行の要素 $d_1$ を計算する.
$$d_1 = \frac{4 \cdot 2 - 2 \cdot 0}{4} = 2$$

完成した配列の第1列の要素は $[2, 1, 2, 4, 2]$ である.すべての要素が正であり,符号変化はない.

\item 安定性の最終評価:\\
第1列に符号変化がないことから,右半平面に極は存在しないことがわかる.しかし,補助多項式が現れたことは,システムが厳密に安定ではないことを示している.このシステムの挙動を正確に理解するために,補助方程式 $P(s)=0$ を解く.
$$2s^2+2=0 \Rightarrow s^2=-1 \Rightarrow s=\pm j$$

これは,システムが虚軸上に一対の極を持つことを意味する.
\end{enumerate}

\textbf{結論:}\\
右半平面に極はないが,虚軸上に極 ($s=\pm j$) が存在するため,このシステムは安定限界の状態にある.

\section{フルビッツの安定判別法}

\subsection{問題4(1) の解答と解説}

\textbf{問題:} 特性方程式が $s^3+7s^2+4s+6=0$ で与えられる伝達関数の安定性を,フルビッツの安定判別法から判定せよ.

\textbf{解答プロセス:}

\begin{enumerate}
\item 係数の特定:\\
$n=3$ であり,係数は $a_3=1, a_2=7, a_1=4, a_0=6$ である.

\item フルビッツ行列の構成:\\
$3 \times 3$ のフルビッツ行列 $H$ を構成する.
$$H = \begin{pmatrix}
a_2 & a_3 & 0 \\
a_0 & a_1 & a_2 \\
0 & 0 & a_0
\end{pmatrix} = \begin{pmatrix}
7 & 1 & 0 \\
6 & 4 & 7 \\
0 & 0 & 6
\end{pmatrix}$$

\item \textbf{首座小行列式の計算:}
\begin{itemize}
\item $\Delta_1$:
$$\Delta_1 = 7$$
\item $\Delta_2$:
$$\Delta_2 = \det\begin{pmatrix} 7 & 1 \\ 6 & 4 \end{pmatrix} = (7 \cdot 4) - (6 \cdot 1) = 28 - 6 = 22$$
\item $\Delta_3$:
$$\Delta_3 = \det(H) = \det\begin{pmatrix} 7 & 1 & 0 \\ 6 & 4 & 7 \\ 0 & 0 & 6 \end{pmatrix}$$
第3列について余因子展開を行うと計算が容易になる.
$$\Delta_3 = 6 \cdot \det\begin{pmatrix} 7 & 1 \\ 6 & 4 \end{pmatrix} = 6 \cdot 22 = 132$$
\end{itemize}

\item 安定性の判定:\\
計算した小行列式を評価する.
\begin{itemize}
\item $\Delta_1 = 7 > 0$
\item $\Delta_2 = 22 > 0$
\item $\Delta_3 = 132 > 0$
\end{itemize}
すべての首座小行列式が正である.
\end{enumerate}

\textbf{結論:}\\
フルビッツの安定条件をすべて満たしているため,このシステムは安定である.

\subsection{問題4(2) の解答と解説}

\textbf{問題:} 特性方程式が $3s^4+s^3+5s^2+2s+6=0$ で与えられる伝達関数の安定性を,フルビッツの安定判別法から判定せよ.

\textbf{解答プロセス:}

\begin{enumerate}
\item 係数の特定:\\
$n=4$ であり,係数は $a_4=3, a_3=1, a_2=5, a_1=2, a_0=6$ である.

\item フルビッツ行列の構成:\\
$4 \times 4$ のフルビッツ行列 $H$ を構成する.
$$H = \begin{pmatrix}
a_3 & a_4 & 0 & 0 \\
a_1 & a_2 & a_3 & a_4 \\
0 & a_0 & a_1 & a_2 \\
0 & 0 & 0 & a_0
\end{pmatrix} = \begin{pmatrix}
1 & 3 & 0 & 0 \\
2 & 5 & 1 & 3 \\
0 & 6 & 2 & 5 \\
0 & 0 & 0 & 6
\end{pmatrix}$$

\item \textbf{首座小行列式の計算:}
\begin{itemize}
\item $\Delta_1$:
$$\Delta_1 = a_3 = 1$$
この時点では条件 $\Delta_1 > 0$ を満たしている.
\item $\Delta_2$:
$$\Delta_2 = \det\begin{pmatrix} 1 & 3 \\ 2 & 5 \end{pmatrix} = (1 \cdot 5) - (2 \cdot 3) = 5 - 6 = -1$$
\end{itemize}

\item 安定性の判定:\\
計算した小行列式を評価する.
\begin{itemize}
\item $\Delta_1 = 1 > 0$
\item $\Delta_2 = -1 < 0$
\end{itemize}
$\Delta_2$ が負の値となったため,フルビッツの安定条件は満たされない.この時点で,システムが安定でないことが確定する.したがって,$\Delta_3$ と $\Delta_4$ を計算する必要はない.
\end{enumerate}

\textbf{結論:}\\
第2首座小行列式 $\Delta_2$ が負であるため,このシステムは不安定である.

\end{document}
