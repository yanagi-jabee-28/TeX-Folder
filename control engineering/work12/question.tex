\documentclass[11pt,a4paper]{ltjsarticle}
\usepackage{luatexja}
\usepackage{luatexja-fontspec}
\usepackage{amsmath,amssymb}
\usepackage{geometry}
\geometry{left=2.5cm,right=2.5cm,top=3cm,bottom=3cm}
\usepackage{graphicx}
\usepackage{booktabs}
\usepackage{siunitx}
\sisetup{detect-all,detect-weight=true,detect-family=true}
\usepackage{hyperref}
\usepackage{url}
\usepackage{fancyhdr}
\usepackage{fontspec}
\usepackage{unicode-math}
\usepackage{pgfplots}
\pgfplotsset{compat=1.18}

% 欧文フォント設定
\setmainfont{Times New Roman}
\setsansfont{Arial}
\setmonofont{Consolas}

% 日本語フォント設定
\setmainjfont{Yu Mincho}
\setsansjfont{Yu Gothic}

% 数式フォント設定
\setmathfont{XITS Math}

% 参考文献番号を右肩に上付き表示するためのカスタムコマンド
\newcommand{\supcite}[1]{\textsuperscript{\cite{#1}}}

% ヘッダー設定
\pagestyle{fancy}
\fancyhead{}
\fancyhead[R]{\footnotesize
  制御工学 安定判別問題 \\
  長野高専 電気電子工学科 5年 XX番 氏名 \\
  2025年7月17日
}
\setlength{\headheight}{34.832pt}

\begin{document}

\title{制御工学における安定判別問題}
\author{長野高専 電気電子工学科 5年 XX番 氏名}
\date{2025年7月17日}
\maketitle
\thispagestyle{fancy}

\section{問題1:極による安定判別}

以下の伝達関数の安定性を,極を求めて判定せよ.

\subsection{問題1(1)}
\begin{equation}
\frac{1}{s^2 - 2s + 3}
\end{equation}

\subsection{問題1(2)}
\begin{equation}
\frac{1}{s^3 + 4s^2 + 7s + 6}
\end{equation}

\section{問題2:ラウスの安定判別法}

特性方程式が以下の式で与えられる伝達関数の安定性を,ラウスの安定判別法から判定せよ.ただし,不安定の場合は不安定極の数も求めよ.

\subsection{問題2(1)}
\begin{equation}
s^4 + 3s^3 + 4s^2 + 3s + 2 = 0
\end{equation}

\subsection{問題2(2)}
\begin{equation}
2s^4 + 4s^3 + s^2 + 2s + 3 = 0
\end{equation}

\subsection{問題2(3)}
\begin{equation}
2s^4 + s^3 + 4s^2 + s + 2 = 0
\end{equation}

\section{問題3:フルビッツの安定判別法}

特性方程式が以下の式で与えられる伝達関数の安定性を,フルビッツの安定判別法から判定せよ.

\subsection{問題3(1)}
\begin{equation}
s^3 + 7s^2 + 4s + 6 = 0
\end{equation}

\subsection{問題3(2)}
\begin{equation}
3s^4 + s^3 + 5s^2 + 2s + 6 = 0
\end{equation}

\end{document}
