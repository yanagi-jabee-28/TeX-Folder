
\documentclass[11pt,a4paper]{ltjsarticle}
\usepackage{luatexja}
\usepackage{luatexja-fontspec}
\usepackage{amsmath,amssymb}
\usepackage{geometry}
\geometry{left=2.5cm,right=2.5cm,top=3cm,bottom=3cm}
\usepackage{graphicx}
\usepackage{booktabs}
\usepackage{fancyhdr}
\usepackage{fontspec}
% 欧文フォント
\setmainfont{Times New Roman}
\setsansfont{Arial}
\setmonofont{Consolas}
% 日本語フォント
\setmainjfont{Yu Mincho}
\setsansjfont{Yu Gothic}
\pagestyle{fancy}
\fancyhead{}
\fancyhead[R]{\footnotesize
  制御工学における安定判別法の演習 \\
  長野高専 電気電子工学科 5年 34番 栁原魁人 \\
  2025年7月22日
}
\setlength{\headheight}{34.832pt}

\begin{document}
\title{制御工学演習解答}
\author{長野高専 電気電子工学科 5年 34番 栁原魁人}
\date{2025年7月22日}
\maketitle

\section{問題1}

\subsection{(1)}
$$\frac{1}{s^2-2s+3}$$

特性方程式: $s^2-2s+3=0$

解の公式より:
$$s = \frac{2 \pm \sqrt{4-12}}{2} = \frac{2 \pm j2\sqrt{2}}{2} = 1 \pm j\sqrt{2}$$

極: $s_1=1+j\sqrt{2}$, $s_2=1-j\sqrt{2}$

実数部が正なので右半平面にあり,システムは不安定.

\subsection{(2)}
$$\frac{1}{s^3+4s^2+7s+6}$$

特性方程式: $s^3+4s^2+7s+6=0$

$s=-2$を代入: $(-2)^3+4(-2)^2+7(-2)+6=-8+16-14+6=0$

$(s+2)$で割ると: $(s+2)(s^2+2s+3)=0$

$s_1=-2$

$s^2+2s+3=0$より: $s = \frac{-2 \pm \sqrt{4-12}}{2} = -1 \pm j\sqrt{2}$

極: $s_1=-2$, $s_2=-1+j\sqrt{2}$, $s_3=-1-j\sqrt{2}$

すべての極の実数部が負なので,システムは安定.

\section{問題3}

\subsection{(1)}
特性方程式: $s^4+3s^3+4s^2+3s+2=0$

係数: $a_4=1, a_3=3, a_2=4, a_1=3, a_0=2$

ラウス配列の基本形:
\begin{align*}
s^4: & \quad a_4 \quad a_2 \quad a_0 \\
s^3: & \quad a_3 \quad a_1 \quad 0 \\
s^2: & \quad b_1 \quad b_2 \\
s^1: & \quad c_1 \\
s^0: & \quad d_1
\end{align*}

数値を代入したラウス配列:
\begin{align*}
s^4: & \quad 1 \quad 4 \quad 2 \\
s^3: & \quad 3 \quad 3 \quad 0 \\
s^2: & \quad b_1 \quad b_2 \\
s^1: & \quad c_1 \\
s^0: & \quad d_1
\end{align*}

計算:
$$b_1 = \frac{a_3 \cdot a_2 - a_4 \cdot a_1}{a_3} = \frac{3 \cdot 4 - 1 \cdot 3}{3} = 3$$

$$b_2 = \frac{a_3 \cdot a_0 - a_4 \cdot 0}{a_3} = \frac{3 \cdot 2 - 1 \cdot 0}{3} = 2$$

$$c_1 = \frac{b_1 \cdot a_1 - a_3 \cdot b_2}{b_1} = \frac{3 \cdot 3 - 3 \cdot 2}{3} = 1$$

$$d_1 = \frac{c_1 \cdot b_2 - b_1 \cdot 0}{c_1} = \frac{1 \cdot 2 - 3 \cdot 0}{1} = 2$$

完成したラウス配列:
\begin{align*}
s^4: & \quad 1 \quad 4 \quad 2 \\
s^3: & \quad 3 \quad 3 \quad 0 \\
s^2: & \quad 3 \quad 2 \\
s^1: & \quad 1 \\
s^0: & \quad 2
\end{align*}

第1列: $[1, 3, 3, 1, 2]$ すべて正,符号変化なし.

システムは安定.

\subsection{(2)}
特性方程式: $2s^4+4s^3+s^2+2s+3=0$

係数: $a_4=2, a_3=4, a_2=1, a_1=2, a_0=3$

ラウス配列の基本形:
\begin{align*}
s^4: & \quad a_4 \quad a_2 \quad a_0 \\
s^3: & \quad a_3 \quad a_1 \quad 0 \\
s^2: & \quad b_1 \quad b_2 \\
s^1: & \quad c_1 \\
s^0: & \quad d_1
\end{align*}

数値を代入したラウス配列:
\begin{align*}
s^4: & \quad 2 \quad 1 \quad 3 \\
s^3: & \quad 4 \quad 2 \quad 0 \\
s^2: & \quad b_1 \quad b_2 \\
\end{align*}

$$b_1 = \frac{a_3 \cdot a_2 - a_4 \cdot a_1}{a_3} = \frac{4 \cdot 1 - 2 \cdot 2}{4} = 0 \text{ (特殊ケース)}$$

$$b_2 = \frac{a_3 \cdot a_0 - a_4 \cdot 0}{a_3} = \frac{4 \cdot 3 - 2 \cdot 0}{4} = 3$$

計算結果を代入したラウス配列:
\begin{align*}
s^4: & \quad 2 \quad 1 \quad 3 \\
s^3: & \quad 4 \quad 2 \quad 0 \\
s^2: & \quad 0 \quad 3 \\
\end{align*}

$\epsilon$法適用: $b_1 = \epsilon$

$$c_1 = \frac{b_1 \cdot a_1 - a_3 \cdot b_2}{b_1} = \frac{\epsilon \cdot 2 - 4 \cdot 3}{\epsilon} = \frac{2\epsilon - 12}{\epsilon} \approx \frac{-12}{\epsilon} \text{ (負)}$$

$$d_1 = \frac{c_1 \cdot b_2 - b_1 \cdot 0}{c_1} = \frac{\frac{-12}{\epsilon} \cdot 3 - \epsilon \cdot 0}{\frac{-12}{\epsilon}} = \frac{\frac{-36}{\epsilon}}{\frac{-12}{\epsilon}} = 3 \text{ (正)}$$

完成したラウス配列(ε法適用後):
\begin{align*}
s^4: & \quad 2 \quad 1 \quad 3 \\
s^3: & \quad 4 \quad 2 \quad 0 \\
s^2: & \quad \epsilon \quad 3 \\
s^1: & \quad \frac{-12}{\epsilon} \\
s^0: & \quad 3
\end{align*}

第1列符号: $[+, +, +, -, +]$ 符号変化2回

右半平面に2個の不安定極.システムは不安定.

\subsection{(3)}
特性方程式: $2s^4+s^3+4s^2+s+2=0$

係数: $a_4=2, a_3=1, a_2=4, a_1=1, a_0=2$

ラウス配列の基本形:
\begin{align*}
s^4: & \quad a_4 \quad a_2 \quad a_0 \\
s^3: & \quad a_3 \quad a_1 \quad 0 \\
s^2: & \quad b_1 \quad b_2 \\
s^1: & \quad c_1 \\
s^0: & \quad d_1
\end{align*}

数値を代入したラウス配列:
\begin{align*}
s^4: & \quad 2 \quad 4 \quad 2 \\
s^3: & \quad 1 \quad 1 \quad 0 \\
s^2: & \quad b_1 \quad b_2 \\
s^1: & \quad c_1 \\
\end{align*}

計算:
$$b_1 = \frac{a_3 \cdot a_2 - a_4 \cdot a_1}{a_3} = \frac{1 \cdot 4 - 2 \cdot 1}{1} = 2$$

$$b_2 = \frac{a_3 \cdot a_0 - a_4 \cdot 0}{a_3} = \frac{1 \cdot 2 - 2 \cdot 0}{1} = 2$$

$$c_1 = \frac{b_1 \cdot a_1 - a_3 \cdot b_2}{b_1} = \frac{2 \cdot 1 - 1 \cdot 2}{2} = 0$$

計算結果を代入したラウス配列:
\begin{align*}
s^4: & \quad 2 \quad 4 \quad 2 \\
s^3: & \quad 1 \quad 1 \quad 0 \\
s^2: & \quad 2 \quad 2 \\
s^1: & \quad 0 \\
\end{align*}

$$s^1\text{行がすべて}0 \text{ (特殊ケース)}$$

補助多項式の構成:
ゼロの行の直前の行($s^2$行)の係数 $[2, 2]$ を使用。
$s^2$行なので $s^2$ の項から始まり,係数を2つ飛ばしで配置:
$$P(s) = 2s^2 + 2s^0 = 2s^2 + 2$$

補助多項式の微分:
$$\frac{dP(s)}{ds} = 4s + 0 = 4s$$

微分の係数 $[4, 0]$ を $s^1$ 行に代入

修正後のラウス配列:
\begin{align*}
s^4: & \quad 2 \quad 4 \quad 2 \\
s^3: & \quad 1 \quad 1 \quad 0 \\
s^2: & \quad 2 \quad 2 \\
s^1: & \quad 4 \quad 0 \\
s^0: & \quad 2
\end{align*}

$$d_1 = \frac{c_1 \cdot b_2 - b_1 \cdot 0}{c_1} = \frac{4 \cdot 2 - 2 \cdot 0}{4} = 2$$

第1列: $[2, 1, 2, 4, 2]$ すべて正,符号変化なし

補助方程式: $2s^2+2=0 \Rightarrow s=\pm j$

虚軸上に極があるため,システムは安定限界.

\section{問題4}

\subsection{(1)}
特性方程式: $s^3+7s^2+4s+6=0$

係数: $a_3=1, a_2=7, a_1=4, a_0=6$

フルビッツ行列:
$$H = \begin{pmatrix} a_2 & a_0 & 0 \\ a_3 & a_1 & 0 \\ 0 & a_2 & a_0 \end{pmatrix} = \begin{pmatrix} 7 & 6 & 0 \\ 1 & 4 & 0 \\ 0 & 7 & 6 \end{pmatrix}$$

小行列式の計算:

**第1首座小行列式:**
$$\Delta_1 = a_2 = 7 > 0$$

**第2首座小行列式:**
$$\Delta_2 = \begin{vmatrix} 7 & 6 \\ 1 & 4 \end{vmatrix} = 7 \times 4 - 6 \times 1 = 28 - 6 = 22 > 0$$

**第3首座小行列式:**
$$\Delta_3 = \begin{vmatrix} 7 & 6 & 0 \\ 1 & 4 & 0 \\ 0 & 7 & 6 \end{vmatrix}$$

第3列で余因子展開すると:
$$\Delta_3 = 0 \times (\text{小行列式}) + 0 \times (\text{小行列式}) + 6 \times \begin{vmatrix} 7 & 6 \\ 1 & 4 \end{vmatrix}$$

$$= 6 \times (7 \times 4 - 6 \times 1) = 6 \times 22 = 132 > 0$$

すべて正なので,システムは安定.

\subsection{(2)}
特性方程式: $3s^4+s^3+5s^2+2s+6=0$

係数: $a_4=3, a_3=1, a_2=5, a_1=2, a_0=6$

フルビッツ行列:
$$H = \begin{pmatrix} 
a_3 & a_1 & 0 & 0 \\
a_4 & a_2 & a_0 & 0 \\
0 & a_3 & a_1 & 0 \\
0 & a_4 & a_2 & a_0
\end{pmatrix} = \begin{pmatrix} 
1 & 2 & 0 & 0 \\
3 & 5 & 6 & 0 \\
0 & 1 & 2 & 0 \\
0 & 3 & 5 & 6
\end{pmatrix}$$

小行列式の計算:

**第1首座小行列式:**
$$\Delta_1 = a_3 = 1 > 0$$

**第2首座小行列式:**
$$\Delta_2 = \begin{vmatrix} 1 & 2 \\ 3 & 5 \end{vmatrix}$$

2×2行列式の計算:
$$\Delta_2 = (1 \times 5) - (2 \times 3) = 5 - 6 = -1 < 0$$

$\Delta_2 < 0$なので,システムは不安定.

\end{document}
