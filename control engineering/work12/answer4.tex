\documentclass[11pt,a4paper]{ltjsarticle}

% --- 基本パッケージ ---
\usepackage{luatexja}
\usepackage{luatexja-fontspec}
\usepackage{amsmath,amssymb}
\usepackage{geometry}
\usepackage{graphicx}
\usepackage{booktabs}
\usepackage{fancyhdr}
\usepackage{lmodern} % スケーラブルなフォント
\usepackage{bm} % 太字の数式用

% --- レイアウト・デザイン ---
\usepackage[most]{tcolorbox}

% --- フォント設定 ---
\setmainfont{Times New Roman}
\setsansfont{Arial}
\setmonofont{Consolas}
\setmainjfont{Yu Mincho}
\setsansjfont{Yu Gothic}

% --- ページ設定 ---
\geometry{left=2cm,right=2cm,top=2.5cm,bottom=2.5cm}
\pagestyle{fancy}
\fancyhf{}
\fancyhead[L]{\small 制御工学演習:安定判別法}
\fancyhead[R]{\small \thepage}
\fancyfoot[C]{\small 長野高専 電気電子工学科 5年 34番 栁原魁人}
\renewcommand{\headrulewidth}{0.4pt}
\renewcommand{\footrulewidth}{0.4pt}
\setlength{\headheight}{15pt}

% --- tcolorbox の設定 ---
\tcbset{
    colback=blue!5!white,
    colframe=blue!75!black,
    fonttitle=\bfseries,
    arc=2mm,
    boxrule=1pt
}

\newtcolorbox{problem}[1]{
    title={#1},
    colback=orange!5!white,
    colframe=orange!80!black
}

\newtcolorbox{solution}{
    colback=green!5!white,
    colframe=green!50!black,
    title=解法プロセス
}

\newtcolorbox{conclusion}[1]{
    title={#1},
    colback=yellow!10!white,
    colframe=yellow!75!black,
    fonttitle=\bfseries
}

\begin{document}

\begin{center}
    {\Huge \bfseries 制御工学演習:安定判別法 解答レポート} \\
    \vspace{2mm}
    {\Large 極の配置、ラウス・フルビッツ法、フルビッツ法の実践}
\end{center}

\vspace{3mm}
\begin{flushright}
    長野高専 電気電子工学科 5年 34番 栁原魁人 \\
    2025年7月22日
\end{flushright}
\vspace{5mm}

\section{問題1:極の配置による安定性判別}
\textit{システムの安定性は、伝達関数の極(特性方程式の根)の実数部の符号によって決まります。すべての極の実数部が負であれば安定、一つでも正の極があれば不安定となります。}

\begin{problem}{問題1 (1)}
    伝達関数が $G(s) = \frac{1}{s^2-2s+3}$ で与えられるシステムの安定性を判別せよ。
\end{problem}

\begin{solution}
    \subsubsection*{Step 1: 特性方程式の導出}
    システムの安定性は、特性方程式 $s^2-2s+3=0$ の根(極)によって決まります。

    \subsubsection*{Step 2: 極の計算}
    解の公式を用いて極を計算します。
    $$ s = \frac{-(-2) \pm \sqrt{(-2)^2 - 4 \cdot 1 \cdot 3}}{2 \cdot 1} = \frac{2 \pm \sqrt{4-12}}{2} = \frac{2 \pm \sqrt{-8}}{2} = \frac{2 \pm j2\sqrt{2}}{2} = 1 \pm j\sqrt{2} $$
    したがって、極は $s_1 = 1 + j\sqrt{2}$ と $s_2 = 1 - j\sqrt{2}$ です。

    \subsubsection*{Step 3: 安定性の判定}
    安定であるためには、すべての極の実数部が負である必要があります。このシステムの極の実数部は $+1$ であり正であるため、極は複素平面の右半面に存在します。
\end{solution}

\begin{conclusion}{結論}
    極の実数部が正であるため、このシステムは\textbf{不安定}です。
\end{conclusion}

\hrulefill
\vspace{1cm}

\begin{problem}{問題1 (2)}
    伝達関数が $G(s) = \frac{1}{s^3+4s^2+7s+6}$ で与えられるシステムの安定性を判別せよ。
\end{problem}

\begin{solution}
    \subsubsection*{Step 1: 特性方程式の導出}
    特性方程式は $s^3+4s^2+7s+6=0$ です。

    \subsubsection*{Step 2: 極の計算}
    因数分解を試みます。$s=-2$ を代入すると、
    $$ (-2)^3 + 4(-2)^2 + 7(-2) + 6 = -8 + 16 - 14 + 6 = 0 $$
    となるため、$s+2$ は因数の一つです。方程式を $(s+2)$ で割ると、$(s+2)(s^2+2s+3)=0$ となります。
    
    残りの二次方程式 $s^2+2s+3=0$ を解の公式で解くと、
    $$ s = \frac{-2 \pm \sqrt{2^2 - 4 \cdot 1 \cdot 3}}{2} = \frac{-2 \pm \sqrt{-8}}{2} = -1 \pm j\sqrt{2} $$
    したがって、3つの極は $s_1 = -2$, $s_2 = -1+j\sqrt{2}$, $s_3 = -1-j\sqrt{2}$ です。

    \subsubsection*{Step 3: 安定性の判定}
    すべての極 ($s_1, s_2, s_3$) の実数部 ($-2, -1, -1$) は負です。
\end{solution}

\begin{conclusion}{結論}
    すべての極の実数部が負であるため、このシステムは\textbf{安定}です。
\end{conclusion}

\clearpage

\section{問題3:ラウス・フルビッツ法による安定性判別}
\textit{ラウス・フルビッツ法は、特性方程式の係数からラウス配列を作成し、その第1列の符号を調べることで安定性を判別します。第1列の要素がすべて正であればシステムは安定です。}

\begin{problem}{問題3 (1)}
    特性方程式が $s^4+3s^3+4s^2+3s+2=0$ で与えられるシステムの安定性を判別せよ。
\end{problem}

\begin{solution}
    \subsubsection*{Step 1: ラウス配列の構築}
    係数 $(a_4=1, a_3=3, a_2=4, a_1=3, a_0=2)$ でラウス配列を構築します。
    
    \subsubsection*{Step 2: 配列の要素を計算}
    \begin{align*}
        b_1 &= \frac{3 \cdot 4 - 1 \cdot 3}{3} = 3 \\
        b_2 &= \frac{3 \cdot 2 - 1 \cdot 0}{3} = 2 \\
        c_1 &= \frac{b_1 \cdot 3 - 3 \cdot b_2}{b_1} = \frac{3 \cdot 3 - 3 \cdot 2}{3} = 1 \\
        d_1 &= \frac{c_1 \cdot b_2 - b_1 \cdot 0}{c_1} = \frac{1 \cdot 2 - 3 \cdot 0}{1} = 2
    \end{align*}

    \subsubsection*{Step 3: 完成したラウス配列と判定}
    \begin{center}
    \begin{tabular}{c|ccc}
        \toprule
        $s^4$ & \textbf{1} & 4 & 2 \\
        $s^3$ & \textbf{3} & 3 & 0 \\
        $s^2$ & \textbf{3} & 2 & \\
        $s^1$ & \textbf{1} & & \\
        $s^0$ & \textbf{2} & & \\
        \bottomrule
    \end{tabular}
    \end{center}
    第1列の要素 $[1, 3, 3, 1, 2]$ はすべて正で、符号の変化はありません。
\end{solution}

\begin{conclusion}{結論}
    ラウス配列の第1列の要素がすべて正であるため、このシステムは\textbf{安定}です。
\end{conclusion}

\hrulefill
\vspace{1cm}

\begin{problem}{問題3 (2)}
    特性方程式が $2s^4+4s^3+s^2+2s+3=0$ で与えられるシステムの安定性を判別せよ。
\end{problem}

\begin{solution}
    \subsubsection*{Step 1: ラウス配列の構築と計算}
    係数 $(a_4=2, a_3=4, a_2=1, a_1=2, a_0=3)$ で計算を進めます。
    $$ b_1 = \frac{4 \cdot 1 - 2 \cdot 2}{4} = 0 $$
    $s^2$行の第1要素が0になりました。これは特殊ケースであり、このままでは計算を続行できません。

    \subsubsection*{Step 2: 特殊ケースの処理(ε法)}
    $b_1=0$ を微小な正の数 $\epsilon$ に置き換えて計算を続けます。
    $$ c_1 = \frac{b_1 \cdot a_1 - a_3 \cdot b_2}{b_1} = \frac{\epsilon \cdot 2 - 4 \cdot 3}{\epsilon} = \frac{2\epsilon - 12}{\epsilon} $$
    $\epsilon \to +0$ の極限を考えると、$c_1 \approx -12/\epsilon$ となり、これは大きな負の値です。
    $$ d_1 = b_2 = 3 $$
    
    \subsubsection*{Step 3: 符号変化の確認}
    第1列の符号は $[2, 4, \epsilon, -12/\epsilon, 3]$ となり、符号は $[+, +, +, -, +]$ となります。
    符号の変化は $(+$ から $-$ へ) と ($-$ から $+$ へ) の2回です。
\end{solution}

\begin{conclusion}{結論}
    第1列の符号変化が2回あるため、システムは\textbf{不安定}であり、複素平面の右半面に2つの不安定な極を持ちます。
\end{conclusion}

\clearpage

\begin{problem}{問題3 (3)}
    特性方程式が $2s^4+s^3+4s^2+s+2=0$ で与えられるシステムの安定性を判別せよ。
\end{problem}

\begin{solution}
    \subsubsection*{Step 1: ラウス配列の構築と計算}
    係数 $(a_4=2, a_3=1, a_2=4, a_1=1, a_0=2)$ で計算を進めます。
    \begin{align*}
        b_1 &= \frac{1 \cdot 4 - 2 \cdot 1}{1} = 2 \\
        b_2 &= \frac{1 \cdot 2 - 2 \cdot 0}{1} = 2 \\
        c_1 &= \frac{2 \cdot 1 - 1 \cdot 2}{2} = 0
    \end{align*}
    $s^1$行の要素がすべて0になりました。これも特殊ケースです。

    \subsubsection*{Step 2: 特殊ケースの処理(補助多項式)}
    ゼロの行の直前($s^2$行)の係数 $[2, 2]$ を使って補助多項式 $P(s)$ を作ります。
    $$ P(s) = 2s^2 + 2s^0 = 2s^2 + 2 $$
    これを $s$ で微分します。
    $$ \frac{dP(s)}{ds} = 4s $$
    この微分の係数 $[4, 0]$ を $s^1$行に代入してラウス配列を完成させます。
    $$ d_1 = \frac{4 \cdot 2 - 2 \cdot 0}{4} = 2 $$

    \subsubsection*{Step 3: 修正されたラウス配列と判定}
    \begin{center}
    \begin{tabular}{c|ccc}
        \toprule
        $s^4$ & \textbf{2} & 4 & 2 \\
        $s^3$ & \textbf{1} & 1 & 0 \\
        $s^2$ & \textbf{2} & 2 & \\
        $s^1$ & \textbf{4} & 0 & \\
        $s^0$ & \textbf{2} & & \\
        \bottomrule
    \end{tabular}
    \end{center}
    第1列の符号に変化はありませんが、補助多項式 $P(s)=2s^2+2=0$ の根、すなわち $s = \pm j$ が元の特性方程式の根に含まれます。
\end{solution}

\begin{conclusion}{結論}
    第1列に符号変化はありませんが、補助多項式ができたため、システムは虚軸上に極を持ちます。したがって、このシステムは\textbf{安定限界}の状態です。
\end{conclusion}

\clearpage

\section{問題4:フルビッツ法による安定性判別}
\textit{フルビッツ法は、特性方程式の係数からフルビッツ行列を構成し、その主座小行列式がすべて正であるかを確認することで安定性を判別します。}

\begin{problem}{問題4 (1)}
    特性方程式が $s^3+7s^2+4s+6=0$ で与えられるシステムの安定性を判別せよ。
\end{problem}

\begin{solution}
    \subsubsection*{Step 1: フルービッツ行列の作成}
    係数 $a_3=1, a_2=7, a_1=4, a_0=6$ からフルビッツ行列 $H$ を作成します。
    $$ H = \begin{pmatrix} a_2 & a_0 & 0 \\ a_3 & a_1 & 0 \\ 0 & a_2 & a_0 \end{pmatrix} = \begin{pmatrix} 7 & 6 & 0 \\ 1 & 4 & 0 \\ 0 & 7 & 6 \end{pmatrix} $$

    \subsubsection*{Step 2: 主座小行列式の計算}
    \textbf{第1主座小行列式}: $\Delta_1 = 7 > 0$
    
    \textbf{第2主座小行列式}: $\Delta_2 = \begin{vmatrix} 7 & 6 \\ 1 & 4 \end{vmatrix} = 7 \cdot 4 - 6 \cdot 1 = 22 > 0$
    
    \textbf{第3主座小行列式}: $\Delta_3 = \det(H) = 6 \cdot \begin{vmatrix} 7 & 6 \\ 1 & 4 \end{vmatrix} = 6 \cdot 22 = 132 > 0$
\end{solution}

\begin{conclusion}{結論}
    すべての主座小行列式が正であるため、このシステムは\textbf{安定}です。
\end{conclusion}

\hrulefill
\vspace{1cm}

\begin{problem}{問題4 (2)}
    特性方程式が $3s^4+s^3+5s^2+2s+6=0$ で与えられるシステムの安定性を判別せよ。
\end{problem}

\begin{solution}
    \subsubsection*{Step 1: フルービッツ行列の作成}
    係数 $a_4=3, a_3=1, a_2=5, a_1=2, a_0=6$ からフルビッツ行列 $H$ を作成します。
    $$ H = \begin{pmatrix} a_3 & a_1 & 0 & 0 \\ a_4 & a_2 & a_0 & 0 \\ 0 & a_3 & a_1 & 0 \\ 0 & a_4 & a_2 & a_0 \end{pmatrix} = \begin{pmatrix} 1 & 2 & 0 & 0 \\ 3 & 5 & 6 & 0 \\ 0 & 1 & 2 & 0 \\ 0 & 3 & 5 & 6 \end{pmatrix} $$

    \subsubsection*{Step 2: 主座小行列式の計算}
    \textbf{第1主座小行列式}: $\Delta_1 = 1 > 0$
    
    \textbf{第2主座小行列式}: $\Delta_2 = \begin{vmatrix} 1 & 2 \\ 3 & 5 \end{vmatrix} = 1 \cdot 5 - 2 \cdot 3 = -1 < 0$
\end{solution}

\begin{conclusion}{結論}
    第2主座小行列式が負であるため、このシステムは\textbf{不安定}です。
\end{conclusion}

\end{document}