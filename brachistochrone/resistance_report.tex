\documentclass{ltjsarticle}
\usepackage{amsmath, amssymb}
\usepackage{graphicx}
\usepackage{url}
\usepackage{hyperref}
\usepackage{geometry}
\usepackage{luatexja}
\geometry{a4paper, margin=25mm}

\title{抵抗の電圧・電流特性の測定実験}
\author{氏名:山田 太郎 学籍番号:C012345}
\date{\today}

\begin{document}
\maketitle
\pagestyle{plain} % ページ番号を全ページに表示

\section{目的}
本実験は,未知抵抗に加える電圧を変化させ,その際に流れる電流を測定することによって抵抗の電圧・電流特性を理解し,オームの法則が成立することを確認することを目的として実施した.

\section{原理}
導体に流れる電流$I$は,導体の両端の電位差$V$に比例する.この関係はオームの法則として知られ,式(1)で表される.
\begin{equation}
V = RI
\end{equation}
ここで,$R$は抵抗であり,電圧と電流の比例定数である.本実験では,電圧$V$と電流$I$を測定し,この関係が成立するかを確認する.$V$を縦軸,$I$を横軸にとったグラフを作成すれば,測定点は傾き$R$の直線上に分布するはずである.

\section{実験方法}
図\ref{fig:circuit}に示す回路を組んだ.直流電源の電圧を0 Vから5 Vまで約1 V間隔で変化させ,それぞれの電圧設定において,デジタルマルチメータを用いて抵抗の両端の電圧$V$と回路に流れる電流$I$を測定した.

% 回路図のプレースホルダー
% 注意:以下のfigure環境を有効にするには、`circuit.png`という名前の
%      回路図ファイルをこの.texファイルと同じフォルダに配置してください。
\begin{figure}[h]
\centering
% \includegraphics[width=0.6\textwidth]{circuit.png}
\framebox[0.6\textwidth][c]{ここに回路図(circuit.png)を挿入}
\caption{電圧・電流測定回路}
\label{fig:circuit}
\end{figure}

\section{使用機器}
本実験で使用した機器を表\ref{tab:equipment}にまとめる.

\begin{table}[h]
\centering
\caption{使用機器一覧}
\label{tab:equipment}
\begin{tabular}{lll}
\hline
機器名 & 型番 (メーカ名) & 備品番号 \\
\hline
直流安定化電源 & PW18-1.3ATS (KENWOOD) & E-1234 \\
デジタルマルチメータ & PC710 (SANWA) & E-5678 \\
未知抵抗 & -- & -- \\
ブレッドボード & -- & -- \\
\hline
\end{tabular}
\end{table}

\section{結果および考察}
測定によって得られた電圧$V$と電流$I$の値,および式(1)を用いて算出した抵抗値$R$を表\ref{tab:results}に示す.

\begin{table}[h]
\centering
\caption{電圧・電流測定結果と算出抵抗値}
\label{tab:results}
\begin{tabular}{ccc}
\hline
電圧 $V$ (V) & 電流 $I$ (mA) & 抵抗 $R$ ($\Omega$) \\
\hline
1.01 & 10.2 & 99.0 \\
2.05 & 20.3 & 101.0 \\
3.02 & 30.5 & 99.0 \\
4.08 & 40.6 & 100.5 \\
5.10 & 50.9 & 100.2 \\
\hline
\multicolumn{2}{r}{平均値} & 100.0 \\
\hline
\end{tabular}
\end{table}

次に,測定結果をグラフにまとめたものを図\ref{fig:vi_graph}に示す.

% グラフのプレースホルダー
% 注意:以下のfigure環境を有効にするには、`vi_graph.png`という名前の
%      グラフ画像をこの.texファイルと同じフォルダに配置してください。
\begin{figure}[h]
\centering
% \includegraphics[width=0.8\textwidth]{vi_graph.png}
\framebox[0.8\textwidth][c]{ここにV-Iグラフ(vi_graph.png)を挿入}
\caption{未知抵抗の電圧-電流特性グラフ}
\label{fig:vi_graph}
\end{figure}

表\ref{tab:results}より,測定された抵抗値は平均で100.0 $\Omega$となった.図\ref{fig:vi_graph}を見ると,測定点はほぼ一直線上に並んでおり,電圧と電流の間に強い比例関係があることがわかる.これはオームの法則($V=RI$)と合致する結果である.グラフの近似直線の傾きからも抵抗値が約100 $\Omega$であることが読み取れ,表\ref{tab:results}の計算結果と一致する.

理論的には,抵抗値は一定であるはずだが,測定値には最大で$\pm 1.0 \%$程度のばらつきが見られる.これは,測定機器の内部抵抗や読み取り誤差,接触抵抗などが原因として考えられる.しかし,全体としてはオームの法則をよく満たしており,本実験の目的は達成された.

\section{報告事項}
\subsection*{課題:カラーコードが「茶黒茶金」の抵抗の抵抗値を求め,本実験の未知抵抗と同一と見なせるか考察する}
カラーコード「茶黒茶金」は,それぞれ1 (茶), 0 (黒), 1 (10の乗数, 茶), $\pm$5% (金) を示す.したがって,その抵抗値は $10 \times 10^1 \, \Omega = 100 \, \Omega$ であり,許容差は$\pm 5 \%$である.
よって,抵抗値の範囲は $95 \, \Omega$ から $105 \, \Omega$ となる.

本実験で測定した抵抗値の平均は100.0 $\Omega$であり,この範囲内に収まっている.したがって,本実験で用いた未知抵抗は,カラーコードが「茶黒茶金」の抵抗である可能性が非常に高いと結論付けられる.

\begin{thebibliography}{9}
\bibitem{hayashi} 岡本市太郎, 電気電子計測, コロナ社, 2005.
\end{thebibliography}

\end{document}
