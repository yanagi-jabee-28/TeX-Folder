\documentclass[a4paper,12pt]{bxjsarticle}
\usepackage[no-guess-jfm]{luatexja}
\usepackage{luatexja-preset}
\usepackage{amsmath}
\usepackage{amsfonts}
\usepackage{amssymb}
\usepackage{graphicx}
\usepackage{hyperref}
\usepackage{longtable}
\usepackage{booktabs}
\usepackage{array}
\usepackage{geometry}
\geometry{margin=2.5cm}
\usepackage{fontspec}
\setmainfont{Yu Mincho}
\setsansfont{Yu Gothic}

\title{\textbf{時を制する曲線:ゼロから学ぶ最速降下線の旅}}
\author{ }
\date{ }

\begin{document}

\maketitle

\section*{序章:史上最も美しい数学の問題への招待}

物語は1696年6月、スイスの数学者ヨハン・ベルヌーイが、当時のヨーロッパ科学界で最も権威ある学術誌の一つ『学者論叢(Acta Eruditorum)』に、挑戦状を叩きつけたことから始まります\footnote{The Brachistochrone Curve : 18 Steps (with Pictures) - Instructables}。その挑戦は、単なる知的なパズルではありませんでした。それは、同時代の最も輝かしい頭脳を持つ数学者たちに向けられた、公然たる挑戦でした\footnote{Brachistochrone problem - MacTutor History of Mathematics}。ベルヌーイは、その挑戦状を次のような言葉で飾りました。「私は、ヨハン・ベルヌーイ、世界の最も優れた数学者たちに呼びかける。知的な人々にとって、その解が名声をもたらし、永続する記念碑として残るであろう、誠実で挑戦的な問題ほど魅力的なものはない」。

その問題とは、こうです。

「鉛直な平面上にある2つの点AとBが与えられたとき(ただし、BはAの真下になく、Aより低い位置にあるものとする)、重力のみの作用を受け、点Aを初速度ゼロで出発した質点が、点Bに最短時間で到達するような曲線は何か?」\footnote{Brachistochrone problem - MacTutor History of Mathematics}

この問題は、ギリシャ語で「最短の時間」を意味する「ブラキストス($\beta\rho\alpha\chi\iota\sigma\tau o\sigma$)」と「クロノス($\chi\rho o\nu o\sigma$)」を組み合わせ、「最速降下曲線(Brachistochrone curve)」と名付けられました\footnote{Brachistochrone curve - Wikipedia}。

この挑戦状は、ヨーロッパの知性界に大きな波紋を広げました。これは単なる数学の競技会ではなく、当時生まれたばかりの強力な新兵器「微積分学」の真価を問う、壮大な試金石だったのです。この挑戦に応じたのは、アイザック・ニュートン、ゴットフリート・ライプニッツ、ギヨーム・ド・ロピタル、そしてヨハンの実の兄であり、才能あるライバルでもあったヤコブ・ベルヌーイといった、まさに時代の巨人たちでした\footnote{The Brachistochrone Curve : 18 Steps (with Pictures) - Instructables}。

この物語には、単なる学問的探求を超えた、人間ドラマの側面がありました。特に、大陸の数学者(ライプニッツやベルヌーイ家)とイギリスのニュートンとの間には、微積分学の発見の優先権を巡る激しい論争がくすぶっていました。ヨハンの挑戦状には、「(彼らが)誰にも知られていないと思っていた黄金の定理によって、その領域を驚くほど広げたと自慢しているが、実際にはずっと以前に他の人々によって発表されていたものだ」という、ニュートンを明らかに意識した痛烈な皮肉が含まれていました\footnote{Brachistochrone problem - MacTutor History of Mathematics}。

伝説によれば、この挑戦状はニュートンの元にも届けられました。ニュートンの伝記作家が記すところによると、彼は王立造幣局での長い一日の仕事を終え、「ひどく疲れて」帰宅したその夜、眠りにつく前にこの難問を解き明かしてしまったといいます\footnote{Brachistochrone problem - MacTutor History of Mathematics}。ニュートンは自身の解答を匿名で提出しましたが、ヨハン・ベルヌーイはその卓越した解法を一目見るや、「私は爪によってライオンを知る(tanquam ex ungue leonem)」と叫んだと伝えられています。その解答に現れた紛れもない天才の痕跡を、彼は見抜いたのです\footnote{The Brachistochrone: A Mathematical Journey Through Time and Space | by Priyanshu Pansari | Medium}。この逸話は、最速降下曲線問題を、単なる抽象的な計算ではなく、天才たちの誇りと知性がぶつかり合う人間ドラマとして歴史に刻み込みました。

しかし、この問題の真の重要性は、その歴史的背景だけにあるのではありません。実は、ベルヌーイ以前にも、1638年にガリレオ・ガリレイが同様の問題を考察していましたが、彼はその答えを「円弧」であると誤って結論づけていました\footnote{Brachistochrone problem - MacTutor History of Mathematics}。この歴史的な誤りは、答えがいかに直感に反するものであるかを物語っています。最速降下曲線問題が画期的だったのは、それが特定の「数値」を最適化するのではなく、最適な「経路」や「曲線」、すなわち「関数そのもの」を求めることを要求した点にありました。これは、当時の標準的な微積分学が直接扱うようには設計されていなかった、全く新しい種類の問いでした\footnote{The Brachistochrone Problem: A Deep Dive - Number Analytics}。

天才たちが提示した個々の解答は正しかったものの、それらは特定の技巧に頼るものでした。この状況は、このような「関数を最適化する」問題を一般的に解くための、より強力で体系的な方法論の必要性を浮き彫りにしました。最速降下曲線問題と、それに続いてヤコブ・ベルヌーイが提示したさらに難解な「等周問題」が引き金となり、レオンハルト・オイラーやジョゼフ=ルイ・ラグランジュといった後世の数学者たちによって、「変分法」という全く新しい数学分野が創設されることになります\footnote{The Brachistochrone Problem: A Deep Dive - Number Analytics}。したがって、この問題は単に既存の微積分学の応用例であったのではなく、その限界を暴き、数学の新たな地平を切り拓くための決定的な起爆剤となったのです。そこでは、人間同士の競争心が、数学の進歩を加速させるエンジンとして機能したのでした。

このレポートでは、あなたをこの歴史的な冒険へと誘います。数学の専門知識がなくても、一歩一歩、直感と論理を頼りに、この「時を制する曲線」の謎を解き明かしていきます。読み終える頃には、あなた自身の手で、この美しい問題を解くための知恵と道具を手にしていることでしょう。

\section{第1部:直感で解き明かす「最速」の謎}

数学的な詳細に立ち入る前に、まずは私たちの直感を頼りに、この問題の核心に迫ってみましょう。スタート地点Aからゴール地点Bまで、玉を転がして最も速く到達する坂道は、どのような形をしているでしょうか?

\subsection{直線はなぜ最速ではないのか?}

多くの人が最初に思い浮かべるのは、「2点間を結ぶ最短距離は直線なのだから、直線コースが一番速いに決まっている」という考えでしょう\footnote{An Introduction To The Brachistochrone Problem - The Journal of Young Physicists}。これは非常に自然な推論ですが、実は落とし穴があります。

この推論が成り立つのは、物体が「一定の速度」で動く場合に限られます\footnote{An Introduction To The Brachistochrone Problem - The Journal of Young Physicists}。しかし、今考えているのは重力によって坂道を下る玉です。その速度は、坂の傾きによって刻一刻と変化します。

ここで、スキーヤーが山を滑り降りる場面を想像してみてください。できるだけ速く麓にたどり着きたいとき、彼らはどのようなコースを選ぶでしょうか?スタートからゴールまでを結ぶ、なだらかで距離の短い直線的な斜面でしょうか?それとも、最初は急な斜面で一気に加速し、スピードに乗ってからゴールを目指す、カーブしたコースでしょうか?

答えは明らかです。速く進むためには、まず速度を稼ぐ必要があります。直線コースは距離こそ最短ですが、傾斜が一定で緩やかであるため、加速が鈍く、なかなかスピードが上がりません。一方で、最初の部分が急なカーブを描くコースは、距離は少し長くなるかもしれませんが、重力を効率よく利用して素早く高速度に達することができます。この速度の「儲け」が、少しばかり長くなった距離の「損」を補って余りあるのです\footnote{An Introduction To The Brachistochrone Problem - The Journal of Young Physicists}。

最速降下曲線問題の本質は、この「距離」と「速度」の間の最適なトレードオフを見つけることにあります。最短距離に固執することは、より速い速度を得るチャンスを逃すことになり、結果としてゴールへの到着が遅れてしまうのです。

\subsection{では、最も急な坂道は?}

「なるほど、ならば最初に一気に加速するのが鍵なのか。それなら、スタート直後に垂直に落下し、その後、水平なコースでゴールに向かうのが一番速いのではないか?」という考えも浮かぶかもしれません。これもまた、もっともらしい直感です。このコースは、確かに最速で最高速度に達することができます。

しかし、この戦略にも欠点があります。垂直に落下して最高速度を得た後、コースは水平になってしまいます。水平な部分では、重力はもはや加速のために働かず、玉は摩擦がなければ等速で進むだけです\footnote{An Introduction To The Brachistochrone Problem - The Journal of Young Physicists}。せっかく稼いだ最高速度も、それ以上は伸びません。序盤で得た大きなアドバンテージを、加速のない長い水平区間で使い果たしてしまうのです。

ここからわかるのは、最適なコースは、単に距離が短いだけでも、単に序盤の加速が最大になるだけでもない、ということです。それは、急すぎて後半の伸びを失うことなく、かつ、緩やかすぎて加速が鈍くならないような、絶妙な「短さと急さのバランス」の上に成り立つ、完璧な妥協点なのです\footnote{Brachistochrone — are things really how you expect them to be? | by Atonu Roy Chowdhury | Medium}。

私たちの直感は、距離や初期加速度といった、単一の静的な変数を最適化しようとする傾向があります。しかし、この問題は、玉の速度という状態が、それまでに通過した経路全体に依存して動的に変化するシステムです。問われているのは「どの選択が最善か?」ではなく、「最善の結果を生む一連の選択(すなわち、コースの形状)は何か?」ということです。この動的で全体的な視点こそが、後に登場する変分法を理解するための鍵となります。

\subsection{究極のカーブ「サイクロイド」との出会い}

では、その完璧なバランスを持つ究極のカーブとは、一体どのような形をしているのでしょうか。ここで、この物語の主役が登場します。その名は「サイクロイド(cycloid)」\footnote{最速降下問題について}。

サイクロイドとは、「直線上を円が滑ることなく転がるとき、円周上の一つの点が描く軌跡」のことです\footnote{The Brachistochrone Curve : 18 Steps (with Pictures) - Instructables}。夜道を走る自転車のタイヤの端に、小さな光る点がついていたと想像してみてください。その光る点が宙に描く、美しいアーチの連なりがサイクロイドです\footnote{サイクロイドの超解説【数学Ⅲ】(トロコイド・エピサイクロイド・ハイポサイクロイドetc) | 理系ラボ}。

驚くべきことに、最速降下曲線問題の答えは、このサイクロイド曲線を上下逆さまにした形なのです\footnote{名古屋市科学館}。それは、最短距離である直線でもなく、ガリレオが考えた円弧でもありません。自転車の車輪が生み出す、この一見単純な曲線こそが、「時を制する曲線」だったのです。

この事実は、初めて聞くと非常に不思議で、神秘的にさえ感じられるかもしれません。なぜ、数ある曲線の中で、このサイクロイドだけが「最速」の栄誉を担うのでしょうか?この「なぜ」に答えることこそが、私たちの旅の目的です。この美しくも不思議な事実を心に留め、その謎を解き明かすための冒険に出発しましょう。

\section{第2部:冒険の準備〜物理と数学の道具箱〜}

サイクロイドがなぜ最速降下曲線なのかを理解するためには、いくつかの強力な道具が必要です。ここでは、物理学と数学の世界から、冒険に不可欠な2つの「黄金の道具」を取り出し、その使い方を学びましょう。

\subsection{物理学の黄金律:エネルギー保存則}

最初の道具は、物理学における最も基本的で強力な法則の一つ、「エネルギー保存則」です。

\subsubsection{位置エネルギーと運動エネルギー}

まず、物体のエネルギーには大きく分けて2つの種類があることを理解しましょう。「位置エネルギー」と「運動エネルギー」です\footnote{【理科】位置エネルギーと運動エネルギー、その保存則について - 家庭教師のやる気アシスト}。

\begin{itemize}
\item \textbf{位置エネルギー}:高い場所にある物体が持つ、「蓄えられた」エネルギーです。丘の上に置かれたボールを想像してください。ボールはまだ動いていませんが、坂を転がり落ちる「可能性」を秘めています。この可能性が位置エネルギーです。高い場所にあるほど、位置エネルギーは大きくなります\footnote{力学的エネルギーとは?力学的エネルギー保存の法則って何? - Lab BRAINS}。

\item \textbf{運動エネルギー}:動いている物体が持つエネルギーです。坂を転がり落ちているボールは、速さを持っています。この動きそのものがエネルギーであり、運動エネルギーと呼ばれます。速く動いているほど、運動エネルギーは大きくなります\footnote{【理科】位置エネルギーと運動エネルギー、その保存則について - 家庭教師のやる気アシスト}。
\end{itemize}

\subsubsection{エネルギーの交換}

エネルギー保存則とは、摩擦や空気抵抗がない理想的な世界では、これら2つのエネルギーの合計(これを「力学的エネルギー」と呼びます)は、常に一定に保たれるという法則です\footnote{【理科】位置エネルギーと運動エネルギー、その保存則について - 家庭教師のやる気アシスト}。

坂道を転がり落ちる玉を例にとると、玉が下に落ちるにつれて「高さ」は失われ、位置エネルギーは減少します。その減少した位置エネルギーは、どこかへ消えてしまうわけではありません。それは運動エネルギーに姿を変え、玉の「速さ」を増やします。つまり、位置エネルギーが運動エネルギーへと交換されるのです\footnote{力学的エネルギーとは?力学的エネルギー保存の法則って何? - Lab BRAINS}。

\subsubsection{速度を求める}

このエネルギー保存則を使うと、驚くほど簡単に、坂の途中での玉の速さを計算することができます。これは、私たちの冒険における極めて重要なステップです。

座標軸を、スタート地点Aを原点 $(0,0)$ とし、鉛直下向きを $y$ 軸の正の方向とします。

\begin{enumerate}
\item \textbf{スタート地点 (A) でのエネルギー}:
\begin{itemize}
\item 玉はまだ動いていないので、初速度はゼロです。したがって、運動エネルギーは $0$ です。
\item $y=0$ の高さを基準と定めるので、位置エネルギーも $0$ です。
\item よって、この系の力学的エネルギーの合計は $0+0=0$ で、この値は常に一定に保たれます。
\end{itemize}

\item \textbf{坂の途中($y$だけ落下した点)でのエネルギー}:
\begin{itemize}
\item 玉は速さ $v$ で動いているので、運動エネルギーは $\frac{1}{2}mv^2$ です($m$ は玉の質量)。
\item 玉は基準点から $y$ だけ下にいるので、位置エネルギーは $-mgy$ となります($g$ は重力加速度)。
\item ここでの力学的エネルギーの合計は $\frac{1}{2}mv^2-mgy$ です。
\end{itemize}

\item エネルギー保存則を適用:\\
スタート地点でのエネルギーと、坂の途中でのエネルギーは等しくなければなりません。
\begin{equation}
\frac{1}{2}mv^2-mgy=0
\end{equation}

\item 速度 $v$ を解く:\\
この式を $v$ について解いてみましょう。
\begin{equation}
\frac{1}{2}mv^2=mgy
\end{equation}

両辺にある質量 $m$ は、互いに打ち消し合います。
\begin{equation}
\frac{1}{2}v^2=gy
\end{equation}

両辺を2倍して、
\begin{equation}
v^2=2gy
\end{equation}

最後に平方根をとると、次の非常に重要な式が得られます。
\begin{equation}
v=\sqrt{2gy}
\end{equation}
\end{enumerate}

この式が教えてくれるのは、玉の速さ $v$ は、その時点での垂直方向の落下距離 $y$ だけで決まるということです。驚くべきことに、玉の質量 $m$ や、そこに至るまでの坂道の形(水平方向にどれだけ進んだか)には一切関係ありません\footnote{Brachistochrone curve - Wikipedia}。このエレガントな関係式が、複雑な運動の問題を、よりシンプルな幾何学の問題へと変換してくれるのです。物理法則(エネルギー保存則)を用いて、動的な問題を静的な(経路に関する)問題へと再定式化する。これは物理学における強力な戦略であり、私たちの最初の道具箱の中身です。

\subsection{「関数のための微分」:変分法への第一歩}

第二の道具は、この問題のために生まれたと言っても過言ではない、数学の強力な分野「変分法」です。

\subsubsection{汎関数:経路を評価する機械}

まず、「汎関数(はんかんすう、functional)」という新しい概念に触れましょう\footnote{変分法について - Mathlog}。

\begin{itemize}
\item \textbf{普通の関数} $f(x)$:これは、ある「数」$x$ を入力すると、別の「数」$f(x)$ を出力する機械です。例えば $f(x)=x^2$ なら、$x=2$ を入れると $f(2)=4$ が出てきます。

\item \textbf{汎関数} $J[y(x)]$:これは、もっと高レベルな機械です。入力として「数」ではなく、「関数そのもの(例えば、一本の曲線や経路を表す関数 $y(x)$)」を受け取り、出力として一つの「数」を返します\footnote{変分法について - Mathlog}。
\end{itemize}

最速降下曲線問題において、私たちが最小化したいのは「ゴールまでの所要時間 $T$」です。この所要時間 $T$ は、私たちが選ぶ坂道の「形 $y(x)$」によって変わります。直線コースを選べばある時間がかかり、円弧コースを選べば別の時間がかかります。つまり、この所要時間 $T$ は、入力として経路の関数 $y(x)$ を受け取り、出力として時間という数値を返す「汎関数」なのです。私たちは、この汎関数 $T[y(x)]$ の値を最小にするような入力関数 $y(x)$ を見つけ出したいのです。

\subsubsection{微分と変分}

この目標を達成するために、変分法は普通の微積分と似たような戦略をとりますが、対象が異なります。

\begin{itemize}
\item \textbf{普通の微積分}:関数 $f(x)$ の最小値を見つけるために、その導関数 $df/dx$ がゼロになる点 $x$ を探します。これは、グラフの谷底では接線の傾きがゼロになる、という考えに基づいています。

\item \textbf{変分法}:汎関数 $J[y]$ の最小値を見つけるために、「変分 $\delta J$」と呼ばれるものがゼロになる「関数 $y(x)$」を探します\footnote{Calculus of Variations For Dummies: An Intuitive Introduction}。これは、最適な経路(関数)からほんの少しだけ形をずらしても、汎関数の値(この場合は所要時間)は一次のオーダーでは変化しない、という考えに基づいています。
\end{itemize}

この違いを明確にするために、以下の表にまとめます。

\begin{longtable}{|p{3cm}|p{5cm}|p{5cm}|}
\hline
\textbf{特徴} & \textbf{通常の微積分} & \textbf{変分法} \\
\hline
\textbf{目標} & \textbf{関数} $f(x)$ を最小・最大にする\textbf{点} $x$ を見つける。 & \textbf{汎関数} $J[y]$ を最小・最大にする\textbf{関数} $y(x)$ を見つける。 \\
\hline
\textbf{入力} & 数 $x$ & 関数全体 $y(x)$ \\
\hline
\textbf{出力} & 数 $f(x)$ & 数 $J[y]$ \\
\hline
\textbf{中心的な問い} & どの $x$ の値が $f(x)$ を最小にするか? & どの形の $y(x)$ が $J[y]$ を最小にするか? \\
\hline
\textbf{手法} & 導関数がゼロになる点を探す:$df/dx=0$ & 変分がゼロになる関数を探す:$\delta J=0$ \\
\hline
\textbf{核心となる方程式} & $df/dx=0$ & オイラー・ラグランジュ方程式 \\
\hline
\end{longtable}

この表からわかるように、変分法は、関数の谷底を探す微積分の考え方を、いわば「関数の海」へと拡張し、最も低い場所にある「経路(関数)」そのものを見つけ出すための、壮大な航海術なのです。そして、その航海の羅針盤となるのが、次に登場する「オイラー・ラグランジュ方程式」です。

\section{第3部:数学の頂へ〜最速降下曲線の導出〜}

さあ、準備は整いました。第2部で手に入れた物理と数学の道具を手に、いよいよ最速降下曲線を導出するという、この冒険のクライマックスに挑みます。一見すると複雑に見えるかもしれませんが、一歩一歩、論理の足場を確かめながら進めば、必ずや山頂にたどり着くことができます。

\subsection{問題を数式で表現する}

最初のステップは、私たちの目標である「所要時間を最小にする」という問題を、厳密な数学の言葉、すなわち汎関数の形で書き表すことです。

\begin{enumerate}
\item \textbf{時間の基本式}:ゴールまでの総所要時間 $T$ は、ごく短い時間 $dt$ を、スタートからゴールまで全て足し合わせたもの(積分したもの)です\footnote{最速降下曲線 - date-physics - 伊達の物理}。
\begin{equation}
T=\int_A^B dt
\end{equation}

\item \textbf{時間と距離、速度の関係}:物理の基本に立ち返ると、「時間 = 距離 ÷ 速度」です。ごく短い区間においては、$dt=ds/v$ と書けます。ここで $ds$ はその短い区間の道のりの長さ、$v$ はその区間での速さです\footnote{最速降下問題 - ScienceTime}。
\begin{equation}
T=\int_A^B \frac{ds}{v}
\end{equation}

\item \textbf{道具の代入}:ここで、第2部で準備した道具を代入します。
\begin{itemize}
\item 短い道のりの長さ $ds$ は、三平方の定理から $ds=\sqrt{1+(y')^2}dx$ と表せました($y'$ は坂の傾き $dy/dx$ です)。
\item 速さ $v$ は、エネルギー保存則から $v=\sqrt{2gy}$ と求められました。
\end{itemize}

\item 最終的な汎関数:これらを代入すると、所要時間 $T$ を計算するための、私たちの最終目標である汎関数が完成します。
\begin{equation}
T[y]=\int_A^B \frac{\sqrt{1+(y')^2}}{\sqrt{2gy}} dx
\end{equation}
\end{enumerate}

この式の意味を改めて確認しましょう。これは、坂道の形を表す任意の関数 $y(x)$ をこの積分に入れると、その坂道を玉が滑り落ちるのにかかる総時間が計算できる、ということを示しています。私たちの目標は、この $T[y]$ の値を最小にするような、魔法の関数 $y(x)$ を見つけ出すことです。

\subsection{魔法の杖:オイラー・ラグランジュ方程式}

汎関数を最小化する関数を見つけるための魔法の杖、それが「オイラー・ラグランジュ方程式」です。

積分の中身の関数を $L(y,y')=\frac{\sqrt{1+(y')^2}}{\sqrt{2gy}}$ と置きます($\sqrt{2g}$ は定数なので、簡単のため一旦無視して $L=\frac{\sqrt{1+(y')^2}}{\sqrt{y}}$ とします)。オイラー・ラグランジュ方程式の一般形は次の通りです。

\begin{equation}
\frac{\partial L}{\partial y}-\frac{d}{dx}\left(\frac{\partial L}{\partial y'}\right)=0
\end{equation}

これを直接計算することも可能ですが、骨の折れる作業になります。しかし幸運なことに、私たちの $L$ 関数には特別な性質があります。それは、式の中に変数 $x$ が陽に含まれていない、ということです。このような場合、「ベルトラミの恒等式」として知られる、オイラー・ラグランジュ方程式のよりシンプルなバージョンを使うことができます。これは計算を劇的に簡単にしてくれる、まさに救いの手です\footnote{Brachistochrone Problem -- from Wolfram MathWorld}。

ベルトラミの恒等式:
\begin{equation}
L-y'\frac{\partial L}{\partial y'}=C
\end{equation}

ここで $C$ は何らかの定数です。この強力な恒等式を使って、微分方程式を解く冒険へと進みましょう。

\subsection{微分方程式を解く冒険}

ここからが、数学的な計算の核心部分です。一つ一つのステップを丁寧に解説していきます。

\begin{enumerate}
\item ベルトラミの恒等式を適用する:\\
まず、$L=\frac{\sqrt{1+(y')^2}}{\sqrt{y}}$ を恒等式に適用するために、偏微分 $\frac{\partial L}{\partial y'}$ を計算します。合成関数の微分法を用いると、
\begin{equation}
\frac{\partial L}{\partial y'} = \frac{1}{\sqrt{y}} \cdot \frac{\partial}{\partial y'} \left( \sqrt{1+(y')^2} \right) = \frac{1}{\sqrt{y}} \cdot \frac{1}{2\sqrt{1+(y')^2}} \cdot (2y') = \frac{y'}{\sqrt{y(1+(y')^2)}}
\end{equation}

これをベルトラミの恒等式 $L-y'\frac{\partial L}{\partial y'}=C$ に代入します。
\begin{equation}
\frac{\sqrt{1+(y')^2}}{\sqrt{y}}-y'\left(\frac{y'}{\sqrt{y(1+(y')^2)}}\right)=C
\end{equation}

\item 式を単純化する:\\
この式を整理していきましょう。共通の分母 $\sqrt{y(1+(y')^2)}$ で通分します。
\begin{equation}
\frac{\sqrt{1+(y')^2}-(y')^2}{\sqrt{y(1+(y')^2)}}=C
\end{equation}

左辺の分子は $(y')^2$ が打ち消しあって $1$ になります。
\begin{equation}
\frac{1}{\sqrt{y(1+(y')^2)}}=C
\end{equation}

両辺を逆数にして整理すると、
\begin{equation}
\sqrt{y(1+(y')^2)}=\frac{1}{C}
\end{equation}

両辺を2乗して、根号を取り除きます。
\begin{equation}
y(1+(y')^2)=\frac{1}{C^2}
\end{equation}

ここで $1/C^2$ を新しい定数 $C_1$ と置くと、非常にすっきりした形の一階微分方程式が得られます。
\begin{equation}
y(1+(y')^2)=C_1
\end{equation}

\item 傾き $y'$ を分離する:\\
この方程式を $y'$ について解きます。
\begin{align}
1+(y')^2&=\frac{C_1}{y}\\
(y')^2&=\frac{C_1}{y}-1=\frac{C_1-y}{y}
\end{align}

よって、坂の傾き $y'=dy/dx$ は次のようになります。
\begin{equation}
\frac{dy}{dx}=\sqrt{\frac{C_1-y}{y}}
\end{equation}

\item 鍵となる変数変換:\\
この微分方程式は変数分離形なので、$x$ について積分することができます。
\begin{equation}
dx=\sqrt{\frac{y}{C_1-y}}dy \Rightarrow x=\int\sqrt{\frac{y}{C_1-y}}dy
\end{equation}

しかし、この積分を直接計算するのは困難です。ここで、問題を一気に解決する「魔法の」変数変換を導入します。サイクロイドが転がる円から生まれることを思い出してください。円運動に関連する三角関数を使った置換が有効そうです。そこで、次のような変数変換を行います\footnote{最速降下線(Brachistocrone)}。
\begin{equation}
y=a(1-\cos\theta)
\end{equation}

ここで $a=C_1/2$ とします。このとき、$dy$ は連鎖律により次のようになります。
\begin{equation}
\frac{dy}{d\theta}=a\sin\theta \Rightarrow dy=a\sin\theta d\theta
\end{equation}

\item 積分して解を求める:\\
この変数変換を積分に適用すると、驚くほど積分が簡単になります。まず、積分の中身の根号部分を変形します。
\begin{align}
\sqrt{\frac{y}{C_1 - y}} &= \sqrt{\frac{a(1-\cos\theta)}{2a - a(1-\cos\theta)}} \\
&= \sqrt{\frac{a(1-\cos\theta)}{a(1+\cos\theta)}} \\
&= \sqrt{\frac{1-\cos\theta}{1+\cos\theta}}
\end{align}

三角関数の半角の公式 $1-\cos\theta=2\sin^2(\theta/2)$ と $1+\cos\theta=2\cos^2(\theta/2)$ を使うと、
\begin{align}
\sqrt{\frac{2\sin^2(\theta/2)}{2\cos^2(\theta/2)}} &= \sqrt{\tan^2(\theta/2)} \\
&= \tan(\theta/2) \\
&= \frac{\sin(\theta/2)}{\cos(\theta/2)}
\end{align}

これを $x$ の積分式に代入します。
\begin{equation}
x=\int\frac{\sin(\theta/2)}{\cos(\theta/2)} \cdot (a\sin\theta)d\theta
\end{equation}

ここで倍角の公式 $\sin\theta=2\sin(\theta/2)\cos(\theta/2)$ を用いると、
\begin{align}
x &= \int \frac{\sin(\theta/2)}{\cos(\theta/2)} \cdot a \cdot (2\sin(\theta/2)\cos(\theta/2)) d\theta \\
&= \int 2a \sin^2(\theta/2) d\theta
\end{align}

再び半角の公式(今度は逆向きに)$2\sin^2(\theta/2)=1-\cos\theta$ を使うと、
\begin{equation}
x=\int a(1-\cos\theta)d\theta
\end{equation}

これを $\theta$ で積分すると、最終的な解が得られます(積分定数は原点を通るように調整します)。
\begin{equation}
x=a(\theta-\sin\theta)
\end{equation}
\end{enumerate}

\subsection{答え合わせ:サイクロイドの凱旋}

今、私たちの手元には、微分方程式を解いて得られた曲線の方程式があります。

\begin{align}
x&=a(\theta-\sin\theta)\\
y&=a(1-\cos\theta)
\end{align}

この媒介変数表示された方程式をよく見てください。これは、私たちが第1部で出会った、転がる円が描く軌跡「サイクロイド」の定義そのものです\footnote{サイクロイドの超解説【数学Ⅲ】(トロコイド・エピサイクロイド・ハイポサイクロイドetc) | 理系ラボ}。数学の厳密な手続きを通して、私たちは再びサイクロイドにたどり着きました。直感が示唆し、歴史が語り継いできた答えが、今、私たちの手によって証明されたのです。これは、この冒険の山頂に到達した瞬間です。

この導出の過程で、物理的な洞察も得られます。導かれた微分方程式 $y(1+(y')^2)=C_1$ をスタート地点 $(0,0)$ で考えると、$y=0$ のため、$C_1=0$ となってしまい、解が $y=0$ という水平な直線になってしまいます。これは明らかに間違いです。この矛盾は、この方程式が $y>0$ の領域で成り立つことを示唆しています。スタート地点 $(0,0)$ では速度がゼロであり、方程式が成り立つためには、傾き $y'$ が無限大でなければならないことが、より詳細な解析からわかります。傾きが無限大とは、接線が垂直であることを意味します。つまり、最速降下曲線は、スタート地点で垂直に、つまり真下に落下し始めるのです\footnote{Brachistochrone curve - Wikipedia}。これは、最初に最大の加速を得るためには、まず自由落下するのが最も効率的であるという、私たちの直感とも一致します。数学は、この直感を厳密に裏付けてくれたのです。

この長い導出の道のりを、以下の表に要約します。

\begin{longtable}{|p{2.5cm}|p{4cm}|p{6cm}|}
\hline
\textbf{ステップ} & \textbf{目標} & \textbf{主要な方程式・操作} \\
\hline
\textbf{1. 定式化} & 総所要時間を最小化すべき汎関数として表現する。 & $T[y]=\int\frac{\sqrt{1+y'^2}}{\sqrt{y}}dx$ \\
\hline
\textbf{2. 単純化} & 汎関数の最小化問題を微分方程式に変換する。 & ベルトラミの恒等式を適用:$L-y'\frac{\partial L}{\partial y'}=C$ \\
\hline
\textbf{3. 解法} & 得られた微分方程式を解き、経路 $y(x)$ を求める。 & $y(1+(y')^2)=C_1$ を変数変換 $y=a(1-\cos\theta)$ で解く。 \\
\hline
\textbf{4. 同定} & 解が既知の曲線であることを確認する。 & 解 $x=a(\theta-\sin\theta)$, $y=a(1-\cos\theta)$ はサイクロイドである。 \\
\hline
\end{longtable}

\section{第4部:もう一つの道筋〜光のアナロジー〜}

私たちが変分法という険しい山道を登ってようやくたどり着いた山頂へ、実はもっとエレガントで、驚くほど直感的な別のルートが存在します。それは、この問題を最初に出題したヨハン・ベルヌーイ自身が用いた、天才的な発想に基づく解法です。彼は、重力に引かれる玉の問題を、全く異なる物理現象である「光の屈折」の問題として捉え直したのです。

\subsection{フェルマーの原理とスネルの法則}

この解法の鍵となるのは、17世紀のフランスの数学者ピエール・ド・フェルマーが発見した「フェルマーの原理」です。これは、「光は、2つの点の間を移動するとき、常に所要時間が最短となる経路を通る」という、自然界の驚くべき経済原則です\footnote{The Brachistochrone Curve : 18 Steps (with Pictures) - Instructables}。

この原理から、光が水やガラスのような異なる媒質の境界面を通過するときに曲がる「屈折」という現象を説明できます。屈折の法則は「スネルの法則」として知られており、次のように表されます。

\begin{equation}
\frac{\sin\theta_1}{v_1}=\frac{\sin\theta_2}{v_2}
\end{equation}

ここで、$v_1, v_2$ はそれぞれの媒質中での光の速さ、$\theta_1, \theta_2$ は境界面の法線(境界面に垂直な線)と光の進路がなす角度です\footnote{The Brachistochrone - Whistler Alley Mathematics}。この法則は、光が速く進める媒質へ入る際には、より長い距離をそちらで進むように経路を「曲げる」ことで、全体の時間を最小化しようとすることを示しています。

\subsection{ベルヌイの天才的ひらめき}

ここからがベルヌーイの独壇場です。彼は、重力下を滑り落ちる玉と、媒質中を進む光との間に、深遠なアナロジー(類推)を見出しました。

彼の発想はこうです。玉が滑り落ちる空間が、空っぽの空間ではなく、屈折率が連続的に変化する、無限に多くの薄いガラスの層で満たされていると想像してください\footnote{Brachistochrone problem - MacTutor History of Mathematics}。

\begin{itemize}
\item \textbf{速度と屈折率のアナロジー}:玉は落下するにつれて、エネルギー保存則 $v=\sqrt{2gy}$ に従って速度を増していきます。これは、光が、より「速く」進める(つまり、屈折率が低い)媒質の層へと次々に入っていく状況にそっくりです。落下距離 $y$ が増えるほど、その「媒質」は光にとって進みやすくなるのです。

\item スネルの法則の連続的な適用:もし、光がこれらの無限に薄い層を次々と通過していくなら、スネルの法則は各境界面で成り立ち続けなければなりません。これを極限まで考えると、経路全体を通して、ある関係式が常に一定に保たれるはずです。その関係式とは、
\begin{equation}
\frac{\sin\theta}{v}=\text{一定}
\end{equation}
です\footnote{Brachistochrone — are things really how you expect them to be? | by Atonu Roy Chowdhury | Medium}。ここで $\theta$ は、経路の接線と鉛直方向がなす角度、$v$ はその点での玉の速さです。

\item 最終的な方程式:このアナロジーが正しければ、最速降下曲線もこの条件を満たすはずです。ここで、物理法則から得られた速度の式 $v=\sqrt{2gy}$ を代入すると、
\begin{equation}
\frac{\sin\theta}{\sqrt{2gy}}=\text{一定}
\end{equation}
という、最速降下曲線が満たすべき微分方程式が得られます。
\end{itemize}

そして、この方程式を満たす曲線こそが、まさにサイクロイドなのです。ベルヌーイは、変分法の複雑な計算を経ることなく、物理的な洞察とアナロジーの力だけで、同じ結論にたどり着きました。

この解法は、単なる巧妙なトリックではありません。それは、一見すると全く無関係に見える二つの物理領域(力学と光学)の間に、深い構造的類似性(同型性)が存在することを示しています。自然は、異なる現象において、同じ根本的な数学的原理(この場合は最適化の原理)に従って振る舞うことがあるのです。このベルヌーイの洞察は、物理学の異なる分野を統一する数学という「言語」の普遍性と美しさを、鮮やかに描き出しています。

\section{終章:サイクロイドのさらなる神秘と、その先へ}

私たちは、歴史的な挑戦から始まり、直感的な考察、物理法則の準備、そして厳密な数学的証明を経て、最速降下曲線がサイクロイドであることを突き止めました。しかし、この魅力的な曲線の物語は、まだ終わりません。サイクロイドは、さらなる驚くべき性質を秘めており、この問題自体が、物理学のより広大で深遠な世界への扉を開いてくれるのです。

\subsection{等時性:サイクロイドのもう一つの魔法}

実は、サイクロイドには「最速」であることとは別に、もう一つの魔法のような性質があります。それは、「等時性(とうじせい)」と呼ばれる性質です。この性質を持つことから、サイクロイドは「等時曲線(Tautochrone curve)」とも呼ばれます\footnote{Tautochrone curve - Wikipedia}。

等時性とは、「逆さまにしたサイクロイドの坂道の、どの地点から玉を放しても、最下点に到達するまでにかかる時間は全く同じである」という驚くべき性質です\footnote{名古屋市科学館}。

直感的には、高い位置からスタートした玉は、低い位置からスタートした玉よりも長い距離を移動しなければなりません。しかし、サイクロイドの形状は、高い位置ほど傾斜が急になっています。そのため、高い位置からスタートした玉は、より大きな初期加速度を得て、速くスピードに乗ることができます。サイクロイドにおいては、この「長い距離」という不利な点と、「大きな初期加速」という有利な点が、互いに完璧に打ち消し合うのです\footnote{A Tour Through Some Curves – Part 4 [The Tautochrone Curve] - Nilabha Saha}。

この驚くべき性質は、単なる偶然ではありません。サイクロイドの坂道を滑る物体の運動方程式を立てると、その形が物理学で最も基本的な振動である「単振動」の運動方程式と全く同じになることが示せます。単振動の最大の特徴は、その周期が振幅(振れ幅)に依存しないことです。サイクロイド上の運動が単振動であるからこそ、スタート地点の高さ(振幅)によらず、最下点までの時間が一定になるのです。

この等時性は、歴史的にも非常に重要でした。17世紀の科学者クリスチャン・ホイヘンスは、振り子の揺れる周期が振れ幅によってわずかに変わってしまうという問題に悩まされていました。彼は、このサイクロイドの等時性を利用して、どのような振れ幅でも正確な時を刻む「サイクロイド振り子時計」を発明し、精密な時間測定の技術を大きく前進させたのです\footnote{Tautochrone curve - Wikipedia}。

\subsection{変分原理の世界}

最速降下曲線問題は、孤立した面白いパズルではありません。それは、物理学の最も根源的な指導原理の一つである「最小作用の原理」へと続く、壮大な道の入り口なのです\footnote{Principle of Least Action: Derivation, Example \& Application - StudySmarter}。

最小作用の原理とは、「物理的な系が、ある状態から別の状態へ変化するとき、実際にたどる経路は、『作用』と呼ばれる量が停留する(多くの場合、最小になる)ような経路である」という、自然界の根本的な法則です\footnote{The Feynman Lectures on Physics Vol. II Ch. 19: The Principle of Least Action}。作用とは、大まかに言えば、運動エネルギーと位置エネルギーの差を時間で積分した量です。

この原理は、自然が根源的に「経済的」あるいは「効率的」であることを示唆しています\footnote{The Feynman Lectures on Physics Vol. II Ch. 19: The Principle of Least Action}。投げ上げられたボールが描く放物線も、惑星が太陽の周りを回る軌道も、さらには量子力学の世界で光の粒子がたどる経路さえも、すべてこの最小作用の原理によって支配されています\footnote{Principle of Least Action: Derivation, Example \& Application - StudySmarter}。私たちが解き明かした最速降下曲線問題は、この壮大な原理の、最もわかりやすく美しい現れの一つだったのです。

変分法という強力な道具は、物理学だけでなく、人間が作り出した世界にも応用されています。経済学や金融工学における最適ポートフォリオの決定、あるいは人工知能(AI)が学習する際に、予測誤差(損失関数)を最小化するプロセスなど、様々な分野で「何かを最適化する」という問題は普遍的に存在し、変分的な考え方がその解決の鍵を握っています\footnote{汎関数と変分法について、わかりやすく解説 - U-知能デバイス研究所}。

ヨハン・ベルヌーイが投げかけた一つの問いから始まった私たちの旅は、サイクロイドという美しい曲線の発見につながり、さらには物理学の根幹をなす深遠な原理へと私たちを導いてくれました。それは、一見複雑で多様に見える自然現象や社会現象の背後に、驚くほどシンプルで統一的な数学的法則が横たわっていることを教えてくれます。この世界を支配する、深く、美しい秩序の一端に触れることができた、この知的な興奮こそが、科学の探求がもたらす最大の喜びなのかもしれません。

\end{document}
