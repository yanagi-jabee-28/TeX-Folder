% !TEX program = lualatex
%==============================================================================
% プリアンブル (Preamble)
%==============================================================================

\documentclass[a4paper, 11pt]{ltjsarticle}

%------------------------------------------------------------------------------
% パッケージ読み込み
%------------------------------------------------------------------------------
\usepackage[margin=2.5cm]{geometry}
\usepackage{amsmath}           % 数式
\usepackage{booktabs}          % 表
\usepackage{siunitx}           % 単位
\usepackage{graphicx}          % 画像読み込み
\usepackage{float}             % 画像配置の制御
\usepackage{luatexja-fontspec} % 和文フォント
\usepackage{listings}          % ソースコード
\usepackage{xcolor}            % 色

% listings の設定
\lstset{
	basicstyle=\ttfamily\small,
	keywordstyle=\color{blue}\bfseries,
	commentstyle=\color{green!40!black},
	stringstyle=\color{red!60!black},
	numbers=left,
	numberstyle=\tiny,
	stepnumber=1,
	numbersep=5pt,
	frame=single,
	breaklines=true,
	breakatwhitespace=true,
	columns=fullflexible,
	showstringspaces=false,
	language=Python,
	inputencoding=utf8,
	captionpos=b
}

%------------------------------------------------------------------------------
% 各種設定
%------------------------------------------------------------------------------

% --- 和文フォント設定 ---
\setmainjfont[Renderer=HarfBuzz]{Yu Mincho}
\setsansjfont[Renderer=HarfBuzz]{Yu Gothic}

% --- ドキュメント情報 ---
\title{画像処理・画像処理工学 レポート課題1}
\author{画像処理工学科 学籍番号: 21239 組番号:234 5E 氏名:栁原 魁人}
\date{\today}

%==============================================================================
% ドキュメント本体 (Body)
%==============================================================================
\begin{document}

\maketitle
\thispagestyle{empty}
\clearpage

\section{課題2概要}
このレポートでは、画像処理・画像処理工学の課題2に取り組みます。以下の4つの問題について、理論、プログラムリスト、結果、考察を含めて報告します。

\section{問題1: モルフォロジー処理によるノイズ除去}

\subsection{理論}
モルフォロジー処理は、二値画像に対して幾何学的な形態操作を行う処理です。主な操作には以下があります:
\begin{itemize}
	\item \textbf{膨張(Dilation)}:画像内の白領域を拡大する処理
	\item \textbf{収縮(Erosion)}:画像内の白領域を縮小する処理
	\item \textbf{開処理(Opening)}:収縮の後に膨張を行う処理。小さなノイズを除去する
	\item \textbf{閉処理(Closing)}:膨張の後に収縮を行う処理。小さな黒いノイズを埋める
\end{itemize}


\subsection{結果}
開処理により小さな孤立ノイズが効果的に除去されました。閉処理により、黒いノイズも埋められました。

\subsection{考察}
開処理と閉処理を組み合わせることで、効果的にノイズを除去できます。構造要素のサイズを調整することで、除去するノイズの大きさを制御できます。

\clearpage

\section{問題2: JPEG品質と圧縮率の関係}

\subsection{理論}
JPEG圧縮では、品質パラメータ(0-100)により圧縮率と画質が変化します。品質が低いほど圧縮率は高くなりますが、画質が低下します。SSIM(Structural Similarity Index)を用いて画質を定量的に評価できます:
\[
\text{SSIM} = \frac{(2\mu_x\mu_y + C_1)(2\sigma_{xy} + C_2)}{(\mu_x^2 + \mu_y^2 + C_1)(\sigma_x^2 + \sigma_y^2 + C_2)}
\]


\subsection{結果}
JPEG品質と圧縮率の関係を分析すると、品質70〜80では十分な画質を保ちながら高い圧縮率を達成できます。

\subsection{考察}
推奨されるJPEG品質は75〜85の範囲です。この範囲では、視覚的な品質低下が最小限に抑えられながら、データ圧縮率が30\%程度達成できます。

\clearpage

\section{問題3: 2次元FFTと振幅スペクトル}

\subsection{理論}
2次元フーリエ変換は、画像を周波数領域に変換します。振幅スペクトルは周波数成分の大きさを表します。対数スケール変換により、弱い周波数成分も可視化できます:
\[
S(\omega_x, \omega_y) = \log(1 + |F(\omega_x, \omega_y)|)
\]


\subsection{結果}
振幅スペクトルより、低周波成分が中心に集中していることが観察されました。

\subsection{考察}
自然画像では通常、低周波成分が支配的です。スペクトルの分布から、画像の周波数特性が理解できます。

\clearpage

\section{問題4: 周波数フィルタの応用}

\subsection{理論}
周波数フィルタは周波数領域で画像を処理します。主なフィルタ種類:
\begin{itemize}
	\item \textbf{ローパスフィルタ}:低周波を通す。ノイズ除去に使用
	\item \textbf{ハイパスフィルタ}:高周波を通す。エッジ検出に使用
	\item \textbf{理想フィルタ}:遮断周波数で急峻に変化
	\item \textbf{ガウシアンフィルタ}:滑らかに変化。リンギングが少ない
\end{itemize}

\subsection{結果}
ローパスフィルタはノイズを除去して画像を平滑化し、ハイパスフィルタはエッジを強調しました。

\subsection{考察}
ガウシアンフィルタは理想フィルタと異なり、周波数応答が滑らかに変化するため、逆FFT後のリンギング成分が少なく、実用的です。カットオフ周波数の選択により、処理結果の特性を制御できます。

\clearpage

\clearpage

\section*{付録: プログラムリスト}
本レポートの課題2で使用したPythonプログラムを以下に示す.

\subsection*{問題1: モルフォロジー処理}

\lstinputlisting[caption={問題1 モルフォロジー処理によるノイズ除去}, label={lst:code1}, language=Python]{../問題2_1.py}

\clearpage

\subsection*{問題2: JPEG品質と圧縮率}

\lstinputlisting[caption={問題2 JPEG品質と圧縮率の関係調査}, label={lst:code2}, language=Python]{../問題2_2.py}

\clearpage

\subsection*{問題3: 2次元FFTと振幅スペクトル}

\lstinputlisting[caption={問題3 2次元FFTと振幅スペクトル}, label={lst:code3}, language=Python]{../問題2_3.py}

\clearpage

\subsection*{問題4: 周波数フィルタの応用}

\lstinputlisting[caption={問題4 周波数フィルタの応用}, label={lst:code4}, language=Python]{../問題2_4.py}

\end{document}