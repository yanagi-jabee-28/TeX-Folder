% !TEX program = lualatex
%==============================================================================
% プリアンブル (Preamble)
%==============================================================================

\documentclass[a4paper, 11pt]{ltjsarticle}

%------------------------------------------------------------------------------
% パッケージ読み込み
%------------------------------------------------------------------------------
\usepackage[margin=2.5cm]{geometry}
\usepackage{amsmath}           % 高度な数式環境 (aligned, pmatrixなど)
\usepackage{booktabs}          % 見栄えの良い表 (\toprule, \midrule, \bottomrule)
\usepackage{siunitx}           % 物理単位の記述 (\SI{}{})
\usepackage{luatexja-fontspec} % LuaLaTeX向け和文フォント設定

%------------------------------------------------------------------------------
% 各種設定
%------------------------------------------------------------------------------

% --- 和文フォント設定 ---
% ご利用の環境にインストールされているフォント名を指定してください。
\setmainjfont[Renderer=HarfBuzz]{Yu Mincho}
\setsansjfont[Renderer=HarfBuzz]{Yu Gothic}

% --- ドキュメント情報 ---
\title{画像処理・画像処理工学 レポート課題1 解答}
\author{} % 氏名などを記載
\date{\today}

%==============================================================================
% ドキュメント本体 (Body)
%==============================================================================
\begin{document}

\maketitle
\thispagestyle{empty}
\clearpage

\section{問題1}

\subsection{複合レンズによる像の位置と大きさ}

\subsubsection{計算・導出過程}
\paragraph{ステップ1:凸レンズAによる中間像の計算}
まず,凸レンズAによってできる中間像の位置 $b_A$ をレンズの公式 $\frac{1}{a} + \frac{1}{b} = \frac{1}{f}$ を用いて求める.
与えられた値は,物体距離 $a_A = \SI{16}{cm}$,焦点距離 $f_A = \SI{12}{cm}$ である.
\[
\frac{1}{16} + \frac{1}{b_A} = \frac{1}{12}
\]
\[
\frac{1}{b_A} = \frac{1}{12} - \frac{1}{16} = \frac{4}{48} - \frac{3}{48} = \frac{1}{48}
\]
\[
b_A = \SI{48}{cm}
\]
次に,中間像の大きさを求めるために倍率 $m_A$ を計算する.
\[
m_A = -\frac{b_A}{a_A} = -\frac{48}{16} = -3
\]
元の物体の大きさが $h = \SI{2.0}{cm}$ なので,中間像の大きさ $h_A'$ は,
\[
h_A' = h \times |m_A| = 2.0 \times 3 = \SI{6.0}{cm}
\]

\paragraph{ステップ2:凸レンズBによる最終的な像の計算}
ステップ1でできた中間像を,凸レンズBの物体と考える.
凸レンズBから中間像までの距離(物体距離 $a_B$)は,AB間の距離 $\SI{63}{cm}$ と $b_A$ から計算できる.
\[
a_B = 63 - b_A = 63 - 48 = \SI{15}{cm}
\]
凸レンズBの焦点距離 $f_B = \SI{10}{cm}$ を用いて,最終的な像の位置 $b_B$ を求める.
\[
\frac{1}{15} + \frac{1}{b_B} = \frac{1}{10}
\]
\[
\frac{1}{b_B} = \frac{1}{10} - \frac{1}{15} = \frac{3}{30} - \frac{2}{30} = \frac{1}{30}
\]
\[
b_B = \SI{30}{cm}
\]
最後に,最終的な像の大きさを求めるために倍率 $m_B$ を計算する.
\[
m_B = -\frac{b_B}{a_B} = -\frac{30}{15} = -2
\]
中間像の大きさ $h_A' = \SI{6.0}{cm}$ から,最終的な像の大きさ $h_B'$ は,
\[
h_B' = h_A' \times |m_B| = 6.0 \times 2 = \SI{12.0}{cm}
\]

\subsubsection*{答え}
\begin{itemize}
    \item 像の位置: 凸レンズBの後方 \SI{30}{cm}
    \item 像の大きさ: \SI{12.0}{cm}
\end{itemize}

\begin{center}\rule{0.5\linewidth}{0.4pt}\end{center}

\subsection{各デバイスのppi計算と比較}

\subsubsection{計算・導出過程}
ppi (pixels per inch) は,画面の横 ($w_p$) と縦 ($h_p$) の画素数,および画面の対角線の長さ $d_i$ (インチ) を用いて,以下の公式で計算する.
\[
\text{ppi} = \frac{\sqrt{w_p^2 + h_p^2}}{d_i}
\]
\paragraph{A. Google Pixel 10} ($w_p=1080$, $h_p=2424$, $d_i=6.3$)
\[
\text{ppi} = \frac{\sqrt{1080^2 + 2424^2}}{6.3} = \frac{\sqrt{1166400 + 5875776}}{6.3} = \frac{\sqrt{7042176}}{6.3} \approx \frac{2653.7}{6.3} \approx 421.22
\]
\paragraph{B. iPad (A16)} ($w_p=2360$, $h_p=1640$, $d_i=10.9$)
\[
\text{ppi} = \frac{\sqrt{2360^2 + 1640^2}}{10.9} = \frac{\sqrt{5569600 + 2689600}}{10.9} = \frac{\sqrt{8259200}}{10.9} \approx \frac{2873.9}{10.9} \approx 263.66
\]
\paragraph{C. EIZO EV2740S} ($w_p=3840$, $h_p=2160$, $d_i=27.0$)
\[
\text{ppi} = \frac{\sqrt{3840^2 + 2160^2}}{27.0} = \frac{\sqrt{14745600 + 4665600}}{27.0} = \frac{\sqrt{19411200}}{27.0} \approx \frac{4405.8}{27.0} \approx 163.18
\]

\subsubsection*{答え}
各デバイスのppiを整数値で求めると以下の通りである.
\begin{center}
    \begin{tabular}{lr}
        \toprule
        デバイス名 & ppi (整数値) \\
        \midrule
        Google Pixel 10 & 421 ppi \\
        iPad (A16)      & 264 ppi \\
        EIZO EV2740S    & 163 ppi \\
        \bottomrule
    \end{tabular}
\end{center}

\subsection{8近傍鮮鋭化フィルタの導出}

\subsubsection{導出過程}
鮮鋭化フィルタは,「元の画像」に「エッジ成分」を足し合わせることで得られる.これをフィルタオペレータで表現すると以下の関係になる.
\[
W_S (\text{鮮鋭化}) = W_I (\text{単位}) - W_L (\text{ラプラシアン})
\]
この式におけるラプラシアンフィルタ $W_L$ を,その基本原理である2次差分から導出する.

\paragraph{ステップ1:各方向の2次差分の導出}
デジタル画像における微分は,隣接する画素値の差分(変化量)で近似される.2次微分は,その「変化量の変化量」を計算することで求められる.注目画素を $f(i, j)$ とする.
\begin{itemize}
    \item \textbf{x方向2次差分 ($f_{xx}$)}
    \begin{enumerate}
        \item まず,注目画素 $f(i, j)$ 周辺のx方向の一次差分(変化量)を考える.
        \begin{itemize}
            \item 右隣との差分: $f(i+1, j) - f(i, j)$
            \item 左隣との差分: $f(i, j) - f(i-1, j)$
        \end{itemize}
        \item 次に,これらの差分のさらに差分を取ることで,2次差分を求める.
        \[
        \begin{aligned}
        f_{xx} & = \{f(i+1, j) - f(i, j)\} - \{f(i, j) - f(i-1, j)\} \\
               & = f(i+1, j) - 2f(i, j) + f(i-1, j)
        \end{aligned}
        \]
    \end{enumerate}
    \item \textbf{y方向2次差分 ($f_{yy}$)}
    \begin{enumerate}
        \item 同様に,y方向の一次差分を考える.
        \begin{itemize}
            \item 下隣との差分: $f(i, j+1) - f(i, j)$
            \item 上隣との差分: $f(i, j) - f(i, j-1)$
        \end{itemize}
        \item これらの差分から2次差分を求める.
        \[
        \begin{aligned}
        f_{yy} & = \{f(i, j+1) - f(i, j)\} - \{f(i, j) - f(i, j-1)\} \\
               & = f(i, j+1) - 2f(i, j) + f(i, j-1)
        \end{aligned}
        \]
    \end{enumerate}
    \item \textbf{斜め方向1 (左上-右下) 2次差分 ($f_{d1}$)}
    \begin{enumerate}
        \item 左上から右下への対角線方向の一次差分を考える.
        \begin{itemize}
            \item 右下隣との差分: $f(i+1, j+1) - f(i, j)$
            \item 左上隣との差分: $f(i, j) - f(i-1, j-1)$
        \end{itemize}
        \item これらの差分から2次差分を求める.
        \[
        \begin{aligned}
        f_{d1} & = \{f(i+1, j+1) - f(i, j)\} - \{f(i, j) - f(i-1, j-1)\} \\
               & = f(i+1, j+1) - 2f(i, j) + f(i-1, j-1)
        \end{aligned}
        \]
    \end{enumerate}
    \item \textbf{斜め方向2 (右上-左下) 2次差分 ($f_{d2}$)}
    \begin{enumerate}
        \item 右上から左下への対角線方向の一次差分を考える.
        \begin{itemize}
            \item 左下隣との差分: $f(i-1, j+1) - f(i, j)$
            \item 右上隣との差分: $f(i, j) - f(i+1, j-1)$
        \end{itemize}
        \item これらの差分から2次差分を求める.
        \[
        \begin{aligned}
        f_{d2} & = \{f(i-1, j+1) - f(i, j)\} - \{f(i, j) - f(i+1, j-1)\} \\
               & = f(i-1, j+1) - 2f(i, j) + f(i+1, j-1)
        \end{aligned}
        \]
    \end{enumerate}
\end{itemize}

\paragraph{ステップ2:8近傍ラプラシアンフィルタ ($W_L$) の導出}
8近傍ラプラシアン $\nabla^2 f$ は,これら4方向すべての2次差分の総和として定義される.
\[
\nabla^2 f = f_{xx} + f_{yy} + f_{d1} + f_{d2}
\]
ステップ1で導出した4つの式をすべて足し合わせると,注目画素 $f(i,j)$ の係数は-8,周囲8画素の係数はすべて1となる.この係数を3x3の行列に配置することで,8近傍ラプラシアンフィルタ $W_L$ が得られる.
\[
W_L =
\begin{pmatrix}
1 & 1 & 1 \\
1 & -8 & 1 \\
1 & 1 & 1
\end{pmatrix}
\]

\paragraph{ステップ3:8近傍鮮鋭化フィルタ ($W_S$) の導出}
単位オペレータ $W_I$(適用しても画像が変化しないフィルタ)と,$W_L$ を用いて $W_S$ を計算する.
\[
W_I =
\begin{pmatrix}
0 & 0 & 0 \\
0 & 1 & 0 \\
0 & 0 & 0
\end{pmatrix}
\]
\[
W_S = W_I - W_L =
\begin{pmatrix}
0 & 0 & 0 \\
0 & 1 & 0 \\
0 & 0 & 0
\end{pmatrix}
-
\begin{pmatrix}
1 & 1 & 1 \\
1 & -8 & 1 \\
1 & 1 & 1
\end{pmatrix}
=
\begin{pmatrix}
0-1 & 0-1 & 0-1 \\
0-1 & 1-(-8) & 0-1 \\
0-1 & 0-1 & 0-1
\end{pmatrix}
=
\begin{pmatrix}
-1 & -1 & -1 \\
-1 & 9 & -1 \\
-1 & -1 & -1
\end{pmatrix}
\]

\subsubsection*{答え}
導出された8近傍鮮鋭化フィルタのオペレータは以下の通りである.
\[
W_S =
\begin{pmatrix}
-1 & -1 & -1 \\
-1 & 9 & -1 \\
-1 & -1 & -1
\end{pmatrix}
\]

\subsection{Sobelフィルタによるエッジの強度と方向の算出}

\subsubsection{計算・導出過程}
Sobelフィルタは,x方向とy方向のオペレータを用いて一次微分を計算する.
\[
S_x =
\begin{pmatrix}
-1 & 0 & 1 \\
-2 & 0 & 2 \\
-1 & 0 & 1
\end{pmatrix}
, \quad
S_y =
\begin{pmatrix}
-1 & -2 & -1 \\
0 & 0 & 0 \\
1 & 2 & 1
\end{pmatrix}
\]
エッジ強度 $G$ と方向 $\theta$ は以下の式で求める.
\[
G = \sqrt{G_x^2 + G_y^2}
\]
\[
\theta = \arctan\left(\frac{G_y}{G_x}\right)
\]
\paragraph{注目画素 A}
Aを中心とする3x3領域:
\[
\text{Image}_A =
\begin{pmatrix}
50 & 50 & 100 \\
50 & 50 & 100 \\
100 & 100 & 150
\end{pmatrix}
\]
\begin{itemize}
    \item 一次微分:\\
    $G_x = (100 \times 1 + 100 \times 2 + 150 \times 1) - (50 \times 1 + 50 \times 2 + 100 \times 1) = (100+200+150) - (50+100+100) = 450 - 250 = 200$\\
    $G_y = (100 \times 1 + 100 \times 2 + 150 \times 1) - (50 \times 1 + 50 \times 2 + 100 \times 1) = (100+200+150) - (50+100+100) = 450 - 250 = 200$
    \item エッジ強度:\\
    $G_A = \sqrt{200^2 + 200^2} = \sqrt{40000 + 40000} = \sqrt{80000} \approx 282.84$
    \item エッジ方向:\\
    $\theta_A = \arctan\left(\frac{200}{200}\right) = \arctan(1) = 45^{\circ}$
\end{itemize}

\paragraph{注目画素 B}
Bを中心とする3x3領域:
\[
\text{Image}_B =
\begin{pmatrix}
150 & 250 & 250 \\
150 & 150 & 250 \\
100 & 100 & 200
\end{pmatrix}
\]
\begin{itemize}
    \item 一次微分:\\
    $G_x = (250 \times 1 + 250 \times 2 + 200 \times 1) - (150 \times 1 + 150 \times 2 + 100 \times 1) = (250+500+200) - (150+300+100) = 950 - 550 = 400$\\
    $G_y = (100 \times 1 + 100 \times 2 + 200 \times 1) - (150 \times 1 + 250 \times 2 + 250 \times 1) = (100+200+200) - (150+500+250) = 500 - 900 = -400$
    \item エッジ強度:\\
    $G_B = \sqrt{400^2 + (-400)^2} = \sqrt{160000 + 160000} = \sqrt{320000} \approx 565.69$
    \item エッジ方向:\\
    $\theta_B = \arctan\left(\frac{-400}{400}\right) = \arctan(-1) = -45^{\circ}$
\end{itemize}

\subsubsection*{答えと両者の比較}
\begin{center}
    \begin{tabular}{lrr}
        \toprule
         & 注目画素 A & 注目画素 B \\
        \midrule
        x方向微分 $G_x$ & 200 & 400 \\
        y方向微分 $G_y$ & 200 & -400 \\
        エッジ強度 $G$ & 約 282.8 & 約 565.7 \\
        エッジ方向 $\theta$ & $45^{\circ}$ & $-45^{\circ}$ \\
        \bottomrule
    \end{tabular}
\end{center}
\paragraph{比較}
\begin{itemize}
    \item \textbf{エッジ強度}: 注目画素Bのエッジ強度はAの約2倍であり,B周辺の方がより急峻な輝度変化(強いエッジ)があることを示している.
    \item \textbf{エッジ方向}: Aのエッジは輝度が左下から右上($45^{\circ}$)に向かって最も強くなる向きであるのに対し,Bのエッジは左上から右下($-45^{\circ}$)に向かって最も強くなる向きであり,両者のエッジの傾きが異なることを示している.
\end{itemize}

\end{document}