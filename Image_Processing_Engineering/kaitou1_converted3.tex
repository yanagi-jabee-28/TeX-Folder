% !TEX program = lualatex
%==============================================================================
% プリアンブル (Preamble)
%==============================================================================

\documentclass[a4paper, 11pt]{ltjsarticle}

%------------------------------------------------------------------------------
% パッケージ読み込み
%------------------------------------------------------------------------------
\usepackage[margin=2.5cm]{geometry}
\usepackage{amsmath}           % 数式
\usepackage{booktabs}          % 表
\usepackage{siunitx}           % 単位
\usepackage{graphicx}          % 画像読み込み
\usepackage{float}             % 画像配置の制御
\usepackage{luatexja-fontspec} % 和文フォント
\usepackage{listings}          % ソースコード
\usepackage{xcolor}            % 色

% listings の設定
\lstset{
    basicstyle=\ttfamily\small,
    keywordstyle=\color{blue}\bfseries,
    commentstyle=\color{green!40!black},
    stringstyle=\color{red!60!black},
    numbers=left,
    numberstyle=\tiny,
    stepnumber=1,
    numbersep=5pt,
    frame=single,
    breaklines=true,
    breakatwhitespace=true,
    columns=fullflexible,
    showstringspaces=false,
    language=Python,
    inputencoding=utf8,
    captionpos=b
}

%------------------------------------------------------------------------------
% 各種設定
%------------------------------------------------------------------------------

% --- 和文フォント設定 ---
\setmainjfont[Renderer=HarfBuzz]{Yu Mincho}
\setsansjfont[Renderer=HarfBuzz]{Yu Gothic}

% --- ドキュメント情報 ---
\title{画像処理・画像処理工学 レポート課題1}
\author{画像処理工学科 学生番号: 25X0000 氏名: 画像 太郎}
\date{\today}

%==============================================================================
% ドキュメント本体 (Body)
%==============================================================================
\begin{document}

\maketitle
\thispagestyle{empty}
\clearpage

\section{問題1}

\subsection{1) 複合レンズ系による結像}

本問では、凸レンズの公式 $\frac{1}{a} + \frac{1}{b} = \frac{1}{f}$ を用い、レンズAによる像を求めた後、その像をレンズBの物体として再度計算を行う。

\subsubsection{凸レンズAによる像(中間像)の計算}
レンズAについて、物体距離 $a_A = 16\,\text{cm}$、焦点距離 $f_A = 12\,\text{cm}$ である。像の位置 $b_A$ を求める。
\[
\frac{1}{16} + \frac{1}{b_A} = \frac{1}{12}
\]
\[
\frac{1}{b_A} = \frac{1}{12} - \frac{1}{16}
\]
\[
\frac{1}{b_A} = \frac{4}{48} - \frac{3}{48}
\]
\[
\frac{1}{b_A} = \frac{1}{48}
\]
よって、$b_A = 48\,\text{cm}$ である。これはレンズAの後方 $48\,\text{cm}$ の位置に実像ができることを示す。

次に倍率 $m_A$ を求める。
\[
m_A = -\frac{b_A}{a_A} = -\frac{48}{16} = -3
\]
物体の大きさ $h = 2.0\,\text{cm}$ より、中間像の大きさ $h'_A$ は以下のようになる。
\[
h'_A = |m_A| \times h = 3 \times 2.0 = 6.0\,\text{cm}
\]

\subsubsection{凸レンズBによる像(最終像)の計算}
レンズAとレンズBの間隔は $63\,\text{cm}$ である。レンズAによる像(レンズAの後方 $48\,\text{cm}$)をレンズBに対する物体とみなす。
レンズBから見た物体距離 $a_B$ は以下の通りである。
\[
a_B = 63 - 48 = 15\,\text{cm}
\]
レンズBの焦点距離 $f_B = 10\,\text{cm}$ を用い、最終像の位置 $b_B$ を求める。
\[
\frac{1}{15} + \frac{1}{b_B} = \frac{1}{10}
\]
移項して通分を行う。
\[
\frac{1}{b_B} = \frac{1}{10} - \frac{1}{15}
\]
\[
\frac{1}{b_B} = \frac{3}{30} - \frac{2}{30}
\]
\[
\frac{1}{b_B} = \frac{1}{30}
\]
よって、$b_B = 30\,\text{cm}$ である。

倍率 $m_B$ を求める。
\[
m_B = -\frac{b_B}{a_B} = -\frac{30}{15} = -2
\]
最終的な像の大きさ $h'_{B}$ は、中間像の大きさ $h'_A$ に倍率を乗じて求められる。
\[
h'_{B} = |m_B| \times h'_A = 2 \times 6.0 = 12.0\,\text{cm}
\]

\subsubsection*{解答}
\begin{itemize}
    \item 像の位置: 凸レンズBの後方 \textbf{30 cm}
    \item 像の大きさ: \textbf{12.0 cm}
\end{itemize}
\hrulefill

\subsection{2) デバイスのppi比較}

画素密度 ppi (pixels per inch) は、画面解像度の幅 $w$、高さ $h$、および対角線インチ数 $d$ を用いて以下の式で定義される。
\[
\text{ppi} = \frac{\sqrt{w^2 + h^2}}{d}
\]

\subsubsection{Google Pixel 10}
$d=6.3$インチ, $w=1080$, $h=2424$ を代入する。
まず対角線の画素数を計算する。
\[
\sqrt{1080^2 + 2424^2} = \sqrt{1166400 + 5875776} = \sqrt{7042176} \approx 2653.71
\]
インチ数で除算する。
\[
\text{ppi} = \frac{2653.71}{6.3} \approx 421.22
\]

\subsubsection{iPad (A16)}
$d=10.9$インチ, $w=2360$, $h=1640$ を代入する。
\[
\sqrt{2360^2 + 1640^2} = \sqrt{5569600 + 2689600} = \sqrt{8259200} \approx 2873.88
\]
\[
\text{ppi} = \frac{2873.88}{10.9} \approx 263.65
\]

\subsubsection{EIZO EV2740S}
$d=27.0$インチ, $w=3840$, $h=2160$ を代入する。
\[
\sqrt{3840^2 + 2160^2} = \sqrt{14745600 + 4665600} = \sqrt{19411200} \approx 4405.81
\]
\[
\text{ppi} = \frac{4405.81}{27.0} \approx 163.17
\]

\subsubsection*{解答}
整数値で比較すると以下の通りとなる。
\begin{itemize}
    \item Google Pixel 10: \textbf{421 ppi}
    \item iPad (A16): \textbf{264 ppi}
    \item EIZO EV2740S: \textbf{163 ppi}
\end{itemize}
\hrulefill

\subsection{3) 8近傍鮮鋭化フィルタの導出}

\subsubsection{ラプラシアンフィルタの導出}
画像の2次微分(ラプラシアン)$\nabla^2 f$ は、注目画素を中心とした近傍画素との差分によって近似できる。8近傍の場合、水平($x$)、垂直($y$)に加え、対角2方向($d_1, d_2$)の2階差分の和をとる。
\[
\nabla^2 f \approx \frac{\partial^2 f}{\partial x^2} + \frac{\partial^2 f}{\partial y^2} + \frac{\partial^2 f}{\partial d_1^2} + \frac{\partial^2 f}{\partial d_2^2}
\]
各方向の2階差分係数は $(1, -2, 1)$ であるため、中心画素 $(i,j)$ における係数の総和は $-2 \times 4 = -8$ となる。周辺の8画素はそれぞれ係数 $1$ を持つ。
よって、8近傍ラプラシアンフィルタのカーネル $H_L$ は以下のようになる。
\[
H_L =
\begin{pmatrix}
1 & 1 & 1 \\
1 & -8 & 1 \\
1 & 1 & 1
\end{pmatrix}
\]

\subsubsection{鮮鋭化フィルタの導出}
鮮鋭化(アンシャープマスキング)は、元画像からラプラシアン(エッジ成分の2次微分)を引くことで行われる。
\[
g(x,y) = f(x,y) - \nabla^2 f(x,y)
\]
これを行列演算として記述する。元画像を表す単位インパルスフィルタ(恒等写像)を $H_I$ とすると、鮮鋭化フィルタ $H_S$ は $H_I - H_L$ で求められる。
\[
H_S = H_I - H_L =
\begin{pmatrix}
0 & 0 & 0 \\
0 & 1 & 0 \\
0 & 0 & 0
\end{pmatrix}
-
\begin{pmatrix}
1 & 1 & 1 \\
1 & -8 & 1 \\
1 & 1 & 1
\end{pmatrix}
\]
各要素ごとの引き算を行う。
\[
H_S =
\begin{pmatrix}
0-1 & 0-1 & 0-1 \\
0-1 & 1-(-8) & 0-1 \\
0-1 & 0-1 & 0-1
\end{pmatrix}
=
\begin{pmatrix}
-1 & -1 & -1 \\
-1 & 9 & -1 \\
-1 & -1 & -1
\end{pmatrix}
\]

\subsubsection*{解答}
導出された8近傍鮮鋭化フィルタのオペレータ:
\[
\begin{pmatrix}
-1 & -1 & -1 \\
-1 & 9 & -1 \\
-1 & -1 & -1
\end{pmatrix}
\]
\hrulefill

\subsection{4) Sobelフィルタによるエッジ解析}

図A-3より、各注目画素の近傍領域における画素値分布を読み取り、計算を行う。
Sobelフィルタのカーネル $S_x$(水平微分)、$S_y$(垂直微分)を以下のように定義する。
\[
S_x = \begin{pmatrix} -1 & 0 & 1 \\ -2 & 0 & 2 \\ -1 & 0 & 1 \end{pmatrix}, \quad
S_y = \begin{pmatrix} -1 & -2 & -1 \\ 0 & 0 & 0 \\ 1 & 2 & 1 \end{pmatrix}
\]

\subsubsection{注目画素A (1,1) の計算}
画素Aの中心画素値は50である。図より、この領域は左右で輝度が変化する縦エッジ(左が暗く右が明るい)であると読み取れるため、以下の画素値を仮定して計算する。
\[
\text{近傍}_A = \begin{pmatrix} 0 & 50 & 100 \\ 0 & 50 & 100 \\ 0 & 50 & 100 \end{pmatrix}
\]

\textbf{x方向微分 $g_x$ の計算(畳み込み):}
\[
g_x = (-1 \cdot 0) + (0 \cdot 50) + (1 \cdot 100) + (-2 \cdot 0) + (0 \cdot 50) + (2 \cdot 100) + (-1 \cdot 0) + (0 \cdot 50) + (1 \cdot 100)
\]
\[
g_x = 100 + 200 + 100 = 400
\]

\textbf{y方向微分 $g_y$ の計算:}
上段と下段の画素値が等しいため、相殺されて0となる。
\[
g_y = (-1 \cdot 0 -2 \cdot 50 -1 \cdot 100) + (1 \cdot 0 + 2 \cdot 50 + 1 \cdot 100) = -200 + 200 = 0
\]

\textbf{エッジ強度 $G$ と方向 $\theta$:}
\[
G_A = \sqrt{400^2 + 0^2} = 400
\]
\[
\theta_A = \arctan\left(\frac{0}{400}\right) = 0^\circ \quad (\text{垂直エッジ})
\]

\subsubsection{注目画素B (3,3) の計算}
画素Bの中心画素値は150である。図より、この領域は上下で輝度が変化する横エッジ(上が暗く下が明るい)であると読み取れるため、以下の画素値を仮定する。
\[
\text{近傍}_B = \begin{pmatrix} 50 & 50 & 50 \\ 150 & 150 & 150 \\ 250 & 250 & 250 \end{pmatrix}
\]

\textbf{x方向微分 $g_x$ の計算:}
左列と右列の画素値が等しいため、相殺されて0となる。
\[
g_x = (-1 \cdot 50 + 1 \cdot 50) + (-2 \cdot 150 + 2 \cdot 150) + (-1 \cdot 250 + 1 \cdot 250) = 0
\]

\textbf{y方向微分 $g_y$ の計算:}
\[
g_y = (-1 \cdot 50 -2 \cdot 50 -1 \cdot 50) + (0) + (1 \cdot 250 + 2 \cdot 250 + 1 \cdot 250)
\]
\[
g_y = (-50 -100 -50) + (250 + 500 + 250) = -200 + 1000 = 800
\]

\textbf{エッジ強度 $G$ と方向 $\theta$:}
\[
G_B = \sqrt{0^2 + 800^2} = 800
\]
\[
\theta_B = \arctan\left(\frac{800}{0}\right) = 90^\circ \quad (\text{水平エッジ})
\]

\subsubsection*{比較結果}
\begin{itemize}
    \item \textbf{強度}: 画素Bのエッジ強度(800)は、画素A(400)の2倍である。画素B周辺の方が輝度変化が急峻である。
    \item \textbf{方向}: 画素Aは $0^\circ$(垂直方向のエッジ)、画素Bは $90^\circ$(水平方向のエッジ)である。
\end{itemize}

\clearpage

\section{問題2}

\subsection{1) 線形変換によるコントラスト強調}
図A-4のグラフから、入力画素値 $f$ と出力画素値 $g$ の関係式(トーンカーブ)を導出した。
グラフは入力範囲 $0 \le f \le 200$ において直線的に増加し、$200 < f \le 255$ では飽和している。
直線の始点は $(0, 15)$、終点は $(200, 255)$ であるため、傾き $a$ は以下のようになる。
\[
a = \frac{255 - 15}{200 - 0} = \frac{240}{200} = 1.2
\]
切片は15であるため、変換式は以下の通りである。
\[
g(f) = \begin{cases} 1.2f + 15 & (0 \le f \le 200) \\ 255 & (200 < f \le 255) \end{cases}
\]
この変換を適用した結果、元画像で低輝度に集中していた分布が全体的に明るい方向へシフトし、ヒストグラムの幅が広がった。これにより画像のコントラストが強調され、視認性が向上したことが確認できた。

\subsection{2) 輝度反転}
画像のネガポジ反転を行うため、以下の変換式を用いた。
\[
g(f) = 255 - f
\]
これは8ビット画像の最大輝度255から入力値を引く操作であり、黒(0)は白(255)に、白(255)は黒(0)に変換される。
処理結果として、元画像の明暗が完全に逆転した画像が得られた。ヒストグラムにおいても、輝度分布が左右反転(高輝度側と低輝度側が入れ替わる)していることが確認でき、理論通りの結果となった。

\subsection{3) 空間フィルタリングによるノイズ除去}
ノイズ除去には\textbf{メディアンフィルタ}(サイズ $3 \times 3$)を採用した。
選択の理由は、対象画像に含まれるノイズが、ランダムに白や黒の点が現れる「ごま塩ノイズ(インパルスノイズ)」であったためである。平均化フィルタではノイズ成分が周囲にぼやけて広がってしまうのに対し、メディアンフィルタは注目画素周辺の中央値を出力するため、突発的な外れ値であるインパルスノイズを効果的に除去しつつ、物体の輪郭(エッジ)を比較的保存できる特性がある。
適用結果として、エッジの鋭さを保ったままノイズのみがきれいに除去された画像が得られた。

\subsection{4) Canny法によるエッジ検出}
ノイズ除去後の画像に対し、Canny法を用いてエッジ検出を行った。
Canny法は、ガウシアンフィルタによる平滑化、Sobelフィルタによる勾配計算、非極大抑制、およびヒステリシス閾値処理の4段階からなるアルゴリズムである。
本課題では、ヒステリシス処理における2つの閾値(min\_val, max\_val)の調整が重要であった。
実験の結果、\textbf{min\_val=100, max\_val=200} に設定した際に、建物の主要な輪郭線が途切れることなく、かつ不要なテクスチャや微細なノイズを拾わずに検出できた。
この結果から、Canny法はパラメータ調整により、目的とする構造的特徴のみを抽出する能力に優れていることが確認できた。

\section*{付録: プログラムリスト}
本レポートの課題2で使用したPythonプログラムを以下に示す.

\lstinputlisting[caption={画像処理プログラム (report\_image\_analysis3.py)}, label={lst:code}]{report_image_analysis3.py}

\end{document}