% !TEX program = lualatex
%==============================================================================
% プリアンブル (Preamble)
%==============================================================================

\documentclass[a4paper, 11pt]{ltjsarticle}

%------------------------------------------------------------------------------
% パッケージ読み込み
%------------------------------------------------------------------------------
\usepackage[margin=2.5cm]{geometry}
\usepackage{amsmath}           % 高度な数式環境 (aligned, pmatrixなど)
\usepackage{amssymb}           % 数学記号 (\therefore など)
\usepackage{booktabs}          % 見栄えの良い表 (\toprule, \midrule, \bottomrule)
\usepackage{siunitx}           % 物理単位の記述 (\SI{}{})
\usepackage{luatexja-fontspec} % LuaLaTeX向け和文フォント設定

%------------------------------------------------------------------------------
% 各種設定
%------------------------------------------------------------------------------

% --- 和文フォント設定 ---
% ご利用の環境にインストールされているフォント名を指定してください。
\setmainjfont[Renderer=HarfBuzz]{Yu Mincho}
\setsansjfont[Renderer=HarfBuzz]{Yu Gothic}

% --- ドキュメント情報 ---
\title{画像処理・画像処理工学 レポート課題1 解答}
\author{} % 氏名などを記載
\date{\today}

%==============================================================================
% ドキュメント本体 (Body)
%==============================================================================
\begin{document}

\maketitle
\thispagestyle{empty}
\clearpage

\section{問題1}

\subsection{複合レンズによる像の位置と大きさ}

\subsubsection{計算・導出過程}
\paragraph{凸レンズAによる中間像}
凸レンズAによる中間像の位置 $b_A$ は,レンズの公式より以下のように求まる.
\[
\frac{1}{16} + \frac{1}{b_A} = \frac{1}{12}
\]
\[
\frac{1}{b_A} = \frac{1}{12} - \frac{1}{16} = \frac{4}{48} - \frac{3}{48} = \frac{1}{48}
\]
\[
\therefore b_A = \SI{48}{cm}
\]
このときの倍率 $m_A$ および中間像の大きさ $h_A'$ は,
\[
m_A = -\frac{48}{16} = -3
\]
\[
h_A' = 2.0 \times |-3| = \SI{6.0}{cm}
\]
となる.

\paragraph{凸レンズBによる最終的な像}
中間像は凸レンズBの前方に位置するため,これを物体として扱う.凸レンズBまでの物体距離 $a_B$ は,
\[
a_B = 63 - 48 = \SI{15}{cm}
\]
となる.凸レンズBの焦点距離 $f_B = \SI{10}{cm}$ を用いて,最終的な像の位置 $b_B$ を計算する.
\[
\frac{1}{15} + \frac{1}{b_B} = \frac{1}{10}
\]
\[
\frac{1}{b_B} = \frac{1}{10} - \frac{1}{15} = \frac{3}{30} - \frac{2}{30} = \frac{1}{30}
\]
\[
\therefore b_B = \SI{30}{cm}
\]
最終的な倍率 $m_B$ と像の大きさ $h_B'$ は以下の通り.
\[
m_B = -\frac{30}{15} = -2
\]
\[
h_B' = 6.0 \times |-2| = \SI{12.0}{cm}
\]

\subsubsection*{答え}
\begin{itemize}
    \item 像の位置: 凸レンズBの後方 \SI{30}{cm}
    \item 像の大きさ: \SI{12.0}{cm}
\end{itemize}

\begin{center}\rule{0.5\linewidth}{0.4pt}\end{center}

\subsection{各デバイスのppi計算と比較}

\subsubsection{計算・導出過程}
画面解像度 $(w_p, h_p)$ および対角線インチ数 $d_i$ から,画素密度 (ppi) は次式で得られる.
\[
\text{ppi} = \frac{\sqrt{w_p^2 + h_p^2}}{d_i}
\]
各デバイスについて計算を行う.

\paragraph{A. Google Pixel 10}
\[
\text{ppi} = \frac{\sqrt{1080^2 + 2424^2}}{6.3} = \frac{\sqrt{7042176}}{6.3} \approx \frac{2653.7}{6.3} \approx 421.22
\]

\paragraph{B. iPad (A16)}
\[
\text{ppi} = \frac{\sqrt{2360^2 + 1640^2}}{10.9} = \frac{\sqrt{8259200}}{10.9} \approx \frac{2873.9}{10.9} \approx 263.66
\]

\paragraph{C. EIZO EV2740S}
\[
\text{ppi} = \frac{\sqrt{3840^2 + 2160^2}}{27.0} = \frac{\sqrt{19411200}}{27.0} \approx \frac{4405.8}{27.0} \approx 163.18
\]

\subsubsection*{答え}
計算結果を整数値でまとめると下表の通りとなる.
\begin{center}
    \begin{tabular}{lr}
        \toprule
        デバイス名 & ppi (整数値) \\
        \midrule
        Google Pixel 10 & 421 ppi \\
        iPad (A16)      & 264 ppi \\
        EIZO EV2740S    & 163 ppi \\
        \bottomrule
    \end{tabular}
\end{center}

\subsection{8近傍鮮鋭化フィルタの導出}

\subsubsection{導出過程}
鮮鋭化フィルタオペレータ $W_S$ は,単位オペレータ $W_I$ とラプラシアンフィルタ $W_L$ を用いて次式で表される.
\[
W_S = W_I - W_L
\]
ここで,ラプラシアンフィルタ $W_L$ を構成するための2次差分を導出する.

\paragraph{各方向の2次差分の近似}
注目画素を $f(i, j)$ とし,隣接画素間の変化量の差分をとることで2次微分(ラプラシアン)を近似する.

\textbf{1) x方向およびy方向 ($f_{xx}, f_{yy}$)}
\[
\begin{aligned}
f_{xx} & = \{f(i+1, j) - f(i, j)\} - \{f(i, j) - f(i-1, j)\} \\
       & = f(i+1, j) - 2f(i, j) + f(i-1, j)
\end{aligned}
\]
\[
\begin{aligned}
f_{yy} & = \{f(i, j+1) - f(i, j)\} - \{f(i, j) - f(i, j-1)\} \\
       & = f(i, j+1) - 2f(i, j) + f(i, j-1)
\end{aligned}
\]

\textbf{2) 斜め方向 ($f_{d1}, f_{d2}$)}
同様に,対角成分についても差分をとる.
\[
\begin{aligned}
f_{d1} & = \{f(i+1, j+1) - f(i, j)\} - \{f(i, j) - f(i-1, j-1)\} \\
       & = f(i+1, j+1) - 2f(i, j) + f(i-1, j-1)
\end{aligned}
\]
\[
\begin{aligned}
f_{d2} & = \{f(i-1, j+1) - f(i, j)\} - \{f(i, j) - f(i+1, j-1)\} \\
       & = f(i-1, j+1) - 2f(i, j) + f(i+1, j-1)
\end{aligned}
\]

\paragraph{8近傍ラプラシアンフィルタ $W_L$}
全方向の2次差分の和 $\nabla^2 f = f_{xx} + f_{yy} + f_{d1} + f_{d2}$ を計算すると,注目画素 $f(i,j)$ の係数は $-2 \times 4 = -8$,周囲8近傍の画素係数はすべて $1$ となる.
したがって,対応するオペレータ $W_L$ は以下のようになる.
\[
W_L =
\begin{pmatrix}
1 & 1 & 1 \\
1 & -8 & 1 \\
1 & 1 & 1
\end{pmatrix}
\]

\paragraph{8近傍鮮鋭化フィルタ $W_S$}
単位オペレータ $W_I$ から $W_L$ を減じることで,$W_S$ が得られる.
\[
W_S =
\begin{pmatrix}
0 & 0 & 0 \\
0 & 1 & 0 \\
0 & 0 & 0
\end{pmatrix}
-
\begin{pmatrix}
1 & 1 & 1 \\
1 & -8 & 1 \\
1 & 1 & 1
\end{pmatrix}
=
\begin{pmatrix}
-1 & -1 & -1 \\
-1 & 9 & -1 \\
-1 & -1 & -1
\end{pmatrix}
\]

\subsubsection*{答え}
\[
W_S =
\begin{pmatrix}
-1 & -1 & -1 \\
-1 & 9 & -1 \\
-1 & -1 & -1
\end{pmatrix}
\]

\subsection{Sobelフィルタによるエッジの強度と方向の算出}

\subsubsection{計算・導出過程}
Sobelフィルタのオペレータ $S_x, S_y$ を用いて各方向の一次微分 $G_x, G_y$ を求め,エッジ強度 $G$ と方向 $\theta$ を算出する.
\[
G = \sqrt{G_x^2 + G_y^2}, \quad \theta = \arctan\left(\frac{G_y}{G_x}\right)
\]

\paragraph{注目画素 A}
周辺画素の値を用いて微分値を計算する.
\[
G_x = (100 + 200 + 150) - (50 + 100 + 100) = 200
\]
\[
G_y = (100 + 200 + 150) - (50 + 100 + 100) = 200
\]
よって,エッジ強度および方向は,
\[
G_A = \sqrt{200^2 + 200^2} \approx 282.84
\]
\[
\theta_A = \arctan(1) = 45^{\circ}
\]

\paragraph{注目画素 B}
同様に微分値を計算する.
\[
G_x = (250 + 500 + 200) - (150 + 300 + 100) = 400
\]
\[
G_y = (100 + 200 + 200) - (150 + 500 + 250) = -400
\]
エッジ強度および方向は以下の通り.
\[
G_B = \sqrt{400^2 + (-400)^2} \approx 565.69
\]
\[
\theta_B = \arctan(-1) = -45^{\circ}
\]

\subsubsection*{答えと考察}
\begin{center}
    \begin{tabular}{lrr}
        \toprule
         & 注目画素 A & 注目画素 B \\
        \midrule
        x方向微分 $G_x$ & 200 & 400 \\
        y方向微分 $G_y$ & 200 & -400 \\
        エッジ強度 $G$ & 約 282.8 & 約 565.7 \\
        エッジ方向 $\theta$ & $45^{\circ}$ & $-45^{\circ}$ \\
        \bottomrule
    \end{tabular}
\end{center}
算出結果を比較すると,注目画素Bのエッジ強度 $G_B$ は $G_A$ の約2倍の値を示している.これは画素B周辺の方がより急峻な輝度変化を持つことを意味する.
また,エッジ方向については,Aが左下から右上(正の傾き)であるのに対し,Bは左上から右下(負の傾き)であり,エッジの特性が異なることが確認できる.

\end{document}