% !TEX program = lualatex
%==============================================================================
% プリアンブル (Preamble)
%==============================================================================

\documentclass[a4paper, 11pt]{ltjsarticle}

%------------------------------------------------------------------------------
% パッケージ読み込み
%------------------------------------------------------------------------------
\usepackage[margin=2.5cm]{geometry}
\usepackage{amsmath}           % 数式
\usepackage{booktabs}          % 表
\usepackage{siunitx}           % 単位
\usepackage{graphicx}          % 画像読み込み (追加)
\usepackage{float}             % 画像配置の制御
\usepackage{luatexja-fontspec} % 和文フォント
\usepackage{listings}          % ソースコード
\usepackage{xcolor}            % 色

% listings の設定
\lstset{
    basicstyle=\ttfamily\small,
    keywordstyle=\color{blue}\bfseries,
    commentstyle=\color{green!40!black},
    stringstyle=\color{red!60!black},
    numbers=left,
    numberstyle=\tiny,
    stepnumber=1,
    numbersep=5pt,
    frame=single,
    breaklines=true,
    breakatwhitespace=true,
    columns=fullflexible,
    showstringspaces=false,
    language=Python,
    inputencoding=utf8,
    captionpos=b
}

%------------------------------------------------------------------------------
% 各種設定
%------------------------------------------------------------------------------

% --- 和文フォント設定 ---
\setmainjfont[Renderer=HarfBuzz]{Yu Mincho}
\setsansjfont[Renderer=HarfBuzz]{Yu Gothic}

% --- ドキュメント情報 ---
\title{画像処理・画像処理工学 レポート課題1}
\author{クラス: 〇〇 番号: 〇〇 氏名: 〇〇〇〇} % 提出要領に従い記入
\date{\today}

%==============================================================================
% ドキュメント本体 (Body)
%==============================================================================
\begin{document}

\maketitle
\thispagestyle{empty}
\clearpage

\section{問題1}

\subsection{複合レンズによる像の位置と大きさ}

\subsubsection{計算・導出過程}
\paragraph{凸レンズAによる中間像}
レンズの公式 $\frac{1}{a} + \frac{1}{b} = \frac{1}{f}$ より,中間像の位置 $b_A$ を求める.
物体距離 $a_A = \SI{16}{cm}$,焦点距離 $f_A = \SI{12}{cm}$ を代入すると以下の通りとなる.
\[
\frac{1}{16} + \frac{1}{b_A} = \frac{1}{12}
\]
\[
\frac{1}{b_A} = \frac{1}{12} - \frac{1}{16} = \frac{4}{48} - \frac{3}{48} = \frac{1}{48}
\]
\[
b_A = \SI{48}{cm}
\]
倍率 $m_A$ は以下の通りである.
\[
m_A = -\frac{b_A}{a_A} = -\frac{48}{16} = -3
\]
物体の大きさ $h = \SI{2.0}{cm}$ より,中間像の大きさ $h_A'$ は次式となる.
\[
h_A' = h \times |m_A| = 2.0 \times 3 = \SI{6.0}{cm}
\]

\paragraph{凸レンズBによる最終像}
中間像を凸レンズBの物体とみなす.
AB間の距離は $\SI{63}{cm}$ であるため,凸レンズBにおける物体距離 $a_B$ は以下となる.
\[
a_B = 63 - b_A = 63 - 48 = \SI{15}{cm}
\]
凸レンズBの焦点距離 $f_B = \SI{10}{cm}$ より,最終像の位置 $b_B$ を求める.
\[
\frac{1}{15} + \frac{1}{b_B} = \frac{1}{10}
\]
\[
\frac{1}{b_B} = \frac{1}{10} - \frac{1}{15} = \frac{3}{30} - \frac{2}{30} = \frac{1}{30}
\]
\[
b_B = \SI{30}{cm}
\]
倍率 $m_B$ は以下の通りである.
\[
m_B = -\frac{b_B}{a_B} = -\frac{30}{15} = -2
\]
中間像の大きさ $h_A' = \SI{6.0}{cm}$ より,最終像の大きさ $h_B'$ は次式となる.
\[
h_B' = h_A' \times |m_B| = 6.0 \times 2 = \SI{12.0}{cm}
\]

\subsubsection*{結果}
\begin{itemize}
    \item 像の位置: 凸レンズBの後方 \SI{30}{cm}
    \item 像の大きさ: \SI{12.0}{cm}
\end{itemize}

\begin{center}\rule{0.5\linewidth}{0.4pt}\end{center}

\subsection{各デバイスのppi計算と比較}

\subsubsection{計算・導出過程}
ppiは,画面の画素数 ($w_p, h_p$) および対角線長 $d_i$ (インチ) より次式で定義される.
\[
\text{ppi} = \frac{\sqrt{w_p^2 + h_p^2}}{d_i}
\]

\paragraph{A. Google Pixel 10} ($w_p=1080$, $h_p=2424$, $d_i=6.3$)
\[
\text{ppi} = \frac{\sqrt{1080^2 + 2424^2}}{6.3} = \frac{\sqrt{1166400 + 5875776}}{6.3} \approx \frac{2653.7}{6.3} \approx 421.22
\]

\paragraph{B. iPad (A16)} ($w_p=2360$, $h_p=1640$, $d_i=10.9$)
\[
\text{ppi} = \frac{\sqrt{2360^2 + 1640^2}}{10.9} = \frac{\sqrt{5569600 + 2689600}}{10.9} \approx \frac{2873.9}{10.9} \approx 263.66
\]

\paragraph{C. EIZO EV2740S} ($w_p=3840$, $h_p=2160$, $d_i=27.0$)
\[
\text{ppi} = \frac{\sqrt{3840^2 + 2160^2}}{27.0} = \frac{\sqrt{14745600 + 4665600}}{27.0} \approx \frac{4405.8}{27.0} \approx 163.18
\]

\subsubsection*{結果}
各デバイスのppi(整数値)を以下に示す.
\begin{center}
    \begin{tabular}{lr}
        \toprule
        デバイス名 & ppi \\
        \midrule
        Google Pixel 10 & 421 ppi \\
        iPad (A16)      & 264 ppi \\
        EIZO EV2740S    & 163 ppi \\
        \bottomrule
    \end{tabular}
\end{center}

\subsection{8近傍鮮鋭化フィルタの導出}

\subsubsection{導出過程}
鮮鋭化フィルタ $W_S$ は,単位オペレータ $W_I$ とラプラシアンフィルタ $W_L$ を用いて次式で表される.
\[
W_S = W_I - W_L
\]
以下,ラプラシアンフィルタ $W_L$ を2次差分より導出する.

\paragraph{各方向の2次差分}
注目画素を $f(i, j)$ とし,隣接画素との差分(1次微分)の差分により2次微分を近似する.
\begin{itemize}
    \item \textbf{x方向2次差分 ($f_{xx}$)}
        \[
        f_{xx} = f(i+1, j) - 2f(i, j) + f(i-1, j)
        \]
    \item \textbf{y方向2次差分 ($f_{yy}$)}
        \[
        f_{yy} = f(i, j+1) - 2f(i, j) + f(i, j-1)
        \]
    \item \textbf{斜め方向1 (左上-右下) 2次差分 ($f_{d1}$)}
        \[
        f_{d1} = f(i+1, j+1) - 2f(i, j) + f(i-1, j-1)
        \]
    \item \textbf{斜め方向2 (右上-左下) 2次差分 ($f_{d2}$)}
        \[
        f_{d2} = f(i-1, j+1) - 2f(i, j) + f(i+1, j-1)
        \]
\end{itemize}

\paragraph{8近傍ラプラシアンフィルタ ($W_L$)}
8近傍ラプラシアン $\nabla^2 f$ は,上記4方向の2次差分の総和と定義される.
\[
\nabla^2 f = f_{xx} + f_{yy} + f_{d1} + f_{d2}
\]
式を整理すると,注目画素 $f(i,j)$ の係数は-8,周囲8画素の係数は1となるため,以下のカーネルが得られる.
\[
W_L =
\begin{pmatrix}
1 & 1 & 1 \\
1 & -8 & 1 \\
1 & 1 & 1
\end{pmatrix}
\]

\paragraph{8近傍鮮鋭化フィルタ ($W_S$)}
$W_S = W_I - W_L$ より計算する.
\[
W_S =
\begin{pmatrix}
0 & 0 & 0 \\
0 & 1 & 0 \\
0 & 0 & 0
\end{pmatrix}
-
\begin{pmatrix}
1 & 1 & 1 \\
1 & -8 & 1 \\
1 & 1 & 1
\end{pmatrix}
=
\begin{pmatrix}
-1 & -1 & -1 \\
-1 & 9 & -1 \\
-1 & -1 & -1
\end{pmatrix}
\]

\subsubsection*{結果}
\[
W_S =
\begin{pmatrix}
-1 & -1 & -1 \\
-1 & 9 & -1 \\
-1 & -1 & -1
\end{pmatrix}
\]

\subsection{Sobelフィルタによるエッジ解析}

\subsubsection{計算・導出過程}
Sobelオペレータを $S_x, S_y$ とし,エッジ強度 $G$ と方向 $\theta$ を以下で定義する.
\[
G = \sqrt{G_x^2 + G_y^2}, \quad \theta = \arctan\left(\frac{G_y}{G_x}\right)
\]

\paragraph{注目画素 A}
\[
G_x = 200, \quad G_y = 200
\]
\[
G_A = \sqrt{200^2 + 200^2} \approx 282.84, \quad \theta_A = \arctan(1) = 45^{\circ}
\]

\paragraph{注目画素 B}
\[
G_x = 400, \quad G_y = -400
\]
\[
G_B = \sqrt{400^2 + (-400)^2} \approx 565.69, \quad \theta_B = \arctan(-1) = -45^{\circ}
\]

\subsubsection*{結果と考察}
\begin{itemize}
    \item \textbf{エッジ強度}: 画素Bの強度はAの約2倍であり,B周辺の方がより急峻な輝度変化を有している.
    \item \textbf{エッジ方向}: Aは$45^{\circ}$,Bは$-45^{\circ}$方向へのエッジであり,傾きが直交に近い関係にある.
\end{itemize}

\clearpage

\section{問題2}

\subsection{線形変換関数による濃度変換}

\paragraph{理論}
図A-4より,入力濃度 $f$ と出力濃度 $g$ の関係式を導出する.
グラフは点 $(0, 15)$ と 点 $(200, 255)$ を結ぶ直線,および $f > 200$ における定数値である.
直線の傾き $a$ は以下の通りとなる.
\[
a = \frac{255 - 15}{200 - 0} = \frac{240}{200} = 1.2
\]
したがって,求める線形変換関数は次式で表される.
\[
g(f) = 
\begin{cases}
1.2 f + 15 & (0 \le f \le 200) \\
255 & (200 < f \le 255)
\end{cases}
\]

\paragraph{結果}
「gray\_image.png」に対し上記変換を適用した結果を図\ref{fig:task2-1}に示す.

\begin{figure}[H]
    \centering
    \includegraphics[width=0.85\linewidth]{transformed_image.png}
    \caption{線形変換の結果 (左: 元画像, 右: 変換後画像) および各ヒストグラム}
    \label{fig:task2-1}
\end{figure}

\paragraph{考察}
ヒストグラムを確認すると,元画像では0付近にあった低輝度の画素が,変換後には15付近にシフトしている.また,輝度分布の幅が広がり(コントラストが強調され),全体的に画像が明るくなっている.これは,変換式の傾きが1.2(増幅)であり,かつ切片15(オフセット)が加算されていることと整合する.

\subsection{輝度反転}

\paragraph{理論}
8bit濃淡画像の輝度反転における濃度変換関数は,入力 $f$ が0のとき出力 $g$ が255,入力255のとき出力0となる直線である.
\[
g(f) = -f + 255 \quad (= 255 - f)
\]
グラフは傾き-1,切片255の右下がりの直線となる.

\paragraph{結果}
「building.png」に対し輝度反転を適用した結果を図\ref{fig:task2-2}に示す.

\begin{figure}[H]
    \centering
    \includegraphics[width=0.85\linewidth]{inverted_image.png}
    \caption{輝度反転の結果とヒストグラム}
    \label{fig:task2-2}
\end{figure}

\paragraph{考察}
画像の明暗が完全に反転している.ヒストグラムにおいても,分布の形状が左右(高輝度・低輝度)で反転していることが確認できる.これは変換式 $g = 255 - f$ により,画素値の大小関係が逆転した結果である.

\subsection{空間フィルタリングによるノイズ除去}

\paragraph{使用したフィルタ}
\begin{itemize}
    \item \textbf{種類}: メディアンフィルタ (Median Filter)
    \item \textbf{サイズ}: $5 \times 5$
    \item \textbf{選定理由}: 元画像「noisy\_image.png」には,白黒の斑点状のノイズ(ごま塩ノイズ/インパルスノイズ)が確認された.メディアンフィルタは注目画素周辺の中央値を採用するため,平均化フィルタ等と比較して,このような極端な値を持つ外れ値(ノイズ)の除去能力が高く,かつエッジを保存しやすい特性があるため採用した.
\end{itemize}

\paragraph{結果}
ノイズ除去の結果を図\ref{fig:task2-3}に示す.

\begin{figure}[H]
    \centering
    \includegraphics[width=0.85\linewidth]{denoised_image.png}
    \caption{メディアンフィルタによるノイズ除去結果}
    \label{fig:task2-3}
\end{figure}

\paragraph{考察}
処理後の画像では,建物の輪郭などのエッジ情報を大きく損なうことなく,ごま塩ノイズが良好に除去されている.これにより,後段のエッジ検出処理に適した画像が得られたといえる.

\subsection{Canny法によるエッジ検出}

\paragraph{処理概要}
前項でノイズ除去を行った画像に対し,Cannyのエッジ検出アルゴリズムを適用した.
物体の輪郭を適切に抽出するため,ヒステリシスしきい値処理における2つのしきい値を調整した.

\paragraph{結果}
\begin{itemize}
    \item \textbf{設定したしきい値}: threshold1 = 100, threshold2 = 130
\end{itemize}
エッジ検出の結果を図\ref{fig:task2-4}に示す.

\begin{figure}[H]
    \centering
    \includegraphics[width=0.85\linewidth]{canny_edge_image.png}
    \caption{Canny法によるエッジ検出結果}
    \label{fig:task2-4}
\end{figure}

\paragraph{考察}
適切なしきい値設定を行うことで,建物の主要な輪郭線や窓の枠組みが明瞭に検出された.しきい値をこれより低く設定するとノイズによる微細な線が多数検出され,高く設定すると建物の輪郭が途切れる現象が確認されたため,上記の値が最適であると判断した.

\clearpage
\section*{付録: プログラムリスト}
本レポートの課題2で使用したPythonプログラムを以下に示す.

\lstinputlisting[caption={画像処理プログラム (report\_image\_analysis3.py)}, label={lst:code}]{report_image_analysis3.py}

\end{document}